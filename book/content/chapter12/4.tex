
上一个示例中,介绍了概念和约束的话题,借助几个完全基于现有类型特征的例子了解了它们。还使用了更简洁的语法来指定概念,即在模板声明中使用概念名称替代 typename 或 class 关键字。借助 requires 表达式,可以定义更复杂的概念。这些表达式是 bool 类型的 prvalue,描述了一些模板参数上的约束。

本示例中,将介绍如何编写 requires 表达式,以及指定模板参数约束的另一种方法。

\mySubsubsection{}{Getting ready}

本示例中呈现的代码,将使用在上一个示例中定义的类模板 NumericalValue<T> 和函数模板 wrap()。

\mySubsubsection{}{How to do it...}

为了指定模板参数的要求,可以使用 requires 关键字引入的 requires 表达式:

\begin{itemize}
\item
使用编译器验证正确性的简单表达式。下面的代码中,操作符 + 必须为 T 模板参数重载:

\begin{cpp}
template <typename T>
concept Addable = requires (T a, T b) {a + b;};
template <Addable T>
T add(T a, T b)
{
    return a + b;
}
add(1, 2);        // OK, integers
add("1"s, "2"s);  // OK, std::string user-defined literals
NumericalValue<int> a{1};
NumericalValue<int> b{2};
add(a, b); // error: no matching function for call to 'add'
// 'NumericalValue<int>' does not satisfy 'Addable'

\end{cpp}

\item
使用简单的表达式来要求特定函数的存在。在下面的代码中,必须存在一个以 T 模板参数作为参数的 wrap() 函数:

\begin{cpp}
template <typename T>
concept Wrapable = requires(T x) { wrap(x); };
template <Wrapable T>
void do_wrap(T x)
{
    [[maybe_unused]] auto v = wrap(x);
}
do_wrap(42);    // OK, can wrap an int
do_wrap(42.0);  // OK, can wrap a double
do_wrap("42"s); // error, cannot wrap a std::string
\end{cpp}

\item
使用 typename 关键字后跟类型名称(可选限定符)指定类型要求,以指定成员名称、类模板特化或别名模板替换等要求。下面的代码中,T 模板参数必须具有两个内部类型,分别称为 \verb|value_type| 和 iterator。此外,必须提供接受 T 参数的 begin() 和 end() 函数:

\begin{cpp}
template <typename T>
concept Container = requires(T x)
{
    typename T::value_type;
    typename T::iterator;
    begin(x);
    end(x);
};
template <Container T>
void pass_container(T const & c)
{
    for(auto const & x : c)
        std::cout << x << '\n';
}
std::vector<int> v { 1, 2, 3};
std::array<int, 3> a {1, 2, 3};
int arr[] {1,2,3};
pass_container(v);   // OK
pass_container(a);   // OK
pass_container(arr); // error: 'int [3]' does not satisfy
// 'Container'
\end{cpp}

\item
使用复合要求来指定表达式的要求及其求值结果。下面的示例中,必须存在一个可以接受 T 模板参数类型的 wrap() 函数,而且调用该函数的结果必须是 NumericalValue<T> 类型:

\begin{cpp}
template <typename T>
concept NumericalWrapable =
requires(T x)
{
    {wrap(x)} -> std::same_as<NumericalValue<T>>;
};
template <NumericalWrapable T>
void do_wrap_numerical(T x)
{
    [[maybe_unused]] auto v = wrap(x);
}
template <typename T>
class any_wrapper
{
    public:
    T value;
};
any_wrapper<std::string> wrap(std::string s)
{
    return any_wrapper<std::string>{s};
}
// OK, wrap(int) returns NumericalValue<int>
do_wrap_numerical(42);
// error, wrap(string) returns any_wrapper<string>
do_wrap_numerical("42"s);
\end{cpp}
\end{itemize}

模板参数上的约束也可以使用涉及 requires 关键字的语法指定,这称为 requires 子句:

\begin{itemize}
\item
模板参数列表之后使用 requires 子句:

\begin{cpp}
template <typename T> requires Addable<T>
T add(T a, T b)
{
    return a + b;
}
\end{cpp}

\item
或者,在函数声明符的最后一个元素之后使用 requires 子句:

\begin{cpp}
template <typename T>
T add(T a, T b) requires Addable<T>
{
    return a + b;
}
\end{cpp}

\item
将 requires 子句与 requires 表达式结合使用,而不是使用命名概念。这时,requires 关键字会出现两次:

\begin{cpp}
template <typename T>
T add(T a, T b) requires requires (T a, T b) {a + b;}
{
    return a + b;
}
\end{cpp}
\end{itemize}

\mySubsubsection{}{How it works...}

新 requires 关键字有多种用途。一方面,引入一个指定模板参数约束的 requires 子句。另一方面,定义一个 bool 类型的 prvalue 形式的 requires 表达式,用于定义模板参数上的约束。

\begin{myNotic}
如果不熟悉 C++ 的值类别(lvalue、rvalue、prvalue、xvalue、glvalue),建议查阅 \url{https://en.cppreference.com/w/cpp/language/value_category}。术语 prvalue 指的是纯 rvalue,不是 xvalue(即将到期的值)。prvalue 的例子包括字面量、返回类型不是引用类型的函数调用、枚举器或 this 指针。
\end{myNotic}

在 requires 子句中,requires 关键字后面必须跟着一个 bool 类型的常量表达式。表达式必须是一个主表达式(如 \verb|std::is_arithmetic_v<T>| 或 \verb|std::integral<T>|)、括号内的表达式,或者是用 \verb|&&| 或 || 操作符连接的此类表达式序列。

requires 表达式的形式为 \verb|requires (parameters-list) { requirements }|。参数列表是可选的,可以完全省略(包括括号)。指定的要求可能涉及:

\begin{itemize}
\item
范围内的模板参数

\item
参数列表中引入的局部参数

\item
从封闭上下文可见的任何其他声明
\end{itemize}

requires 表达式的要求序列可以包含以下类型的要求:

\begin{itemize}
\item
简单要求:这些不以 requires 关键字开头的表达式。编译器仅检查其语言正确性。

\item
类型要求:这些表达式以关键字 typename 开头,后跟一个类型名称,该名称必须有效。这使得编译器能够验证某个嵌套名称是否存在,或者某个类模板特化或别名模板替换是否存在。

\item
复合要求:形式为 \verb|{expression} noexcept -> type-constraint|。noexcept 关键字可选,如果不存在,则表达式不得可能抛出异常。返回类型要求(以 -> 引入)也是可选的。但如果存在,则 decltype(expression) 必须满足 type-constraint 施加的约束。

\item
嵌套要求:这些是更复杂的表达式,定义为 requires 表达式,后者可以是另一个嵌套要求。以 requires 关键字开头的要求被认为是嵌套要求。
\end{itemize}

计算之前,每个命名概念和每个 requires 表达式都会进行替换,直到获得一系列原子约束的合取和析取,这一过程称为规范化。实际的规范化细节和编译器执行的分析则超出了本书的范畴。

