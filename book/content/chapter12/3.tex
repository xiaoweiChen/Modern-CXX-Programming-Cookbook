

模板元编程是C++语言的重要组成部分,促进了通用库(包括标准库)的发展,但模板元编程并不简单。相反,复杂的任务在没有大量经验的情况下,可能乏味且难以正确完成。实际上,《C++核心指南》——由Bjarne Stroustrup和Herb Sutter发起的一项倡议——有一条规则叫做“仅在需要时才使用模板元编程”,其理由是:

\begin{myTip}
模板元编程很难正确掌握,会减慢编译速度,并且往往非常难以维护。
\end{myTip}

关于模板元编程的一个重要方面是类型模板参数的约束规范,以便对可用于实例化模板的类型施加限制。C++20概念库旨在解决这个问题。概念是一组约束的名称,而约束是模板参数的要求,用于选择适当的函数重载和模板特化。

本示例中,将介绍如何使用C++20概念来指定模板参数的要求。

\mySubsubsection{}{Getting ready}

开始学习概念之前,考虑以下名为 NumericalValue 的类型模板,该模板应该保存一个整数或浮点类型的值。C++11使用了 \verb|std::enable_if| 来指定对 T 模板参数的要求:

\begin{cpp}
template <typename T>,
          typename = typename std::enable_if_t<std::is_arithmetic_v<T>, T>>
struct NumericalValue
{
    T value;
};
template <typename T>
NumericalValue<T> wrap(T value) { return { value }; }
template <typename T>
T unwrap(NumericalValue<T> t) { return t.value; }
auto nv = wrap(42);
std::cout << nv.value << '\n';   // prints 42
auto v = unwrap(nv);
std::cout << v << '\n';          // prints 42
using namespace std::string_literals;
auto ns = wrap("42"s);           // error
\end{cpp}

这段代码将成为本示例中的基础。

\mySubsubsection{}{How to do it...}

可以按照以下方式指定模板参数的要求:

\begin{itemize}
\item
使用 concept 关键字创建概念:

\begin{cpp}
template <class T>
concept Numerical = std::is_arithmetic_v<T>;
\end{cpp}

\item
或者,可以使用 <concepts> 头文件(或标准库的其他头文件之一)中定义的标准概念:

\begin{cpp}
template <class T>
concept Numerical = std::integral<T> || std::floating_point<T>;
\end{cpp}

\item
函数模板、类型模板或变量模板中使用概念名称代替 class 或 typename 关键字:

\begin{cpp}
template <Numerical T>
struct NumericalValue
{
    T value;
};
template <Numerical T>
NumericalValue<T> wrap(T value) { return { value }; }
template <Numerical T>
T unwrap(NumericalValue<T> t) { return t.value; }
\end{cpp}

\item
使用相同的语法实例化类型模板和调用函数模板:

\begin{cpp}
auto nv = wrap(42);
std::cout << nv.value << '\n';   // prints 42
auto v = unwrap(nv);
std::cout << v << '\n';          // prints 42
using namespace std::string_literals;
auto ns = wrap("42"s);           // error
\end{cpp}
\end{itemize}

\mySubsubsection{}{How it works...}

概念是在命名空间作用域中定义的一组一个或多个约束,概念的定义类似于变量模板。下面的代码展示了概念如何用于变量模板:

\begin{cpp}
template <class T>
concept Real = std::is_floating_point_v<T>;
template<Real T>
constexpr T pi = T(3.1415926535897932385L);
std::cout << pi<double> << '\n';
std::cout << pi<int>    << '\n'; // error
\end{cpp}

概念本身不能约束,也不能递归引用自身。迄今为止的示例中,Numerical 和 Real 概念由单个原子约束组成。然而,概念可以从多个约束中创建。使用逻辑操作符 \verb|&&| 创建的两个约束称为合取,而使用 || 创建的两个约束称为析取。

“How to do it...”部分中定义的 Numerical 概念使用了 \verb|std::is_arithmetic_v| 类型特征,但可以有两个概念,Real 和 Integral:

\begin{cpp}
template <class T>
concept Integral = std::is_integral_v<T>;
template <class T>
concept Real = std::is_floating_point_v<T>;
\end{cpp}

从这两个概念起,可以使用逻辑操作符 || 组合成 Numerical 概念。结果是一个析取:

\begin{cpp}
template <class T>
concept Numerical = Integral<T> || Real<T>;
\end{cpp}

从语义上讲,这两个版本的 Numerical 概念之间没有区别,尽管以不同的方式定义。

为了理解合取,来看另一个例子。考虑两个基类 IComparableToInt 和 IConvertibleToInt,应该是由支持与 int 比较或转换为 int 类型的派生类型:

\begin{cpp}
struct IComparableToInt
{
    virtual bool CompareTo(int const o) = 0;
};
struct IConvertibleToInt
{
    virtual int ConvertTo() = 0;
};
\end{cpp}

一些类可以同时实现这两个接口,而另一些类只能实现其中一个。SmartNumericalValue<T> 类型就是这种实现,而 DullNumericalValue<T> 类型只实现了 IConvertibleToInt 接口:

\begin{cpp}
template <typename T>
struct SmartNumericalValue : public IComparableToint, IConvertibleToInt
{
    T value;
    SmartNumericalValue(T v) :value(v) {}
    bool CompareTo(int const o) override
    { return static_cast<int>(value) == o; }
    int ConvertTo() override
    { return static_cast<int>(value); }
};
template <typename T>
struct DullNumericalValue : public IConvertibleToInt
{
    T value;
    DullNumericalValue(T v) :value(v) {}
    int ConvertTo() override
    { return static_cast<int>(value); }
};
\end{cpp}

想要做的是编写一个函数模板,该模板只接受既可比较又可以转换为 int 的参数。这里展示的 IntComparableAndConvertible 概念是 IntComparable 和 IntConvertible 概念的合取。可以这样实现:

\begin{cpp}
template <class T>
concept IntComparable = std::is_base_of_v<IComparableToInt, T>;
template <class T>
concept IntConvertible = std::is_base_of_v<IConvertibleToInt, T>;
template <class T>
concept IntComparableAndConvertible = IntComparable<T> && IntConvertible<T>;
template <IntComparableAndConvertible T>
void print(T o)
{
    std::cout << o.value << '\n';
}
int main()
{
    auto snv = SmartNumericalValue<double>{ 42.0 };
    print(snv);                      // prints 42
    auto dnv = DullNumericalValue<short>{ 42 };
    print(dnv);                      // error
}
\end{cpp}

合取和析取是从左到右评估并且短路的。对于合取,只有当左边的约束满足时,右边的约束才会计算;而对于析取,只有当左边的约束不满足时,右边的约束才会计算。

\begin{myNotic}
约束的第三种类别是原子约束,由表达式 E 和从 E 中的类型参数到受限实体的模板参数之间的映射组成,称为参数映射。原子约束是在约束规范化过程中形成的,这是将约束表达式转换为原子约束的合取和析取序列的过程。原子约束通过将参数映射和模板参数代入表达式 E 来检查。结果必须是一个有效的 bool 类型的 prvalue 常量表达式;否则,不满足约束。
\end{myNotic}

标准库定义了一系列用于定义模板参数编译时要求的概念。虽然大多数概念都施加了语法和语义要求,但编译器通常只能确保前者。当语义要求未得到满足时,程序视为非法,编译器不必提供有关问题的诊断信息。标准概念在以下几个地方可用:

\begin{itemize}
\item
在 <concepts> 头文件和 std 命名空间中的概念库中。这包括核心语言概念(如 \verb|same_as|、\verb|integral|、\verb|floating_point|、\verb|copy_constructible| 和 \verb|move_constructible|)、比较概念(如 \verb|equality_comparable| 和 \verb|totally_ordered|)、对象概念(如可复制、可移动和常规),以及可调用概念(如可调用和谓词)。

\item
在 <iterator> 头文件和 std 命名空间中的算法库中。这包括算法要求(如可排序、可排列和可合并)和间接可调用概念(如 \verb|indirect_unary_predicate| 和 \verb|indirect_binary_predicate|)。

\item
在 <ranges> 头文件和 std::ranges 命名空间中的范围库中。这包括特定于范围的概念,如 \verb|range|、\verb|view|、\verb|input_range|、\verb|output_range|、\verb|forward_range| 和 \verb|random_access_range|。
\end{itemize}

\mySubsubsection{}{There's more...}

本示例中定义的概念使用了已有的类型特征。许多情况下,模板参数的要求无法用这种方式描述。为此,概念可以使用 requires 表达式定义,这是一种 bool 类型的 prvalue 表达式,描述了模板参数的要求,这将是下一个示例的主题。


