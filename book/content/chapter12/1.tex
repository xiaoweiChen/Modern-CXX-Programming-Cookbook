
模块是C++20标准中最重要的一项变化,代表了对C++语言及其编写和使用方式的根本性改变。模块在单独编译的源文件中提供,使用模块的翻译单元中编译。

模块提供了多项优势,尤其是在与头文件的使用相比:

\begin{itemize}
\item
只需导入一次,且导入顺序无关紧要。

\item
不需要将接口和实现分割到不同的源文件中(这仍然可能)。

\item
模块可减少编译时间,某些情况下减少显著。从模块导出的具体实现在一个二进制文件中,编译器可以比传统预编译头文件更快地处理这个文件。

\item
此外,该文件有可能用于构建与其他语言中的C++代码的集成和互操作性。
\end{itemize}

本示例中,将介绍如何使用模块。

\mySubsubsection{}{Getting ready}

撰写本文时,主要编译器(VC++、Clang和GCC)对模块的支持程度不同,构建系统(如CMake)在采用模块方面滞后(这种情况很可能在未来发生改变)。因为不同的编译器有不同的支持方式和不同的编译选项,本书不会提供如何构建这些示例的详细信息,建议查阅特定编译器的在线文档。

\begin{myNotic}
随书提供的源代码包含使用Visual Studio 2019 16.8和Visual Studio 2022 17.x中的MSVC编译器(cl.exe)构建此示例及下一个配方中源码的脚本。
\end{myNotic}

模块文件有几种类型:模块接口单元、模块接口分区和模块实现分区。本示例中,仅讨论第一种;其他两种将在下一个示例中介绍。

\mySubsubsection{}{How to do it...}

当模块化代码时,可以执行以下操作:

\begin{itemize}
\item
使用import指令后跟模块名来导入模块。标准库可在std模块中找到,但这只有从C++23开始才可用。可以在C++23中编写如下代码:

\begin{cpp}
import std;
int main()
{
    std::println("Hello, world!");
}
\end{cpp}

C++23之前,标准库可能作为编译器特定的模块提供。下面的代码使用了Visual C++的std.core模块,包含了标准库的大部分功能,包括流库:

\begin{cpp}
import std.core;
int main()
{
    std::cout << "Hello, World!\n";
}
\end{cpp}

\item
创建一个模块接口单元(MIU)以导出模块,可以包含函数、类型、常量甚至宏。声明必须以关键字export开头。对于VC++,模块接口单元文件必须具有.ixx扩展名。Clang接受不同的扩展名,包括.cpp、.cppm甚至是.ixx。以下示例导出了一个名为point的类模板、一个计算两点之间距离的函数distance(),以及一个用户定义的字面值操作符\_ip,该操作符可以从形如“0,0”或“12,-3”的字符串创建point类型的对象:

\begin{cpp}
// --- geometry.ixx/.cppm ---
export module geometry;
#ifdef __cpp_lib_modules
import std;
#else
import std.core;
#endif
export template <class T,
    typename = typename std::enable_if_t<std::is_arithmetic_v<T>, T>>
struct point
{
    T x;
    T y;
};
export using int_point = point<int>;
export constexpr int_point int_point_zero{ 0,0 };
export template <class T>
double distance(point<T> const& p1, point<T> const& p2)
{
    return std::sqrt((p2.x - p1.x) * (p2.x - p1.x) +
                     (p2.y - p1.y) * (p2.y - p1.y));
}
namespace geometry_literals
{
    export int_point operator ""_ip(const char* ptr, std::size_t size)
    {
        int x = 0, y = 0;
        if(ptr)
        {
            while (*ptr != ',' && *ptr != ' ')
                x = x * 10 + (*ptr++ - '0');
            while (*ptr == ',' || *ptr == ' ') ptr++;
            while (*ptr != 0)
                y = y * 10 + (*ptr++ - '0');
        }
        return { x, y };
    }
}
// --- main.cpp ---
#ifdef __cpp_lib_modules
import std;
#else
import std.core;
#endif
import geometry;
int main()
{
    int_point p{ 3, 4 };
    std::cout << distance(int_point_zero, p) << '\n';
    {
        using namespace geometry_literals;
        std::cout << distance("0,0"_ip, "30,40"_ip) << '\n';
    }
}
\end{cpp}

\item
使用import指令还可以导入头文件的内容。这里展示的例子使用了前面示例中相同的类型和函数:

\begin{cpp}
// --- geometry.h ---
#pragma once
#include <cmath>
template <class T,
    typename = typename std::enable_if_t<std::is_arithmetic_v<T>, T>>
struct point
{
    T x;
    T y;
};
using int_point = point<int>;
constexpr int_point int_point_zero{ 0,0 };
template <class T>
double distance(point<T> const& p1, point<T> const& p2)
{
    return std::sqrt((p2.x – p1.x) * (p2.x – p1.x) +
                     (p2.y – p1.y) * (p2.y – p1.y));
}
namespace geometry_literals
{
    int_point operator ""_ip(const char* ptr, std::size_t)
    {
        int x = 0, y = 0;
        if(ptr)
        {
            while (*ptr != ',' && *ptr != ' ')
                x = x * 10 + (*ptr++ - '0');
            while (*ptr == ',' || *ptr == ' ') ptr++;
            while (*ptr != 0)
                y = y * 10 + (*ptr++ - '0');
        }
        return { x, y };
    }
}
// --- main.cpp ---
#ifdef __cpp_lib_modules
import std;
#else
import std.core;
#endif
import "geometry.h";
int main()
{
    int_point p{ 3, 4 };
    std::cout << distance(int_point_zero, p) << '\n';
    {
        using namespace geometry_literals;
        std::cout << distance("0,0"_ip, "30,40"_ip) << '\n';
    }
}
\end{cpp}
\end{itemize}

\mySubsubsection{}{How it works...}

模块单元由几个部分组成,这些部分可以是必需的或是可选的:

\begin{itemize}
\item
全局模块,通过module;引入。这一部分可选,如果存在,则只能包含预处理器指令。在此添加的所有内容都属于全局模块,这是一个包含所有全局模块和所有非模块翻译单元的集合。

\item
模块声明,这是必需的形式为export module name;语句。

\item
模块前言,这是可选的,只能包含导入声明。

\item
模块范围,这是单元的内容,从模块声明开始,一直延伸到模块单元的末尾。
\end{itemize}

下图显示了一个包含上述所有部分的模块单元。左侧是模块的源码,右侧解释了各个模块:

\myGraphic{0.8}{content/chapter12/images/1.png}{图12.1:模块示例(在左侧),每个部分都突出显示并进行了说明(在右侧)}

模块可以导出实体,如函数、类和常量。每个导出都必须以export关键字开头。这个关键字总是第一个关键字,位于其他关键字(如class/struct、template或using)之前。在上一节所示的geometry模块中提供了几个例子:

\begin{itemize}
\item
一个表示二维空间中点的类模板point

\item
一个point<int>的类型别名int\_point

\item
一个编译时常量int\_point\_zero

\item
一个计算两个点之间距离的函数模板distance()

\item
一个用户定义的字面值操作符\_ip,可以从像“3,4”这样的字符串创建int\_point实例
\end{itemize}

使用模块的翻译单元不需要任何更改,除了用import指令替换\#include预处理器指令。此外,也可以使用相同的import指令将头文件作为模块导入,如前所述的例子所示。

模块和命名空间之间没有关系。这两个概念是独立的。geometry模块在geometry\_literals命名空间中导出了用户定义的字面值操作符""\_ip,而模块中的所有其他导出都在全局命名空间中可用。

模块名称与单元文件名之间也没有关系。虽然geometry模块是在名为geometry.ixx/.cppm的文件中定义的,但文件名都会有相同的结果。建议遵循一致的命名方案,并使用模块名称作为模块文件名。另一方面,用于模块单元的扩展名因编译器而异,尽管随着模块支持达到成熟阶段,这可能会有所变化。

C++23之前,标准库尚未模块化。然而,编译器已经以模块形式提供了标准库。Clang编译器为每个头文件提供了一个不同的模块。而Visual C++编译器则为标准库提供了以下模块:

\begin{itemize}
\item
std.regex: <regex>头文件的内容

\item
std.filesystem: <filesystem>头文件的内容

\item
std.memory: <memory>头文件的内容

\item
std.threading: <atomic>、<condition\_variable>、<future>、<mutex>、<shared\_mutex>和<thread>头文件的内容

\item
std.core: 其余的C++标准库内容
\end{itemize}

正如从 std.core 或 std.regex 这样的模块名称中看到的,模块的名称可以是一系列由点(.)连接的标识符。点的作用只是帮助将名称分割成表示逻辑层次结构的部分,例如 company.project.module。使用点号可以说比使用下划线(如 \verb|std_core| 或 \verb|std_regex|)提供更好的可读性,后者也是合法的。

另一方面,C++23 标准提供了两个标准化的命名模块:

\begin{itemize}
\item
std,将 C++ 标准头文件(如 <vector>、<string>、<algorithm> 等)和 C 语言包装头文件(如 <cstdio>)中的所有内容导入 std 命名空间。如果想使用 std 限定所有内容,而不污染全局命名空间,应该使用这个模块。

\item
std.compat,导入 std 导入的所有内容,并且还包括 C 语言包装头文件的全局命名空间对应部分。例如,如果 std 从 <cstdio> 导入 std::fopen 和 std::fclose(以及其他所有内容),那么 std.compat 将导入 ::fopen 和 ::fclose。如果希望更方便迁移代码而无需用 std 命名空间来限定名称(如 std::fopen 而不是 fopen,\verb|std::size_t| 而不是 \verb|size_t| 等),应该使用这个模块。
\end{itemize}

作为一名开发者,熟悉编程语言中典型的入门程序——称为“Hello, world!”,简单地将这段文本输出到控制台。在 C++ 中,这个程序的传统形式如下:

\begin{cpp}
#include <iostream>
int main()
{
    std::cout << "Hello, world!\n";
}
\end{cpp}

C++23 中,借助对标准化模块的支持和文本格式库的新打印功能,这个程序可以看起来如下:

\begin{cpp}
import std;
int main()
{
    std::println("Hello, world!");
}
\end{cpp}

可以使用 \verb|__cpp_lib_modules| 特性宏检查标准模块 std 和 std.compat 是否可用。



