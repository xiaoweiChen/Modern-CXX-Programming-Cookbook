
将各种范围适配器应用于范围(例如容器)的结果是一个复杂且难以输入或记忆的类型。通常,会使用 auto 关键字来指示链式调用适配器的结果类型。范围是惰性求值的,只有在对其进行迭代时,才会计算并产生结果。然而,经常需要将一个或多个范围适配器的应用结果存储在一个容器中,比如vector或map。C++23 之前,这需要显式的编码。但是,C++23 提供了一个范围转换函数,称为 std::ranges::to,使得这项任务变得简单,还允许在不同的容器之间进行转换。这个示例中,将介绍如何使用。

\mySubsubsection{}{How to do it...}

可以使用 std::ranges::to 范围转换函数将范围转换为容器:

\begin{itemize}
\item
将范围转换为 std::vector:

\begin{cpp}
std::vector<int> numbers{ 1, 1, 2, 3, 5, 8, 13 };
std::vector<int> primes = numbers |
                          std::views::filter(is_prime) |
                          std::ranges::to<std::vector>();
std::ranges::copy(primes,
                  std::ostream_iterator<int>(std::cout, " "));
std::println("");
std::string text{ "server=demo123;db=optimus" };
auto parts = text |
             std::views::lazy_split(';') |
             std::ranges::to<std::vector<std::string>>();
std::ranges::copy(parts,
                  std::ostream_iterator<std::string>(std::cout, " "));
std::println("");
\end{cpp}

\item
将范围转换为映射类型,如 \verb|std::unordered_multimap|:

\begin{cpp}
std::vector<int> numbers{ 1, 1, 2, 3, 5, 8, 13 };
std::vector<std::string> words{"one", "two", "three", "four"};
auto zipped = std::views::zip(numbers, words) |
              std::ranges::to<
                std::unordered_multimap<int, std::string>>();
for (auto const [number, word] : zipped)
{
    std::println("{} = {}", number, word);
}
\end{cpp}

\item
将范围转换为 std::string:

\begin{cpp}
std::string text{ "server=demo123;db=optimus" };
std::string text2 = text |
                    std::views::stride(3) |
                    std::ranges::to<std::string>();
std::println("{}", text2);
\end{cpp}
\end{itemize}

\mySubsubsection{}{How it works...}

std::ranges::to 范围转换函数从 C++23 开始在 <ranges> 头文件中可用(使用特性测试宏 \verb|__cpp_lib_ranges_to_container| 来检查是否支持)。

尽管前面的例子显示了如何从一个范围转换到另一个范围,但 std::ranges::to 还可用于不同类型容器之间的转换,例如从数组到列表,或者从映射到成对的数组:

\begin{cpp}
std::vector<int> numbers{ 1, 1, 2, 3, 5, 8, 13 };
std::list<int> list = numbers | std::ranges::to<std::list>();
\end{cpp}

编译器可能能够推断出容器元素的类型,例如在上述代码中,只指定了 std::list,而非 std::list<int>。但存在一些情况,编译器无法推断出类型,除非显式地提供了完整类型,否则会出现编译错误:

\begin{cpp}
std::map<int, std::string> m{ {1, "one"}, {2, "two"}, {3, "three"} };
std::vector<std::pair<int, std::string>> words =
    m | rg::to<std::vector<std::pair<int, std::string>>>();
\end{cpp}

当使用管道 (|) 语法时,圆括号是必需的;否则,会遇到编译错误(这些错误往往难以理解):

\begin{cpp}
std::vector<int> v {1, 1, 2, 3, 5, 8};
auto r = v | std::ranges::to<std::vector>;   // error
\end{cpp}

正确的语法如下:

\begin{cpp}
auto r = v | std::ranges::to<std::vector>();  // OK
\end{cpp}


