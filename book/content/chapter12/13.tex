

C++20 标准包括了对标准库的两大更新:范围库(ranges library)和协程(coroutines)。关于后者,提供的支持是最小的。C++20 标准仅仅定义了一个构建协程的框架,像 libcoro 这样的库创建出来以提供实际的协程,如前两个示例中看到的任务(task)和生成器(generator)。C++23 标准引入了首个标准协程,称为 std::generator。这将范围和协程结合在一起,因为 std::generator 是一个表示同步协程生成器的视图。这是一个标准实现,对应于上一个示例。本示例中,来介绍其是如何工作的。

\begin{myNotic}
写作本文时,只有 GCC 14 支持这一标准协程。
\end{myNotic}

\mySubsubsection{}{How to do it...}

为了以惰性的方式生成一系列元素,编写一个协程:

\begin{itemize}
\item
std::generator<T> 作为返回类型。

\item
\verb|co_yield| 语句返回一个值。
\end{itemize}

\begin{cpp}
std::generator<int> iota(int start = 0, int step = 1) noexcept
{
    auto value = start;
    for (int i = 0;; ++i)
    {
        co_yield value;
        value += step;
    }
}
int main()
{
    for (auto i : iota())
    {
        std::cout << i << ' ';
        if (i >= 10) break;
    }
}
\end{cpp}

\mySubsubsection{}{How it works...}

std::generator 类型模板位于其自己的头文件 <generator> 中,派生自 \verb|std::ranges::view_interface|,因此是从协程(一种可中断的函数)计算中产生的元素的视图。该类定义如下:

\begin{cpp}
template<class Ref, class V = void, class Allocator = void>
class generator
    : public ranges::view_interface<generator<Ref, V, Allocator>>;
\end{cpp}

每当协程恢复并且 \verb|co_yield| 语句计算时,就会产生序列的一个新元素。以下是一个包含一系列 \verb|co_yield| 语句的例子(而不是一个循环)。总共来说,这个协程产生三个元素。如果 \verb|get_values()| 协程只计算一次,就只会产生一个元素。可称此为延迟计算:

\begin{cpp}
std::generator<int> get_values() noexcept
{
    co_yield 1;
    co_yield 2;
    co_yield 3;
}
\end{cpp}

std::generator 类型是一个同步生成器;协程只能使用 \verb|co_yield| 语句来返回值。不可能在协程内使用 \verb|co_await| 操作符,为此需要一个不同类型的生成器,而这样的生成器目前不可用。

另一个使用 std::generator 类型生成值序列的例子如下,可生成斐波那契数列。前一个示例中有相同的例子,唯一的改变是用 C++23 标准提供的 std::generator<int> 替换了自己编写的 generator<int>:

\begin{cpp}
std::generator<int> fibonacci() noexcept
{
    int a = 0, b = 1;
    while (true)
    {
        co_yield b;
        auto tmp = a;
        a = b;
        b += tmp;
    }
}
int main()
{
    int c = 1;
    for (auto i : fibonacci())
    {
        std::cout << i << ' ';
        if (++c > 10) break;
    }
}
\end{cpp}


