
上一个示例中,介绍了如何创建一个协程任务,该任务能够进行异步计算。使用了 \verb|co_await| 操作符来暂停执行直到恢复,以及 \verb|co_return| 关键字来完成执行并返回一个值。还有一个关键字 \verb|co_yield| 也可以定义一个函数为协程,会暂停协程的执行并返回一个值。这使得协程可以返回多个值,每次恢复时返回一个值。为了支持这一特性,需要另一种类型的协程,这种类型称为生成器(generator)。它就像一个流,以惰性的方式(当迭代时)产生一系列类型为 T 的值。本示例中,将介绍如何实现一个简单的生成器。

\mySubsubsection{}{Getting ready}

本示例的目标是创建一个生成器协程类型,能够编写如下所示的代码:

\begin{cpp}
generator<int> iota(int start = 0, int step = 1) noexcept
{
    auto value = start;
    for (int i = 0;; ++i)
    {
        co_yield value;
        value += step;
    }
}
generator<std::optional<int>> iota_n(
int start = 0, int step = 1,
int n = std::numeric_limits<int>::max()) noexcept
{
    auto value = start;
    for (int i = 0; i < n; ++i)
    {
        co_yield value;
        value += step;
    }
}
generator<int> fibonacci() noexcept
{
    int a = 0, b = 1;
    while (true)
    {
        co_yield b;
        auto tmp = a;
        a = b;
        b += tmp;
    }
}
int main()
{
    for (auto i : iota())
    {
        std::cout << i << ' ';
        if (i >= 10) break;
    }
    for (auto i : iota_n(0, 1, 10))
    {
        if (!i.has_value()) break;
        std::cout << i.value() << ' ';
    }
    int c = 1;
    for (auto i : fibonacci())
    {
        std::cout << i << ' ';
        if (++c > 10) break;
    }
}
\end{cpp}

\mySubsubsection{}{How to do it...}

要创建一个支持同步延迟生产值序列的生成器协程类型,请按照以下步骤操作:

\begin{itemize}
\item
创建一个名为 generator 的类型模板,包含以下内容(每个部分的详细内容将在下面的项目符号中呈现):

\begin{cpp}
template <typename T>
struct generator
{
    // struct promise_type
    // struct iterator
    // member functions
    // iterators
private:
    std::coroutine_handle<promise_type> handle_ = nullptr;
};
\end{cpp}

\item
创建一个内部类 promise\_type(名称是强制的),包含以下内容:

\begin{cpp}
struct promise_type
{
    T const*             value_;
    std::exception_ptr   eptr_;
    auto get_return_object()
    { return generator{ *this }; }
    auto initial_suspend() noexcept
    { return std::suspend_always{}; }
    auto final_suspend() noexcept
    { return std::suspend_always{}; }
    void unhandled_exception() noexcept
    {
        eptr_ = std::current_exception();
    }
    void rethrow_if_exception()
    {
        if (eptr_)
        {
            std::rethrow_exception(eptr_);
        }
    }
    auto yield_value(T const& v)
    {
        value_ = std::addressof(v);
        return std::suspend_always{};
    }
    void return_void() {}
    template <typename U>
    U&& await_transform(U&& v)
    {
        return std::forward<U>(v);
    }
};
\end{cpp}

\item
创建一个名为 iterator 的内部类型:

\begin{cpp}
struct iterator
{
    using iterator_category = std::input_iterator_tag;
    using difference_type   = ptrdiff_t;
    using value_type        = T;
    using reference         = T const&;
    using pointer           = T const*;
    std::coroutine_handle<promise_type> handle_ = nullptr;
    iterator() = default;
    iterator(nullptr_t) : handle_(nullptr) {}
    iterator(std::coroutine_handle<promise_type> arg)
    : handle_(arg)
    {}
    iterator& operator++()
    {
        handle_.resume();
        if (handle_.done())
        {
            std::exchange(handle_, {}).promise()
                                      .rethrow_if_exception();
        }
        return *this;
    }
    void operator++(int)
    {
        ++*this;
    }
    bool operator==(iterator const& _Right) const
    {
        return handle_ == _Right.handle_;
    }
    bool operator!=(iterator const& _Right) const
    {
        return !(*this == _Right);
    }
    reference operator*() const
    {
        return *handle_.promise().value_;
    }
    pointer operator->() const
    {
        return std::addressof(handle_.promise().value_);
    }
};
\end{cpp}

\item
提供一个默认构造函数,从 promise\_type 实例显式构造的构造函数,一个移动构造函数和一个移动赋值操作符,以及一个析构函数。删除复制构造函数和复制赋值操作符,使类型仅能移动:

\begin{cpp}
explicit generator(promise_type& p)
    : handle_(
        std::coroutine_handle<promise_type>::from_promise(p))
{}
generator() = default;
generator(generator const&) = delete;
generator& operator=(generator const&) = delete;
generator(generator&& other) : handle_(other.handle_)
{
    other.handle_ = nullptr;
}
generator& operator=(generator&& other)
{
    if (this != std::addressof(other))
    {
        handle_ = other.handle_;
        other.handle_ = nullptr;
    }
    return *this;
}
~generator()
{
    if (handle_)
    {
        handle_.destroy();
    }
}
\end{cpp}

\item
提供 begin() 和 end() 函数以启用对生成器序列的迭代:

\begin{cpp}
iterator begin()
{
    if (handle_)
    {
        handle_.resume();
        if (handle_.done())
        {
            handle_.promise().rethrow_if_exception();
            return { nullptr };
        }
    }
    return { handle_ };
}
iterator end()
{
    return { nullptr };
}
\end{cpp}
\end{itemize}

\mySubsubsection{}{How it works...}

本示例中实现的 promise\_type 类似于前一个示例中的实现,尽管有一些不同点:

\begin{itemize}
\item
作为一个内部类型实现,名称为 \verb|promise_type|,协程框架要求协程类型具有此名称的内部promise类型。

\item
支持处理未捕获而离开协程块的异常。前一个示例中,这种情况未得到处理,\verb|unhandled_exception()| 调用了 \verb|std::terminate()| 来异常终止进程。然而,此实现捕获当前异常的指针并将其存储在一个 \verb|std::exception_ptr| 实例中。当遍历生成的序列时(无论是调用 \verb|begin()| 还是递增迭代器时),该异常会重新抛出。

\item
不提供 \verb|return_value()| 和 \verb|return_void()| 函数,而是替换为 \verb|yield_value()|,当解析 \verb|co_yield| 表达式时调用它。
\end{itemize}

生成器类型与前一示例中的任务类型有一些相似之处:

\begin{itemize}
\item
默认可构造的。

\item
可以从promise实例构造。

\item
不是复制可构造的或可复制的。

\item
是移动可构造的和可移动的。

\item
析构函数可销毁协程帧。
\end{itemize}

此类不会重载 \verb|co_await| 操作符,因为等待生成器没有意义;相反,提供了 \verb|begin()| 和 \verb|end()| 函数,这两个函数返回迭代器实例,用于遍历值序列。这个生成器为惰性,不会在协程恢复之前生成新的值,恢复可以由调用 \verb|begin()| 或递增迭代器触发。协程是在挂起状态下创建的,其首次执行仅在调用 \verb|begin()| 函数时开始。执行要么持续到第一个 \verb|co_yield| 语句,要么持续到协程完成执行。同样地,递增迭代器将恢复协程的执行,执行要么持续到下一个 \verb|co_yield| 语句,要么持续到协程完成。

以下示例展示了一个产生多个整数值的协程,不是通过使用循环而是通过重复 \verb|co_yield| 语句来实现的:

\begin{cpp}
generator<int> get_values() noexcept
{
    co_yield 1;
    co_yield 2;
    co_yield 3;
}
int main()
{
    for (auto i : get_values())
    {
        std::cout << i << ' ';
    }
}
\end{cpp}

需要注意的一个重要点是,协程只能使用 \verb|co_yield| 关键字并同步生成值。在协程内使用 \verb|co_await| 操作符不支持。要能够使用 \verb|co_await| 操作符来挂起执行,需要一个不同的实现。

\mySubsubsection{}{There's more...}

前一个示例中提到的 libcoro 库有一个 \verb|generator<T>| 类型,可以用来替代这里创建的类型。实际上,将 \verb|generator<T>| 替换为 \verb|coro::generator<T>|,前面的代码将继续按预期工作。
