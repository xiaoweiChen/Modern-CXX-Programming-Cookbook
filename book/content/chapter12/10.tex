

C++ 标准库包含了超过 100 个通用目的算法(大部分在 <algorithm> 头文件中,一些在 <numeric> 头文件中)。在第 5 章的几个示例中,已经了解了一些算法,学习了如何在范围内搜索元素、排序范围、初始化范围等。算法的通用性来自于使用迭代器(一个序列元素的开始和结束迭代器——即一个范围),但这也有缺点,即需要更明确的代码,并且这些代码需要一遍又一遍地重复。为了简化这些算法的使用,C++20 标准在 std::ranges 命名空间中提供了匹配的算法,这些算法既可以与范围一起工作(但也有接受迭代器的重载)。范围库中的这些算法称为受限算法,可以在 <algorithm> 头文件中找到。虽然不可能在这里详细查看所有算法,本示例中将介绍如何使用其中的一些算法来初始化、排序和查找范围内的元素。


\mySubsubsection{}{How to do it...}

可以在范围内执行各种操作,包括初始化、查找和排序:

\begin{itemize}
\item
使用 \verb|std::ranges::fill()| 向范围的所有元素分配一个值:

\begin{cpp}
std::vector<int> v(5);
std::ranges::fill(v, 42);
// v = {42, 42, 42, 42, 42}
\end{cpp}

\item
使用 \verb|std::ranges::fill_n()| 向范围的指定数量的元素分配一个值。要分配的第一个元素由输出迭代器指示:

\begin{cpp}
std::vector<int> v(10);
std::ranges::fill_n(v.begin(), 5, 42);
// v = {42, 42, 42, 42, 42, 0, 0, 0, 0, 0}
\end{cpp}

\item
使用 \verb|std::ranges::generate_n()| 向范围的一定数量的元素分配由给定函数连续调用返回的值。要分配的第一个元素由迭代器指示:

\begin{cpp}
std::vector<int> v(5);
auto i = 1;
std::ranges::generate_n(v.begin(), v.size(),
                        [&i] { return I * i++; });
// v = {1, 4, 9, 16, 25}
\end{cpp}

\item
使用 \verb|std::ranges::iota()| 向范围的元素分配依次增加的值。值从指定的初始值开始,使用前缀 operator++ 增加:

\begin{cpp}
std::vector<int> v(5);
std::ranges::iota(v, 1);
// v = {1, 2, 3, 4, 5}
\end{cpp}

\item
使用 \verb|std::ranges::find()| 在范围内查找一个值;此算法返回一个指向第一个等于提供的值的元素的迭代器(如果存在这样的值),或一个等于范围末尾的迭代器:

\begin{cpp}
std::vector<int> v{ 1, 1, 2, 3, 5, 8, 13 };
auto it = std::ranges::find(v, 3);
if (it != v.cend()) std::cout << *it << '\n';
\end{cpp}

\item
使用 \verb|std::ranges::find_if()| 查找符合一元谓词定义条件的范围内的值。该算法返回一个指向范围内第一个谓词返回 true 的元素的迭代器,或者没有这样的元素,则返回一个指向范围末尾的迭代器:

\begin{cpp}
std::vector<int> v{ 1, 1, 2, 3, 5, 8, 13 };
auto it = std::ranges::find_if(v, [](int const n) { return n > 10; });
if (it != v.cend()) std::cout << *it << '\n';
\end{cpp}

\item
使用 \verb|std::ranges::find_first_of()| 在另一个范围内查找来自一个范围的任何值的出现;该算法返回一个指向第一个找到的元素(在被搜索的范围内)的迭代器,否则返回一个等于范围末尾的迭代器:

\begin{cpp}
std::vector<int> v{ 1, 1, 2, 3, 5, 8, 13 };
std::vector<int> p{ 5, 7, 11 };
auto it = std::ranges::find_first_of(v, p);
if (it != v.cend())
    std::cout << "found " << *it
    << " at index " << std::ranges::distance(v.cbegin(), it)
    << '\n';
\end{cpp}

\item
使用 \verb|std::ranges::sort()| 对范围进行排序您可以提供一个应用于元素的比较函数。这可以包括 std::ranges::greater、std::ranges::less 和 <functional> 头文件中对应于 <、<=、>、>=、== 和 != 操作符的其他函数实例:

\begin{cpp}
std::vector<int> v{ 3, 13, 5, 8, 1, 2, 1 };
std::ranges::sort(v);
// v = {1, 1, 2, 3, 5, 8, 13}
std::ranges::sort(v, std::ranges::greater());
// v = {13, 8, 5, 3, 2, 1 ,1}
\end{cpp}

\item
使用 \verb|std::ranges::is_sorted()| 检查范围是否已排序:

\begin{cpp}
std::vector<int> v{ 1, 1, 2, 3, 5, 8, 13 };
auto sorted = std::ranges::is_sorted(v);
sorted = std::ranges::is_sorted(v, std::ranges::greater());
\end{cpp}

\item
使用 \verb|std::ranges::is_sorted_until()| 从范围的开始处找到已排序的子范围:

\begin{cpp}
std::vector<int> v{ 3, 13, 5, 8, 1, 2, 1 };
auto it = std::ranges::is_sorted_until(v);
auto length = std::ranges::distance(v.cbegin(), it);
// length = 2
\end{cpp}
\end{itemize}

\mySubsubsection{}{How it works...}

几乎所有受限算法都位于 <algorithm> 头文件中。唯一的例外是 std::ranges::iota(),其位于 <numeric> 头文件中。在“How to do it...”列出的算法只是可用受限算法的一小部分。它们称为受限算法,是因为它们的参数中有定义的要求,这些要求通过概念和约束来帮助。这里是前面使用的 std::ranges::find() 的一个重载定义:

\begin{cpp}
template <ranges::input_range R, class T, class Proj = std::identity>
requires std::indirect_binary_predicate<
    ranges::equal_to,
    std::projected<ranges::iterator_t<R>, Proj>,
    const T*>
constexpr ranges::borrowed_iterator_t<R>
    find( R&& r, const T& value, Proj proj = {} );
\end{cpp}

如果查看前面给出的代码以及第 5 章中示例的代码,会注意到它们几乎是相同的。唯一的区别是命名空间(std::ranges 而不是 std)和参数;传统算法接受定义范围边界的迭代器,而受限算法则与范围库中定义的范围一起工作。因此,可以直接传递一个 std::vector、std::list 或 std::array。然而,受限算法有接受一对迭代器的重载。下面的代码展示了 std::ranges::find() 算法,也出现在前面的部分中,使用开始和结束迭代器调用:

\begin{cpp}
std::vector<int> v{ 1, 1, 2, 3, 5, 8, 13 };
auto it = std::ranges::find(v.begin(), v.end(), 3);
if (it != v.cend()) std::cout << *it << '\n';
\end{cpp}

另一方面,有些算法,如之前提到的 \verb|std::ranges::fill_n()| 和 \verb|std::ranges::generate_n()|,只有一个接受从范围开始的迭代器的重载。

传统算法和受限算法之间的另一个区别在于后者没有用于指定执行策略的重载,而前者有。

受限算法比传统算法有几个优势:

\begin{itemize}
\item
编写的代码更少,不必检索范围的开始和结束迭代器。

\item
使用概念和约束进行了限制,这有助于在误用时提供更好的错误消息。

\item
可以与范围库中定义的范围/视图一起使用。

\item
其中一些算法有重载,允许指定应用于元素的投影,然后在投影结果上应用指定的谓词。
\end{itemize}

首先看看受限算法如何与范围交互,考虑以下示例:

\begin{cpp}
std::vector<int> v{ 3, 13, 5, 8, 1, 2, 1 };
auto range =
    v |
    std::views::filter([](int const n) {return n % 2 == 1; }) |
    std::views::transform([](int const n) {return n * n; }) |
    std::views::take(4);
std::ranges::for_each(range,
                      [](int const n) {std::cout << n << ' '; });
std::cout << '\n';
auto it = std::ranges::find_if(range,
                               [](int const n) {return n > 10; });
if (it != range.end())
    std::cout << *it << '\n';
\end{cpp}

这个例子中,有一个整数vector。从这个vector中,过滤掉偶数,将其余数字转换为其平方,并最终保留四个结果数字。结果是一个范围。其类型太复杂,难以记忆或书写;因此,使用 auto 说明符并让编译器对其类型进行推断。

\begin{myNotic}
对于那些想知道实际类型是什么的人(前面的例子中),可以看看这里\\std::ranges::take\_view<\\
\hspace*{0.5cm}std::ranges::transform\_view<\\
\hspace*{1.0cm}std::ranges::filter\_view<\\
\hspace*{1.5cm}std::ranges::ref\_view<\\
\hspace*{2.0cm}std::vector<int>\\
\hspace*{1.5cm}>\\
\hspace*{1.0cm}, lambda [](int n)->bool\\
\hspace*{1.0cm}>\\
\hspace*{0.5cm}, lambda [](int n)->int\\
\hspace*{0.5cm}>\\
>
\end{myNotic}

目标是将结果值打印到控制台,并找到大于 10 的第一个值(如果有)。为此,使用 \verb|std::ranges::for_each()| 和 \verb|std::ranges::find_if()|,传递范围实例而不必直接处理迭代器。

前面列表中提到的最后一个优点是指定投影的能力,投影是一个可调用对象(例如,函数对象或成员引用)。此投影应用于范围的元素。投影结果上,再应用另一个谓词。

为了理解这是如何工作的,假设一个具有 ID、名称和价格的产品列表。从这个列表中,想要找到某个价格的产品并输出其名称。列表定义如下:

\begin{cpp}
struct Product
{
    int         id;
    std::string name;
    double      price;
};
std::vector<Product> products
{
    {1, "pen", 15.50},
    {2, "pencil", 9.99},
    {3, "rubber", 5.0},
    {4, "ruler", 5.50},
    {5, "notebook", 12.50}
};
\end{cpp}

使用传统算法时,需要使用 \verb|std::find_if()| 并传递一个执行每个元素检查的 lambda 函数:

\begin{cpp}
auto pos = std::find_if(
    products.begin(), products.end(),
    [](Product const& p) { return p.price == 12.5; });
if (pos != products.end())
    std::cout << pos->name << '\n';
\end{cpp}

使用受限算法时,可以使用 std::ranges::find() 的一个接受范围、值和投影的重载:

\begin{cpp}
auto it = std::ranges::find(products, 12.50, &Product::price);
if (it != products.end())
    std::cout << it->name << '\n';
\end{cpp}

另一个类似的例子是对产品按名称进行字母顺序(升序)排序:

\begin{cpp}
std::ranges::sort(products, std::ranges::less(), &Product::name);
std::ranges::for_each(products, [](Product const& p) {
    std::cout << std::format("{} = {}\n", p.name, p.price); });
\end{cpp}

希望这些例子能够展示出通常情况下选择新的 C++20 受限算法,而非传统算法的有力理由。请记住,当想要指定执行策略(例如,为了并行化或向量化算法的执行)时,不能使用受限算法(因为没有提供这样的重载)。


