
模块的源码可能会变得庞大且难以维护,模块可能由逻辑上分离的部分组成。为了帮助处理这样的场景,模块由分区部分组合而成。一个导出实体的模块单元,如果是分区的话,称为模块接口分区。

然而,也可能存在不导出任何东西的内部分区,这样的分区单元称为模块实现分区。本示例中,将介绍如何使用接口和实现分区。

\mySubsubsection{}{Getting ready}

继续本示例之前,应该先阅读上一个示例。您

\begin{myNotic}
接下来的代码中,将使用仅在 C++23 中可用的 std 模块。对于之前的版本,请在 VC++ 中使用 std.core 或者编译器支持的其他特定模块。
\end{myNotic}

\mySubsubsection{}{How to do it...}

可以将一个模块拆分为多个分区,方法如下:

\begin{itemize}
\item
每个分区单元必须以 export module modulename:partitionname; 形式的声明开始。只有全局模块片段可以前置于此类声明:

\begin{cpp}
// --- geometry-core.ixx/.cppm ---
export module geometry:core;
import std;
export template <class T,
    typename = typename std::enable_if_t<std::is_arithmetic_v<T>, T>>
struct point
{
    T x;
    T y;
};
export using int_point = point<int>;
export constexpr int_point int_point_zero{ 0,0 };
export template <class T>
double distance(point<T> const& p1, point<T> const& p2)
{
    return std::sqrt((p2.x - p1.x) * (p2.x - p1.x) +
                     (p2.y - p1.y) * (p2.y - p1.y));
}
// --- geometry-literals.ixx/.cppm ---
export module geometry:literals;
import :core;
namespace geometry_literals
{
    export int_point operator ""_ip(const char* ptr, std::size_t)
    {
        int x = 0, y = 0;
        if(ptr)
        {
            while (*ptr != ',' && *ptr != ' ')
                x = x * 10 + (*ptr++ - '0');
            while (*ptr == ',' || *ptr == ' ') ptr++;
            while (*ptr != 0)
                y = y * 10 + (*ptr++ - '0');
        }
        return { x, y };
    }
}
\end{cpp}

\item
主模块接口单元中,使用 export import :partitionname; 形式的语句导入并导出分区:

\begin{cpp}
// --- geometry.ixx/.cppm ---
export module geometry;
export import :core;
export import :literals;
\end{cpp}

\item
导入由多个分区组成的模块的代码,只会看到作为一个整体构建的模块(前提是源自单一模块单元):

\begin{cpp}
// --- main.cpp ---
import std;
import geometry;
int main()
{
    int_point p{ 3, 4 };
    std::cout << distance(int_point_zero, p) << '\n';
    {
        using namespace geometry_literals;
        std::cout << distance("0,0"_ip, "30,40"_ip) << '\n';
    }
}
\end{cpp}

\item
可以创建不导出任何东西,但包含可在同一模块中使用的代码的内部分区。这样的分区必须以 module modulename:partitionname; 形式(不带 export 关键字)的声明开始。不同的编译器可能还需要分区文件具有不同的扩展名。对于 VC++,扩展名必须是 .cpp:

\begin{cpp}
// --- geometry-details.cpp --
module geometry:details;
import std;
std::pair<int, int> split(const char* ptr)
{
    int x = 0, y = 0;
    if(ptr)
    {
        while (*ptr != ',' && *ptr != ' ')
            x = x * 10 + (*ptr++ - '0');
        while (*ptr == ',' || *ptr == ' ') ptr++;
        while (*ptr != 0)
            y = y * 10 + (*ptr++ - '0');
    }
    return { x, y };
}
// --- geometry-literals.ixx/.cppm ---
export module geometry:literals;
import :core;
import :details;
namespace geometry_literals
{
    export int_point operator ""_ip(const char* ptr, std::size_t)
    {
        auto [x, y] = split(ptr);
        return {x, y};
    }
}
\end{cpp}
\end{itemize}

\mySubsubsection{}{How it works...}

前面展示的代码是对前一个示例中介绍模块的延续。geometry 模块已经拆分为两个不同的分区,分别命名为 \verb|core| 和 \verb|literals|。

当声明分区时,必须使用 modulename:partitionname 形式的名称,例如 geometry:core 和 geometry:literals。模块导入分区时,这并不必要。可以在主分区单元 geometry.ixx 和模块接口分区 geometry-literals.ixx 中看到。为了清晰起见,再次列出这些代码:

\begin{cpp}
// --- geometry-literals.ixx/.cppm ---
export module geometry:literals;
// import the core partition
import :core;
// --- geometry.ixx/.cppm ---
export module geometry;
// import the core partition and then export it
export import :core;
// import the literals partition and then export it
export import :literals;
\end{cpp}

虽然模块分区是不同的文件,但并不会作为单独的模块或子模块出现在使用模块的翻译单元中。它们会作为一个单一的聚合模块导出。如果将 main.cpp 文件中的源代码与前一个示例中的源代码进行比较,则看不到任何差异。

\begin{myNotic}
与模块接口单元一样,没有关于包含分区的文件命名规则。但编译器可能要求不同的扩展名,或支持某些特定的命名方案。例如,VC++ 使用 <module-name>-<partition-name>.ixx 方案,这简化了构建命令。
\end{myNotic}

分区,就像模块一样,可以包含不从模块导出的代码。一个分区可能完全不包含导出,只是一个内部分区。这样的分区称为模块实现分区,在模块声明中定义时不使用 export 关键字。

一个内部分区的例子是前面展示的 geometry:details 分区,提供了一个名为 split() 的辅助函数,用于解析字符串中的两个由逗号分隔的整数。然后,该分区导入到 geometry:literals 分区中,split() 函数用来实现用户定义的字面值操作符 \_ip。

\mySubsubsection{}{There's more...}

分区是模块的划分,并不是子模块,其在逻辑上不能脱离模块而存在。C++ 语言中没有子模块的概念。本示例中使用分区编写的代码,也可以使用模块来编写:

\begin{cpp}
// --- geometry-core.ixx ---
export module geometry.core;
import std;
export template <class T,
    typename = typename std::enable_if_t<std::is_arithmetic_v<T>, T>>
struct point
{
    T x;
    T y;
};
export using int_point = point<int>;
export constexpr int_point int_point_zero{ 0,0 };
export template <class T>
double distance(point<T> const& p1, point<T> const& p2)
{
    return std::sqrt(
        (p2.x - p1.x) * (p2.x - p1.x) +
        (p2.y - p1.y) * (p2.y - p1.y));
}
// --- geometry-literals.ixx ---
export module geometry.literals;
import geometry.core;
namespace geometry_literals
{
    export int_point operator ""_ip(const char* ptr, std::size_t)
    {
        int x = 0, y = 0;
        if(ptr)
        {
            while (*ptr != ',' && *ptr != ' ')
                x = x * 10 + (*ptr++ - '0');
            while (*ptr == ',' || *ptr == ' ') ptr++;
            while (*ptr != 0)
                y = y * 10 + (*ptr++ - '0');
        }
        return { x, y };
    }
}
// --- geometry.ixx ---
export module geometry;
export import geometry.core;
export import geometry.literals;
\end{cpp}

例子中,有三个模块:geometry.core、geometry.literals 和 geometry。这里,geometry 导入并重新导出了前两个模块的全部内容。因此,main.cpp 文件中的代码不需要改变。

仅通过导入 geometry 模块,就能访问 geometry.core 和 geometry.literals 模块的内容。

然而,如果不再定义 geometry 模块,则需要显式地导入这两个模块:

\begin{cpp}
import std;
import geometry.core;
import geometry.literals;
int main()
{
    int_point p{ 3, 4 };
    std::cout << distance(int_point_zero, p) << '\n';
    {
        using namespace geometry_literals;
        std::cout << distance("0,0"_ip, "30,40"_ip) << '\n';
    }
}
\end{cpp}

选择使用分区还是多个模块来组件化源码,应取决于项目特点。如果使用多个较小的模块,可以为导入提供更好的粒度。这对于开发大型库很重要,用户应该只导入他们使用的东西(而不需要导入一个非常大的模块,即使他们只需要其中的一些功能)。


