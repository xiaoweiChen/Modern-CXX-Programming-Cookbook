
C++11引入了统一初始化,是一种初始化数据的新方法,也称为大括号初始化。这种初始化方式旨在减少不同初始化语法之间的差异,提高代码的一致性和可读性。

\mySubsubsection{}{Getting ready}

要继续了解本示例,读者们需要熟悉直接初始化,即通过一组显式的构造函数参数来初始化对象;以及复制初始化,即通过另一个对象来初始化对象。下面是一个简单的示例,展示了两种初始化的方式:

\begin{cpp}
std::string s1("test");   // 直接初始化
std::string s2 = "test";  // 复制初始化
\end{cpp}

\mySubsubsection{}{How to do it...}

为了统一地初始化不同类型的对象,可以使用大括号初始化的形式 \{\},其既适用于直接初始化也适用于复制初始化。当与大括号初始化一起使用时,分别称为直接列表初始化和复制列表初始化:

\begin{cpp}
T object {other};   // 直接列表初始化
T object = {other}; // 复制列表初始化
\end{cpp}

统一初始化的例子:

\begin{itemize}
\item
标准容器:

\begin{cpp}
std::vector<int> v { 1, 2, 3 };
std::map<int, std::string> m { {1, "one"}, { 2, "two" }};
\end{cpp}

\item
动态分配的数组:

\begin{cpp}
int* arr2 = new int[3]{ 1, 2, 3 };
\end{cpp}

\item
数组:

\begin{cpp}
int arr1[3] { 1, 2, 3 };
\end{cpp}

\item
内置类型:

\begin{cpp}
int i { 42 };
double d { 1.2 };
\end{cpp}

\item
自定义类型:

\begin{cpp}
class foo
{
    int a_;
    double b_;
    public:
    foo():a_(0), b_(0) {}
    foo(int a, double b = 0.0):a_(a), b_(b) {}
};
foo f1{};
foo f2{ 42, 1.2 };
foo f3{ 42 };
\end{cpp}

\item
自户定义的普通数据类型(POD):

\begin{cpp}
struct bar { int a_; double b_;};
bar b{ 42, 1.2 };
\end{cpp}
\end{itemize}

\mySubsubsection{}{How it works...}

C++11之前,不同类型的实例需要基于其类型采用不同的初始化方式:

\begin{itemize}
\item
基本类型可以使用赋值方式进行初始化:

\begin{cpp}
int a = 42;
double b = 1.2;
\end{cpp}

\item
如果类型实例有转换构造函数(C++11前,单个参数的构造函数称为转换构造函数),也可以通过从单个值赋值进行初始化:

\begin{cpp}
class foo
{
    int a_;
    public:
    foo(int a):a_(a) {}
};
foo f1 = 42;
\end{cpp}

\item
非聚合类型在提供参数时可以用括号(函数形式)初始化,而在执行默认初始化时则只能在没有括号的情况下初始化(调用默认构造函数)。下面的例子中,foo 是在“How to do it...”中定义的类型:

\begin{cpp}
foo f1;           // default initialization
foo f2(42, 1.2);
foo f3(42);
foo f4();         // function declaration
\end{cpp}

\item
聚合类型和POD类型可以用大括号初始化。下面的例子中,bar是“How to do it...”中定义的结构体:

\begin{cpp}
bar b = {42, 1.2};
int a[] = {1, 2, 3, 4, 5};
\end{cpp}
\end{itemize}

\begin{myNotic}
普通数据类型(POD)是一种既简单(具有编译器提供的特殊成员或显式默认成员,并占用连续内存区域)又符合标准布局(不包含与C语言不兼容的语言特性,如虚函数,且所有成员具有相同的访问控制)的类型。POD类型的概念在C++20中已弃用,取而代之的是简单类型和标准布局类型。
\end{myNotic}

除了不同的初始化方法外,还有一些限制。例如,初始化标准容器(除了复制构造之外)唯一的方法是先声明一个实例,然后插入元素;std::vector 是例外,其可以从预先使用聚合初始化初始化的数组中赋值。另外,动态分配的聚合类型无法直接初始化。

“How to do it...”中所有示例中都使用了直接初始化,但使用大括号初始化也可以进行复制初始化。这两种形式——直接初始化和复制初始化——大多数情况下可能等价,但复制初始化由于其隐式转换序列中不考虑显式构造函数,所以不如直接初始化宽容,后者期望从初始化器到构造函数参数有一个隐式转换。因此,动态分配的数组只能使用直接初始化。

前面示例中显示的类型中,foo 是唯一一个既有默认构造函数,又有带参数构造函数的类型。要使用默认构造函数执行默认初始化,需要使用空的大括号——\{\}。要使用带参数的构造函数,需要在大括号 \{\} 中提供所有参数的值。对于非聚合类型而言,默认初始化会调用默认构造函数;而对于聚合类型,默认初始化会使用零值初始化。

标准容器(如 vector 和 map)也可以统一初始化,所有标准容器在 C++11 中多了一个构造函数,该构造函数接受一个 std::initializer\_list<T> 类型的参数。这是一个轻量级的代理,用于包装一个 T const 类型元素的数组。这些构造函数会使用初始化列表中的值,初始化内部数据。

使用 std::initializer\_list 进行初始化的方式:

\begin{itemize}
\item
编译器解析初始化列表中元素的类型(所有元素必须为相同的类型)。

\item
编译器创建一个包含初始化列表中元素的数组。

\item
编译器创建一个 std::initializer\_list<T> 对象来封装先前创建的数组。

\item
std::initializer\_list<T> 对象作为参数传递给构造函数。
\end{itemize}

当使用大括号初始化时,初始化列表总是优先于其他构造函数。如果有这样的构造函数,则在执行大括号初始化时会调用这个构造函数:

\begin{cpp}
class foo
{
    int a_;
    int b_;
public:
    foo() :a_(0), b_(0) {}
    foo(int a, int b = 0) :a_(a), b_(b) {}
    foo(std::initializer_list<int> l) {}
};
foo f{ 1, 2 }; // calls constructor with initializer_list<int>
\end{cpp}

优先规则不仅适用于构造函数,也适用于任何函数。下面的例子中,存在同一个函数的两个版本。用初始化列表调用函数时,会解析为调用带有 std::initializer\_list 参数的版本:

\begin{cpp}
void func(int const a, int const b, int const c)
{
    std::cout << a << b << c << '\n';
}
void func(std::initializer_list<int> const list)
{
    for (auto const & e : list)
    std::cout << e << '\n';
}
func({ 1,2,3 }); // calls second overload
\end{cpp}

但这种情况有可能导致错误,以 std::vector 类型为例。vector 的构造函数中,有一个接受单个参数表示要分配的初始元素数量的构造函数,还有一个接受 std::initializer\_list 参数的构造函数。如果要创建一个预分配大小的 vector,因为带有 std::initializer\_list 参数的构造函数会优先调用,所以使用大括号初始化将不起作用:

\begin{cpp}
std::vector<int> v {5};
\end{cpp}

上述代码并没有创建一个包含五个元素的 vector,而是创建了一个只有一个元素且值为 5 的 vector。为了真正创建一个包含五个元素的 vector,必须使用圆括号形式的初始化:

\begin{cpp}
std::vector<int> v (5);
\end{cpp}

另一点需要注意的是,大括号初始化不允许窄化转换。根据 C++ 标准(参见标准第 9.4.5 段,文档版本 N4917),窄化转换是一种隐式转换:

\begin{myTip}
* 从浮点类型转换到整数类型。

* 从 long double 转换到 double 或 float,或者从 double 转换到 float。除非是常量表达式,且转换后的实际值在目标类型可以表示的范围内(即使不能精确表示)。

* 从整数类型或未限定的作用域枚举类型转换到浮点类型。除非是常量表达式,且转换后的实际值适合目标类型,并且在转换回原始类型时能够产生原始值。

* 从整数类型或未限定的作用域枚举类型,转换到不能表示原始类型所有值的整数类型。除非是常量表达式,且转换后的实际值适合目标类型,并且在转换回原始类型时能够产生原始值。
\end{myTip}

下列声明会触发编译器错误(因为窄化转换):

\begin{cpp}
int i{ 1.2 };           // 错误
double d = 47 / 13;
float f1{ d };          // 错误, 在 gcc 中仅产生警告
\end{cpp}

要解决这个错误,必须进行显式转换:

\begin{cpp}
int i{ static_cast<int>(1.2) };
double d = 47 / 13;
float f1{ static_cast<float>(d) };
\end{cpp}

\begin{myNotic}
大括号初始化列表不是一个表达式,也没有类型。decltype 不能用于大括号初始化列表,模板类型推导也无法从大括号初始化列表中推断出匹配的类型。
\end{myNotic}

再看一个例子:

\begin{cpp}
float f2{47/13};        // OK, f2=3
\end{cpp}

上述声明正确,存在从 int 到 float 的隐式转换。表达式 47/13 首先计算为整数值 3,然后赋值给类型为 float 的变量 f2。

\mySubsubsection{}{There’s more...}

以下示例展示了几个直接列表初始化和复制列表初始化的例子。C++11后,所有这些表达式的推导类型都是 std::initializer\_list<int>:

\begin{cpp}
auto a = {42};   // std::initializer_list<int>
auto b {42};     // std::initializer_list<int>
auto c = {4, 2}; // std::initializer_list<int>
auto d {4, 2};   // std::initializer_list<int>
\end{cpp}

C++17 修改了列表初始化的规则,区分了直接列表初始化和复制列表初始化。新类型的推导规则:

\begin{itemize}
\item
对于复制列表初始化,如果列表中的所有元素具有相同的类型,auto 推导将得出一个 std::initializer\_list<T>,否则形式非法。

\item
对于直接列表初始化,如果列表中只有一个元素,auto 推导将得出一个 T;如果有多个元素,则形式非法。
\end{itemize}

根据这些新规则,前面的例子将变化如下(注释中提到推导的类型):

\begin{cpp}
auto a = {42};   // std::initializer_list<int>
auto b {42};     // int
auto c = {4, 2}; // std::initializer_list<int>
auto d {4, 2};   // error, too many
\end{cpp}

a 和 c 推导为 std::initializer\_list<int>,b 推导为 int,而 d 使用直接初始化并且大括号初始化列表中有多个值,从而触发编译器错误。































