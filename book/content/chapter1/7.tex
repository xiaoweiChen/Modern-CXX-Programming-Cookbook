C++ 没有专门用于声明接口(只有纯虚函数的类)的语法,而且在声明虚函数方面也存在一些不足之处。在 C++ 中,虚函数通过 virtual 关键字引入。然而,virtual 关键字在派生类中声明重载方法时是可选的,这在处理大型类或层次结构时可能导致混淆。可能需要遍历整个层次结构,直至基类来确定一个函数是否是虚函数。另一方面,有时确保一个虚函数或派生类不能再重载或进一步派生也可用。本节中,将介绍如何使用 C++11 引入的特殊标识符 override 和 final 来声明虚函数或类型。

\mySubsubsection{}{Getting ready}

应该熟悉 C++ 中的继承和多态概念,以及抽象类型、纯虚函数、虚函数和重载方法的概念。

\mySubsubsection{}{How to do it...}

为了确保正确声明基类和派生类中的虚方法,同时提高可读性,请遵循以下步骤:

\begin{itemize}
\item
派生类中声明用于覆盖基类虚函数的虚函数时使用 virtual 关键字。

\item
在虚函数声明或定义的声明符部分后面使用 override 特殊标识符:

\begin{cpp}
class Base
{
    virtual void foo() = 0;
    virtual void bar() {}
    virtual void foobar() = 0;
};
void Base::foobar() {}
class Derived1 : public Base
{
    virtual void foo() override = 0;
    virtual void bar() override {}
    virtual void foobar() override {}
};
class Derived2 : public Derived1
{
    virtual void foo() override {}
};
\end{cpp}
\end{itemize}

\begin{myNotic}
声明符是指函数类型中除了返回类型的部分。
\end{myNotic}

为了确保函数不能再进一步覆盖,或类不能再进一步派生,可以使用 final 标识符:

\begin{itemize}
\item
虚函数声明或定义的声明符部分后面使用 final 防止派生类中的覆盖:

\begin{cpp}
class Derived2 : public Derived1
{
    virtual void foo() final {}
};
\end{cpp}

\item
类型声明的类型名后面使用 final 防止类型的派生:

\begin{cpp}
class Derived4 final : public Derived1
{
    virtual void foo() override {}
};
\end{cpp}

\end{itemize}

\mySubsubsection{}{How it works...}

override 的工作方式非常简单;在虚函数声明或定义中,确保该函数实际上覆盖了基类中的函数;如果不是这样,编译器将报错。

override 和 final 这两个特殊标识符仅在成员函数声明或定义中具有特殊含义。不是保留关键字,可以在程序的其他地方作为用户定义的标识符使用。

使用 override 特殊标识符可以帮助编译器检测虚方法没有覆盖另一个方法的情况,如下例所示:

\begin{cpp}
class Base
{
    public:
    virtual void foo() {}
    virtual void bar() {}
};
class Derived1 : public Base
{
    public:
    void foo() override {}
    // 为了提高可读性,建议使用 virtual 关键字
    virtual void bar(char const c) override {}
    // 错误,Base 中没有 bar(char const) 函数
};
\end{cpp}

如果没有 override 标识符的存在,Derived1 类中的 virtual bar(char const) 将不是一个覆盖函数,而是 Base 类型中 bar() 函数的一个重载。

另一个特殊标识符 final 用于成员函数声明或定义中,表明该函数是虚函数并且不能在派生类中覆盖。如果派生类试图覆盖虚函数,编译器将报错:

\begin{cpp}
class Derived2 : public Derived1
{
    virtual void foo() final {}
};
class Derived3 : public Derived2
{
    virtual void foo() override {} // 错误
};
\end{cpp}

final 标识符也可以用于类声明中,表明该类不能派生:

\begin{cpp}
class Derived4 final : public Derived1
{
    virtual void foo() override {}
};
class Derived5 : public Derived4 // 错误
{
};
\end{cpp}

由于 override 和 final 在特定上下文中使用时具有特殊含义,事实上并不是保留关键字,可以在 C++ 代码的其他地方使用。这确保了在 C++11 之前的代码,不会因为这些名称当作标识符而出现问题:

\begin{cpp}
class foo
{
    int final = 0;
    void override() {}
};
\end{cpp}

虽然前面的建议提到了在声明重载的虚方法时应同时使用 virtual 和 override,但 virtual 关键字实际上是可选的,可以省略以缩短声明。override 标识符的存在应该足以向读者表明该方法是虚方法。这更多是一个个人偏好问题,并不影响语义。










