
基于范围的循环(其他编程语言中称为 for-each)允许迭代一个范围的元素,相对于标准的 for 循环提供了一个简化的语法。但基于范围的循环并不适用于所有表示范围的类型,其要求存在 begin() 和 end() 函数(对于非数组类型),这些函数可以是成员函数也可以是独立函数。本示例中,将介绍如何让自定义类型能够在基于范围的循环中使用。

\mySubsubsection{}{Getting ready}

为了展示如何使表示序列的自定义类型支持基于范围的循环,可以使用以下简单的数组实现:

\begin{cpp}
template <typename T, size_t const Size>
class dummy_array
{
    T data[Size] = {};
    public:
    T const & GetAt(size_t const index) const
    {
        if (index < Size) return data[index];
        throw std::out_of_range("index out of range");
    }
    void SetAt(size_t const index, T const & value)
    {
        if (index < Size) data[index] = value;
        else throw std::out_of_range("index out of range");
    }
    size_t GetSize() const { return Size; }
};
\end{cpp}

本示例的目标是让如下代码正常工作:

\begin{cpp}
dummy_array<int, 3> arr;
arr.SetAt(0, 1);
arr.SetAt(1, 2);
arr.SetAt(2, 3);
for(auto&& e : arr)
{
    std::cout << e << '\n';
}
\end{cpp}

\mySubsubsection{}{How to do it...}

要使自定义类型支持给予范围的循环,需要做以下几件事:

\begin{itemize}
\item
创建该类型的可变和常量迭代器,这些迭代器必须实现以下操作符:

\begin{itemize}
\item
operator++ (前置和后置版本)用于递增迭代器。

\item
operator* 用于解引用迭代器并访问迭代器指向的实际元素。

\item
operator!= 用于将迭代器与其他迭代器进行不等比较。
\end{itemize}

\item
为该类型提供独立的 begin() 和 end() 函数。
\end{itemize}

基于上述示例,需要提供以下内容:

\begin{itemize}
\item
下面是最小化迭代器的类型实现:

\begin{cpp}
template <typename T, typename C, size_t const Size>
class dummy_array_iterator_type
{
    public:
    dummy_array_iterator_type(C& collection,
    size_t const index) :
    index(index), collection(collection)
    { }
    bool operator!= (dummy_array_iterator_type const & other) const
    {
        return index != other.index;
    }
    T const & operator* () const
    {
        return collection.GetAt(index);
    }
    dummy_array_iterator_type& operator++()
    {
        ++index;
        return *this;
    }
    dummy_array_iterator_type operator++(int)
    {
        auto temp = *this;
        ++*this;
        return temp;
    }
    private:
    size_t   index;
    C&       collection;
};
\end{cpp}

\item
可变和常量迭代器的别名模板:

\begin{cpp}
template <typename T, size_t const Size>
using dummy_array_iterator =
    dummy_array_iterator_type<
T, dummy_array<T, Size>, Size>;
template <typename T, size_t const Size>
using dummy_array_const_iterator =
    dummy_array_iterator_type<
        T, dummy_array<T, Size> const, Size>;
\end{cpp}

\item
返回相应开始和结束迭代器的独立 begin() 和 end() 函数,包括两种别名模板的重载:

\begin{cpp}
template <typename T, size_t const Size>
inline dummy_array_iterator<T, Size> begin(
dummy_array<T, Size>& collection)
{
    return dummy_array_iterator<T, Size>(collection, 0);
}
template <typename T, size_t const Size>
inline dummy_array_iterator<T, Size> end(
dummy_array<T, Size>& collection)
{
    return dummy_array_iterator<T, Size>(
    collection, collection.GetSize());
}
template <typename T, size_t const Size>
inline dummy_array_const_iterator<T, Size> begin(
dummy_array<T, Size> const & collection)
{
    return dummy_array_const_iterator<T, Size>(
    collection, 0);
}
template <typename T, size_t const Size>
inline dummy_array_const_iterator<T, Size> end(
dummy_array<T, Size> const & collection)
{
    return dummy_array_const_iterator<T, Size>(
    collection, collection.GetSize());
}
\end{cpp}

\end{itemize}

\mySubsubsection{}{How it works...}

有了上述实现,先前展示的基于范围的循环就能编译和执行。当执行依赖查找时,编译器会找到编写的两个 begin() 和 end() 函数(接受 dummy\_array 的引用),因此生成的代码就有效了。

上述示例中,定义了迭代器类模板和两个别名模板,分别为 dummy\_array\_iterator 和 dummy\_array\_const\_iterator。begin() 和 end() 函数都有这两种迭代器类型的重载。以便使用基于范围的循环,来处理常量和非常量实例的容器:

\begin{cpp}
template <typename T, const size_t Size>
void print_dummy_array(dummy_array<T, Size> const & arr)
{
    for (auto && e : arr)
    {
        std::cout << e << '\n';
    }
}
\end{cpp}

使自定义类型支持基于范围的循环,一种替代方案是提供成员 begin() 和 end() 函数,这只有在拥有并可以修改源码的情况下才有意义。













