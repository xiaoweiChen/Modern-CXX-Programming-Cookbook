
自动类型推导是现代C++中最重要的特性,新的C++标准中可以使用auto作为类型的占位符,让编译器来推断实际的类型。C++11,auto可以用来声明局部变量,以及使用后置返回类型表示函数的返回类型。C++14,auto可以用于不需要指定后置类型函数的返回类型,以及lambda表达式中的参数声明。C++17,auto还可以用于声明结构化绑定,这部分内容将在本章末尾讨论。C++20允许使用auto简化函数模板的语法。C++23允许使用auto执对prvalue的副本进行显式转换。未来标准版本可能会将auto的应用扩展到更多情况。C++11和C++14引入auto,开发者应该了解这些优点,并尽可能地使用auto。Andrei Alexandrescu提出了一个术语并由Herb Sutter推广——尽可能地使用auto(AAA)(\url{ https://herbsutter.com/2013/08/12/gotw-94-solution-aaa-style-almost-always-auto/})。

\mySubsubsection{}{How to do it...}

以下情况,使用auto作为实际类型的占位符:

\begin{itemize}
\item
不想限定于某个特定类型时,局部变量的声明形式可为auto name = expression:

\begin{cpp}
auto i = 42;          // int
auto d = 42.5;        // double
auto s = "text";      // char const *
auto v = { 1, 2, 3 }; // std::initializer_list<int>
\end{cpp}

\item
需要限定于某个特定类型时,局部变量的声明形式可为auto name = type-id \{ expression \}:

\begin{cpp}
auto b  = new char[10]{ 0 };            // char*
auto s1 = std::string {"text"};         // std::string
auto v1 = std::vector<int> { 1, 2, 3 }; // std::vector<int>
auto p  = std::make_shared<int>(42);    // std::shared_ptr<int>
\end{cpp}

\item
需要定义命名的lambda函数时,除非该lambda需要传递或返回给另一个函数,否则声明形式为auto name = lambda-expression:

\begin{cpp}
auto upper = [](char const c) {return toupper(c); };
\end{cpp}

\item
定义lambda参数和返回值时:

\begin{cpp}
auto add = [](auto const a, auto const b) {return a + b;};
\end{cpp}

\item
不希望限定于某个特定类型时,用auto声明函数的返回类型:

\begin{cpp}
template <typename F, typename T>
auto apply(F&& f, T value)
{
    return f(value);
}
\end{cpp}
\end{itemize}

\mySubsubsection{}{How it works...}

auto说明符是实际类型的占位符。当使用auto时,编译器会根据以下实例推断实际类型:

\begin{itemize}
\item
auto用于声明变量时,从初始化变量使用的表达式的类型推断。

\item
auto用作函数返回类型的占位符时,从函数的后置返回类型或返回表达式的类型推断。
\end{itemize}

某些情况下,需要限定于特定类型。例如第一个例子中,编译器推断s的类型为char const *。如有一个std::string,则必须显式指定类型,v的类型也会推断为std::initializer\_list<int>。所以,规则指出推断的类型是std::initializer\_list<T>,其中T在我们的例子中为int。但这里可能想要一个std::vector<int>,那就必须在赋值号右侧显式指定类型。

使用auto代替实际类型的好处:

\begin{itemize}
\item
auto不可能出现未初始化的变量。这是开发人员在声明变量,并指定实际类型时常犯的一个错误。然而,使用auto要求初始化变量以推断类型。未初始化的变量会导致未定义行为,所以用已定义的值初始化变量很重要。

\item
使用auto确保始终使用预期的类型,防止隐式转换的发生。以下例子中,获取一个vector的大小,并将其赋值给一个局部变量。第一种情况下,变量的类型是int,尽管size()方法返回的是size\_t,所以会发生从size\_t到int的隐式转换。但对于类型使用auto将推断出正确的类型,即size\_t:

\begin{cpp}
auto v = std::vector<int>{ 1, 2, 3 };
// implicit conversion, possible loss of data
int size1 = v.size();
// OK
auto size2 = v.size();
// ill-formed (warning in gcc, error in clang & VC++)
auto size3 = int{ v.size() };
\end{cpp}

\item
auto促进了良好的面向对象实践,如优先考虑接口而非实现。这在面向对象编程的中很重要,提供了在不同实现之间切换的灵活性、模块化以及可测试性,因为模拟对象变得容易。指定的类型越少,代码就越通用,对未来的变化就越兼容,这是OOP的基本原则。

\item
auto意味着更少的输入(通常),对不关心的类型减少关注,常见的例子就是迭代器。当遍历一个范围时,并不会关心迭代器的具体类型。只会对迭代器本身感兴趣,所以auto可以节省输入(可能很长)名称,并帮助开发者专注于实际代码而不是类型名称。以下示例中,第一个for循环中显式使用了迭代器的类型。使用auto说明符的第二个循环看起来更简单,免去输入和对于实际类型的关注:

\begin{cpp}
std::map<int, std::string> m;
for (std::map<int, std::string>::const_iterator
it = m.cbegin();
it != m.cend(); ++it)
{ /*...*/ }
for (auto it = m.cbegin(); it != m.cend(); ++it)
{ /*...*/ }
\end{cpp}

\item
auto声明变量提供了一致的编码风格,类型始终位于右侧。如果动态分配对象,则需要在赋值的左右两侧都写类型,例如:int* p = new int(42)。使用auto后,类型仅需在右侧指定一次。
\end{itemize}

然而,使用auto也有一些需要注意的地方:

\begin{itemize}
\item
auto说明符只是类型的占位符,不是const/volatile和引用说明符的占位符。如果需要const/volatile和/或引用类型(\&),则需要显式指定。下面的例子中,foo类的get()成员函数返回一个int的引用;当变量x从返回值初始化时,编译器推断的类型是int,而不是int\&,所以对x所做的更改都不会影响到foo.x\_,为了达到目的应使用auto\&:

\begin{cpp}
class foo {
    int x;
public:
    foo(int const value = 0) :x{ value } {}
    int& get() { return x; }
};
foo f(42);
auto x = f.get();
x = 100;
std::cout << f.get() << '\n'; // prints 42
\end{cpp}

\item
对于不可移动的类型,不能使用auto:

\begin{cpp}
auto ai = std::atomic<int>(42); // error
\end{cpp}

\item
对于多词类型,如long long、long double或struct foo,不能使用auto。但第一种情况,可能的解决办法是使用字面量或类型别名;此外,在Clang和GCC(不是MSVC)中,可以将类型名称放在括号内,(long long)\{ 42 \}。至于第二种情况,以这种方式使用struct/class主要是为了兼容C语言,无论如何都应该避免:

\begin{cpp}
auto l1 = long long{ 42 }; // error
using llong = long long;
auto l2 = llong{ 42 };     // OK
auto l3 = 42LL;            // OK
auto l4 = (long long){ 42 }; // OK with gcc/clang
\end{cpp}

\item
如果使用了auto说明符,仍需要知道类型,在大多数IDE中可以通过将光标放在变量上来查看。离开了IDE,就很难知道实际类型了,只能自己从初始化表达式中推断出来,这需要在代码中查找函数的返回类型。
\end{itemize}

auto可用于指定函数的返回类型。C++11中,需要在函数声明中使用后置返回类型。而在C++14中,这一限制放宽了,返回值的类型由编译器从返回表达式中推断得出。如果有多个返回值,则具有相同的类型:

\begin{cpp}
// C++11
auto func1(int const i) -> int
{ return 2*i; }
// C++14
auto func2(int const i)
{ return 2*i; }
\end{cpp}

auto不会保留const/volatile和引用修饰符,auto作为函数返回类型占位符时的问题。以先前的foo.get()为例,这次有一个名为proxy\_get()的包装函数,接受foo的引用,调用get()并返回get()返回的值,这是一个int\&。但编译器会推断proxy\_get()的返回类型为int,而不是int\&。

尝试将该值赋给int\&会出错:

\begin{cpp}
class foo
{
    int x_;
public:
    foo(int const x = 0) :x_{ x } {}
    int& get() { return x_; }
};
auto proxy_get(foo& f) { return f.get(); }
auto f = foo{ 42 };
auto& x = proxy_get(f); // cannot convert from 'int' to 'int &'
\end{cpp}

要解决这个问题,需要返回auto\&。但对于模板和完美转发返回类型,不了解它是值还是引用的情况,存在一个问题。C++14中的解决方案是decltype(auto),这样可以正确推断类型:

\begin{cpp}
decltype(auto) proxy_get(foo& f)
{ return f.get(); }
auto f = foo{ 42 };
decltype(auto) x = proxy_get(f);
\end{cpp}

decltype说明符用于检查实体或表达式的声明类型。当声明类型很麻烦或者根本无法用标准符号声明就很有用,例如:声明lambda类型和依赖于模板参数的类型。

最后一个重要案例是在lambda中使用auto。自C++14起,既可以在lambda返回类型中也可以在lambda参数类型中使用auto。这样的lambda称为泛型lambda,因为lambda定义的闭包类型具有模板化的调用操作符。以下是一个泛型lambda的例子,接受两个auto参数并返回应用operator+到实际类型的结果:

\begin{cpp}
auto ladd = [] (auto const a, auto const b) { return a + b; };
\end{cpp}

编译器生成的函数对象具有以下形式,其中调用操作符是一个函数模板:

\begin{cpp}
struct
{
    template<typename T, typename U>
    auto operator () (T const a, U const b) const { return a+b; }
} L;
\end{cpp}

此lambda可以用于添加定义了operator+的操作数:

\begin{cpp}
auto i = ladd(40, 2);            // 42
auto s = ladd("forty"s, "two"s); // "fortytwo"s
\end{cpp}

这个例子中,使用ladd lambda来加两个整数,并且将它们连接成std::string对象(使用C++14的用户定义字面量操作符""s)。























