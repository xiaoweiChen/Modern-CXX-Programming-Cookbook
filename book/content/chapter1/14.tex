模板在C++中无处不在,但每次都要指定模板参数可能会令人厌烦。幸运的是,有些情况下编译器可以从上下文中推断出模板参数。这项C++17中引入的功能称为类型模板参数推导(Class Template Argument Deduction,CTAD),它使得编译器可以根据初始化器的类型推导出缺失的模板参数。在本示例中,将介绍如何利用这一特性。

\mySubsubsection{}{How to do it...}

C++17中遇到以下情况时,可以跳过指定模板参数,让编译器推导:

\begin{itemize}
\item
声明变量或变量模板并初始化时:

\begin{cpp}
std::pair   p{ 42, "demo" };  // 推导为 std::pair<int, char const*>
std::vector v{ 1, 2 };        // 推导为 std::vector<int>
std::less   l;                // 推导为 std::less<void>
\end{cpp}

\item
使用new表达式创建类型实例时:

\begin{cpp}
template <class T>
struct foo
{
    foo(T v) :data(v) {}
    private:
    T data;
};
auto f = new foo(42);
\end{cpp}

\item
执行函数风格的类型转换表达式时:

\begin{cpp}
std::mutex mx;
// 推导为 std::lock_guard<std::mutex>
auto lock = std::lock_guard(mx);
std::vector<int> v;
// 推导为 std::back_insert_iterator<std::vector<int>>
std::fill_n(std::back_insert_iterator(v), 5, 42);
\end{cpp}
\end{itemize}

\mySubsubsection{}{How it works...}

C++17之前,初始化变量时必须指定所有的模板参数,这些参数都是实例化类模板所必需:

\begin{cpp}
std::pair<int, char const*> p{ 42, "demo" };
std::vector<int>            v{ 1, 2 };
foo<int>                    f{ 42 };
\end{cpp}

通过函数模板(如std::make\_pair())可以避免显式指定模板参数的问题:

\begin{cpp}
auto p = std::make_pair(42, "demo");
\end{cpp}

对于这里展示的foo类模板,可以通过编写类似的make\_foo()函数模板来实现同样的行为:

\begin{cpp}
template <typename T>
constexpr foo<T> make_foo(T&& value)
{
    return foo{ value };
}
auto f = make_foo(42);
\end{cpp}

C++17中,这种做法不再必要。以以下声明为例:

\begin{cpp}
std::pair p{ 42, "demo" };
\end{cpp}

std::pair并不是一个具体的类型,而是充当激活类模板参数推导的占位符。当编译器在带有初始化的变量声明或函数风格的类型转换时,会构建一组推导指南。这些推导指南就像是假设类型构造函数的集合。

用户可以通过自定义推导规则来补充这组推导指南,这组推导指南用于执行模板参数推导和重载解析。

以std::pair为例,编译器将构建一组推导指南,其中包括但不限于以下假设的函数模板:

\begin{cpp}
template <class T1, class T2>
std::pair<T1, T2> F();
template <class T1, class T2>
std::pair<T1, T2> F(T1 const& x, T2 const& y);
template <class T1, class T2, class U1, class U2>
std::pair<T1, T2> F(U1&& x, U2&& y);
\end{cpp}

这些由编译器生成的推导指南是从类型模板的构造函数中创建的,如果没有构造函数,则为假设的默认构造函数创建推导指南。此外,都会为假设的复制构造函数创建推导指南。

自定义推导指南是带有尾随返回类型的函数签名,但没有auto关键字(没有返回值的假设构造函数),必须定义在应用的类型模板的命名空间中。

为了理解其工作原理,来看一个std::pair实例的例子:

\begin{cpp}
std::pair p{ 42, "demo" };
\end{cpp}

编译器推导出的类型是std::pair<int, char const*>。如果希望编译器将字符串 "demo" 的类型推导为 std::string,则需要几个自定义推导规则:

\begin{cpp}
namespace std {
    template <class T>
    pair(T&&, char const*)->pair<T, std::string>;
    template <class T>
    pair(char const*, T&&)->pair<std::string, T>;
    pair(char const*, char const*)->pair<std::string, std::string>;
}
\end{cpp}

这将使我们能够执行以下声明,其中字符串 "demo" 的类型始终推导为 std::string:

\begin{cpp}
std::pair  p1{ 42, "demo" };    // std::pair<int, std::string>
std::pair  p2{ "demo", 42 };    // std::pair<std::string, int>
std::pair  p3{ "42", "demo" };  // std::pair<std::string, std::string>
\end{cpp}

\begin{myNotic}{}
推导指南不一定是函数模板!
\end{myNotic}

如果存在模板参数列表,无论指定了多少参数,都不会发生类模板参数推导。例如:

\begin{cpp}
std::pair<>    p1 { 42, "demo" };
std::pair<int> p2 { 42, "demo" };
\end{cpp}

因为这两行声明都指定了模板参数列表,所以无效,会产生编译错误。

还有一些已知的情况下,类模板参数推导不起作用:

\begin{itemize}
\item
聚合模板,通过编写自定义推导指南来解决这个问题:

\begin{cpp}
template<class T>
struct Point3D { T x; T y; T z; };

Point3D p{1, 2, 2};   // error, requires Point3D<int>
\end{cpp}

\item
类型别名,如以下示例所示(GCC,使用 -std=c++20 编译时可以工作):

\begin{cpp}
template <typename T>
using my_vector = std::vector<T>;
std::vector v{1,2,3}; // OK
my_vector mv{1,2,3};  // error
\end{cpp}

\item
继承的构造函数,无论是隐式的还是用户定义的推导指南,在构造函数继承时都不会继承:

\begin{cpp}
template <typename T>
struct box
{
    box(T&& t) : content(std::forward<T>(t)) {}
    virtual void unwrap()
    { std::cout << "unwrapping " << content << '\n'; }
    T content;
};
template <typename T>
struct magic_box : public box<T>
{
    using box<T>::box;
    virtual void unwrap() override
    { std::cout << "unwrapping " << box<T>::content << '\n'; }
};
int main()
{
    box b(42);        // OK
    b.unwrap();
    magic_box m(21);  // error, requires magic_box<int>
    m.unwrap();
}
\end{cpp}

\end{itemize}

上述最后一个限制在C++23中已经移除,此时当构造函数继承时,推导指南也会继承。















