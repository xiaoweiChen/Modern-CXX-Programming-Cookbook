某些程序必须以各种方式将数据持久化到磁盘文件中,包括在数据库或平面文件中存储数据,无论是文本还是二进制数据。本示例专注于从二进制文件中持久化和加载原始数据和对象。

原始数据指的是未结构化的数据,这里将考虑写入和读取一个缓冲区(即连续的内存序列)的内容,这可以是一个数组、一个std::vector或一个std::array。

\mySubsubsection{}{Getting ready}

应该熟悉标准流I/O库,尽管接下来会提供一些必要的解释来帮助理解本示例。应该了解二进制文件和文本文件之间的区别。

本示例中,将使用<fstream>头文件中std命名空间下的ofstream和ifstream类。

\mySubsubsection{}{How to do it...}

要将缓冲区的内容(例子中是一个std::vector)写入二进制文件,应该按照以下步骤操作:

\begin{enumerate}
\item
创建一个std::ofstream类的实例,以二进制模式打开一个用于写入的文件流:

\begin{cpp}
std::ofstream ofile("sample.bin", std::ios::binary);
\end{cpp}

\item
向文件写入数据之前确保文件已打开:

\begin{cpp}
if(ofile.is_open())
{
    // streamed file operations
}
\end{cpp}

\item
提供指向字符数组的指针和要写入的字符数来写入数据。下面的例子中,将vector的部分内容写入文件:

\begin{cpp}
std::vector<unsigned char> output {0,1,2,3,4,5,6,7,8,9};
ofile.write(reinterpret_cast<char*>(output.data()), output.size());
\end{cpp}

\item
可选地,可以通过调用flush()方法将流的输出缓冲区的内容刷新到实际的磁盘文件。这使得流中的未提交更改与外部目标同步,这种情况下是磁盘文件。

\item
通过调用close()关闭流。反过来又会调用flush(),所以前一步骤好像没那么必要:

\begin{cpp}
ofile.close();
\end{cpp}
\end{enumerate}

要将整个二进制文件的内容读取到缓冲区,应该按照以下步骤操作:

\begin{enumerate}
\item
创建std::ifstream类型的实例,以二进制模式打开一个用于读取的文件流。文件路径可以是相对于当前工作目录的绝对路径或相对路径(不是相对于可执行文件的路径)。本例中,是相对路径:

\begin{cpp}
std::ifstream ifile("sample.bin", std::ios::binary);
\end{cpp}

\item
从文件读取数据之前确保文件已打开:

\begin{cpp}
if(ifile.is_open())
{
    // streamed file operations
}
\end{cpp}

\item
将输入位置指示器定位到文件末尾,读取其值。将指示器移动到开头,以确定文件长度:

\begin{cpp}
ifile.seekg(0, std::ios_base::end);
auto length = ifile.tellg();
ifile.seekg(0, std::ios_base::beg);
\end{cpp}

\item
分配内存以读取文件内容:

\begin{cpp}
std::vector<unsigned char> input;
input.resize(static_cast<size_t>(length));
\end{cpp}

\item
提供指向字符数组的指针,以接收数据和要读取的字符数,从而将文件内容读取到分配的缓冲区:

\begin{cpp}
ifile.read(reinterpret_cast<char*>(input.data()), length);
\end{cpp}

\item
检查读取操作是否成功完成:

\begin{cpp}
auto success = !ifile.fail() && length == ifile.gcount();
\end{cpp}

\item
最后,关闭文件流:

\begin{cpp}
ifile.close();
\end{cpp}
\end{enumerate}

\mySubsubsection{}{Hot it works...}

标准流基I/O库提供了多个实现高级输入、输出或同时输入输出文件流、字符串流和字符数组操作的类,控制这些流行为的操纵符,以及几个预定义的流对象(cin/wcin、cout/wcout、cerr/wcerr和clog/wclog)。

这些流作为类模板实现。对于文件,库提供了几个(不可复制的)类型:

\begin{itemize}
\item
\verb|basic_filebuf|实现了针对原始文件的I/O操作,其语义类似于C语言中的FILE流。

\item
\verb|basic_ifstream|实现了由\verb|basic_istream|流接口定义的高级文件流输入操作,内部使用了一个\verb|basic_filebuf|实例。

\item
\verb|basic_ofstream|实现了由\verb|basic_ostream|流接口定义的高级文件流输出操作,内部使用了一个\verb|basic_filebuf|实例。

\item
\verb|basic_fstream|实现了由\verb|basic_iostream|流接口定义的高级文件流输入和输出操作,内部使用了一个\verb|basic_filebuf|实例。
\end{itemize}

为了更好地理解这些类型的关系,在下面的类图中进行展示:

\myGraphic{0.8}{content/chapter7/images/1.png}{图7.1: 流的类型图}

该图还展示了几个设计为与基于字符串的流一起工作的类型,这里就不多加讨论了。

<fstream>头文件中,std命名空间下也定义了前面提到的类模板的一些类型定义。前面的例子中使用的ofstream和ifstream实例就是这些类型定义的同义词:

\begin{cpp}
typedef basic_ifstream<char>    ifstream;
typedef basic_ifstream<wchar_t> wifstream;
typedef basic_ofstream<char>    ofstream;
typedef basic_ofstream<wchar_t> wofstream;
typedef basic_fstream<char>     fstream;
typedef basic_fstream<wchar_t>  wfstream;
\end{cpp}

上一节中,了解了如何将原始数据写入和从文件流中读取。现在,将更详细地介绍这一过程。

要将数据写入文件,创建了一个std::ofstream类型的实例。构造函数中,传递了要打开的文件名和流的打开模式,其中指定了std::ios::binary表示二进制模式。以这种方式打开文件会丢弃先前的文件内容。如果想向现有文件追加内容,可以使用std::ios::app标志(即std::ios::app | std::ios::binary)。此构造函数内部对其底层的原始文件实例(\verb|basic_filebuf|实例)调用了open()。如果此操作失败,则会设置一个错误位。为了检查流是否已成功关联到文件设备,使用了\verb|is_open()|(内部调用了底层\verb|basic_filebuf|方法的同名方法)。通过write()方法将数据写入文件流,该方法接受一个指向要写入的字符字符串的指针和要写入的字符数。由于此方法以字符字符串操作,如果数据是其他类型,例子中的unsigned char,则需要进行\verb|reinterpret_cast|。写入操作在失败时不设置错误位,但会抛出\verb|std::ios_base::failure|异常。然而,数据不直接写入文件设备,而是存储在\verb|basic_filebuf|实例中。为了将其写入文件,需要刷新缓冲区,可通过调用flush()完成。关闭文件流时,此操作会自动完成,如前面的例子所示。

要从文件读取数据,可以创建了一个std::ifstream类型的实例。构造函数中,传递了与打开文件写入时相同的参数,以指定文件名和打开模式(即std::ios::binary)。构造函数内部对其底层的\verb|std::basic_filebuf|实例调用了open()。为了检查流是否已成功关联到文件设备,这里使用了\verb|is_open()|(内部调用了底层\verb|basic_filebuf|方法的同名方法)。这个例子中,将整个文件内容读取到了内存缓冲区,特别是std::vector中。在读取数据之前,必须知道文件的大小,以便分配足够大的缓冲区来容纳这些数据。为此,使用seekg()将输入位置指示器移动到文件末尾。

然后,调用tellg()返回当前位置,以表示文件的大小(以字节为单位),然后将输入位置指示器移动回文件开头,以便可以从开头开始读取。通过在构造函数(或open()方法)中使用std::ios::ate打开标志,可以避免调用seekg()将位置指示器移动到末尾的操作,这样可以立即将位置指示器移动到文件末尾。在为文件内容分配足够的内存后,使用read()方法将数据从文件复制到内存。可接受一个指向接收从流中读取数据的字符串指针,和要读取的字符长度。由于流以字符操作,如果缓冲区包含其他类型的数据,例如例子中的unsigned char,则需要\verb|reinterpret_cast|对类型进项转换。

如果发生错误,此操作会抛出\verb|std::basic_ios::failure|异常。为了确定从流中成功读取的字符数,可以使用gcount()方法。完成读取操作后,关闭文件流。

\begin{myNotic}
确定已打开文件大小的替代方案是使用文件系统库中的\verb|std::filesystem::file_size()|函数。这只需要一个路径,不需要打开文件。还可以确定目录的大小,但这取决于具体实现。
\end{myNotic}

这些示例中显示的操作,是将数据写入和从文件流中读取所需的最小操作。然而,重要的是要对操作的成功性进行适当的检查,并捕获可能发生的异常。

需要注意的是代表要写入或读取的字符数的参数值。在例子中,使用了unsigned char类型的缓冲区。unsigned char的大小为1,就像char一样,字符的数量等于缓冲区中的元素数量。但如果缓冲区包含int类型的元素,情况就会发生变化。int通常是32位的,重新解释为char时,相当于4个字符。所以,当写入任何大于1的大小的数据时,需要将元素数量乘以元素的大小:

\begin{cpp}
std::vector<int> numbers{ 0, 1, 2, 3, 4, 5, 6, 7, 8, 9 };
std::ofstream ofile("sample.bin", std::ios::binary);
if (ofile.is_open())
{
    ofile.write(reinterpret_cast<char*>(numbers.data()),
                numbers.size() * sizeof(int));
    ofile.close();
}
\end{cpp}

读取数据时,也需要考虑到从文件中读取的元素的大小,这在接下来的例子中有所体现:

\begin{cpp}
std::vector<int> input;
std::ifstream ifile("sample.bin", std::ios::binary);
if (ifile.is_open())
{
    ifile.seekg(0, std::ios_base::end);
    auto length = ifile.tellg();
    ifile.seekg(0, std::ios_base::beg);
    input.resize(static_cast<size_t>(length) / sizeof(int));
    ifile.read(reinterpret_cast<char*>(input.data()), length);
    assert(!ifile.fail() && length == ifile.gcount());
    ifile.close();
}
\end{cpp}

本示例中讨论的示例代码可以重组为两个通用函数的形式,用于将数据写入文件和从文件中读取数据:

\begin{cpp}
bool write_data(char const * const filename,
    char const * const data,
    size_t const size)
{
    auto success = false;
    std::ofstream ofile(filename, std::ios::binary);
    if(ofile.is_open())
    {
        try
        {
            ofile.write(data, size);
            success = true;
        }
        catch(std::ios_base::failure &)
        {
            // handle the error
        }
        ofile.close();
    }
    return success;
}
size_t read_data(char const * const filename,
std::function<char*(size_t const)> allocator)
{
    size_t readbytes = 0;
    std::ifstream ifile(filename, std::ios::ate | std::ios::binary);
    if(ifile.is_open())
    {
        auto length = static_cast<size_t>(ifile.tellg());
        ifile.seekg(0, std::ios_base::beg);
        auto buffer = allocator(length);
        try
        {
            ifile.read(buffer, length);
            readbytes = static_cast<size_t>(ifile.gcount());
        }
        catch (std::ios_base::failure &)
        {
            // handle the error
        }
        ifile.close();
    }
    return readbytes;
}
\end{cpp}

\verb|write_data()| 是一个函数,接受文件名、指向字符数组的指针和该数组的长度作为参数,并将字符写入指定的文件。\verb|read_data()| 是一个函数,接受文件名和一个分配缓冲区的函数,并将文件的全部内容读取到由分配函数返回的缓冲区中。以下是这两个函数的使用示例:

\begin{cpp}
std::vector<int> output {0, 1, 2, 3, 4, 5, 6, 7, 8, 9};
std::vector<int> input;
if(write_data("sample.bin",
               reinterpret_cast<char*>(output.data()),
               output.size() * sizeof(int)))
{
    auto lalloc = [&input](size_t const length)
    {
        input.resize(length) / sizeof(int);
        return reinterpret_cast<char*>(input.data());
    };
    if(read_data("sample.bin", lalloc) > 0)
    {
        std::cout << (output == input ? "equal": "not equal")
        << '\n';
    }
}
\end{cpp}

或者,可以使用动态分配的缓冲区,而不是std::vector:

\begin{cpp}
std::vector<int> output {0, 1, 2, 3, 4, 5, 6, 7, 8, 9};
std::unique_ptr<int[]> input = nullptr;
size_t readb = 0;
if(write_data("sample.bin",
              reinterpret_cast<char*>(output.data()),
              output.size() * sizeof(int)))
{
    if((readb = read_data(
    "sample.bin",
    [&input](size_t const length) {
        input.reset(new int[length / sizeof(int)]);
        return reinterpret_cast<char*>(input.get()); })) > 0)
    {
        auto cmp = memcmp(output.data(), input.get(), output.size());
        std::cout << (cmp == 0 ? "equal": "not equal") << '\n';
    }
\end{cpp}

然而,提供这种替代方法只是为了说明\verb|read_data()|可以与不同种类的输入缓冲区一起使用。建议尽可能避免显式的动态内存分配。

\mySubsubsection{}{There's more...}

本示例中从文件读取数据到内存的方式,只是多种方式的一种。以下是读取文件流数据的几种可能的替代方法:

\begin{itemize}
\item
使用\verb|std::istreambuf_iterator|迭代器直接初始化一个std::vector(这也可以用于std::string):

\begin{cpp}
std::vector<unsigned char> input;
std::ifstream ifile("sample.bin", std::ios::binary);
if(ifile.is_open())
{
    input = std::vector<unsigned char>(
        std::istreambuf_iterator<char>(ifile),
        std::istreambuf_iterator<char>());
    ifile.close();
}
\end{cpp}

\item
使用\verb|std::istreambuf_iterator|迭代器给一个std::vector赋值:

\begin{cpp}
std::vector<unsigned char> input;
std::ifstream ifile("sample.bin", std::ios::binary);
if(ifile.is_open())
{
    ifile.seekg(0, std::ios_base::end);
    auto length = ifile.tellg();
    ifile.seekg(0, std::ios_base::beg);
    input.reserve(static_cast<size_t>(length));
        input.assign(
        std::istreambuf_iterator<char>(ifile),
        std::istreambuf_iterator<char>());
    ifile.close();
}
\end{cpp}

\item
使用\verb|std::istreambuf_iterator|迭代器和\verb|std::back_inserter|适配器将文件流的内容复制到vector中:

\begin{cpp}
std::vector<unsigned char> input;
std::ifstream ifile("sample.bin", std::ios::binary);
if(ifile.is_open())
{
    ifile.seekg(0, std::ios_base::end);
    auto length = ifile.tellg();
    ifile.seekg(0, std::ios_base::beg);
    input.reserve(static_cast<size_t>(length));
    std::copy(std::istreambuf_iterator<char>(ifile),
              std::istreambuf_iterator<char>(),
              std::back_inserter(input));
    ifile.close();
}
\end{cpp}
\end{itemize}

然而,与这些替代方法相比,“How to do it...”部分中描述的方法是最快的。即使从面向对象的角度来看,这些替代方法可能看起来更具吸引力。

