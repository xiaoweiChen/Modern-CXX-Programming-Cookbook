C++17 标准的一个重要补充是文件系统库,能够在层次化的文件系统(如 Windows 或 POSIX 文件系统)中处理路径、文件和目录。这个标准库是基于 boost.filesystem 库开发的。接下来的几个示例中,将探索该库的功能,这些功能能够对文件和目录进行操作,如创建、移动或删除,同时也包括查询属性和搜索。首先,了解这个库如何对路径进行处理。

\mySubsubsection{}{Getting ready}

本示例中,将主要考虑使用 Windows 路径的例子。附带的代码中,所有示例都有 Windows 和 POSIX 的替代方案。

文件系统库位于 std::filesystem 命名空间中,<filesystem> 头文件中。为了简化代码,将在所有示例中使用以下命名空间别名:

\begin{cpp}
namespace fs = std::filesystem;
\end{cpp}

文件系统组件(文件、目录、硬链接或软链接)的路径由 path 类型表示。

\mySubsubsection{}{How to do it...}

以下是路径上最常见的操作列表:

\begin{itemize}
\item
使用构造函数、赋值操作符或 assign() 方法创建路径:

\begin{cpp}
// Windows
auto path = fs::path{"C:\\Users\\Marius\\Documents"};
// POSIX
auto path = fs::path{ "/home/marius/docs" };
\end{cpp}

\item
使用成员操作符 /=、非成员操作符 / 或 append() 添加包含目录分隔符的元素到路径:

\begin{cpp}
path /= "Book";
path = path / "Modern" / "Cpp";
path.append("Programming");
// Windows: C:\Users\Marius\Documents\Book\Modern\Cpp\Programming
// POSIX:   /home/marius/docs/Book/Modern/Cpp/Programming
\end{cpp}

\item
使用成员操作符 +=、非成员操作符 + 或 concat() 方法连接元素到路径,而不包括目录分隔符:

\begin{cpp}
auto path = fs::path{ "C:\\Users\\Marius\\Documents" };
path += "\\Book";
path.concat("\\Modern");
// path = C:\Users\Marius\Documents\Book\Modern
\end{cpp}

\item
使用成员函数如 \verb|root_name()|, \verb|root_dir()|, filename(), stem(), extension() 等将路径的元素分解为其组成部分,如根目录名、根目录、父路径、文件名、扩展名等(所有这些都在以下示例中展示):

\begin{cpp}
auto path =
    fs::path{"C:\\Users\\Marius\\Documents\\sample.file.txt"};
std::cout
    << "root: "        << path.root_name() << '\n'
    << "root dir: "    << path.root_directory() << '\n'
    << "root path: "   << path.root_path() << '\n'
    << "rel path: "    << path.relative_path() << '\n'
    << "parent path: " << path.parent_path() << '\n'
    << "filename: "    << path.filename() << '\n'
    << "stem: "        << path.stem() << '\n'
    << "extension: "   << path.extension() << '\n';
\end{cpp}

\item
使用成员函数如 \verb|has_root_name()|, \verb|has_root_directory()|, \verb|has_filename()|, \verb|has_stem()| 和 \verb|has_extension()| 查询路径是否存在(所有这些都在以下示例中展示):

\begin{cpp}
auto path =
    fs::path{"C:\\Users\\Marius\\Documents\\sample.file.txt"};
std::cout
    << "has root: "        << path.has_root_name() << '\n'
    << "has root dir: "    << path.has_root_directory() << '\n'
    << "has root path: "   << path.has_root_path() << '\n'
    << "has rel path: "    << path.has_relative_path() << '\n'
    << "has parent path: " << path.has_parent_path() << '\n'
    << "has filename: "    << path.has_filename() << '\n'
    << "has stem: "        << path.has_stem() << '\n'
    << "has extension: "   << path.has_extension() << '\n';
\end{cpp}

\item
检查路径是相对的还是绝对的:

\begin{cpp}
auto path2 = fs::path{ "marius\\temp" };
std::cout
    << "absolute: " << path1.is_absolute() << '\n'
    << "absolute: " << path2.is_absolute() << '\n';
\end{cpp}

\item
使用 \verb|replace_filename()| 和 \verb|remove_filename()| 修改路径的文件名部分,使用 \verb|replace_extension()| 修改扩展名部分:

\begin{cpp}
auto path =
    fs::path{"C:\\Users\\Marius\\Documents\\sample.file.txt"};
path.replace_filename("output");
path.replace_extension(".log");
// path = C:\Users\Marius\Documents\output.log
path.remove_filename();
// path = C:\Users\Marius\Documents
\end{cpp}

\item
将目录分隔符转换为系统的分隔符:

\begin{cpp}
// Windows
auto path = fs::path{"Users/Marius/Documents"};
path.make_preferred();
// path = Users\Marius\Documents
// POSIX
auto path = fs::path{ "\\home\\marius\\docs" };
path.make_preferred();
// path = /home/marius/docs
\end{cpp}
\end{itemize}

\mySubsubsection{}{Hot it works...}

std::filesystem::path 类型创建了文件系统组件的路径。但仅处理语法,并不验证路径所表示的组件(如文件或目录)的存在。

该库定义了一种可移植的、通用的路径语法,可以适应各种文件系统,如 POSIX 或 Windows,包括 Microsoft Windows 的通用命名约定(UNC)格式。有几个关键方面会有所不同:

\begin{itemize}
\item
POSIX 系统有一个单一的树状结构,没有根名称,有一个称为 / 的单一根目录,以及单一的当前目录。此外,使用 / 作为目录分隔符。路径表示为以空字符终止的 UTF-8 编码的 char 字符串。

\item
Windows 系统有多个树状结构,每个目录树都有一个根名称(如 C:)、一个根目录(如 \verb||)和一个当前目录(如 \verb|C:\Windows\System32|)。路径表示为以空字符终止的 UTF-16 编码的宽字符字符串。
\end{itemize}

\begin{myNotic}
不应该跨不同系统混合路径格式。虽然 Windows 可以处理 POSIX 路径,但反之则不然。请使用特定于每个系统的路径格式。此外,可以使用文件系统::path 功能,如操作符 /= 和 append() 函数,以及 \verb|preferred_separator| 静态成员,以可移植的方式构建路径。
\end{myNotic}

根据文件系统库定义的路径名具有以下语法:

\begin{itemize}
\item
可选的根名称(如 C: 或 //localhost)

\item
可选的根目录

\item
零个或多个文件名(可以指代文件、目录、硬链接或符号链接)或目录分隔符
\end{itemize}

有两个特殊的文件名可识别:单点(.),表示当前目录;双点(..),表示父目录。目录分隔符可以重复出现,在这种情况下被视为单个分隔符(换句话说,/home////docs 与 /home/marius/docs 相同)。一个没有冗余当前目录名(.)、没有冗余父目录名(..)且没有冗余目录分隔符的路径被认为是处于规范化形式。

上一节介绍的路径操作是最常见的路径操作,实现还定义了查询和修改方法、迭代器、非成员比较操作符等。

以下示例遍历路径的各个部分,并将其输出到控制台:

\begin{cpp}
auto path =
    fs::path{ "C:\\Users\\Marius\\Documents\\sample.file.txt" };
for (auto const & part : path)
{
    std::cout << part << '\n';
}
\end{cpp}

以下是该示例的结果:

\begin{shell}
C:
Users
Marius
Documents
sample.file.txt
\end{shell}

sample.file.txt 是文件名,这是从最后一个目录分隔符到路径结尾的部分,是给定路径上调用成员函数 filename() 会返回的内容。该文件的扩展名为 .txt,这是由成员函数 extension() 返回的字符串。为了获取不带扩展名的文件名,还有一个叫做 stem() 的成员函数可用,该方法返回的字符串是 sample.file。对于所有这些方法,以及所有的其他分解方法,都有一个同名且以 \verb|has_| 为前缀的对应查询方法,如 \verb|has_filename()|, \verb|has_stem()| 和 \verb|has_extension()|。所有这些方法都返回一个 bool 值,以表明路径是否具有相应的部分。

