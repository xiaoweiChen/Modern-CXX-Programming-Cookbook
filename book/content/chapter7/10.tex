

诸如复制、重命名、移动或删除文件的操作直接由文件系统库提供。当涉及到从文件中删除内容时,必须执行明确的动作。

无论处理的是文本文件还是二进制文件,都可以使用以下模式:

\begin{enumerate}
\item
创建一个临时文件。

\item
仅将原始文件中所需的内容复制到临时文件中。

\item
删除原始文件。

\item
将临时文件重命名/移动到原始文件的名称/位置。
\end{enumerate}

本示例中,将介绍如何为文本文件实现这种模式。

\mySubsubsection{}{Getting ready}

本食示例将考虑从文本文件中删除空行或以分号 (;) 开头的行。这个例子中,有一个初始文件,称为 sample.dat,其中包含莎士比亚戏剧的名字,但也包括空行和以分号开头的行。以下是该文件的部分列表(从开头部分):

\begin{shell}
;Shakespeare's plays, listed by genre
;TRAGEDIES
Troilus and Cressida
Coriolanus
Titus Andronicus
Romeo and Juliet
Timon of Athens
Julius Caesar
\end{shell}

下一节中列出的代码示例使用了以下变量:

\begin{cpp}
auto path = fs::current_path();
auto filepath = path / "sample.dat";
auto temppath = path / "sample.tmp";
auto err = std::error_code{};
\end{cpp}

在接下来的部分,会教授如何将这种模式编码实现。

\mySubsubsection{}{How to do it...}

执行以下操作以从文件中删除内容:

\begin{enumerate}
\item
打开文件以读取:

\begin{cpp}
std::ifstream in(filepath);
if (!in.is_open())
{
    std::cout << "File could not be opened!" << '\n';
    return;
}
\end{cpp}

\item
打开另一个临时文件以供写入;如果文件已存在,则截断其内容:

\begin{cpp}
std::ofstream out(temppath, std::ios::trunc);
if (!out.is_open())
{
    std::cout << "Temporary file could not be created!" << '\n';
    return;
}
\end{cpp}

\item
从输入文件中逐行读取,并将选定的内容复制到输出文件中:

\begin{cpp}
auto line = std::string{};
while (std::getline(in, line))
{
    if (!line.empty() && line.at(0) != ';')
    {
        out << line << 'n';
    }
}
\end{cpp}

\item
关闭输入和输出文件:

\begin{cpp}
in.close();
out.close();
\end{cpp}

\item
删除原始文件:

\begin{cpp}
auto success = fs::remove(filepath, err);
if(!success || err)
{
    std::cout << err.message() << '\n';
    return;
}
\end{cpp}

\item
将临时文件重命名/移动到原始文件的名称/位置:

\begin{cpp}
fs::rename(temppath, filepath, err);
if (err)
{
    std::cout << err.message() << '\n';
}
\end{cpp}
\end{enumerate}

\mySubsubsection{}{Hot it works...}

这里描述的模式同样适用于二进制文件,简洁起见,只讨论一个文本文件的例子。这个例子中,临时文件位于与原始文件相同的目录中。作为替代,可以位于单独的目录中,比如:用户临时目录。要获取临时目录的路径,可以使用 \verb|std::filesystem::temp_directory_path()|。Windows 系统上,此函数返回与 GetTempPath() 相同的目录; POSIX 系统上,返回环境变量 TMPDIR、TMP、TEMP 或 TEMPDIR 中指定的路径,如果这些变量均不可用,则返回路径 /tmp。

如何将原始文件的内容复制到临时文件中因需复制的内容而异。前面的例子中,除非行为空或以分号开头,否则将复制整行。

为此,使用 std::getline() 逐行读取原始文件的内容,直到没有更多的行可读。复制了所有必要的内容后,应该关闭文件以便它们可以移动或删除。

为了完成操作,有三种选择:

\begin{itemize}
\item
如果临时文件和原始文件在同一目录中,删除原始文件并将临时文件重命名为与原始文件相同的名称;如果位于不同的目录中,则将临时文件移动到原始文件的位置(这是本示例所采用的方法)。为此,可使用 remove() 函数删除原始文件,并使用 rename() 将临时文件重命名为原始文件名。

\item
将临时文件的内容复制回原始文件(可以使用 \verb|copy()| 或 \verb|copy_file()| 函数),然后删除临时文件(为此使用 \verb|remove()|)。

\item
重命名原始文件(例如,更改扩展名或名称),然后使用原始文件名重命名/移动临时文件。
\end{itemize}

\begin{myTip}
如果采用这里提到的第一种方法,那么必须确保后来替换原始文件的临时文件,具有与原始文件相同的文件权限;否则,根据解决方案的上下文,这可能会有问题。
\end{myTip}

