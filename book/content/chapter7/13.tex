前一个示例中,介绍了如何使用 \verb|directory_iterator| 和 \verb|recursive_directory_iterator| 来枚举目录的内容。展示目录内容,只是需要这样做的情景之一。另一种主要的情景是在目录中搜索特定条目,如具有特定名称、扩展名等的文件。本示例中,将介绍如何使用目录迭代器和前面展示的迭代模式,来查找符合给定条件的文件。

\mySubsubsection{}{How to do it...}

要查找符合特定标准的文件,请使用以下模式:

\begin{enumerate}
\item
使用 \verb|recursive_directory_iterator| 来迭代目录中的所有条目,并递归地迭代其子目录。

\item
考虑常规文件(以及需要处理的其他类型的文件)。

\item
使用函数对象(如lambda表达式)来过滤只匹配标准文件。

\item
将选定的条目添加到容器中(如vector)。
\end{enumerate}

这种模式在如下的 \verb|find_files()| 函数中得到了体现:

\begin{cpp}
std::vector<fs::path> find_files(
    fs::path const & dir,
    std::function<bool(fs::path const&)> filter)
{
    auto result = std::vector<fs::path>{};
    if (fs::exists(dir))
    {
        for (auto const & entry :
        fs::recursive_directory_iterator(
        dir,
        fs::directory_options::follow_directory_symlink))
        {
            if (fs::is_regular_file(entry) &&
            filter(entry))
            {
                result.push_back(entry);
            }
        }
    }
    return result;
}
\end{cpp}

\mySubsubsection{}{Hot it works...}

当想要在目录中查找文件时,目录的结构以及其条目(包括子目录)访问的顺序可能并不重要。因此,可以使用 \verb|recursive_directory_iterator| 来迭代。

\verb|find_files()| 函数接收两个参数:一个路径和一个用于选择应返回的条目的函数包装器。返回类型是一个包含 \verb|filesystem::path| 实例的vector。不过,也可以是一个包含 \verb|filesystem::directory_entry| 实例的vector。例子中,使用的递归目录迭代器不会跟随符号链接,而是返回链接本身而非目标。通过使用具有 \verb|filesystem::directory_options| 类型参数的构造函数重载并传递 \verb|follow_directory_symlink| 参数,可以改变这种行为。

前面的例子中,只考虑常规文件并忽略其他类型的文件系统实例。谓词应用于目录条目,如果返回 true,则该条目会添加到结果中。

下面的例子使用 \verb|find_files()| 函数,来查找测试目录中所有以 \verb|file_| 前缀开始的文件:

\begin{cpp}
auto results = find_files(
        fs::current_path() / "test",
        [](fs::path const & p) {
    auto filename = p.wstring();
    return filename.find(L"file_") != std::wstring::npos;
});
for (auto const & path : results)
{
    std::cout << path << '\n';
}
\end{cpp}

执行这个程序的输出,路径相对于当前路径:

\begin{shell}
test\file_1.txt
test\file_2.txt
test\file_3.log
\end{shell}

第二个例子展示了如何查找具有特定扩展名的文件,扩展名为 .dat:

\begin{cpp}
auto results = find_files(
        fs::current_path() / "test",
        [](fs::path const & p) {
            return p.extension() == L".dat";});
for (auto const & path : results)
{
    std::cout << path << '\n';
}
\end{cpp}

输出再次相对于当前路径,如下所示:

\begin{shell}
test\data\input.dat
test\data\output.dat
\end{shell}

这两个例子非常相似。唯一不同的是lambda函数中的代码,可检查作为参数接收的路径。


