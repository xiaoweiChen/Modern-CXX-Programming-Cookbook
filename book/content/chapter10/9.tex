
单例可能是最著名的设计模式,限制了一个类型实例只能实例化一次。尽管在某些情况下这是必要的,但很多时候单例的使用更像是一种反模式,可以通过其他的设计选择来避免。

由于单例意味着整个程序中只有一个类型实例可用,这样的唯一实例可能会再不同的线程访问。因此,当实现单例时,也应该使其线程安全。

C++11 之前,做到这一点并不容易,通常采用双重检查锁定技术。Scott Meyers 和 Andrei Alexandrescu 在一篇题为《C++ 和双重检查锁定的危险》的论文中指出,使用这一模式并不能保证在可移植的 C++ 中实现线程安全的单例。幸运的是,这种情况在 C++11 中发生了变化,本示例展示了如何在现代 C++ 中编写线程安全的单例。

\mySubsubsection{}{Getting ready}

对于本示例,需要了解静态存储持续时间、内部链接以及删除和默认函数的工作原理。如果还没有阅读过的话,也应该先阅读前一示例关于使用奇特递归模板模式(CRTP)实现静态多态的内容。

\mySubsubsection{}{How to do it...}

要实现线程安全的单例,应该按照以下步骤操作:

\begin{enumerate}
\item
定义 Singleton 类型:

\begin{cpp}
class Singleton
{
};
\end{cpp}

\item
将默认构造函数设为私有:

\begin{cpp}
private:
    Singleton() = default;
\end{cpp}

\item
将复制构造函数和复制赋值操作符设为公有并删除:

\begin{cpp}
public:
    Singleton(Singleton const &) = delete;
    Singleton& operator=(Singleton const&) = delete;
\end{cpp}

\item
创建和返回单一实例的函数应该是静态的,并且应该返回类类型的引用。应该声明一个类类型的静态实例,并返回对该实例的引用:

\begin{cpp}
public:
    static Singleton& instance()
    {
        static Singleton single;
        return single;
    }
\end{cpp}
\end{enumerate}

\mySubsubsection{}{How it works...}

由于不应该直接由用户创建单例实例,所有构造函数要么是私有的,要么是公有的但已删除。默认构造函数是私有的但未删除,因为类型代码中必须实际创建类的一个实例。这个实现中,一个名为 instance() 的静态函数返回类的单一实例。

\begin{myTip}
尽管大多数实现返回的是指针,但实际上返回引用更有意义,不会导致此函数返回空指针(没有对象)。
\end{myTip}

instance() 方法的实现乍看之下,可能显得简单且不线程安全(熟悉双重检查锁定模式(DCLP)的话)。C++11中,这实际上已不再必要,因为静态存储持续时间实例初始化的关键细节发生了变化。即使多个线程同时尝试初始化同一个静态实例,初始化也只会发生一次。DCLP 的责任已经从用户转移到了编译器,尽管编译器可能会使用另一种技术来确保结果。

以下是 C++ 标准文档版本 N4917 第 8.8.3 段落中定义静态实例初始化规则的引文:

\begin{myTip}
具有静态存储持续时间(6.7.5.2)或线程存储持续时间(6.7.5.3)的块变量的动态初始化在其声明首次执行时进行;这样的变量在其初始化完成后即认为已初始化。如果初始化过程中抛出了异常,则初始化未完成,因此下次执行进入声明时将再次尝试初始化。如果在变量正在初始化时并发执行进入声明,则并发执行应当等待初始化完成。

\begin{shell}
[注释:符合标准的实现不应在执行初始化器时引入任何死锁。死锁仍可能由程序逻辑引起;实现只需避免因自身的同步操作而导致的死锁。]
\end{shell}

如果在变量正在初始化时递归地重新进入声明,行为未定义。
\end{myTip}

静态局部实例具有静态存储持续时间,但只有在第一次使用时(即第一次调用 instance() 方法时)才会实例化。当程序退出时,该实例会销毁。顺便提一下,返回指针而不是引用的唯一可能的优势是,在程序退出前的某个时刻删除这个唯一实例,然后可能会重新创建。但这再创建没有太多意义,因为它与类的单一全局实例的概念相冲突,该实例可以在程序中的任何地方访问。

\mySubsubsection{}{There's more...}

在较大的代码库中,可能会出现需要多种单例类型的情况。为了避免多次编写相同的模式,可以以通用方式实现它。为此,需要采用本章前面提到的奇特递归模板模式(CRTP)。实际的单例是作为类模板实现的。instance() 方法创建并返回类型模板参数的示例,该参数将是派生类:

\begin{cpp}
template <class T>
class SingletonBase
{
    protected:
    SingletonBase() {}
public:
    SingletonBase(SingletonBase const &) = delete;
    SingletonBase& operator=(SingletonBase const&) = delete;
    static T& instance()
    {
        static T single;
        return single;
    }
};
class Single : public SingletonBase<Single>
{
    Single() {}
    friend class SingletonBase<Single>;
public:
    void demo() { std::cout << "demo" << '\n'; }
};
\end{cpp}

上一节中的 Singleton 类变成了 SingletonBase 类模板。默认构造函数不再是私有的而是受保护的,因为必须从派生类访问。这个例子中,需要实例化单个对象的类称为 Single,构造函数必须私有,但默认构造函数也必须对基类模板可用;因此,SingletonBase<Single> 是 Single 类型的“朋友”。


