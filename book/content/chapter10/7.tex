
前一个示例中,介绍了一种称为奇特递归模板模式(Curiously Recurring Template Pattern,简称 CRTP)的设计模式,以及如何用于向类型添加通用功能。但这并非其唯一的用途;其他用途还包括限制类型实例化的次数和实现组合模式。与此模式相关的是另一种称为混入(mixins)的模式。混入是小类型,旨在向其他现有类型添加功能,能找到一些文章声称混入模式是通过 CRTP 实现的,这并不正确。事实上,CRTP 和混入模式相似,两者都用于向类型添加功能,但结构不同。CRTP 中,基类向从中派生的类添加功能。而混入类则是向其派生自的类添加功能,所以混入可以说是一种倒置的 CRTP。本示例中,将介绍如何使用混入向现有类添加通用功能。为此,继续研究相同的绘制控件(如按钮和复选框)的例子。这将有助于与 CRTP 进行良好的比较,能够更好地理解两者之间的差异(和相似之处)。

\mySubsubsection{}{How to do it...}

为了使用混入模式向现有类添加通用功能,请按照以下步骤操作(下面的例子中,通用功能是指绘制控件的背景和内容):

\begin{enumerate}
\item
考虑表现出通用功能的类型(可能是不相关的类型):

\begin{cpp}
class button
{
public:
    void erase_background()
    {
        std::cout << "erasing button background..." << '\n';
    }
    void paint()
    {
        std::cout << "painting button..." << '\n';
    }
};
class checkbox
{
public:
    void erase_background()
    {
        std::cout << "erasing checkbox background..." << '\n';
    }
    void paint()
    {
        std::cout << "painting checkbox..." << '\n';
    }
};
\end{cpp}

\item
创建一个从其类型模板参数派生的类模板。这个混入类定义了一些新的功能,这些功能是基于基类的现有功能实现的:

\begin{cpp}
template <typename T>
class control : public T
{
public:
    void draw()
    {
        T::erase_background();
        T::paint();
    }
};
\end{cpp}

\item
实例化并使用混入类的实例来利用添加的功能:

\begin{cpp}
control<button> b;
b.draw();
control<checkbox> c;
c.draw();
\end{cpp}
\end{enumerate}


\mySubsubsection{}{How it works...}

混入是一种向现有类型添加新功能的概念,这个模式在许多编程语言中实现方式不同。C++ 中混入是一个小类,向现有的类添加功能(无需对现有类进行任何修改)。为此,需要:

\begin{itemize}
\item
让混入类成为一个模板。在例子中,这就是 control 类型。如果只有一个类型需要扩展,则无需使用模板,因为不会有代码重复。实际上,这可为多个类似类型添加通用功能。

\item
从其类型模板参数派生,该参数预期将用要扩展的类型实例化。通过重用类型模板参数类的功能来实现添加的功能。在例子中,新功能是 draw(),其使用了 \verb|T::erase_background()| 和 \verb|T::paint()|。
\end{itemize}

由于混入类是一个模板,不能多态地处理。例如,可能希望有一个函数能够绘制按钮、复选框和其他任何可绘制的控件。这样的函数可能像这样:

\begin{cpp}
void draw_all(std::vector<???*> const & controls)
{
    for (auto& c : controls)
    {
        c->draw();
    }
}
\end{cpp}

但在这一代码中,\verb|???| 代表什么?需要一个非模板的基类才能使其多态地工作。这样一个基类可能像这样:

\begin{cpp}
class control_base
{
    public:
    virtual ~control_base() {}
    virtual void draw() = 0;
};
\end{cpp}

混入类(control)也需要从这个基类(\verb|control_base|)派生,而且 draw() 函数将成为一个覆盖的虚函数:

\begin{cpp}
template <typename T>
class control : public control_base, public T
{
public:
    void draw() override
    {
        T::erase_background();
        T::paint();
    }
};
\end{cpp}

这能够多态地处理 control 实例:

\begin{cpp}
void draw_all(std::vector<control_base*> const & controls)
{
    for (auto& c : controls)
    {
        c->draw();
    }
}
int main()
{
    std::vector<control_base*> controls;
    control<button> b;
    control<checkbox> c;
    draw_all({&b, &c});
}
\end{cpp}

正如这个示例和前一个示例中看到的那样,混入和 CRTP 都是为了同一个目的,即向类添加功能。此外,尽管实际模式结构不同,看起来非常相似。


