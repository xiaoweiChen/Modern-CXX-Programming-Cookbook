
多态性提供了对同一接口具有多种形式的能力。虚函数允许派生类重写基类的实现,代表了多态性的一种常见形式,即运行时多态性的最常见元素,在类层次结构中决定调用特定虚函数的操作发生在运行时。这也称为后期绑定,函数调用与函数调用之间的绑定在程序执行期间发生。与此相反的是早期绑定、静态多态性或编译时多态性,函数和操作符重载在编译时进行。

另一方面,一种称为奇特递归模板模式(Curiously Recurring Template Pattern,简称 CRTP)的技术,在编译时通过从派生类为模板参数的基类模板派生类,来模拟基于虚函数的运行时多态性。这种技术在一些库中得到了广泛的应用,包括微软的 Active Template Library (ATL) 和 Windows Template Library (WTL)。本示例中,将介绍 CRTP,并了解如何实现它及其工作原理。

\mySubsubsection{}{Getting ready}

为了展示 CRTP 如何工作,将定义一组具有绘制控件功能的控件类。示例中,这是一个分为两步的操作:擦除背景,然后绘制控件。简单起见,这些将是仅在控制台上进行输出文本的操作。

\mySubsubsection{}{How to do it...}

为了实现奇特递归模板模式以达到静态多态性,请按照以下步骤操作:

\begin{enumerate}
\item
提供一个类模板,将作为其他应在编译时多态处理的类的基类。多态函数从这个类型调用:

\begin{cpp}
template <class T>
class control
{
    public:
    void draw()
    {
        static_cast<T*>(this)->erase_background();
        static_cast<T*>(this)->paint();
    }
};
\end{cpp}

\item
派生类使用类模板作为其基类;派生类也作为基类的模板参数。派生类实现了从基类调用的函数:

\begin{cpp}
class button : public control<button>
{
public:
    void erase_background()
    {
        std::cout << "erasing button background..." << '\n';
    }
    void paint()
    {
        std::cout << "painting button..." << '\n';
    }
};
class checkbox : public control<checkbox>
{
public:
    void erase_background()
    {
        std::cout << "erasing checkbox background..."
        << '\n';
    }
    void paint()
    {
        std::cout << "painting checkbox..." << '\n';
    }
};
\end{cpp}

\item
函数模板可以通过指向基类模板的指针或引用来多态地处理派生类:

\begin{cpp}
template <class T>
void draw_control(control<T>& c)
{
    c.draw();
}
button b;
draw_control(b);
checkbox c;
draw_control(c);
\end{cpp}
\end{enumerate}

\mySubsubsection{}{How it works...}

虚函数可能会带来性能问题,尤其是在其很小并且在一个循环中多次调用的情况下。现代硬件已经使得大多数这种情况变得无关紧要,但仍有一些应用类别,性能至关重要,任何性能提升都很重要。奇特递归模板模式,允许使用元编程在编译时模拟虚函数调用,最终转换为函数重载。

这种模式乍一看可能显得有些奇怪,但完全合法。其思想是从一个模板类作为基类派生一个类,然后将派生类本身传递给基类的类型模板参数。然后,基类调用派生类的函数。在例子中,control<button>::draw() 在编译器知道 button 类之前就声明了。但control 类是一个类模板,所以只有当编译器遇到其使用的代码时,才会实例化,例如,button 类已经定义并为编译器所知,所以可以调用 \verb|button::erase_background()| 和 \verb|button::paint()|。

为了从派生类调用函数,首先需要获得一个指向派生类的指针。这是通过 \verb|static_cast| 转换完成的,如 \verb|static_cast<T*>(this)->erase_background()| 所示。如果这样做很多次,则可以通过提供一个私有函数来简化代码:

\begin{cpp}
template <class T>
class control
{
    T* derived() { return static_cast<T*>(this); }
    public:
    void draw()
    {
        derived()->erase_background();
        derived()->paint();
    }
};
\end{cpp}

使用 CRTP 时需要注意一些陷阱:

\begin{itemize}
\item
所有从基类模板调用的派生类中的函数都必须公有;否则,基类特化必须声明为派生类的友元:

\begin{cpp}
class button : public control<button>
{
    private:
    friend class control<button>;
    void erase_background()
    {
        std::cout << "erasing button background..." << '\n';
    }
    void paint()
    {
        std::cout << "painting button..." << '\n';
    }
};
\end{cpp}

\item
不可能在如vector或list等同质容器中存储 CRTP 类型的示例,每个基类都是一个独特的类型(如 control<button> 和 control<checkbox>)。如果确实需要这样做,可以使用变通方法来实现。这将在下一节中讨论并举例说明。

\item
使用这种方法时,因为模板的实例化方式,程序的大小可能会增加。
\end{itemize}

\mySubsubsection{}{There's more...}

当需要在容器中同质存储实现 CRTP 的类型实例时,必须使用其他惯用法。基类模板本身必须从另一个具有纯虚函数(和虚拟公共析构函数)的类派生。为了在 control 类上举例说明这一点,需要进行更改:

\begin{cpp}
class controlbase
{
public:
    virtual void draw() = 0;
    virtual ~controlbase() {}
};
template <class T>
class control : public controlbase
{
public:
    virtual void draw() override
    {
        static_cast<T*>(this)->erase_background();
        static_cast<T*>(this)->paint();
    }
};
\end{cpp}

不需要对派生类(如 button 和 checkbox)进行更改,可以将指向抽象类的指针存储在容器中,如 std::vector:

\begin{cpp}
void draw_controls(std::vector<std::unique_ptr<controlbase>>& v)
{
    for (auto & c : v)
    {
        c->draw();
    }
}
std::vector<std::unique_ptr<controlbase>> v;
v.emplace_back(std::make_unique<button>());
v.emplace_back(std::make_unique<checkbox>());
draw_controls(v);
\end{cpp}

