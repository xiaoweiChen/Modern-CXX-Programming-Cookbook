本章的第一个示例中,介绍了如何使用线程。还了解到线程函数不能返回值,线程应该使用其他手段(比如:共享数据)来实现,因此需要同步。除了使用共享数据外,另一种在线程间通信返回值或异常的方法是使用 std::promise。本示例将介绍这一机制的工作原理。

\mySubsubsection{}{Getting ready}

本示例使用的 promise 和 future 类型在 <future> 头文件的 std 命名空间中可用。

\mySubsubsection{}{How to do it...}

要通过promise 和 future从一个线程向另一个线程传递值,可以这样做:

\begin{enumerate}
\item
通过参数将promise提供给线程函数:

\begin{cpp}
void produce_value(std::promise<int>& p)
{
    // simulate long running operation
    {
        using namespace std::chrono_literals;
        std::this_thread::sleep_for(2s);
    }
    // continued at 2.
}
\end{cpp}

\item
调用promise上的 \verb|set_value()| 方法设置结果以表示一个值,或调用 \verb|set_exception()| 设置结果以表示异常:

\begin{cpp}
// continued from 1.
p.set_value(42);
\end{cpp}

\item
参数将与promise关联的future提供给另一个线程函数:

\begin{cpp}
void consume_value(std::future<int>& f)
{
    // continued at 4.
}
\end{cpp}

\item
调用future的 get() 方法以获取设置给promise的结果:

\begin{cpp}
// continued from 3.
auto value = f.get();
\end{cpp}

\item
调用线程中,使用promise上的 \verb|get_future()| 方法获取与promise关联的future:

\begin{cpp}
std::promise<int> p;
std::thread t1(produce_value, std::ref(p));
std::future<int> f = p.get_future();
std::thread t2(consume_value, std::ref(f));
t1.join();
t2.join();
\end{cpp}
\end{enumerate}

\mySubsubsection{}{How it works...}

promise-future对是一个通信渠道,允许一个线程通过共享状态向另一个线程通信值或异常。promise 是异步结果的提供者,具有一个关联的 future,后者代表异步返回对象。为了建立这一通道,首先需要创建一个promise。这会创建一个共享状态,稍后可以通过与promise关联的future读取。

要为promise设置结果,可以使用以下任一方法:

\begin{itemize}
\item
使用 \verb|set_value()| 或 \verb|set_value_at_thread_exit()| 方法设置返回值;后者函数将值存储在共享状态中,但只有在线程退出时才能通过关联的future获取。

\item
使用 \verb|set_exception()| 或 \verb|set_exception_at_thread_exit()| 方法设置异常作为返回值。异常会包装在一个 \verb|std::exception_ptr| 实例中。后者函数将异常存储在共享状态中,但只有在线程退出时才能获取。
\end{itemize}

要检索与 promise 关联的future实例,可使用 \verb|get_future()|。要从future中获取值,使用 get() 方法。这会阻塞调用线程,直到共享状态中的值可用。future类型有几个方法可以阻塞线程,直到共享状态中的结果可用:

\begin{itemize}
\item
\verb|wait()| 只有在结果可用时才返回。

\item
\verb|wait_for()| 在结果可用或指定的超时到期时返回。

\item
\verb|wait_until()| 在结果可用或达到指定的时间点时返回。
\end{itemize}

如果异常设置为 promise 的值,则在future实例上调用 get() 方法时将抛出该异常。上一节中的示例已重写如下,以便抛出异常而不是设置结果:

\begin{cpp}
void produce_value(std::promise<int>& p)
{
    // simulate long running operation
    {
        using namespace std::chrono_literals;
        std::this_thread::sleep_for(2s);
    }
    try
    {
        throw std::runtime_error("an error has occurred!");
    }
    catch(...)
    {
        p.set_exception(std::current_exception());
    }
}
void consume_value(std::future<int>& f)
{
    std::lock_guard<std::mutex> lock(g_mutex);
    try
    {
        std::cout << f.get() << '\n';
    }
    catch(std::exception const & e)
    {
        std::cout << e.what() << '\n';
    }
}
\end{cpp}

\verb|consume_value()| 函数中,对 get() 的调用会放在了一个 \verb|try...catch| 块中。如果捕获到了异常——在特定的实现中确实如此——消息就会输出到控制台。

\mySubsubsection{}{There's more...}

以这种方式建立promise-future通道是一种相当明确的操作,可以使用 std::async() 函数来避免;这是一个高级工具,可异步运行一个函数,创建一个内部承诺和共享状态,并返回一个与共享状态关联的future。

