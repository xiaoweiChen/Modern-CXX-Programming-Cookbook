第一个示例中,介绍了线程支持库,并了解了如何使用线程进行一些基本操作。简要讨论了线程函数中的异常处理,并提到线程启动上下文中不能用 try…catch 语句捕获异常。另一方面,异常可以在 \verb|std::exception_ptr| 包装器内跨线程传输。本示例中,将介绍如何处理来自线程函数的异常。

\mySubsubsection{}{Getting ready}

\verb|exception_ptr| 类型在 std 命名空间中可用,该命名空间位于 <exception> 头文件中;互斥锁(之前有更详细的讨论)也在此命名空间中可用,但位于 <mutex> 头文件中。

\mySubsubsection{}{How to do it...}

为了正确处理工作线程中抛出,并在主线程或其加入的线程中处理的异常,请执行以下操作(假设可以从多个线程抛出多个异常):

\begin{enumerate}
\item
使用全局容器来保存 \verb|std::exception_ptr| 的实例:

\begin{cpp}
std::vector<std::exception_ptr> g_exceptions;
\end{cpp}

\item
使用全局互斥锁来同步对共享容器的访问:

\begin{cpp}
std::mutex g_mutex;
\end{cpp}

\item
顶层线程函数中执行的代码周围使用 \verb|try...catch| 块。使用 \verb|std::current_exception()| 捕获当前异常,并将其副本或引用封装进 \verb|std::exception_ptr| 指针中,然后将其添加到异常的共享容器中:

\begin{cpp}
void func1()
{
    throw std::runtime_error("exception 1");
}
void func2()
{
    throw std::runtime_error("exception 2");
}
void thread_func1()
{
    try
    {
        func1();
    }
    catch (...)
    {
        std::lock_guard<std::mutex> lock(g_mutex);
        g_exceptions.push_back(std::current_exception());
    }
}
void thread_func2()
{
    try
    {
        func2();
    }
    catch (...)
    {
        std::lock_guard<std::mutex> lock(g_mutex);
        g_exceptions.push_back(std::current_exception());
    }
}
\end{cpp}

\item
启动线程之前,主线程清除容器:

\begin{cpp}
g_exceptions.clear();
\end{cpp}

\item
主线程中,所有线程执行完毕后,检查捕获的异常并处理每个异常:

\begin{cpp}
std::thread t1(thread_func1);
std::thread t2(thread_func2);
t1.join();
t2.join();
for (auto const & e : g_exceptions)
{
    try
    {
        if(e)
        std::rethrow_exception(e);
    }
    catch(std::exception const & ex)
    {
        std::cout << ex.what() << '\n';
    }
}
\end{cpp}
\end{enumerate}

\mySubsubsection{}{How it works...}

前面的例子中,假设多个线程可能会抛出异常,因此需要一个容器来保存所有异常。如果一次只有一个线程抛出单个异常,那么不需要共享容器和用于同步访问的互斥锁。可以使用一个全局的 \verb|std::exception_ptr| 类型实例来保存在各线程间传输的异常。

\verb|std::current_exception()| 是一个通常在 catch 子句中使用的函数,用于捕获当前异常并创建 \verb|std::exception_ptr| 的实例。这是为了保持对原始异常的副本或引用(取决于实现),只要存在指向 \verb|std::exception_ptr| 指针,该副本或引用就一直有效。如果此函数在没有处理异常的情况下调用,则会创建一个空的 \verb|std::exception_ptr|。

\verb|std::exception_ptr| 使用 \verb|std::current_exception()| 捕获的异常包装器。如果默认构造,则不持有任何异常,是一个空指针。如果这两个对象都是空或指向相同的异常实例,则这两种类型实例相等。\verb|std::exception_ptr| 实例可以传递给其他线程,可以重新抛出并在 \verb|try...catch| 块中捕获。

\verb|std::rethrow_exception()| 是一个接受 \verb|std::exception_ptr| 作为参数的函数,会抛出由其参数引用的异常对象。

\begin{myNotic}
\verb|std::current_exception()|, \verb|std::rethrow_exception()| 和 \verb|std::exception_ptr| 在 C++11 中可用。
\end{myNotic}

前一节的例子中,每个线程函数都使用了整个代码执行的 \verb|try...catch|。当处理异常时,会获取全局互斥锁实例上的锁,并将持有当前异常的 \verb|std::exception_ptr| 实例添加到共享容器中。通过这种方法,线程函数会在遇到第一个异常时停止;然而,即使前一个操作抛出了异常,也可能需要执行多个操作,需要有多个 \verb|try...catch| 语句,并且可能只将某些异常传输到线程外部。

主线程中,所有线程执行完毕后,迭代容器,每个非空异常都会重新抛出,并在 \verb|try...catch| 块中捕获并处理。









