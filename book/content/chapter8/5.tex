
互斥锁是用于保护对共享数据访问的同步原语,标准库还提供了一种称为条件变量的同步原语,一个线程能够向其他线程发出信号,表明某个条件已经发生。等待条件变量的线程或线程组会阻塞,直到条件变量通知、超时或发生伪唤醒为止。本示例中,将介绍如何使用条件变量在生产线程和消费线程之间发送通知。

\mySubsubsection{}{Getting ready}

对于本示例,需要熟悉线程、互斥锁和锁。条件变量在 \verb|<condition_variable>| 头文件的 std 命名空间中可用。

\mySubsubsection{}{How to do it...}

使用以下模式通过条件变量上的通知同步线程:

\begin{enumerate}
\item
定义一个条件变量(适当的上下文中):

\begin{cpp}
std::condition_variable cv;
\end{cpp}

\item
定义一个互斥锁供线程锁定,使用第二个互斥锁来同步从不同线程对标准控制台的访问:

\begin{cpp}
std::mutex data_mutex; // data mutex
std::mutex io_mutex;   // I/O mutex
\end{cpp}

\item
定义线程间共享的数据:

\begin{cpp}
int data = 0;
\end{cpp}

\item
生产者线程中,修改数据之前锁定互斥锁:

\begin{cpp}
std::thread producer([&](){
    // simulate long running operation
    {
        using namespace std::chrono_literals;
        std::this_thread::sleep_for(2s);
    }
    // produce
    {
        std::unique_lock lock(data_mutex);
        data = 42;
    }
    // print message
    {
        std::lock_guard l(io_mutex);
        std::cout << "produced " << data << '\n';
    }
    // continued at 5.
});
\end{cpp}

\item
生产者线程中,使用调用 \verb|notify_one()| 或 \verb|notify_all()| 来通知条件变量(用于保护共享数据的互斥锁解锁后执行此操作):

\begin{cpp}
// continued from 4.
cv.notify_one();
\end{cpp}

\item
在消费者线程中,获取数据互斥锁的唯一锁,并用它来等待条件变量。注意,伪唤醒可能发生,这将在“How it works...”部分详细讨论:

\begin{cpp}
std::thread consumer([&](){
    // wait for notification
    {
        std::unique_lock lock(data_mutex);
        cv.wait(lock);
    }
    // continued at 7.
});
\end{cpp}

\item
消费者线程中,条件通知后使用共享数据:

\begin{cpp}
// continued from 6.
{
    std::lock_guard lock(io_mutex);
    std::cout << "consumed " << data << '\n';
}
\end{cpp}
\end{enumerate}

\mySubsubsection{}{How it works...}

前面的例子展示了两个线程共享共同数据(例子中,是一个整型变量)。一个线程在长时间计算后产生数据(用休眠模拟),而另一个线程仅在数据生成后才消费。为此,使用了一种同步机制,即使用互斥锁和条件变量来阻止消费者线程,直到生产者线程发出通知,表明数据已准备好。这条通信渠道的关键是消费者线程等待的条件变量。两个线程几乎同时开始,生产者线程开始一个长时间的计算,这个计算应该为消费者线程生成数据。与此同时,消费者线程不能运行,直到数据可用;必须保持阻塞状态,直到接收到数据已产生的通知。接收到通知时,可以继续执行。整个机制的工作方式为:

\begin{itemize}
\item
至少有一个线程等待条件变量的通知。

\item
至少有一个线程正在通知条件变量。

\item
等待的线程必须首先获得一个互斥锁(\verb|std::unique_lock<std::mutex>|),并将其传递给条件变量的 \verb|wait()|, \verb|wait_for()| 或 \verb|wait_until()|。所有等待的方法都会原子地释放互斥锁并阻塞线程,直到条件变量通知。此时,线程解除阻塞,互斥锁再次原子地获取(涉及的操作视为整体,线程在执行这些操作时不会中断)。

\item
通知条件变量的线程可以使用 \verb|notify_one()|,其中只有一个阻塞的线程解除阻塞,或者使用 \verb|notify_all()|,其中所有等待条件变量的被阻塞线程都解除阻塞。
\end{itemize}

\begin{myNotic}
条件变量在多处理器系统上无法完全预测,伪唤醒可能发生,即使没有通知条件变量,线程也会解锁。线程解除阻塞后,有必要检查条件是否成立。然而,伪唤醒可能多次发生,因此有必要在一个循环中检查条件变量。关于伪唤醒的更多信息,参见 \url{https://en.wikipedia.org/wiki/Spurious_wakeup}。
\end{myNotic}

C++ 标准提供了两种条件变量的实现:

\begin{itemize}
\item
\verb|std::condition_variable|,本示例中使用的是与 \verb|std::unique_lock| 关联的条件变量。

\item
\verb|std::condition_variable_any| 表示一种更通用的实现,可以与满足基础锁要求(实现 lock() 和 unlock() 方法)的锁一起工作。这种实现的一个可能用途是在接收到中断信号时提供可中断的等待,正如 Anthony Williams 在《C++ In Action》(2012年)中的解释:

\begin{myTip}
自定义锁操作不仅会像预期那样锁定关联的互斥锁,会在接收到中断信号时通知这个条件变量。
\end{myTip}
\end{itemize}

条件变量的所有等待方法都有两个重载:

\begin{itemize}
\item
第一个重载接受 \verb|std::unique_lock<std::mutex>|(基于类型,即持续时间或时间点),并导致线程保持阻塞,直到条件变量被通知。这个重载会原子地释放互斥锁并阻塞当前线程,然后将其添加到等待条件变量的线程列表中。当条件 \verb|notify_one()| 或 \verb|notify_all()| 通知、发生伪唤醒或超时时(取决于函数重载),线程解除阻塞,互斥锁再次原子地获取。

\item
第二个重载除了其他重载的参数外,还接受一个谓词。这个谓词可用于在等待条件变为真时避免伪唤醒。这个重载等价于以下代码:

\begin{cpp}
while(!pred())
    wait(lock);
\end{cpp}
\end{itemize}

以下代码展示了一个比前一节提供的例子更复杂的示例。生产者线程在一个循环中生成数据(这个例子中,是一个有限循环),而消费者线程等待新数据可用并消费它(将其打印到控制台)。生产者线程在完成数据生成后终止,而消费者线程在没有更多数据可消费时终止。数据可添加到 queue<int> 中,一个布尔变量用于指示给消费者线程数据生成过程已完成。以下是生产者线程实现的代码:

\begin{cpp}
std::mutex g_lockprint;
std::mutex g_lockqueue;
std::condition_variable g_queuecheck;
std::queue<int> g_buffer;
bool g_done;
void producer(
int const id,
std::mt19937& generator,
std::uniform_int_distribution<int>& dsleep,
std::uniform_int_distribution<int>& dcode)
{
    for (int i = 0; i < 5; ++i)
    {
        // 模拟工作
        std::this_thread::sleep_for(
        std::chrono::seconds(dsleep(generator)));
        // 生成数据
        {
            std::unique_lock<std::mutex> locker(g_lockqueue);
            int value = id * 100 + dcode(generator);
            g_buffer.push(value);
            {
                std::unique_lock<std::mutex> locker(g_lockprint);
                std::cout << "[produced(" << id << ")]: " << value << '\n';
            }
        }
        // 通知消费者
        g_queuecheck.notify_one();
    }
}
\end{cpp}

另一方面,消费者线程的实现:

\begin{cpp}
void consumer()
{
    // 循环直到接收到数据生成结束的信号
    while (!g_done)
    {
        std::unique_lock<std::mutex> locker(g_lockqueue);
        g_queuecheck.wait_for(
        locker,
        std::chrono::seconds(1),
        [&]() {return !g_buffer.empty(); });
        // 如果队列中有值则处理它们
        while (!g_done && !g_buffer.empty())
        {
            std::unique_lock<std::mutex> locker(g_lockprint);
            std::cout << "[consumed]: " << g_buffer.front() << '\n';
            g_buffer.pop();
        }
    }
}
\end{cpp}

消费者线程执行以下操作:

\begin{itemize}
\item
循环直到接收到数据生成结束的信号。

\item
获取与条件变量关联的互斥锁实例的唯一锁。

\item
使用带有谓词的 \verb|wait_for()| 重载,检查在唤醒时缓冲区是否为空(以避免伪唤醒)。此方法使用1秒的超时,并在超时发生后返回,即使已经发出通知。

\item
条件变量接收到信号后,从队列中消费所有数据。
\end{itemize}

为了测试,可以启动几个生产者线程和一个消费者线程。生产者线程生成随机数据,共享伪随机数生成引擎和分布。所有这些都在下面的代码样本中展示:

\begin{cpp}
auto seed_data = std::array<int, std::mt19937::state_size> {};
std::random_device rd {};
std::generate(std::begin(seed_data), std::end(seed_data),
              std::ref(rd));
std::seed_seq seq(std::begin(seed_data), std::end(seed_data));
auto generator = std::mt19937{ seq };
auto dsleep = std::uniform_int_distribution<>{ 1, 5 };
auto dcode = std::uniform_int_distribution<>{ 1, 99 };
std::cout << "start producing and consuming..." << '\n';
std::thread consumerthread(consumer);
std::vector<std::thread> threads;
for (int i = 0; i < 5; ++i)
{
    threads.emplace_back(producer,
                         i + 1,
                         std::ref(generator),
                         std::ref(dsleep),
                         std::ref(dcode));
}
// 等待工作者线程完成工作
for (auto& t : threads)
    t.join();
// 通知消费者线程完成并等待它
g_done = true;
consumerthread.join();
std::cout << "done producing and consuming" << '\n';
\end{cpp}

这个程序的可能输出如下所示(每次执行的实际输出会有所不同):

\begin{shell}
start producing and consuming...
[produced(5)]: 550
[consumed]: 550
[produced(5)]: 529
[consumed]: 529
[produced(5)]: 537
[consumed]: 537
[produced(1)]: 122
[produced(2)]: 224
[produced(3)]: 326
[produced(4)]: 458
[consumed]: 122
[consumed]: 224
[consumed]: 326
[consumed]: 458
...
done producing and consuming
\end{shell}

标准库还提供了一个名为 \verb|notify_all_at_thread_exit()| 的辅助函数,提供了一种方法,让线程通过 \verb|condition_variable| 实例通知其他线程,包括销毁所有 \verb|thread_local| 实例在内的完整执行已经结束。这个函数有两个参数:一个 \verb|condition_variable| 和一个与条件变量关联的 \verb|std::unique_lock<std::mutex>|(会接管该锁的所有权)。这个函数的典型用例是运行一个分离的线程,该线程在即将结束前调用此函数。








