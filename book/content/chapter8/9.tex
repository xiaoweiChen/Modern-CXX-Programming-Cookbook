第3章中,接扫了两个高阶函数:map,通过转换范围或将产生新范围来将函数应用于范围的元素;fold(也称为reduce),将范围的元素组合成单个值。那时都是串行实现,但在并发、线程和异步任务的上下文中,可以利用硬件运行这些函数的并行版本,以加速对大数据量范围的操作,或当转换和聚合耗时较长时。本示例中,将介绍使用线程实现map和fold的一种可能方案。

\mySubsubsection{}{Getting ready}

需要熟悉map和fold函数的概念。为了测量这些函数的执行时间并与串行替代方案进行比较,将使用在第6章中的引入的 \verb|perf_timer| 类型模板。

\begin{myNotic}
算法的并行版本有可能加快执行时间,但这不是在所有情况下都成立。线程之间的上下文切换和对共享数据的同步访问,可能会引入显著的开销。对于某些实现和特定的数据集,可能导致并行版本实际上比串行版本的运行时间更长。
\end{myNotic}

为了确定拆分工作的线程数量,将使用以下函数:

\begin{cpp}
unsigned get_no_of_threads()
{
    return std::thread::hardware_concurrency();
}
\end{cpp}

接下来,将介绍map和fold函数并行版本的第一种可能实现。

\mySubsubsection{}{How to do it...}

要实现map函数的并行版本,请执行以下步骤:

\begin{enumerate}
\item
定义一个函数模板,接受范围的开始和结束迭代器,以及要应用到所有元素上的函数:

\begin{cpp}
template <typename Iter, typename F>
void parallel_map(Iter begin, Iter end, F&& f)
{
}
\end{cpp}

\item
检查范围的大小。如果元素的数量小于预定义的阈值(对于本实现,阈值为10,000),则以串行方式执行映射:

\begin{cpp}
auto size = std::distance(begin, end);
if(size <= 10000)
    std::transform(begin, end, begin, std::forward<F>(f));
\end{cpp}

\item
对于更大的范围,将工作分配到多个线程上,让每个线程映射范围的一部分。这些部分不应该重叠,以避免同步共享数据的访问:

\begin{cpp}
else
{
    auto no_of_threads = get_no_of_threads();
    auto part = size / no_of_threads;
    auto last = begin;
    // continued at 4. and 5.
}
\end{cpp}

\item
启动线程,并在每个线程上运行映射的串行版本:

\begin{cpp}
std::vector<std::thread> threads;
for(unsigned i = 0; i < no_of_threads; ++i)
{
    if(i == no_of_threads - 1) last = end;
    else std::advance(last, part);
    threads.emplace_back(
        [=,&f]{std::transform(begin, last,
                              begin, std::forward<F>(f));});
    begin = last;
}
\end{cpp}

\item
等待所有线程完成执行:

\begin{cpp}
for(auto & t : threads) t.join();
\end{cpp}
\end{enumerate}

将上述步骤合并后,得到以下实现:

\begin{cpp}
template <typename Iter, typename F>
void parallel_map(Iter begin, Iter end, F&& f)
{
    auto size = std::distance(begin, end);
    if(size <= 10000)
    std::transform(begin, end, begin, std::forward<F>(f));
    else
    {
        auto no_of_threads = get_no_of_threads();
        auto part = size / no_of_threads;
        auto last = begin;
        std::vector<std::thread> threads;
        for(unsigned i = 0; i < no_of_threads; ++i)
        {
            if(i == no_of_threads - 1) last = end;
            else std::advance(last, part);
            threads.emplace_back(
                [=,&f]{std::transform(begin, last,
                                      begin, std::forward<F>(f));});
            begin = last;
        }
        for(auto & t : threads) t.join();
    }
}
\end{cpp}

要实现左折叠函数的并行版本,请执行以下步骤:

\begin{enumerate}
\item
定义一个函数模板,接受范围的开始和结束迭代器、初始值以及应用于范围元素的二元函数:

\begin{cpp}
template <typename Iter, typename R, typename F>
auto parallel_fold(Iter begin, Iter end, R init, F&& op)
{
}
\end{cpp}

\item
检查范围的大小。如果元素的数量小于预定义的阈值(对于本实现,阈值为10,000),则以串行方式执行折叠:

\begin{cpp}
auto size = std::distance(begin, end);
if(size <= 10000)
    return std::accumulate(begin, end,
                           init, std::forward<F>(op));
\end{cpp}

\item
对于更大的范围,将工作拆分为多个线程,让每个线程折叠范围的一部分。这些部分不应重叠,以避免对共享数据的线程同步。结果可以通过传递给线程函数的引用返回,以避免数据同步:

\begin{cpp}
else
{
    auto no_of_threads = get_no_of_threads();
    auto part = size / no_of_threads;
    auto last = begin;
    // continued with 4. and 5.
}
\end{cpp}

\item
启动线程,在每个线程上执行折叠的串行版本:

\begin{cpp}
std::vector<std::thread> threads;
std::vector<R> values(no_of_threads);
for(unsigned i = 0; i < no_of_threads; ++i)
{
    if(i == no_of_threads - 1) last = end;
    else std::advance(last, part);
    threads.emplace_back(
        [=,&op](R& result){
            result = std::accumulate(begin, last, R{},
                                     std::forward<F>(op));},
    std::ref(values[i]));
    begin = last;
}
\end{cpp}

\item
等待所有线程完成执行,并将部分结果折叠成最终结果:

\begin{cpp}
for(auto & t : threads) t.join();
return std::accumulate(std::begin(values), std::end(values),
                       init, std::forward<F>(op));
\end{cpp}
\end{enumerate}

将上述步骤合并后,得到以下实现:

\begin{cpp}
template <typename Iter, typename R, typename F>
auto parallel_fold(Iter begin, Iter end, R init, F&& op)
{
    auto size = std::distance(begin, end);
    if(size <= 10000)
        return std::accumulate(begin, end, init, std::forward<F>(op));
    else
    {
        auto no_of_threads = get_no_of_threads();
        auto part = size / no_of_threads;
        auto last = begin;
        std::vector<std::thread> threads;
        std::vector<R> values(no_of_threads);
        for(unsigned i = 0; i < no_of_threads; ++i)
        {
            if(i == no_of_threads - 1) last = end;
            else std::advance(last, part);
            threads.emplace_back(
                [=,&op](R& result){
                    result = std::accumulate(begin, last, R{},
                                             std::forward<F>(op));},
            std::ref(values[i]));
            begin = last;
        }
        for(auto & t : threads) t.join();
        return std::accumulate(std::begin(values), std::end(values),
                               init, std::forward<F>(op));
    }
}
\end{cpp}

\mySubsubsection{}{How it works...}

map和fold的这些并行实现有几个相似之处:

\begin{itemize}
\item
如果范围内的元素数量小于10,000,都会回退到串行版本。

\item
都启动相同数量的线程。这些线程的数量是通过静态函数 \verb|std::thread::hardware_concurrency()| 确定的,该函数返回实现支持的并发线程数。不过,这个值更多是一个提示而不是准确值,应谨慎使用。

\item
为了避免访问同步,不使用共享数据。即使所有线程都在同一范围的元素上工作,处理的范围部分也不会重叠。

\item
这两个函数都是作为函数模板实现的,接受开始和结束迭代器以定义要处理的范围。为了将范围拆分为多个部分以便由不同线程独立处理,在范围内使用迭代器。为此,使用 \verb|std::advance()| 将迭代器向前移动特定的位置。这适用于vector或array,但对于list等容器来说非常低效。因此,这种实现只适合具有随机访问迭代器的范围。
\end{itemize}

map和fold的串行版本,可以用C++中的 \verb|std::transform()| 和 \verb|std::accumulate()| 实现。为了验证并行算法的正确性,并检查是否提供了任何执行速度提升,可以将它们与这些通用算法的执行情况进行比较。

为了测试,将在一个大小从10,000到5000万元素的向量上使用map和fold。首先对范围进行映射(即转换),将每个元素的值翻倍,然后通过将范围内的所有元素相加来折叠结果。为了简化,范围中的每个元素等于其基于1的索引(第一个元素为1,第二个元素为2,以此类推)。以下示例运行不同大小向量的map和fold的串行和并行版本,并以表格格式打印执行时间:

\begin{myNotic}
作为一个练习,可以改变元素的数量,以及线程的数量,看看并行版本相对于串行版本的表现如何。
\end{myNotic}

\begin{cpp}
std::vector<int> sizes
{
    10000, 100000, 500000,
    1000000, 2000000, 5000000,
    10000000, 25000000, 50000000
};
std::cout
    << std::right << std::setw(8) << std::setfill(' ') << "size"
    << std::right << std::setw(8) << "s map"
    << std::right << std::setw(8) << "p map"
    << std::right << std::setw(8) << "s fold"
    << std::right << std::setw(8) << "p fold"
    << '\n';
for (auto const size : sizes)
{
    std::vector<int> v(size);
    std::iota(std::begin(v), std::end(v), 1);
    auto v1 = v;
    auto s1 = 0LL;
    auto tsm = perf_timer<>::duration([&] {
        std::transform(std::begin(v1), std::end(v1), std::begin(v1),
        [](int const i) {return i + i; }); });
    auto tsf = perf_timer<>::duration([&] {
        s1 = std::accumulate(std::begin(v1), std::end(v1), 0LL,
                             std::plus<>()); });
    auto v2 = v;
    auto s2 = 0LL;
    auto tpm = perf_timer<>::duration([&] {
        parallel_map(std::begin(v2), std::end(v2),
        [](int const i) {return i + i; }); });
    auto tpf = perf_timer<>::duration([&] {
        s2 = parallel_fold(std::begin(v2), std::end(v2), 0LL,
                           std::plus<>()); });
    assert(v1 == v2);
    assert(s1 == s2);
    std::cout
        << std::right << std::setw(8) << std::setfill(' ') << size
        << std::right << std::setw(8)
        << std::chrono::duration<double, std::micro>(tsm).count()
        << std::right << std::setw(8)
        << std::chrono::duration<double, std::micro>(tpm).count()
        << std::right << std::setw(8)
        << std::chrono::duration<double, std::micro>(tsf).count()
        << std::right << std::setw(8)
        << std::chrono::duration<double, std::micro>(tpf).count()
        << '\n';
}
\end{cpp}

此程序的一个可能输出如下面的图表所示(在Windows 64位操作系统的机器上执行,该机器配备Intel Core i7处理器和4个物理核心,及8个逻辑核心)。并行版本,特别是fold实现,表现优于串行版本。但只有当vector长度超过一定大小时,这一结论才成立。下表显示,对于最多1百万个元素,串行版本仍然更快。当vector中有2百万个或更多元素时,并行版本执行得更快。注意,即使在同一台机器上,每次运行的实际时间也会略有不同。而在不同的机器上,时间差异会更大:

\begin{shell}
    size   s map   p map  s fold  p fold
   10000      11      10       7      10
  100000     108    1573      72     710
  500000     547    2006     361     862
 1000000    1146    1163     749     862
 2000000    2503    1527    1677    1289
 5000000    5937    3000    4203    2314
10000000   11959    6269    8269    3868
25000000   29872   13823   20961    9156
50000000   60049   27457   41374   19075
\end{shell}

为了更好地可视化这些结果,可以用柱状图表示并行版本的速度提升。下面的图表中,蓝色条形代表并行map实现的速度提升,而橙色条形显示并行fold实现的速度提升。正值表明并行版本更快;负值表明串行版本更快:

\myGraphic{1.0}{content/chapter8/images/1.png}{图8.1: 对于不同处理元素数量,map(蓝色)和fold(橙色)并行实现的速度提升}

该图表使更容易看出,只有当元素数量超过某个阈值(基准测试中大约是2百万)时,并行实现才会比串行版本更快。







