
线程允许同时执行多个函数,这些函数通常需要访问共享资源。对共享资源的访问必须同步,以便一次只有一个线程可以读取或写入共享资源。本示例中,将介绍C++标准定义了哪些机制来同步线程对共享数据的访问。

\mySubsubsection{}{Getting ready}

本示例中讨论的互斥锁(mutex)和锁(lock)类分别位于<mutex>头文件中的std命名空间下,以及C++14的共享互斥锁和锁位于<shared\_mutex>头文件中。

\mySubsubsection{}{How to do it...}

使用以下模式同步对单个共享资源的访问:

\begin{enumerate}
\item
适当的作用域(类或全局作用域)中定义一个互斥锁:

\begin{cpp}
std::mutex g_mutex;
\end{cpp}

\item
每个线程访问共享资源之前获取该互斥锁:

\begin{cpp}
void thread_func()
{
    using namespace std::chrono_literals;
    {
        std::lock_guard<std::mutex> lock(g_mutex);
        std::cout << "running thread "
                  << std::this_thread::get_id() << '\n';
    }
    std::this_thread::yield();
    std::this_thread::sleep_for(2s);
    {
        std::lock_guard<std::mutex> lock(g_mutex);
        std::cout << "done in thread "
                  << std::this_thread::get_id() << '\n';
    }
}
\end{cpp}
\end{enumerate}

使用以下模式同步对多个共享资源的同时访问以避免死锁:

\begin{enumerate}
\item
适当的作用域(全局或类作用域)中为每个共享资源定义一个互斥锁:

\begin{cpp}
template <typename T>
struct container
{
    std::mutex     mutex;
    std::vector<T> data;
};
\end{cpp}

\item
使用带有死锁避免算法的std::lock()同时锁定互斥锁:

\begin{cpp}
template <typename T>
void move_between(container<T> & c1, container<T> & c2,
                  T const value)
{
    std::lock(c1.mutex, c2.mutex);
    // continued at 3.
}
\end{cpp}

\item
锁定后,采用每个互斥锁的所有权进入std::lock\_guard类,确保函数结束时(或作用域结束时)安全释放:

\begin{cpp}
// continued from 2.
std::lock_guard<std::mutex> l1(c1.mutex, std::adopt_lock);
std::lock_guard<std::mutex> l2(c2.mutex, std::adopt_lock);
c1.data.erase(
    std::remove(c1.data.begin(), c1.data.end(), value),
    c1.data.end());
c2.data.push_back(value);
\end{cpp}
\end{enumerate}

\mySubsubsection{}{How it works...}

互斥锁(mutex,即互斥)是一种同步原语,可保护多个线程对共享资源的并发访问。C++标准库提供了几种实现:

\begin{itemize}
\item
\verb|std::mutex| 是最常用的互斥锁类型;前面的代码段中有所说明。提供了获取和释放互斥锁的方法。lock() 尝试获取互斥锁,如果不可用则阻塞,\verb|try_lock()| 尝试获取互斥锁,如果不可用则不阻塞并返回,unlock() 释放互斥锁。

\item
\verb|std::timed_mutex| 类似于 std::mutex,提供了两种使用超时获取互斥锁的方法:\verb|try_lock_for()| 尝试获取互斥锁并在指定时间内不可用时返回,\verb|try_lock_until()| 尝试获取互斥锁并在到达指定时间点前不可用时返回。

\item
\verb|std::recursive_mutex| 类似于 std::mutex,可以从同一个线程多次获取互斥锁而不阻塞。

\item
\verb|std::recursive_timed_mutex| 是递归互斥锁和带超时的互斥锁的组合。

\item
\verb|std::shared_timed_mutex|,自C++14起,用于多个读者可以同时访问同一资源,而不会引起数据竞争的情况下,而只允许一个写者。实现了两层访问锁定——共享(多个线程可以共享同一互斥锁的所有权)和独占(只有一个线程可以拥有互斥锁)——并且提供了超时功能。

\item
\verb|std::shared_mutex|,自C++17起,类似于 shared\_timed\_mutex,但没有超时功能。
\end{itemize}

第一个锁定可用互斥锁的线程获得所有权并继续执行。任何线程连续尝试锁定互斥锁的所有后续尝试都会失败,包括已经拥有互斥锁的线程,并且 lock() 方法会阻塞线程,直到通过调用 unlock() 释放互斥锁为止。如果线程需要能够在不阻塞的情况下多次锁定互斥锁,从而进入死锁状态,则应使用 \verb|recursive_mutex| 类模板。

使用互斥锁保护对共享资源的访问的典型方法包括锁定互斥锁、使用共享资源,然后解锁互斥锁:

\begin{cpp}
g_mutex.lock();
// use the shared resource such as std::cout
std::cout << "accessing shared resource" << '\n';
g_mutex.unlock();
\end{cpp}

但这种方法容易出错,因为每次对 lock() 的调用都必须与所有执行路径上的 unlock() 调用配对;也就是说,正常的返回路径和异常返回路径。为了安全地获取和释放互斥锁,C++标准定义了几种锁:

\begin{itemize}
\item
\verb|std::lock_guard| 是前面看到的锁定机制;代表了一个以RAII方式实现的互斥锁包装器。试图在构造时获取互斥锁,并在析构时释放互斥锁,从C++11开始提供的:

\begin{cpp}
template <class M>
class lock_guard
{
public:
    typedef M mutex_type;
    explicit lock_guard(M& Mtx) : mtx(Mtx)
    {
        mtx.lock();
    }
    lock_guard(M& Mtx, std::adopt_lock_t) : mtx(Mtx)
    { }
    ~lock_guard() noexcept
    {
        mtx.unlock();
    }
    lock_guard(const lock_guard&) = delete;
    lock_guard& operator=(const lock_guard&) = delete;
private:
    M& mtx;
};
\end{cpp}

\item
\verb|std::unique_lock| 是一个互斥锁所有权包装器,提供了延迟锁定、时间锁定、递归锁定、所有权转移和支持条件变量的功能。这是从C++11开始提供的。

\item
\verb|std::shared_lock| 是一个互斥锁共享所有权包装器,提供了延迟锁定、时间锁定和所有权转移的支持。这是从C++14开始提供的。

\item
\verb|std::scoped_lock| 是一个多互斥锁的RAII方式实现的包装器。构造时,试图以避免死锁的方式像使用 std::lock() 一样获取互斥锁的所有权,并在析构时以相反的顺序释放互斥锁。这是从C++17开始提供的。
\end{itemize}

\begin{myNotic}
RAII(Resource Acquisition Is Initialization,即资源获取即初始化)是一种编程技术,用于某些编程语言,包括C++,简化资源管理,确保程序正确性,减少代码量。这种技术将资源的生命周期绑定到一个对象上。资源的分配(也称为获取)在实例创建时(构造函数中)完成,资源的释放(解除分配)在实例销毁时(析构函数中)完成。这确保了资源不会泄露,前提是绑定的实例本身没有泄露。有关RAII的更多信息,参见 \url{https://en.cppreference.com/w/cpp/language/raii}。
\end{myNotic}

“How to do it...”的第一个例子中,使用了 \verb|std::mutex| 和 \verb|std::lock_guard| 来保护对程序中所有线程共享的 std::cout 流实例的访问。以下示例展示了如何在多个线程上并发执行 \verb|thread_func()| 函数:

\begin{cpp}
std::vector<std::thread> threads;
for (int i = 0; i < 5; ++i)
    threads.emplace_back(thread_func);
for (auto & t : threads)
    t.join();
\end{cpp}

此程序的一个可能输出如下:

\begin{shell}
running thread 140296854550272
running thread 140296846157568
running thread 140296837764864
running thread 140296829372160
running thread 140296820979456
done in thread 140296854550272
done in thread 140296846157568
done in thread 140296837764864
done in thread 140296820979456
done in thread 140296829372160
\end{shell}

当一个线程需要获取旨在保护多个共享资源的多个互斥锁时,逐一获取它们可能会导致死锁:

\begin{cpp}
template <typename T>
void move_between(container<T> & c1, container<T> & c2, T const value)
{
    std::lock_guard<std::mutex> l1(c1.mutex);
    std::lock_guard<std::mutex> l2(c2.mutex);
    c1.data.erase(
        std::remove(c1.data.begin(), c1.data.end(), value),
        c1.data.end());
    c2.data.push_back(value);
}
container<int> c1;
c1.data.push_back(1);
c1.data.push_back(2);
c1.data.push_back(3);
container<int> c2;
c2.data.push_back(4);
c2.data.push_back(5);
c2.data.push_back(6);
std::thread t1(move_between<int>, std::ref(c1), std::ref(c2), 3);
std::thread t2(move_between<int>, std::ref(c2), std::ref(c1), 6);
t1.join();
t2.join();
\end{cpp}

例子中,container 保存着可以从不同线程同时访问的数据,需要通过获取互斥锁来保护。\verb|move_between()| 函数是一个线程安全的函数,从一个容器中移除一个元素并将其添加到另一个容器中。为此,可按顺序获取两个容器的互斥锁,然后从第一个容器中删除该元素,并将其添加到第二个容器的末尾。

这个函数容易出现死锁,因为在获取锁的过程中可能会触发竞争条件。假设有一个场景,其中两个不同的线程以不同的参数执行此函数:

\begin{itemize}
\item
第一个线程以 c1 和 c2 的顺序开始执行。

\item
第一个线程在获取 c1 容器的锁之后被挂起。第二个线程以 c2 和 c1 的顺序开始执行。

\item
第二个线程在获取 c2 容器的锁之后被挂起。

\item
第一个线程继续执行并试图获取 c2 的互斥锁,但由于互斥锁不可用,因此发生死锁(可以通过在线程获取第一个互斥锁后让其短暂睡眠来模拟)。
\end{itemize}

为了避免这类可能的死锁,应该以死锁避免的方式获取互斥锁,标准库提供了一个 std::lock() 工具函数。\verb|move_between()| 函数需要通过替换两个锁来更改,用下面的代码替换:

\begin{cpp}
std::lock(c1.mutex, c2.mutex);
std::lock_guard<std::mutex> l1(c1.mutex, std::adopt_lock);
std::lock_guard<std::mutex> l2(c2.mutex, std::adopt_lock);
\end{cpp}

互斥锁的所有权仍然必须转移到锁实例,这样才能在函数执行结束后(或者根据情况,在特定作用域结束时)正确释放。

C++17 中,提供了一种新互斥锁包装器 \verb|std::scoped_lock|,可以用来简化代码,比如前面的例子中的代码。这种类型的锁可以以无死锁的方式获取多个互斥锁的所有权。当作用域锁销毁时,这些互斥锁就会释放。前面的代码等同于以下一行代码:

\begin{cpp}
std::scoped_lock lock(c1.mutex, c2.mutex);
\end{cpp}

\verb|scoped_lock| 提供了一种简化的机制,在作用域块期间拥有一个或多个互斥锁,也可编写简单且健壮的代码。







