线程能够同时运行多个函数,可在多处理器或多核系统中充分利用硬件。而线程需要显式的、较低层的操作。线程的一个替代方案是任务,在特定线程中运行的工作单元。C++ 标准没有提供完整的任务库,但允许开发者在不同的线程上异步执行函数,并通过promise-future通道回传结果。本示例中,将介绍如何使用 std::async() 和 std::future 来实现。

\mySubsubsection{}{Getting ready}

对于本示例中的代码,将使用以下函数:

\begin{cpp}
void do_something()
{
    // 模拟长时间运行的操作
    {
        using namespace std::chrono_literals;
        std::this_thread::sleep_for(2s);
    }
    std::lock_guard<std::mutex> lock(g_mutex);
    std::cout << "operation 1 done" << '\n';
}
void do_something_else()
{
    // 模拟长时间运行的操作
    {
        using namespace std::chrono_literals;
        std::this_thread::sleep_for(1s);
    }
    std::lock_guard<std::mutex> lock(g_mutex);
    std::cout << "operation 2 done" << '\n';
}
int compute_something()
{
    // 模拟长时间运行的操作
    {
        using namespace std::chrono_literals;
        std::this_thread::sleep_for(2s);
    }
    return 42;
}
int compute_something_else()
{
    // 模拟长时间运行的操作
    {
        using namespace std::chrono_literals;
        std::this_thread::sleep_for(1s);
    }
    return 24;
}
\end{cpp}

本示例中,将使用future。async() 和 future 都在 <future> 头文件的 std 命名空间中可用。

\mySubsubsection{}{How to do it...}

要在另一个线程上异步执行函数,而当前线程继续执行不需要结果的情况下,请按照以下步骤操作:

\begin{enumerate}
\item
std::async() 启动一个新的线程来执行指定的函数,将创建一个异步提供者并返回一个与之关联的未来。使用 std::launch::async 策略作为 std::async() 函数的第一个参数,以确保函数将异步运行:

\begin{cpp}
auto f = std::async(std::launch::async, do_something);
\end{cpp}

\item
继续当前线程的执行:

\begin{cpp}
do_something_else();
\end{cpp}

\item
需要确保异步操作已完成时,在 std::async() 返回的future实例上调用 wait() 方法:

\begin{cpp}
f.wait();
\end{cpp}

需要在当前线程继续其执行的同时,在工作线程上异步执行函数,直到当前线程需要异步函数的结果时,请按照以下步骤操作:

\item
使用 std::async() 启动一个新的线程来执行指定的函数,创建一个异步提供者,并返回一个与之关联的future。使用函数的第一个参数 std::launch::async 策略以确保函数确实异步运行:

\begin{cpp}
auto f = std::async(std::launch::async, compute_something);
\end{cpp}

\item
继续当前线程的执行:

\begin{cpp}
auto value = compute_something_else();
\end{cpp}

\item
需要异步执行函数的结果时,在 std::async() 返回的future实例上调用 get() 方法:

\begin{cpp}
value += f.get();
\end{cpp}
\end{enumerate}

\mySubsubsection{}{How it works...}

std::async() 是一个可变参数模板函数,有两种重载形式:一种是指定启动策略作为第一个参数,另一种则不指定。传递给 std::async() 的其他参数是要执行的函数及其任何参数。启动策略由 <future> 头文件中提供的一个作用域枚举 std::launch 定义:

\begin{cpp}
enum class launch : /* unspecified */
{
    async = /* unspecified */,
    deferred = /* unspecified */,
    /* implementation-defined */
};
\end{cpp}

两个可用的启动策略指定了以下内容:

\begin{itemize}
\item
使用 async,将启动一个新线程来异步执行任务。

\item
使用 deferred,当第一次请求任务的结果时,任务将在调用线程上执行。
\end{itemize}

当两个标志都指定(\verb|std::launch::async| | \verb|std::launch::deferred|)时,是在一个新线程上异步运行任务,还是在当前线程上同步运行任务由实现决定。这是没有指定启动策略的 std::async() 另一个重载的行为,这种行为不确定。

\begin{myTip}
不要使用非确定性的 std::async() 重载来异步运行任务。始终使用需要启动策略的重载,并且只使用 std::launch::async。
\end{myTip}

std::async() 的两种重载都返回一个future实例,该实例引用 std::async() 内部为建立的promise-future通道创建的共享状态。需要异步操作的结果时,调用future的 get() 方法。这会阻塞当前线程,直到结果值或异常可用。如果future不传输任何值或者你实际上对那个值不感兴趣,但想确保异步操作最终会完成,可以使用 wait() 方法;会阻塞当前线程,直到通过future使共享状态可用。

future还有两个等待方法:\verb|wait_for()| 指定了一个持续时间,即使通过未来共享状态尚未可用,调用也会结束并返回;而 \verb|wait_until()| 指定了一个时间点,即使通过未来共享状态尚未可用,调用也会返回。这些方法可用于创建轮询例程并向用户显示状态消息:

\begin{cpp}
auto f = std::async(std::launch::async, do_something);
while(true)
{
    std::cout << "waiting...\n";
    using namespace std::chrono_literals;
    auto status = f.wait_for(500ms);
    if(status == std::future_status::ready)
        break;
}
std::cout << "done!\n";
\end{cpp}

运行此程序的结果:

\begin{shell}
waiting...
waiting...
waiting...
operation 1 done
done!
\end{shell}

