任务是线程的高级替代方案,用于执行并发计算。std::async()可异步执行函数,无需处理底层线程细节。本示例中,将采用与前一个配方相同的任务,即实现map和fold函数的并行版本,但会使用任务,并观察和比较它与线程版本的表现。

\mySubsubsection{}{How to do it...}

要实现map函数的并行版本,请按照以下步骤操作:

\begin{enumerate}
\item
定义一个接受范围的开始和结束迭代器,以及应用于所有元素的函数的函数模板:

\begin{cpp}
template <typename Iter, typename F>
void parallel_map(Iter begin, Iter end, F&& f)
{
}
\end{cpp}

\item
检查范围的大小。如果元素数量小于预定义的阈值(阈值为10,000),则以顺序方式执行映射:

\begin{cpp}
auto size = std::distance(begin, end);
if(size <= 10000)
    std::transform(begin, end, begin, std::forward<F>(f));
\end{cpp}

\item
对于更大的范围,将工作拆分为多个任务,让每个任务映射范围的一部分。这些部分不应重叠,以避免同步线程对共享数据的访问:

\begin{cpp}
else
{
    auto no_of_tasks = get_no_of_threads();
    auto part = size / no_of_tasks;
    auto last = begin;
    // continued at 4. and 5.
}
\end{cpp}

\item
启动异步函数,并在每个函数上运行映射的串行版本:

\begin{cpp}
std::vector<std::future<void>> tasks;
for(unsigned i = 0; i < no_of_tasks; ++i)
{
    if(i == no_of_tasks - 1) last = end;
    else std::advance(last, part);
    tasks.emplace_back(std::async(
    std::launch::async,
        [=,&f]{std::transform(begin, last, begin,
                              std::forward<F>(f));}));
    begin = last;
}
\end{cpp}

\item
等待所有异步函数完成执行:

\begin{cpp}
for(auto & t : tasks) t.wait();
\end{cpp}
\end{enumerate}

将上述步骤组合起来:

\begin{cpp}
template <typename Iter, typename F>
void parallel_map(Iter begin, Iter end, F&& f)
{
    auto size = std::distance(begin, end);
    if(size <= 10000)
        std::transform(begin, end, begin, std::forward<F>(f));
    else
    {
        auto no_of_tasks = get_no_of_threads();
        auto part = size / no_of_tasks;
        auto last = begin;
        std::vector<std::future<void>> tasks;
        for(unsigned i = 0; i < no_of_tasks; ++i)
        {
            if(i == no_of_tasks - 1) last = end;
            else std::advance(last, part);
            tasks.emplace_back(std::async(
            std::launch::async,
                [=,&f]{std::transform(begin, last, begin,
                                      std::forward<F>(f));}));
            begin = last;
        }
        for(auto & t : tasks) t.wait();
    }
}
\end{cpp}

要实现左\verb|fold|函数的并行版本,请按照以下步骤操作:

\begin{enumerate}
\item
定义一个接受范围的开始和结束迭代器、初始值,以及应用于范围元素的二元函数的函数模板:

\begin{cpp}
template <typename Iter, typename R, typename F>
auto parallel_fold(Iter begin, Iter end, R init, F&& op)
{
}
\end{cpp}

\item
检查范围的大小。如果元素数量小于预定义的阈值(阈值为10,000),则以顺序方式执行折叠:

\begin{cpp}
auto size = std::distance(begin, end);
if(size <= 10000)
    return std::accumulate(begin, end, init, std::forward<F>(op));
\end{cpp}

\item
对于更大的范围,将工作拆分为多个任务,让每个任务折叠范围的一部分。这些部分不应重叠,以避免同步线程对共享数据的访问。结果可以通过传递给异步函数的引用返回,避免同步:

\begin{cpp}
else
{
    auto no_of_tasks = get_no_of_threads();
    auto part = size / no_of_tasks;
    auto last = begin;
    // continued at 4. and 5.
}
\end{cpp}

\item
启动异步函数,并在每个函数上执行折叠的串行版本:

\begin{cpp}
std::vector<std::future<R>> tasks;
for(unsigned i = 0; i < no_of_tasks; ++i)
{
    if(i == no_of_tasks - 1) last = end;
    else std::advance(last, part);
    tasks.emplace_back(
    std::async(
        std::launch::async,
        [=,&op]{return std::accumulate(
                            begin, last, R{},
                            std::forward<F>(op));}));
    begin = last;
}
\end{cpp}

\item
等待所有异步函数完成执行,并将部分结果折叠成最终结果:

\begin{cpp}
std::vector<R> values;
for(auto & t : tasks)
    values.push_back(t.get());
return std::accumulate(std::begin(values), std::end(values),
                       init, std::forward<F>(op));
\end{cpp}
\end{enumerate}

将上述步骤组合起来:

\begin{cpp}
template <typename Iter, typename R, typename F>
auto parallel_fold(Iter begin, Iter end, R init, F&& op)
{
    auto size = std::distance(begin, end);
    if(size <= 10000)
        return std::accumulate(begin, end, init, std::forward<F>(op));
    else
    {
        auto no_of_tasks = get_no_of_threads();
        auto part = size / no_of_tasks;
        auto last = begin;
        std::vector<std::future<R>> tasks;
        for(unsigned i = 0; i < no_of_tasks; ++i)
        {
            if(i == no_of_tasks - 1) last = end;
            else std::advance(last, part);
            tasks.emplace_back(
            std::async(
            std::launch::async,
                [=,&op]{return std::accumulate(
                                    begin, last, R{},
                                    std::forward<F>(op));}));
            begin = last;
        }
        std::vector<R> values;
        for(auto & t : tasks)
        values.push_back(t.get());
        return std::accumulate(std::begin(values), std::end(values),
                               init, std::forward<F>(op));
    }
}
\end{cpp}

\mySubsubsection{}{How it works...}

刚提出的实现仅与前一个配方稍有不同。线程使用异步函数所取代,异步函数从std::async()开始启动,结果通过返回的std::future提供。并发启动的异步函数数量等于实现可支持的线程数。这由静态方法\verb|std::thread::hardware_concurrency()|返回,但这只是一个提示值,不可靠。

采取这种方法主要有两个原因:

\begin{itemize}
\item
了解如何修改为线程并行执行设计的函数,以便使用异步函数,从而避免处理线程的底层细节。

\item
运行等于支持线程数目的异步函数,可以潜在地使每个线程运行一个函数;这可能会因为最小化上下文切换和等待时间开销,而提供最快的并行函数执行速度。
\end{itemize}

可以使用与前一个示例相同的方法来测试新map和fold的实现性能:

\begin{cpp}
std::vector<int> sizes
{
    10000, 100000, 500000,
    1000000, 2000000, 5000000,
    10000000, 25000000, 50000000
};
std::cout
    << std::right << std::setw(8) << std::setfill(' ') << "size"
    << std::right << std::setw(8) << "s map"
    << std::right << std::setw(8) << "p map"
    << std::right << std::setw(8) << "s fold"
    << std::right << std::setw(8) << "p fold"
    << '\n';
for(auto const size : sizes)
{
    std::vector<int> v(size);
    std::iota(std::begin(v), std::end(v), 1);
    auto v1 = v;
    auto s1 = 0LL;
    auto tsm = perf_timer<>::duration([&] {
        std::transform(std::begin(v1), std::end(v1), std::begin(v1),
                       [](int const i) {return i + i; }); });
    auto tsf = perf_timer<>::duration([&] {
        s1 = std::accumulate(std::begin(v1), std::end(v1), 0LL,
                             std::plus<>()); });
    auto v2 = v;
    auto s2 = 0LL;
    auto tpm = perf_timer<>::duration([&] {
        parallel_map(std::begin(v2), std::end(v2),
                    [](int const i) {return i + i; }); });
    auto tpf = perf_timer<>::duration([&] {
        s2 = parallel_fold(std::begin(v2), std::end(v2), 0LL,
                            std::plus<>()); });
    assert(v1 == v2);
    assert(s1 == s2);
    std::cout
        << std::right << std::setw(8) << std::setfill(' ') << size
        << std::right << std::setw(8)
        << std::chrono::duration<double, std::micro>(tsm).count()
        << std::right << std::setw(8)
        << std::chrono::duration<double, std::micro>(tpm).count()
        << std::right << std::setw(8)
        << std::chrono::duration<double, std::micro>(tsf).count()
        << std::right << std::setw(8)
        << std::chrono::duration<double, std::micro>(tpf).count()
        << '\n';
}
\end{cpp}

上述程序的一个可能输出如下,实际输出可能因执行而异,且不同机器之间的差异可能很大:

\begin{shell}
    size   s map   p map  s fold  p fold
   10000      11      11      11      11
  100000     117     260     113      94
  500000     576     303     571     201
 1000000    1180     573    1165     283
 2000000    2371     911    2330     519
 5000000    5942    2144    5841    1886
10000000   11954    4999   11643    2871
25000000   30525   11737   29053    9048
50000000   59665   22216   58689   12942
\end{shell}

类似于用线程展示的解决方案,平行map和fold实现的速度提升可以在下图中看到。

负值表示顺序版本更快:

\myGraphic{1.0}{content/chapter8/images/2.png}{图8.2: 使用异步函数的map(蓝色)和平行实现的fold(橙色)与顺序实现相比的速度提升}

如果将其与使用线程的平行版本的结果进行比较,会发现这些是更快的执行时间,特别是对于fold函数,速度提升显著。下图显示了任务实现相对于线程实现的速度提升。在这张图中,小于1的值意味着线程实现更快:

\myGraphic{1.0}{content/chapter8/images/3.png}{图8.3: 使用异步函数的平行实现与使用线程的平行实现相比,map(蓝色)和fold(橙色)的速度提升}

\mySubsubsection{}{There's more...}

前面展示的实现可以用来并行化map和fold函数的方法之一。

一个可能的替代方案使用以下策略:

\begin{itemize}
\item
将要处理的范围分成两个相等的部分。

\item
异步递归调用平行函数来处理范围的第一部分。

\item
同步递归调用平行函数来处理范围的第二部分。

\item
同步递归调用完成后,等待异步递归调用也结束,然后再完成执行。
\end{itemize}

这种分治算法可能会创建大量的任务。根据范围的大小,异步调用的数量可能会超过线程数量。这种情况下,会有大量的等待时间影响整体执行时间。

map和fold函数可以使用分治算法实现:

\begin{cpp}
template <typename Iter, typename F>
void parallel_map(Iter begin, Iter end, F f)
{
    auto size = std::distance(begin, end);
    if(size <= 10000)
    {
        std::transform(begin, end, begin, std::forward<F>(f));
    }
    else
    {
        auto middle = begin;
        std::advance(middle, size / 2);
        auto result = std::async(
            std::launch::deferred,
            parallel_map<Iter, F>,
            begin, middle, std::forward<F>(f));
        parallel_map(middle, end, std::forward<F>(f));
        result.wait();
    }
}
template <typename Iter, typename R, typename F>
auto parallel_fold(Iter begin, Iter end, R init, F op)
{
    auto size = std::distance(begin, end);
    if(size <= 10000)
        return std::accumulate(begin, end, init, std::forward<F>(op));
    else
    {
        auto middle = begin;
        std::advance(middle, size / 2);
        auto result1 = std::async(
            std::launch::async,
            parallel_reduce<Iter, R, F>,
            begin, middle, R{}, std::forward<F>(op));
        auto result2 = parallel_fold(middle, end, init,
                                     std::forward<F>(op));
        return result1.get() + result2;
    }
}
\end{cpp}

此实现的执行时间列于此处,旁边是先前实现的执行时间:

\begin{shell}
    size   s map p1 map  p2 map  s fold p1 fold p2 fold
   10000      11     11      10       7      10      10
  100000     111    275     120      72      96     426
  500000     551    230     596     365     210    1802
 1000000    1142    381    1209     753     303    2378
 2000000    2411    981    2488    1679     503    4190
 5000000    5962   2191    6237    4177    1969    7974
10000000   11961   4517   12581    8384    2966   15174
\end{shell}

当我们比较这些执行时间时,可以看到这个版本(前文输出中标记为p2)对于map和fold来说都与顺序版本相似,且远不如前面展示的第一个并行版本(标记为p1)。



