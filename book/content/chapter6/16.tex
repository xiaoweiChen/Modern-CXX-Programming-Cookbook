
上一个示例中,了解了什么是类型特征,标准提供了哪些特征。本示例中,将更进一步,了解如何定义自定义特征。

\mySubsubsection{}{Getting ready}

本示例中,将了解如何解决以下问题:我几个支持序列化的类型。不涉及具体细节,假设有些类型提供一种“普通”的序列化到字符串的功能,而另一些类型则是基于指定的编码进行序列化。最终目标是为这些类型中的对象创建一个单一且统一的序列化API。为此,考虑以下两个类:foo,提供简单的序列化:以及 bar,提供带编码的序列化。

来看一下代码:

\begin{cpp}
struct foo
{
    std::string serialize()
    {
        return "plain"s;
    }
};
struct bar
{
    std::string serialize_with_encoding()
    {
        return "encoded"s;
    }
};
\end{cpp}

\mySubsubsection{}{How to do it...}

实现以下类和函数模板:

\begin{itemize}
\item
名为 \verb|is_serializable_with_encoding| 的类型模板中,包含一个设置为 false 的静态 const bool 变量:

\begin{cpp}
template <typename T>
struct is_serializable_with_encoding
{
    static const bool value = false;
};
\end{cpp}

\item
\verb|is_serializable_with_encoding| 模板的全特化版本,针对类型 bar 设置静态 const bool 变量为 true:

\begin{cpp}
template <>
struct is_serializable_with_encoding<bar>
{
    static const bool value = true;
};
\end{cpp}

\item
名为 serializer 的类型模板中,包含一个名为 serialize 的静态模板方法,该方法接受一个模板类型 T 的参数,并调用该实例的 serialize() 方法:

\begin{cpp}
template <bool b>
struct serializer
{
    template <typename T>
    static auto serialize(T& v)
    {
        return v.serialize();
    }
};
\end{cpp}

\item
一个全特化的类模板,针对 true,其 serialize() 静态方法调用参数的 \verb|serialize_with_encoding()| :

\begin{cpp}
template <>
struct serializer<true>
{
    template <typename T>
    static auto serialize(T& v)
    {
        return v.serialize_with_encoding();
    }
};
\end{cpp}

\item
名为 serialize() 的函数模板,使用前面定义的 serializer 类模板和 \verb|is_serializable_with_encoding| 类型特征,选择应该调用哪个实际的序列化方法(普通或带编码):

\begin{cpp}
template <typename T>
auto serialize(T& v)
{
    return serializer<is_serializable_with_encoding<T>::value>::
    serialize(v);
}
\end{cpp}
\end{itemize}

\mySubsubsection{}{How it works...}

\verb|is_serializable_with_encoding| 是一个类型特征,用于检查类型 T 是否支持带编码的序列化。其提供了一个名为 value 的静态成员,类型为 bool,如果 T 支持带编码的序列化,则该值为 true,否则为 false。作为具有单个类型模板参数 T 的类模板实现的;这个类模板对于支持编码序列化的类型进行了全特化,就是针对类型 bar:

\begin{cpp}
std::cout << std::boolalpha;
std::cout <<
    is_serializable_with_encoding<foo>::value << '\n';        // false
std::cout <<
    is_serializable_with_encoding<bar>::value << '\n';        // true
std::cout <<
    is_serializable_with_encoding<int>::value << '\n';        // false
std::cout <<
    is_serializable_with_encoding<std::string>::value << '\n';// false
std::cout << std::boolalpha;
\end{cpp}

serialize() 方法是一个函数模板,表示一个通用的API,用于序列化支持任意类型序列化的实例。接受一个类型为模板参数 T 的单个参数,并使用辅助类模板 serializer 来调用参数的 serialize() 或 \verb|serialize_with_encoding()| 方法。

serializer 类型是带有单个非类型模板参数(类型为 bool)的类模板。此类模板包含一个名为 serialize() 的静态函数模板。该函数模板接受一个类型为模板参数 T 的单个参数,调用参数的 serialize() 方法,并返回该调用返回的值。serializer 类模板对其非类型模板参数值为 true 的情况有一个完全特化版本。这个特化版本中,函数模板 serialize() 的签名保持不变,但调用的是 \verb|serialize_with_encoding()|,不是 serialize()。

serialize() 函数模板中,使用 \verb|is_serializable_with_encoding| 类型特征来决定使用泛型还是完全特化的类模板。类型特征的静态成员 value 被用作 serializer 的非类型模板参数的参数。

定义了所有这些之后,可以编写如下代码:

\begin{cpp}
foo f;
bar b;
std::cout << serialize(f) << '\n'; // plain
std::cout << serialize(b) << '\n'; // encoded
\end{cpp}

使用 foo 参数调用 serialize() 将返回字符串 plain,而使用 bar 参数调用 serialize() 将返回字符串 encoded。

