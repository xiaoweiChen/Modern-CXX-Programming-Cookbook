
上一个示例中,了解了 C++20 引入的一个名为 \verb|std::span| 的类型,表示对连续元素序列的视图(非拥有的封装器)。类似于 C++17 中的 \verb|std::string_view| 类型,后者对字符序列做同样的事情。这两个都是对一维序列的视图。然而,有时需要处理多维序列。这些可以以多种方式实现,比如 C 风格数组(int[2][3][4])、指针数组(int** 或 int***)、数组的数组(或vector的vector,如 \verb|vector<vector<vector<int>>>|)。另一种方法是使用一维对象序列,但定义操作使其呈现为逻辑上的多维序列。这就是 C++23 中的 \verb|std::mdspan| 类型:表示一个多维序列的非拥有视图,底层实际上是一维的连续对象序列。可以说 \verb|std::mdspan| 是 \verb|std::span| 类对多维扩展。

\mySubsubsection{}{Getting ready}

本示例中,将参考以下简单的二维矩阵实现(其大小在编译时已知):

\begin{cpp}
template <typename T, std::size_t ROWS, std::size_t COLS>
struct matrix
{
    T&
    #ifdef __cpp_multidimensional_subscript
    operator[] // C++23
    #else
    operator() // previously
    #endif
    (std::size_t const r, std::size_t const c)
    {
        if (r >= ROWS || c >= COLS)
            throw std::runtime_error("Invalid index");
        return data[r * COLS + c];
    }
    T const &
    #ifdef __cpp_multidimensional_subscript
    operator[] // C++23
    #else
    operator() // previously
    #endif
    (std::size_t const r, std::size_t const c) const
    {
        if (r >= ROWS || c >= COLS)
            throw std::runtime_error("Invalid index");
        return data[r * COLS + c];
    }
    std::size_t size() const { return data.size(); }
    std::size_t empty() const { return data.empty(); }
    template <std::size_t dimension>
    std::size_t extent() const
    {
        static_assert(dimension <= 1,
                      "The matrix only has two dimensions.");
        if constexpr (dimension == 0) return ROWS;
        else if constexpr(dimension == 1) return COLS;
    }
    private:
    std::array<T, ROWS* COLS> data;
};
\end{cpp}

\begin{myTip}
C++23 中,建议使用 \verb|operator[]| 而非 \verb|operator()| 来访问多维数据结构的元素。
\end{myTip}

\mySubsubsection{}{How to do it...}

优先使用 \verb|std::mdspan| 而不是多维 C 风格数组、指针数组或vector/array的数组实现:

\begin{cpp}
void f(int data[2][3]) { /* … */ }
void g(int** data, size_t row, size_t cols) { /* … */ }
void h(std::vector<std::vector<int>> dat, size_t row, size_t cols)
{ /* … */ }
\end{cpp}

替换为:

\begin{cpp}
void f(std::mdspan<int,std::extents<size_t, 2, 3>> data)
{ /* … */ }
\end{cpp}

当使用 \verb|std::mdspan| 时,可以做以下事情:

\begin{itemize}
\item
使用编译时常量长度(称为静态范围)创建一个 mdspan,方法是指定 span 中每个维度的元素数量:

\begin{cpp}
int* data = get_data();
std::mdspan<int, std::extents<size_t, 2, 3>> m(data);
\end{cpp}

\item
使用运行时常量长度(称为动态范围)创建 mdspan,方法是不在编译时指定 span 中某个维度的元素数量,而是在运行时提供:

\begin{cpp}
int* data = get_data();
std::mdspan<int, std::extents<size_t,
                              2,
                              std::dynamic_extent>> mv{v.data(), 3};
\end{cpp}

或

\begin{cpp}
int* data = get_data();
std::mdspan<int, std::extents<size_t,
                              std::dynamic_extent,
                              std::dynamic_extent>>
m(data, 2, 3);
\end{cpp}

或

\begin{cpp}
int* data = get_data();
std::mdspan m(data, 2, 3);
\end{cpp}

\item
为了控制 mdspan 的多维索引映射到底层(连续)数据序列的一维索引,可以使用布局策略,这是第三个模板参数:

\begin{cpp}
std::mdspan<int,
            std::extents<size_t, 2, 3>,
            std::layout_right> mv{ data };
\end{cpp}

或

\begin{cpp}
std::mdspan<int,
            std::extents<size_t, 2, 3>,
            std::layout_left> mv{ data };
\end{cpp}

或

\begin{cpp}
std::mdspan<int,
            std::extents<size_t, 2, 3>,
            std::layout_stride> mv{ data };
\end{cpp}
\end{itemize}

\mySubsubsection{}{How it works...}

mdspan 是一个多维 span,即是一个对一维值序列的非拥有视图,该序列被投射为逻辑上的多维结构,这正是 matrix 类型所做的事情,其定义的操作(如 C++23 中的 \verb|operator[]| 和之前的 \verb|operator()|)是针对二维数据结构的。然而,内部数据是以连续序列的形式排列的,的实现中使用的是 \verb|std::array|:

\begin{cpp}
matrix<int, 2, 3> m;
for (std::size_t r = 0; r < m.extent<0>(); r++)
{
    for (std::size_t c = 0; c < m.extent<1>(); c++)
    {
        m[r, c] = r * m.extent<1>() + c + 1; // m[r,c] in C++23
        // m(r, c) previously
    }
}
\end{cpp}

这段 for-in-for 语句将矩阵元素的值设置为以下内容:

\begin{shell}
1 2 3
4 5 6
\end{shell}

C++23 中,可以直接用 \verb|std::mdspan| 替换整个类型:

\begin{cpp}
std::array<int, 6> arr;
std::mdspan m{arr.data(), std::extents{2, 3}};
for (std::size_t r = 0; r < m.extent(0); r++)
{
    for (std::size_t c = 0; c < m.extent(1); c++)
    {
        m[r, c] = r * m.extent(1) + c + 1;
    }
}
\end{cpp}

这里唯一改变的是使用了 extent() 方法,以前这是 matrix 类型的一个成员函数模板。实际上,可以将 matrix 定义为 \verb|std::mdspan| 的别名模板:

\begin{cpp}
template <typename T, std::size_t ROWS, std::size_t COLS>
using matrix = std::mdspan<T, std::extents<std::size_t, ROWS, COLS>>;
std::array<int, 6> arr;
matrix<int, 2, 3> ma {arr.data()};
\end{cpp}

这个例子中,mdspan 是二维的,但可以定义为任意数量的维度。mdspan 类型的接口包括以下成员:

% Please add the following required packages to your document preamble:
% \usepackage{longtable}
% Note: It may be necessary to compile the document several times to get a multi-page table to line up properly
\begin{longtable}{|l|l|}
\hline
\textbf{名称} & \textbf{描述}                                    \\ \hline
\endfirsthead
%
\endhead
%
operator[]    & 提供对底层数据元素的访问。 \\ \hline
size()        & 返回元素数量。                         \\ \hline
empty()       & 表明元素数量是否为零。       \\ \hline
stride() & 返回指定维度的步长。除非明确自定义,否则默认为 1。 \\ \hline
extents()     & 返回指定维度的大小(范围)。   \\ \hline
\end{longtable}

\begin{center}
表 6.5: mdspan 的一些成员函数列表
\end{center}

如果查看 \verb|std::mdspan| 的定义,会看到以下内容:

\begin{cpp}
template<class T,
         class Extents,
         class LayoutPolicy = std::layout_right,
         class AccessorPolicy = std::default_accessor<T>>
class mdspan;
\end{cpp}

前两个模板参数是元素的类型和每个维度的范围(大小),而后两个是定制点:

\begin{itemize}
\item
布局策略控制 mdspan 的多维索引如何映射到底层一维数据的偏移量。有几个选项可用:\verb|layout_right|(默认值),其中最右边的索引提供对底层内存的步长为一的访问(C/C++ 风格);\verb|layout_left|,其中最左边的索引提供对底层内存的步长为一的访问(Fortran 和 Matlab 风格);以及 \verb|layout_stride|,概括了前两者并允许自定义每个维度的步长。布局策略的原因是为了与其他语言的互操作性,以及在不改变循环结构的情况下更改算法的数据访问模式。

\item
访问策略定义了底层序列,如何存储其元素,以及如何使用布局策略提供的偏移量,来获取对存储元素的引用。这些主要是为第三方库设计的,就像不可能需要为 \verb|std::vector| 定义分配器一样,也不需要为 \verb|std::mdspan| 实现访问策略。
\end{itemize}

举例说明布局策略以理解其工作原理,默认布局策略是 \verb|std::layout_right|。可以考虑以下示例:

\begin{cpp}
std::vector v {1,2,3,4,5,6,7,8,9};
std::mdspan<int,
            std::extents<size_t, 2, 3>,
            std::layout_right> mv{v.data()};
\end{cpp}

这里定义的二维矩阵包含以下内容:

\begin{shell}
1 2 3
4 5 6
\end{shell}

如果把布局策略改为 \verb|std::layout_left|,内容也会相应变化为:

\begin{shell}
1 3 5
2 4 6
\end{shell}

可以使用 \verb|std::layout_stride| 自定义每个维度的步长。以下代码段定义了一个步长等同于 \verb|std::layout_right| 的步长,适用于 2x3 的矩阵:

\begin{cpp}
std::mdspan<int,
            std::extents<size_t,
                        std::dynamic_extent,
                        std::dynamic_extent>,
            std::layout_stride>
mv{ v.data(), { std::dextents<size_t,2>{2, 3},
                std::array<std::size_t, 2>{3, 1}}};
\end{cpp}

不同的步长提供了不同的结果。下表展示了几个例子:

% Please add the following required packages to your document preamble:
% \usepackage{longtable}
% Note: It may be necessary to compile the document several times to get a multi-page table to line up properly
\begin{longtable}{|l|l|}
\hline
\textbf{步长} & \textbf{矩阵}                                       \\ \hline
\endfirsthead
%
\endhead
%
{0, 0}           & \begin{tabular}[c]{@{}l@{}}1 1 1\\ 1 1 1\end{tabular} \\ \hline
{0, 1}           & \begin{tabular}[c]{@{}l@{}}1 2 3\\ 1 2 3\end{tabular} \\ \hline
{1, 0}           & \begin{tabular}[c]{@{}l@{}}1 1 1\\ 2 2 2\end{tabular} \\ \hline
{1, 1}           & \begin{tabular}[c]{@{}l@{}}1 2 3\\ 2 3 4\end{tabular} \\ \hline
{2, 1}           & \begin{tabular}[c]{@{}l@{}}1 2 3\\ 3 4 5\end{tabular} \\ \hline
{1, 2}           & \begin{tabular}[c]{@{}l@{}}1 3 5\\ 2 4 6\end{tabular} \\ \hline
{2, 3}           & \begin{tabular}[c]{@{}l@{}}1 4 7\\ 3 6 9\end{tabular} \\ \hline
\end{longtable}

\begin{center}
表 6.6: 自定义步长及其导致视图内容的例子
\end{center}

让讨论最后一个例子,这可能稍微普遍一点。第一个维度的步长表示行之间的偏移增量。底层序列中的第一个元素位于索引 0 处。这时,步长为 2 表示行从索引 0, 2, 4 等读取。第二个维度的步长表示列之间的偏移增量。第一个元素位于对应行的索引处。这个例子中,第一行的索引是 0,所以列的步长为 3 ,所以第一行的元素将从索引 0, 3 和 6 读取。第二行从索引 2 开始,所以第二行的元素将从索引 2, 5 和 8 读取。这就是上表中显示的最后一个例子。

\mySubsubsection{}{there's more}

mdspan 的原始提案中包含了一个名为 \verb|submdspan()| 的独立函数。此函数创建了 mdspan 的切片,也就是 mdspan 的子集视图。为了使 mdspan 能够纳入 C++23,此函数移除并推迟到 C++26。本书撰写之时,已经纳入 C++26(还没有编译器支持)。


