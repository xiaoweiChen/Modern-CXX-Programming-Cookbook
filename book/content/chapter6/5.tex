
标准库提供了几种无序关联容器:\verb|std::unordered_set|、\verb|std::unordered_multiset|、\verb|std::unordered_map| 和 \verb|std::unordered_multimap|。这些容器不会以特定顺序存储其元素;相反,会分组到桶中,元素所属的桶取决于元素的哈希值。这些标准容器默认使用 \verb|std::hash| 类型模板来计算哈希值。所有基本类型以及一些库类型的特化都是可用的。但对于自定义类型,必须自己特化类模板。本示例将展示如何做到这一点,并解释如何计算一个好的哈希值。一个好的哈希值可以快速计算,并且在整个值域内均匀分布,从而最大限度地减少哈希冲突的概率。

\mySubsubsection{}{Getting ready}

本示例中的例子,将使用以下类型:

\begin{cpp}
struct Item
{
    int         id;
    std::string name;
    double      value;
    Item(int const id, std::string const & name, double const value)
        :id(id), name(name), value(value)
    {}
    bool operator==(Item const & other) const
    {
        return id == other.id && name == other.name &&
        value == other.value;
    }
};
\end{cpp}

本示例涵盖了标准库中的哈希功能,应该熟悉哈希和哈希函数的概念。

\mySubsubsection{}{How to do it...}

为了使自定义类型能够在无序关联容器中使用,必须执行以下步骤:

\begin{enumerate}
\item
std 命名空间中为自定义类型特化 \verb|std::hash| 类型模板。

\item
定义参数和结果类型的同义词。

\item
实现调用操作符,使其接受自定义类型的一个常量引用,并返回一个哈希值。
\end{enumerate}

为了计算一个好的哈希值,应该执行以下操作:

\begin{enumerate}
\item
从一个初始值开始,该值应该是一个质数(例如,17)。

\item
对于每个用于确定类的两个实例是否相等的字段,根据以下公式调整哈希值:

\begin{cpp}
hashValue = hashValue * prime + hashFunc(field);
\end{cpp}

\item
可以使用相同的质数对所有字段应用上述公式,但建议使用与初始值不同的值(例如,31)。

\item
使用 \verb|std::hash| 的特化来确定类数据成员的哈希值。
\end{enumerate}

基于此处描述的步骤,\verb|Item| 类的 \verb|std::hash| 特化看起来像这样:

\begin{cpp}
namespace std
{
    template<>
    struct hash<Item>
    {
        typedef Item argument_type;
        typedef size_t result_type;
        result_type operator()(argument_type const & item) const
        {
            result_type hashValue = 17;
            hashValue = 31 * hashValue + std::hash<int>{}(item.id);
            hashValue = 31 * hashValue + std::hash<std::string>{}(item.name);
            hashValue = 31 * hashValue + std::hash<double>{}(item.value);
            return hashValue;
        }
    };
}
\end{cpp}

这种特化使得可以将 \verb|Item| 类型与无序关联容器一起使用,例如 \verb|std::unordered_set|。下面提供了一个例子:

\begin{cpp}
std::unordered_set<Item> set2
{
    { 1, "one"s, 1.0 },
    { 2, "two"s, 2.0 },
    { 3, "three"s, 3.0 },
};
\end{cpp}

\mySubsubsection{}{How it works...}

类型模板 \verb|std::hash| 是一个函数对象模板,其调用操作符定义了一个具有以下属性的哈希函数:

\begin{itemize}
\item
接受一个模板参数类型的参数并返回一个 \verb|size_t| 值。

\item
不抛出任何异常。

\item
对于两个相等的参数,返回相同的哈希值。

\item
对于两个不相等的参数,返回相同值的概率非常小(应接近 \verb|1.0/std::numeric_limits<size_t>::max()|)。
\end{itemize}

标准为所有基本类型提供了特化,如 bool、char、int、long、float、double(包括所有可能的无符号和长整型变化),以及指针类型,还有库类型,包括 \verb|basic_string| 和 \verb|basic_string_view| 类型、\verb|unique_ptr| 和 \verb|shared_ptr|、bitset 和 vector<bool>、optional 和 variant(C++17)、以及其他几种类型。但对于自定义类型,必须提供相应的特化。这种特化必须位于命名空间 std 中(那是定义类模板 hash 的命名空间),并且必须满足前面列出的要求。

标准没有规定如何计算哈希值。可以使用任何函数,只要它对相等的对象返回相同的值,并且对不相等的对象返回相同值的概率很小即可。本示例中描述的算法出自 Joshua Bloch 的《Effective Java》第二版一书。

计算哈希值时,只考虑那些用于确定类型的两个实例是否相等,必须使用在 \verb|operator==| 中的字段。本例子中,\verb|Item| 类型的所有三个字段都用于确定两个对象的相等性,必须使用所有这些字段来计算哈希值。初始哈希值不应为零,例子中选择了质数 17。

重要的是,这些值不应为零;否则,产生哈希值零的初始字段(即处理顺序中的第一个字段)将不会改变哈希值(因为 x * 0 + 0 = 0)。对于计算哈希值的字段,通过将其先前值乘以一个质数,并加上当前字段的哈希值来更改当前哈希值。为此,这里使用 \verb|std::hash| 类型模板的特化。

使用质数 31 对性能优化有利,编译器可以用 \verb|(x << 5) - x| 替换 31 * x,这更快。同样,可以使用 127,因为 127 * x 等于 \verb|(x << 7) - x| 或者使用 8191,因为 8191 * x 等于 \verb|(x << 13) - x|。

如果自定义类型包含一个数组,并且用于确定两个对象的相等性,需要计算哈希值,可以将数组视为其元素是类的数据成员一样处理,对数组的所有元素使用相同的算法。

