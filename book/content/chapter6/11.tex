
有时需要编写一个函数,该函数既返回一些数据,又返回成功或失败的指示(最简单的情况下作为布尔值,复杂情况下作为错误码)。通常,这个问题可以通过返回状态码,并使用按引用传递的参数来返回数据,或者在失败时返回实际数据但抛出异常来解决。随着 std::optional 和 std::variant 的可用性,出现了新的解决方案。C++23 标准提供了一种新的方法,即 std::expected 类型,这是之前提到的两种类型的某种结合体。这种类型存在于其他编程语言中,如 Rust 中的 Result 和 Haskell 中的 Either。本示例中,将介绍如何使用新的 std::expected 类型。

\mySubsubsection{}{Getting ready}

本示例中所示的例子,将使用这里定义的数据类型:

\begin{cpp}
enum class Status
{
    Success, InvalidFormat, InvalidLength, FilterError,
};
enum class Filter
{
    Pixelize, Sepia, Blur
};
using Image = std::vector<char>;
\end{cpp}

\mySubsubsection{}{How to do it...}

可以使用 <expected> 头文件的 std::expected<T, E> 类型,如下例所示:

\begin{itemize}
\item
当从函数返回数据时,若执行过程中出现错误则返回 std::unexpected<E>,若一切顺利则返回数据(T 类型的值):

\begin{cpp}
bool IsValidFormat(Image const& img) { return true; }
bool IsValidLength(Image const& img) { return true; }
bool Transform(Image& img, Filter const filter)
{
    switch(filter)
    {
        case Filter::Pixelize:
            img.push_back('P');
            std::cout << "Applying pixelize\n";
        break;
        case Filter::Sepia:
            img.push_back('S');
            std::cout << "Applying sepia\n";
        break;
        case Filter::Blur:
            img.push_back('B');
            std::cout << "Applying blur\n";
        break;
    }
    return true;
}
std::expected<Image, Status> ApplyFilter(Image img,
                                         Filter const filter)
{
    if (!IsValidFormat(img))
        return std::unexpected<Status> {Status::InvalidFormat};
    if (!IsValidLength(img))
        return std::unexpected<Status> {Status::InvalidLength};
    if (!Transform(img, filter))
        return std::unexpected<Status> {Status::FilterError};
    return img;
}
std::expected<Image, Status> FlipHorizontally(Image img)
{
    return Image{img.rbegin(), img.rend()};
}
\end{cpp}

\item
当检查返回 std::expected<T, E> 类型的函数的结果时,使用布尔操作符(或 \verb|has_value()| 方法)检查对象是否持有预期值,使用 value() 和 error() 方法分别返回预期值或意外错误:

\begin{cpp}
void ShowImage(Image const& img)
{
    std::cout << "[img]:";
    for(auto const & e : img) std::cout << e;
    std::cout << '\n';
}
void ShowError(Status const status)
{
    std::cout << "Error code: "
              << static_cast<int>(status) << '\n';
}
int main()
{
    Image img{'I','M','G'};

    auto result = ApplyFilter(img, Filter::Sepia);
    if (result)
    {
        ShowImage(result.value());
    }
    else
    {
        ShowError(result.error());
    }
}
\end{cpp}

\item
可以使用单子操作 \verb|and_then()|、\verb|or_else()|、\verb|transform()| 和 \verb|transform_error()| 组合返回 std::expected 值的函数的操作链:

\begin{cpp}
int main()
{
    Image img{'I','M','G'};

    ApplyFilter(img, Filter::Sepia)
    .and_then([](Image result){
        return ApplyFilter(result, Filter::Pixelize);
    })
    .and_then([](Image result){
        return ApplyFilter(result, Filter::Blur);
    })
    .and_then([](Image result){
        ShowImage(result);
        return std::expected<Image, Status>{result};
    })
    .or_else([](Status status){
        ShowError(status);
        return std::expected<Image, Status>{std::unexpect,
            status};
    });
}
\end{cpp}

\end{itemize}

\mySubsubsection{}{How it works...}

std::expected<T, E> 类模板可在 C++23 头文件 <expected> 中找到。这个类是 std::variant 和 std::optional 类型(C++17)的混合体,但设计目的是从函数返回数据或意外值。具有区分联合的逻辑结构,要么持有预期类型 T 的值,要么持有意外类型(错误)E 的值。然而,其接口与 std::optional 类非常相似,具有以下成员:

% Please add the following required packages to your document preamble:
% \usepackage{longtable}
% Note: It may be necessary to compile the document several times to get a multi-page table to line up properly
\begin{longtable}{|l|l|}
\hline
\textbf{函数}                   & \textbf{描述}                                                                                                 \\ \hline
\endfirsthead
%
\endhead
%
has\_value()                        & 返回一个布尔值,指示对象是否包含预期值。                        \\ \hline
operator bool                       &  \begin{tabular}[c]{@{}l@{}} 与 has\_value() 相同。在 if 语句中更简便地使用而提供\\(如 if(result) 而非 if(result.has\_value()))。\end{tabular}  \\ \hline
value() &
\begin{tabular}[c]{@{}l@{}}返回预期值,除非对象包含意外值。这时,将抛出一个包含意\\外值的 std::bad\_expected\_access<E> 异常。\end{tabular} \\ \hline
value\_or() &
\begin{tabular}[c]{@{}l@{}}类似于 value(),但如果对象存储了意外值,则返回提供的替\\代值,而不是抛出异常。\end{tabular}  \\ \hline
error()                             & 返回意外值。如果对象持有预期值,则行为未定义。                  \\ \hline
operator-\textgreater and operator* & 访问预期值。如果对象持有意外值,则行为未定义。                \\ \hline
\end{longtable}

\begin{center}
表 6.2: std::expected 最重要的成员列表
\end{center}

之前提到过 std::expected 类型是两个类型 T(预期)和 E(错误)的区分联合,但这并不完全正确。其实际持有的类型要么是 T,要么是 std::unexpected<E>。后者是一个持有 E 类型对象的帮助类。用于 T 和 E 的可能类型有一些限制:

\begin{itemize}
\item
T 可以是 void 或者可销毁类型(可以调用析构函数的类型)。数组和引用类型不能替换 T。如果类型 T 是 void 类型,则 value\_or() 方法不可用。

\item
E 必须是可销毁类型。数组、引用类型,以及带有 const 和 volatile 限定符的类型不能替换 E。
\end{itemize}

当想对一个值应用多个操作时,会发生这种情况。在例子中,这可能是连续应用不同的滤镜到图像上。但也可能是其他操作,比如调整图像大小、改变格式/类型、不同方向翻转等等。这些操作中的每一个都会返回一个 std::expected 值。这时,可以编写如下代码:

\begin{cpp}
int main()
{
    Image img{'I','M','G'};

    auto result = ApplyFilter(img, Filter::Sepia);
    result = ApplyFilter(result.value(), Filter::Pixelize);
    result = ApplyFilter(result.value(), Filter::Blur);
    result = FlipHorizontally(result.value());
    if (result)
    {
        ShowImage(result.value());
    }
    else
    {
        ShowError(result.error());
    }
}
\end{cpp}

如果没有错误发生,则运行此程序的结果如下:

\begin{shell}
Applying sepia
Applying pixelize
Applying blur
[img]:BPSGMI
\end{shell}

如果在 ApplyFilter() 函数中发生错误,随后调用 value() 方法会导致 std::bad\_expected\_access 异常。实际上,必须在每次操作后检查结果。可以通过使用单子操作来改进。

由于 std::expected 类型类似于 std::optional 类型,因此 C++23 中为 std::optional 提供的单子操作也适用于 std::expected。这些操作包括:

% Please add the following required packages to your document preamble:
% \usepackage{longtable}
% Note: It may be necessary to compile the document several times to get a multi-page table to line up properly
\begin{longtable}{|l|l|}
\hline
\textbf{函数}  & \textbf{描述}                                                                                     \\ \hline
\endfirsthead
%
\endhead
%
and\_then() &
\begin{tabular}[c]{@{}l@{}} std::expected 对象包含预期值(类型为 T),则应用给定函数并返回\\结果。否则,返回 std::expected 值。\end{tabular} \\ \hline
or\_else() &
\begin{tabular}[c]{@{}l@{}} std::expected 对象包含意外值(类型为 E),则应用给定函数并返回\\结果。否则,返回 std::expected 值。\end{tabular} \\ \hline
transform()        & \begin{tabular}[c]{@{}l@{}} 这与 and\_then() 类似,只是返回值包装在 std::expected 值中。\end{tabular} \\ \hline
transform\_error() & \begin{tabular}[c]{@{}l@{}}这与 or\_else() 类似,只是返回值包装在 std::expected 值中。\end{tabular}  \\ \hline
\end{longtable}

\begin{center}
表 6.3: std::expected 的一元操作
\end{center}

可以使用单子操作重新编写最后一段代码:

\begin{cpp}
int main()
{
    Image img{'I','M','G'};

    ApplyFilter(img, Filter::Sepia)
    .and_then([](Image result){
        return ApplyFilter(result, Filter::Pixelize);
    })
    .and_then([](Image result){
        return ApplyFilter(result, Filter::Blur);
    })
    .and_then(FlipHorizontally)
    .and_then([](Image result){
        ShowImage(result);
        return std::expected<Image, Status>{result};
    })
    .or_else([](Status status){
        ShowError(status);
        return std::expected<Image, Status>{std::unexpect, status};
    });
}
\end{cpp}

如果没有错误发生,则输出就是看到的那样。如果发生错误,比如说在使用 sepia 滤镜时,则输出会变为:

\begin{shell}
Applying sepia
Error code: 3
\end{shell}

这个例子只展示了可用的单子操作中的两个,and\_then() 和 or\_else()。另外两个,transform() 和 transform\_error()也是类似,但其旨在将预期值或意外值转换为另一个值。下面的代码段(对前一个代码段的修改)中,无论是在预期值还是意外值的情况下,都连接了一个转换操作,最终返回一个字符串:

\begin{cpp}
int main()
{
    Image img{'I','M','G'};

    auto obj = ApplyFilter(img, Filter::Sepia)
    .and_then([](Image result){
        return ApplyFilter(result, Filter::Pixelize);
    })
    .and_then([](Image result){
        return ApplyFilter(result, Filter::Blur);
    })
    .and_then(FlipHorizontally)
    .and_then([](Image result){
        ShowImage(result);
        return std::expected<Image, Status>{result};
    })
    .or_else([](Status status){
        ShowError(status);
        return std::expected<Image, Status>{std::unexpect, status};
    })
    .transform([](Image result){
        std::stringstream s;
        s << std::quoted(std::string(result.begin(),
        result.end()));
        return s.str();
    })
    .transform_error([](Status status){
        return status == Status::Success ? "success" : "fail";
    });
    if(obj)
        std::cout << obj.value() << '\n';
    else
        std::cout << obj.error() << '\n';
}
\end{cpp}

如果在执行此程序的过程中没有发生任何错误,将输出以下内容:

\begin{shell}
Applying sepia
Applying pixelize
Applying blur
[img]:BPSGMI
"BPSGMI"
\end{shell}

如果在执行过程中发生错误,例如:使用 sepia 滤镜时,输出将变为:

\begin{shell}
Applying sepia
Error code: 3
fail
\end{shell}

上面的 or\_else() 函数中,可能会注意到使用了 std::unexpected。这是一个帮助类,用作 std::expected 构造函数的标签,以表明正在构造一个意外值。参数会完美转发给 E 类型(意外类型)的构造函数。对于新创建的 std::expected 值,has\_value() 方法将返回 false,表示其持有一个意外值。

