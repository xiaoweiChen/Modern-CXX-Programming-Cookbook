
另一种调试机制是断言(asserts),因为经常需要知道导致某个执行点的调用序列,这称为堆栈跟踪(stack trace)。C++23 标准包含了一个新的诊断工具库,可以打印堆栈跟踪。本示例中,将介绍如何使用这些诊断工具。

\mySubsubsection{}{How to do it...}

可以使用 C++23 的堆栈跟踪库来执行以下操作:

\begin{itemize}
\item
打印整个堆栈跟踪的内容:

\begin{cpp}
std::cout << std::stacktrace::current() << '\n';
\end{cpp}

\item
遍历堆栈跟踪中的每个帧并打印:

\begin{cpp}
for (auto const & frame : std::stacktrace::current())
{
    std::cout << frame << '\n';
}
\end{cpp}

\item
遍历堆栈跟踪中的每个帧并检索有关信息:

\begin{cpp}
for (auto const& frame : std::stacktrace::current())
{
    std::cout << frame.source_file()
              << "("<< frame.source_line() << ")"
              << ": " << frame.description()
              << '\n';
}
\end{cpp}
\end{itemize}

\mySubsubsection{}{How it works...}

诊断工具位于一个单独的头文件 <stacktrace> 中。该头文件包含以下两个类:

\begin{itemize}
\item
\verb|std::basic_stacktrace|,这是一个表示堆栈跟踪条目序列容器的类模板。定义了一个类型别名 \verb|std::stacktrace|,即 \verb|std::basic_stacktrace<std::allocatorstd::stacktrace_entry>|。

\item
\verb|std::stacktrace_entry|,表示堆栈跟踪中的一个条目。
\end{itemize}

\begin{myTip}
讨论调用序列时,有两个术语需要正确理解:调用栈(call stack)和堆栈跟踪(stack trace)。调用栈是用于存储运行程序中活动帧(调用)信息的数据结构。堆栈跟踪是在某个时刻对调用栈的快照。
\end{myTip}

尽管 \verb|std::basic_stacktrace| 是一个容器,但并不是设计让用户实例化并填充堆栈条目的。没有用于添加或删除堆栈跟踪序列元素的成员函数;但有一些用于元素访问(如 at() 和 operator[])和检查大小(如 capacity()、size() 和 \verb|max_size()|)的成员函数。为了获取调用栈的快照,必须调用静态成员函数 current():

\begin{cpp}
std::stacktrace trace = std::stacktrace::current();
\end{cpp}

当前的堆栈跟踪可以通过多种方式打印:

\begin{itemize}
\item
使用重载的 \verb|operator<<| 输出到流:

\begin{cpp}
std::cout << std::stacktrace::current() << '\n';
\end{cpp}

\item
使用\verb|to_string()|成员函数转换为std::string:

\begin{cpp}
std::cout << std::to_string(std::stacktrace::current())
          << '\n';
\end{cpp}

\item
使用 \verb|to_string()| 成员函数输出到 \verb|std::string|:

\begin{cpp}
auto str = std::format("{}\n", std::stacktrace::current());
std::cout << str;
\end{cpp}
\end{itemize}

使用格式化函数如 \verb|std::format()|(不允许使用任何格式说明符):

\begin{cpp}
int plus_one(int n)
{
    std::cout << std::stacktrace::current() << '\n';
    return n + 1;
}
int double_n_plus_one(int n)
{
    return plus_one(2 * n);
}
int main()
{
    std::cout << double_n_plus_one(42) << '\n';
}
\end{cpp}

以下代码片段展示了如何将堆栈跟踪输出到标准输出:

\begin{shell}
0> [...]\main.cpp(24): chapter06!plus_one+0x4F
1> [...]\main.cpp(37): chapter06!double_n_plus_one+0xE
2> [...]\main.cpp(61): chapter06!main+0x5F
3> D:\a\_work\1\s\src\vctools\crt\vcstartup\src\startup\exe_common.inl(78): chapter06!invoke_main+0x33
4> D:\a\_work\1\s\src\vctools\crt\vcstartup\src\startup\exe_common.inl(288): chapter06!__scrt_common_main_seh+0x157
5> D:\a\_work\1\s\src\vctools\crt\vcstartup\src\startup\exe_common.inl(331): chapter06!__scrt_common_main+0xD
6> D:\a\_work\1\s\src\vctools\crt\vcstartup\src\startup\exe_main.cpp(17): chapter06!mainCRTStartup+0x8
7> KERNEL32+0x17D59
8> ntdll!RtlInitializeExceptionChain+0x6B
9> ntdll!RtlClearBits+0xBF
\end{shell}

运行此程序的结果会因编译器和目标系统不同而异,以下是一个输出示例:

\begin{shell}
[...]\main.cpp(24): chapter06!main+0x5F
-------------- --   -------------------
source         line description
\end{shell}

这些部分可以独立获取,使用 \verb|std::stacktrace_entry| 的成员函数 \verb|source_file()|、\verb|source_line()| 和 \verb|description()|。可以从堆栈跟踪容器中使用迭代器遍历堆栈跟踪条目序列,或者使用成员函数 at() 和 operator[] 访问。

