上一个示例中,介绍了如何使用标准库 chrono 处理时间段。同样,开发者也经常需要处理时间点。chrono 库提供了这样的组件,代表了一个时钟纪元以来的持续时间(根据时钟定义的时间起点)。本示例中,将介绍如何使用 chrono 库和时间点来测量函数的执行。

\mySubsubsection{}{Getting ready}

对于本示例中的例子,将使用以下函数,除了暂停当前线程一段时间外什么也不做:

\begin{cpp}
void func(int const interval = 1000)
{
    std::this_thread::sleep_for(std::chrono::milliseconds(interval));
}
\end{cpp}

\mySubsubsection{}{How to do it...}

要测量函数的执行,必须执行以下步骤:

\begin{enumerate}
\item
使用标准时钟获取当前时刻:

\begin{cpp}
auto start = std::chrono::high_resolution_clock::now();
\end{cpp}

\item
调用想要测量的函数:

\begin{cpp}
func();
\end{cpp}

\item
再次获取当前时刻;两次时刻之间的差值就是函数的执行时间:

\begin{cpp}
auto diff = std::chrono::high_resolution_clock::now() - start;
\end{cpp}

\item
将差值(以纳秒表示)转换为感兴趣的分辨率:

\begin{cpp}
std::cout
    << std::chrono::duration<double, std::milli>(diff).count()
    << "ms" << '\n';
std::cout
    << std::chrono::duration<double, std::nano>(diff).count()
    << "ns" << '\n';
\end{cpp}


\end{enumerate}

为了在一个可重用的组件中实现这种模式,请执行以下步骤:

\begin{enumerate}
\item
创建一个以分辨率和时钟为参数的类模板。

\item
创建一个静态变参函数模板,接受一个函数及其参数。

\item
实现上述模式,调用带有参数的函数。

\item
返回一个持续时间,而不是时钟周期数。
\end{enumerate}

这在以下代码中得到了体现:

\begin{cpp}
template <typename Time = std::chrono::microseconds,
          typename Clock = std::chrono::high_resolution_clock>
struct perf_timer
{
    template <typename F, typename... Args>
    static Time duration(F&& f, Args... args)
    {
        auto start = Clock::now();
        std::invoke(std::forward<F>(f), std::forward<Args>(args)...);
        auto end = Clock::now();
        return std::chrono::duration_cast<Time>(end - start);
    }
};
\end{cpp}

\mySubsubsection{}{How it works...}

时钟是一个定义了两件事的组件:

\begin{itemize}
\item
一个称为纪元的时间起点;关于纪元是什么没有限制,但典型的实现使用 1970 年 1 月 1 日。

\item
一个定义了两个时间点之间增量的滴答率(如毫秒或纳秒)。
\end{itemize}

时间点是指自某个时钟纪元以来的时间持续。有几个特别重要的时间点:

\begin{itemize}
\item
当前时间,由时钟的静态成员 now() 返回。

\item
纪元,即时间的起点;通过特定时钟的 \verb|time_point| 默认构造函数创建的时间点。

\item
时钟能够表示的最小时间,由 \verb|time_point| 的静态成员 min() 返回。

\item
时钟能够表示的最大时间,由 \verb|time_point| 的静态成员 max() 返回。
\end{itemize}

标准定义了几种时钟:

\begin{itemize}
\item
\verb|system_clock|: 当前系统的实时钟来表示时间点。

\item
\verb|high_resolution_clock|: 使用当前系统上最短可能滴答周期的时钟。

\item
\verb|steady_clock|: 永远不会调整的时钟。与其它时钟不同,随着时光流逝,两个时间点之间的差异始终为正。

\item
\verb|utc_clock|: 这是一个 C++20 中用于协调世界时的时钟。

\item
\verb|tai_clock|: 这是一个 C++20 中用于国际原子时的时钟。

\item
\verb|gps_clock|: 这是一个 C++20 中用于全球定位系统时间的时钟。

\item
\verb|file_clock|: 这是一个 C++20 中用于表示文件时间的时钟。
\end{itemize}

以下示例打印了列表中前三个时钟(C++11 中可用的那些)的精度,无论是否稳定(或单调):

\begin{cpp}
template <typename T>
void print_clock()
{
    std::cout << "precision: "
              << (1000000.0 * double(T::period::num)) /
                 (T::period::den)
              << '\n';
    std::cout << "steady: " << T::is_steady << '\n';
}
print_clock<std::chrono::system_clock>();
print_clock<std::chrono::high_resolution_clock>();
print_clock<std::chrono::steady_clock>();
\end{cpp}

可能的输出如下:

\begin{shell}
precision: 0.1
steady: 0
precision: 0.001
steady: 1
precision: 0.001
steady: 1
\end{shell}

\verb|system_clock| 的分辨率为 0.1 微秒且不是一个单调时钟。另一方面,另外两个时钟 \verb|high_resolution_clock| 和 \verb|steady_clock| 都有 1 纳秒的分辨率,并且是单调时钟。

时钟的稳定性在测量函数执行时间时非常重要,如果函数运行过程中进行时钟调整,结果将不会反映真实的执行时间,甚至可能出现负值。测量函数执行时间应依赖于稳定的时钟。通常的选择是 \verb|high_resolution_clock|,这也是在“How to do it...”部分示例中使用的时钟。

当测量执行时间时,需要在调用之前和调用之后获取当前时间。为此,可以使用时钟的 now() 静态方法。结果是一个 \verb|time_point|;当减去两个时间点时,结果是一个由时钟定义的持续时间。

为了创建一个可以用来测量任何函数执行时间的可重用组件,这里定义了一个名为 \verb|perf_timer| 的类型模板。这个模板以感兴趣的分辨率(默认为微秒)和想要使用的时钟(默认为 \verb|high_resolution_clock|)为参数。模板有一个名为 duration() 的单一静态成员——一个变参函数模板——接受要执行的函数及其可变数量的参数。实现相对简单:获取当前时间,使用 std::invoke 调用函数(以便处理调用任何可调用对象的不同机制),然后再次获取当前时间,返回值是一个时间段(具有定义的分辨率)。以下代码展示了这一点:

\begin{cpp}
auto t = perf_timer<>::duration(func, 1500);
std::cout << std::chrono::duration<double, std::milli>(t).count()
          << "ms" << '\n';
std::cout << std::chrono::duration<double, std::nano>(t).count()
          << "ns" << '\n';
\end{cpp}

需要注意的是,不是从 duration() 函数返回一个时钟周期数,而是返回一个实际时间段的值。原因是返回一个时钟周期数会导致失去分辨率,所以不知道它们实际上代表什么。最好是在确实需要时钟周期数的时候才调用 count()。这在此处举例说明:

\begin{cpp}
auto t1 = perf_timer<std::chrono::nanoseconds>::duration(func, 150);
auto t2 = perf_timer<std::chrono::microseconds>::duration(func, 150);
auto t3 = perf_timer<std::chrono::milliseconds>::duration(func, 150);
std::cout
    << std::chrono::duration<double, std::micro>(t1 + t2 + t3).count()
    << "us" << '\n';
\end{cpp}

这个例子中,使用三种不同的分辨率(纳秒、微秒和毫秒)测量了三个不同函数的执行时间。值 t1、t2 和 t3 表示持续时间,可以轻松地将其相加并将结果转换为微秒。




