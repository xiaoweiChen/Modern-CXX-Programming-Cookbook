
C++11 中提供的 chrono 库支持了时钟、时间点和持续时间的功能,但并不好用来表示时间和日期,特别是与日历和时区有关的时候。新的 C++20 标准通过扩展现有的 chrono 库来解决这些问题,加入了一些新特性:

\begin{itemize}
\item
更多的时钟类型,例如 UTC 时钟、国际原子时钟、GPS 时钟、文件时间时钟,以及一个本地时间的伪时钟。

\item
表示一天中时间的概念,将从午夜开始经过的时间分为小时、分钟和秒。

\item
日历,能够用年、月和日部分来表示日期。

\item
时区,能够相对于某个时区来表示时间点,并且能够在不同的时区之间转换时间。

\item
用于从流中解析 chrono 对象的输入输出支持。
\end{itemize}

\mySubsubsection{}{Getting ready}

新 chrono 功能都在 <chrono> 头文件中的 std::chrono 和 std::chrono\_literals 命名空间中可用。

\mySubsubsection{}{How to do it...}

可以使用 C++20 的 chrono 日历功能:

\begin{itemize}
\item
使用 \verb|year_month_day| 类型的实例表示公历日期,包括年、月和日。使用标准用户定义的字面值、常量和重载的 / 操作符来构造这样的对象:

\begin{cpp}
// format: year / month /day
year_month_day d1 = 2024y / 1 / 15;
year_month_day d2 = 2024y / January / 15;
// format: day / month / year
year_month_day d3 = 15d / 1 / 2024;
year_month_day d4 = 15d / January / 2024
// format: month / day / year
year_month_day d5 = 1 / 15d / 2024;
year_month_day d6 = January / 15 / 2024;
\end{cpp}

\item
使用 \verb|year_month_weekday| 类型的实例表示特定年份和月份的第 n 个星期几:

\begin{cpp}
// format: year / month / weekday
year_month_weekday d1 = 2024y / January / Monday[1];
// format: weekday / month / year
year_month_weekday d2 = Monday[1] / January / 2024;
// format: month / weekday / year
year_month_weekday d3 = January / Monday[1] / 2024;
\end{cpp}

\item
确定当前日期,以及计算其他日期,如明天和昨天的日期:

\begin{cpp}
auto today = floor<days>(std::chrono::system_clock::now());
auto tomorrow = today + days{ 1 };
auto yesterday = today - days{ 1 };
\end{cpp}

\item
确定特定年份和月份的第一天和最后一天:

\begin{cpp}
year_month_day today = floor<days>(std::chrono::system_clock::now());
year_month_day first_day_this_month = today.year() / today.month() / 1;
year_month_day last_day_this_month = today.year() / today.month() / last; // std::chrono::last
year_month_day last_day_feb_2024 = 2024y / February / last;
year_month_day_last ymdl {today.year(), month_day_last{ month{ 2 } }};
year_month_day last_day_feb { ymdl };
\end{cpp}

\item
计算两个日期之间的天数:

\begin{cpp}
inline int number_of_days(std::chrono::sys_days const& first,
                          std::chrono::sys_days const& last)
{
    return (last - first).count();
}
auto days = number_of_days(2024y / April / 1,
2024y / December / 25);
\end{cpp}

\item
检查一个日期是否有效:

\begin{cpp}
auto day = 2024y / January / 33;
auto is_valid = day.ok();
\end{cpp}

\item
使用 \verb|hh_mm_ss<Duration>| 类模板表示一天中的时间,以小时、分钟和秒为单位,其中 Duration 决定了用于分割时间间隔的精度。下面的例子中,std::chrono::seconds 定义了以秒为单位的分割精度:

\begin{cpp}
chrono::hh_mm_ss<chrono::seconds> td(13h+12min+11s);
std::cout << td << '\n';  // 13:12:11
\end{cpp}

\item
创建带有日期和时间部分的时间点:

\begin{cpp}
auto tp = chrono::sys_days{ 2024y / April / 1 } + 12h + 30min + 45s;
std::cout << tp << '\n';  // 2024-04-01 12:30:45
\end{cpp}

\item
确定具体时间并以不同精度表示:

\begin{cpp}
auto tp = std::chrono::system_clock::now();
auto dp = floor<days>(tp);
chrono::hh_mm_ss<chrono::milliseconds> time1 {
    chrono::duration_cast<chrono::milliseconds>(tp - dp) };
std::cout << time1 << '\n';  // 13:12:11.625
chrono::hh_mm_ss<chrono::minutes> time2 {
    chrono::duration_cast<chrono::minutes>(tp - dp) };
std::cout << time2 << '\n';  // 13:12
\end{cpp}

\end{itemize}

\mySubsubsection{}{How it works...}

例子中看到的 \verb|year_month_day| 和 \verb|year_month_weekday| 类型只是为了支持日历而添加到 chrono 库中的许多新类型中的一部分。以下表格列出了 std::chrono 命名空间中的所有这些类型及其代表的意义:

% Please add the following required packages to your document preamble:
% \usepackage{longtable}
% Note: It may be necessary to compile the document several times to get a multi-page table to line up properly
\begin{longtable}{|l|l|}
\hline
\textbf{类型}              & \textbf{表示}                           \\ \hline
\endfirsthead
%
\endhead
%
day                        & 一个月的一天                              \\ \hline
month                      & 一年的一个月                             \\ \hline
year                       & 公历年份              \\ \hline
weekday                    & 公历中的一周中的某一天   \\ \hline
weekday\_indexed & \begin{tabular}[c]{@{}l@{}}一个月中的第 n 个星期几,\\其中 n 在范围 {[}1, 5{]} 内\\(1 是该月的第一个星期几,\\5 是该月的第五个星期几\\——如果存在的话)\end{tabular} \\ \hline
weekday\_last              & 一个月的最后一个星期几                   \\ \hline
month\_day                 & 某个月的某一天             \\ \hline
month\_day\_last           & 某个月的最后一天              \\ \hline
month\_weekday             & 某个月的第 n 个星期几           \\ \hline
month\_weekday\_last       & 某个月的最后一个星期几          \\ \hline
year\_month                & 某年的某个月           \\ \hline
year\_month\_day           & 某年的某月的某一天               \\ \hline
year\_month\_day\_last     & 某年的某月的最后一天     \\ \hline
year\_month\_weekday       & 某年的某月的第 n 个星期几  \\ \hline
year\_month\_weekday\_last & 某年的某月的最后一个星期几  \\ \hline
\end{longtable}

\begin{center}
表 6.1: C++20 chrono 类型用于处理日期
\end{center}

本表中列出的所有类型都具有:

\begin{itemize}
\item
默认构造函数,它使成员字段未初始化

\item
成员函数,用于访问实体的部分

\item
成员函数 ok(),用于检查当前值是否有效

\item
非成员比较操作符,用于比较该类型值

\item
重载的 <{}< 操作符,用于将该类型的值输出到流中

\item
重载的函数模板 \verb|from_stream()|,根据提供的格式从流中解析值

\item
文本格式化库的 std::formatter<T, CharT> 类型模板的特化
\end{itemize}

此外,对于许多这些类型,/ 操作符都进行了重载,以便可以轻松地创建公历日期。创建一个日期(包含年、月和日)时,可以选择三种不同的格式:

\begin{itemize}
\item
年/月/日(在中国、日本、韩国、加拿大等国家使用,有时也与其他日/月/年格式一起使用)

\item
月/日/年(在美国使用)

\item
日/月/年(世界上大多数地方使用)
\end{itemize}

"日" 可以是:

\begin{itemize}
\item
实际的月份中的某一天(值从 1 到 31)

\item
std::chrono::last 表示该月的最后一天

\item
weekday[n],表示该月的第 n 个星期几(其中 n 的取值范围为 1 到 5)

\item
weekday[std::chrono::last],表示该月的最后一个星期几
\end{itemize}

为了区分表示日、月和年的整数,库提供了两个用户定义的字面值:""y 构造 std::chrono::year 类型的字面值,而 ""d 构造 std::chrono::day 类型的字面值。

此外,还有一些常量代表:

\begin{itemize}
\item
std::chrono::month,名称为 January、February 至 December。

\item
std::chrono::weekday,名称为 Sunday、Monday、Tuesday、Wednesday、Thursday、Friday 或 Saturday。
\end{itemize}

可以使用这些来构建 2025y/April/1、25d/December/2025 或 Sunday[last]/May/2025 这样的日期。

\verb|year_month_day| 类型提供了与 \verb|std::chrono::sys_days| 类型之间的隐式转换。\verb|sys_days| 类型是一个精度为一天(24 小时)的 \verb|std::chrono::time_point|。还有一个 \verb|std::chrono::sys_seconds| 的类型,精度为一秒的 \verb|time_point|。可以用 \verb|std::chrono::time_point_cast()| 或 \verb|std::chrono::floor()| 进行 \verb|time_point| 和 \verb|sys_days/sys_seconds| 之间的显式转换。

为了表示一天中的某一时刻,可以使用 \verb|std::chrono::hh_mm_ss| 类型。这个类表示从午夜开始经过的时间,分为小时、分钟、秒和亚秒。这种类型主要用于格式化工具。

还有几个用于在 12 小时制和 24 小时制之间转换的实用函数:

\begin{itemize}
\item
\verb|is_am()| 和 \verb|is_pm()|,检查以 std::chrono::hours 值提供的 24 小时格式的时间是否为上午(中午之前)或下午(午夜之前):

\begin{cpp}
std::cout << is_am(0h)  << '\n'; // true
std::cout << is_am(1h)  << '\n'; // true
std::cout << is_am(12h) << '\n'; // false
std::cout << is_pm(0h)  << '\n'; // false
std::cout << is_pm(12h) << '\n'; // true
std::cout << is_pm(23h) << '\n'; // true
std::cout << is_pm(24h) << '\n'; // false
\end{cpp}

\item
make12() 和 make24() 函数分别返回 24 小时制时间对应的 12 小时制时间,反之亦然。这两个函数都接受一个 std::chrono::hours 值作为输入时间,但 make24() 还有一个参数,即一个布尔值,表明时间是否为下午(P.M.):

\begin{cpp}
for (auto h : { 0h, 1h, 12h, 23h, 24h })
{
    std::cout << make12(h).count() << '\n';
    // prints 12, 1, 12, 11, 12
}
for (auto [h, pm] : {
    std::pair<hours, bool>{ 0h, false},
    std::pair<hours, bool>{ 1h, false},
    std::pair<hours, bool>{ 1h, true},
    std::pair<hours, bool>{12h, false},
    std::pair<hours, bool>{12h, true}, })
{
    std::cout << make24(h, pm).count() << '\n';
    // prints 0, 1, 13, 0, 12
}
\end{cpp}
\end{itemize}

从这些例子中可以看出,这四个函数只处理小时值,因为一天中某一时刻的小时部分决定了它是 12 小时制,还是 24 小时制;或者是上午(A.M.),还是下午(P.M.)时间。

\begin{myNotic}
本书第二版出版时,chrono 库的变化尚未完成。\verb|hh_mm_ss| 类型原本称为 \verb|time_of_day|,而 \verb|make12()/make24()| 函数是其成员函数。这一版本反映了这些变化,并采用了标准化的 API。
\end{myNotic}

\mySubsubsection{}{there's more}

这里描述的日期和时间设施都是基于 \verb|std::chrono::system_clock|。C++20 起,此时钟定义为测量 Unix 时间,即从 1970 年 1 月 1 日 UTC 时间 00:00:00 开始的时间,所以时区是 UTC。大多数情况下,可能更关心特定时区的本地时间。为此,chrono 库增加了对时区的支持,这将在下一个示例中进行介绍。
