
数据经常需要从一种类型转换为另一种类型。有些转换在编译时是必要的(如从双精度浮点数转为整型),有些则在运行时是必要的(如向上或向下转换指向继承体系中类的指针)。语言支持与C风格的类型转换兼容,既可以采用(type)expression的形式,也可以采用type(expression)的形式,但这种类型的转换会破坏C++的类型安全性。

语言还提供了几种转换:\verb|static_cast|、\verb|dynamic_cast|、\verb|const_cast| 和 \verb|reinterpret_cast|,用来更好地表明意图并编写更安全的代码。本示例中,将介绍这些转换如何使用。

\mySubsubsection{}{How to do it...}

使用以下转换执行类型转换:

\begin{itemize}
\item
使用 \verb|static_cast| 对非多态类型进行类型转换,包括整数到枚举的转换、从浮点数到整数的转换,或者从一个指针类型到另一个指针类型的转换,如从基类到派生类(向下转换)或从派生类到基类(向上转换),但不进行运行时检查:

\begin{cpp}
enum options {one = 1, two, three};
int value = 1;
options op = static_cast<options>(value);
int x = 42, y = 13;
double d = static_cast<double>(x) / y;
int n = static_cast<int>(d);
\end{cpp}

\item
使用 \verb|dynamic_cast| 对多态类型的指针或引用进行类型转换,从基类到派生类或反之亦然。这些检查是在运行时完成的,并且可能需要启用运行时类型信息(RTTI):

\begin{cpp}
struct base
{
    virtual void run() {}
    virtual ~base() {}
};
struct derived : public base
{
};
derived d;
base b;
base* pb = dynamic_cast<base*>(&d);         // OK
derived* pd = dynamic_cast<derived*>(&b);   // fail
try
{
    base& rb = dynamic_cast<base&>(d);       // OK
    derived& rd = dynamic_cast<derived&>(b); // fail
}
catch (std::bad_cast const & e)
{
    std::cout << e.what() << '\n';
}
\end{cpp}

\item
使用 \verb|const_cast| 进行不同const和volatile限定符的类型之间的转换,如从非const对象去除const限定符:

\begin{cpp}
void old_api(char* str, unsigned int size)
{
    // do something without changing the string
}
std::string str{"sample"};
old_api(const_cast<char*>(str.c_str()),
        static_cast<unsigned int>(str.size()));
\end{cpp}

\item
使用 \verb|reinterpret_cast| 进行位级重新解释,如整数与指针类型之间的转换、从指针类型到整数的转换,或者从一个指针类型到任何其他指针类型的转换,而不涉及运行时检查:

\begin{cpp}
class widget
{
public:
    typedef size_t data_type;
    void set_data(data_type d) { data = d; }
    data_type get_data() const { return data; }
private:
    data_type data;
};
widget w;
user_data* ud = new user_data();
// write
w.set_data(reinterpret_cast<widget::data_type>(ud));
// read
user_data* ud2 = reinterpret_cast<user_data*>(w.get_data());
\end{cpp}
\end{itemize}

\mySubsubsection{}{How it works...}

显式类型转换,有时也称为C风格的类型转换或静态转换,是C++与C语言兼容性的遗留特性。可以执行各种转换,包括但不限于:

\begin{itemize}
\item
算术类型之间

\item
指针类型之间

\item
整型与指针类型之间

\item
const或volatile限定和非限定类型之间
\end{itemize}

这种类型的转换对于多态类型或模板来说效果不佳,C++提供了前面示例中看到的四种转换。使用这些转换带来了几个重要的好处:

\begin{itemize}
\item
更好地表达了用户的意图,既对编译器也对阅读代码的人。

\item
允许在不同类型之间进行更安全的转换(除了 \verb|reinterpret_cast| 外)。

\item
可以在源代码中轻松搜索。
\end{itemize}

尽管名称可能暗示如此,但 \verb|static_cast| 并不是显式类型转换或静态转换的等价。此转换在编译时完成,可用于执行隐式转换、隐式转换的逆向转换,以及从继承层次结构中的类的指针到类型的转换。但不能用于触发不相关指针类型之间的转换,所以在以下示例中,使用 \verb|static_cast| 将 int* 转换为 double* 会产生编译器错误:

\begin{cpp}
int* pi = new int{ 42 };
double* pd = static_cast<double*>(pi);   // compiler error
\end{cpp}

然而,从 base* 转换为 derived*(其中 base 和 derived 是“How to do it...”部分中显示的类)不会产生编译器错误,但在尝试使用新获得的指针时会产生运行时错误:

\begin{cpp}
base b;
derived* pd = static_cast<derived*>(&b); // compilers OK, runtime error
base* pb1 = static_cast<base*>(pd);      // OK
\end{cpp}

另一方面,\verb|static_cast| 不能用于去除 const 和 volatile 限定符。下面的代码说明了这一点:

\begin{cpp}
int const c = 42;
int* pc = static_cast<int*>(&c);         // compiler error
\end{cpp}

沿继承层次结构向上、向下或横向安全地类型转换表达式可以使用 \verb|dynamic_cast| 完成。这种转换在运行时完成并且要求启用 RTTI,所以会带来运行时开销。动态转换只能用于指针和引用。

当使用 \verb|dynamic_cast| 将表达式转换为指针类型且操作失败时,结果是指针为空;将表达式转换为引用类型且操作失败时,则抛出 \verb|std::bad_cast| 异常。因此,总是应该将 \verb|dynamic_cast| 转换为引用类型放在 try...catch 块中。

\begin{myNotic}
RTTI 是一种机制,运行时公开有关对象数据类型的信息。这仅适用于多态类型(具有至少一个虚拟方法的类型,包括虚拟析构函数,所有基类都应该有)。RTTI 通常是可选的编译器特性(或根本不支持),所以使用此功能可能需要使用编译器开关。
\end{myNotic}

尽管动态转换是在运行时进行的,如果试图在非多态类型之间进行转换,会得到一个编译器错误:

\begin{cpp}
struct struct1 {};
struct struct2 {};
struct1 s1;
struct2* ps2 = dynamic_cast<struct2*>(&s1); // compiler error
\end{cpp}

\verb|reinterpret_cast| 更像是一个编译器指令,不会转换为CPU指令;只是指示编译器将表达式的二进制表示解释为另一种指定的类型。这是一个类型不安全的转换,应谨慎使用。它可以用于整型类型和指针、指针类型和函数指针类型之间的转换,不做任何检查,所以 \verb|reinterpret_cast| 可以成功用于不相关类型之间的转换,如从 int* 到 double*,这会产生未定义行为:

\begin{cpp}
int* pi = new int{ 42 };
double* pd = reinterpret_cast<double*>(pi);
\end{cpp}

使用 \verb|reinterpret_cast| 的典型情况是在使用操作系统或供应商特定API的代码中转换类型之间的表达式。许多API以指针或整型的形式存储用户数据。如果需要将用户定义类型的地址传递给这样的API,就需要将不相关的指针类型值或指针类型值转换为整型类型值。在前一部分提供的类似例子中,widget 是一个类,在数据成员中存储用户定义的数据,并提供了访问它的方法:\verb|set_data()| 和 \verb|get_data()|。如果需要在 widget 中存储一个对象的指针,则使用 \verb|reinterpret_cast|,如本例所示。

\verb|const_cast| 类似于 \verb|reinterpret_cast|,它是一个编译器指令,并不转换为CPU指令。用于去除 const 或 volatile 限定符,这是其他三种转换所不能做的操作。

\begin{myTip}
\verb|const_cast| 应仅用于去除非 const 或非 volatile 对象的 const 或 volatile 限定符。其他情况下会导致未定义行为:

\begin{cpp}
int const a = 42;
int const * p = &a;
int* q = const_cast<int*>(p);
*q = 0; // undefined behavior
\end{cpp}

变量 p 指向了一个声明为常量的对象(变量 a)。通过去除 const 限定符,尝试修改指向的对象引入了未定义行为。
\end{myTip}

\mySubsubsection{}{There's more...}

当使用形如 (type)expression 的显式类型转换时,会选择满足特定转换要求的第一个选项:

\begin{enumerate}
\item
\verb|const_cast<type>(expression)|

\item
\verb|static_cast<type>(expression)|

\item
\verb|static_cast<type>(expression)| + \verb|const_cast<type>(expression)|

\item
\verb|reinterpret_cast<type>(expression)|

\item
\verb|reinterpret_cast<type>(expression)| + \verb|const_cast<type>(expression)|
\end{enumerate}

此外,与特定的 C++ 转换不同,静态转换可以用于不完整类类型之间的转换。如果类型和表达式都是指向不完整类型的指针,那么选择 \verb|static_cast| 或 \verb|reinterpret_cast|都可以。



