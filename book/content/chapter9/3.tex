

尽管没有正式定义,常量正确性意味着那些不应该修改(即不可变对象)保持不变。作为开发人员,可以通过使用const关键字来声明参数、变量和成员函数来强制实现这一点。本示例中,将介绍常量正确性的优点,以及如何实现。

\mySubsubsection{}{How to do it...}

为了确保程序的常量正确性,应该始终将以下内容声明为常量:

\begin{itemize}
\item
不应在函数内部修改函数参数:

\begin{cpp}
struct session {};
session connect(std::string const & uri,
int const timeout = 2000)
{
    /* do something */
    return session { /* ... */ };
}
\end{cpp}

\item
不会改变类型的数据成员:

\begin{cpp}
class user_settings
{
    public:
    int const min_update_interval = 15;
    /* other members */
};
\end{cpp}

\item
从外部看不会修改状态的成员函数:

\begin{cpp}
class user_settings
{
    bool show_online;
public:
    bool can_show_online() const {return show_online;}
    /* other members */
};
\end{cpp}

\item
生命周期内值不会改变的函数局部变量:

\begin{cpp}
user_settings get_user_settings()
{
    return user_settings {};
}
void update()
{
    user_settings const us = get_user_settings();
    if(us.can_show_online()) { /* do something */ }
    /* do more */
}
\end{cpp}

\item
应该绑定到临时实例(右值)以延长临时实例生命周期至(常量)引用生命周期的引用:

\begin{cpp}
std::string greetings()
{
    return "Hello, World!";
}
const std::string & s = greetings(); // must use const
std::cout << s << std::endl;
\end{cpp}
\end{itemize}

\mySubsubsection{}{How it works...}

将实例和成员函数声明为常量有几个好处:

\begin{itemize}
\item
避免实例的意外和故意更改,在某些情况下,这可能导致程序的错误行为。

\item
使编译器能够进行更好的优化。

\item
为其他用户提供代码语义文档。
\end{itemize}

\begin{myNotic}
常量正确性不仅是一种个人风格,而是应该指导C++开发的核心原则。
\end{myNotic}

不幸的是,常量正确性的重要性在过去以及现在都没有得到足够的重视,无论是在书籍、C++社区还是工作环境中。一个经验法则是,所有不应该改变的东西都应该声明为常量。这应该始终这样做,而不仅仅是在开发后期,需要清理和重构代码时才去做。

将参数或变量声明为常量时,可以在类型前面放const关键字(const T c),也可以放在类型后面(T const c)。两种方式等价,无论选择哪种风格,声明的阅读方向都应该是从右向左。const T c 应读作 c 是一个常量 T,而 T const c 则读作 c 是一个 T 类型的常量。对于指针来说,这一点稍微复杂一些。下表展示了各种指针声明及其含义:

% Please add the following required packages to your document preamble:
% \usepackage{longtable}
% Note: It may be necessary to compile the document several times to get a multi-page table to line up properly
\begin{longtable}{|l|l|}
\hline
\textbf{Expression} & \textbf{Description}                                                       \\ \hline
\endfirsthead
%
\endhead
%
T* p                & p 是指向非常量 T 的非常量指针。                           \\ \hline
const T* p          & p 是指向常量 T 的非常量指针。                       \\ \hline
T const * p         & p 是指向常量 T 的非常量指针(同上)。     \\ \hline
const T * const p   & p 是指向常量 T 的常量指针。                           \\ \hline
T const * const p  & p 是指向常量 T 的常量指针(同上)。        \\ \hline
T** p               & p 是指向非常量 T 的非常量指针的非常量指针。 \\ \hline
const T** p         & p 是指向常量 T 的非常量指针的非常量指针。     \\ \hline
T const ** p        & 同 const T** p。                                                       \\ \hline
const T* const * p & p 是指向常量 T 的常量指针的非常量指针。 \\ \hline
T const * const * p & 同 const T* const * p。                                                \\ \hline
\end{longtable}

\begin{center}
表 9.1: 指针声明示例及其含义
\end{center}

\begin{myNotic}
将const关键字放在类型后面更自然,其语法解释的方向——从右向左——保持一致。本书中的所有示例都采用了这种风格。
\end{myNotic}

对于引用来说,情况类似:\verb|const T & c| 和 \verb|T const & c| 等价,所以 c 是一个指向常量 T 的引用。然而,\verb|T const & const c|,表示 c 是一个指向常量 T 的常量引用。没有意义,因为引用——变量的别名——在本质上是隐式常量,即不能修改为指向另一个变量的别名。

指向非常量实例的非常量指针,即 T*,可以隐式转换为指向常量实例的非常量指针,即 T const *。 而,T** 不能隐式转换为 T const** (与 const T** 相同)。可通过指向非常量实例的指针,修改常量对象:

\begin{cpp}
int const c = 42;
int* x;
int const ** p = &x; // this is an actual error
*p = &c;
*x = 0;              // this modifies c
\end{cpp}

如果一个实例是常量,则只能调用其类型的常量成员函数。然而,将成员函数声明为常量并表示该函数只能在常量实例上调用,还表明该函数不会从外部修改实例的状态,而这点通常会被误解。一个类型具有内部状态,可以通过公共接口将其暴露给外部。

然而,并不是所有的内部状态都会暴露出来,而且从公共接口可见的内容可能在内部状态中没有直接的表现。(如果建模订单行,并在内部表示中有项目数量和项目销售价格字段,则可以有一个公开的方法通过乘以数量和价格来暴露订单行金额。)因此,从实例公共接口来看,状态是一个逻辑状态。将方法定义为常量是一种声明,确保函数不会改变逻辑状态。然而,编译器会阻止使用这样的方法修改数据成员。为了避免这个问题,应该将常量方法修改的数据成员声明为mutable。

以下示例中,computation 是一个具有 compute() 方法的类型,该方法执行长时间运行的计算操作。因为这个函数不影响对象的逻辑状态,所以声明为常量。然而,为了避免重新计算相同输入的结果,计算后的值存储在一个缓存中。为了能够在常量函数中修改缓存,声明为mutable:

\begin{cpp}
class computation
{
    double compute_value(double const input) const
    {
        /* long running operation */
        return input + 42;
    }
    mutable std::map<double, double> cache;
public:
    double compute(double const input) const
    {
        auto it = cache.find(input);
        if(it != cache.end()) return it->second;
        auto result = compute_value(input);
        cache[input] = result;
        return result;
    }
};
\end{cpp}

类似的场景由以下类型表示,该类实现了线程安全的容器。对共享内部数据的访问由互斥锁保护。该类提供了诸如添加和删除值的方法,以及 contains() 这样的方法,用于指示容器中是否存在某个项。成员函数不打算修改对象的逻辑状态,所以声明为常量。然而,对共享内部状态的访问必须由互斥锁保护。为了锁定和解锁互斥锁,这两个操作(修改对象状态)和互斥锁本身都必须声明为mutable:

\begin{cpp}
template <typename T>
class container
{
    std::vector<T>     data;
    mutable std::mutex mt;
public:
    void add(T const & value)
    {
        std::lock_guard<std::mutex> lock(mt);
        data.push_back(value);
    }
    bool contains(T const & value) const
    {
        std::lock_guard<std::mutex> lock(mt);
        return std::find(std::begin(data), std::end(data), value)
               != std::end(data);
    }
};
\end{cpp}

mutable 修饰符允许修改类型成员,即使包含实例声明为 const。就是 std::mutex 类型的 mt 成员的情况,即使在声明为 const 的 contains() 方法中也会进行修改。

有时,一个方法或操作符会进行重载,同时具有常量和非常量版本。这通常发生在下标操作符或提供对内部状态直接访问的方法上。这样做的原因是,该方法应该既适用于常量对象,也适用于非常量对象。不过,行为应该不同:对于非常量实例,该方法应允许外部修改提供的数据;而对于常量对象,则不允许。因此,非常量下标操作符返回一个非常量实例的引用,而常量下标操作符则返回一个常量实例的引用:

\begin{cpp}
class contact {};
class addressbook
{
    std::vector<contact> contacts;
public:
    contact& operator[](size_t const index);
    contact const & operator[](size_t const index) const;
};
\end{cpp}

\begin{myNotic}
即使一个对象是常量的,如果一个成员函数是常量的,这个成员函数返回的数据也可能不是常量。
\end{myNotic}

使用const的一个重要用例是定义对临时实例的引用。临时实例是一个右值(rvalue),而非常量左值引用(non-const lvalue reference)不能绑定到一个右值。但通过使左值引用成为const可以做到,这会产生延长临时实例生命周期至常量引用生命周期的效果。但这只适用于基于栈的引用,而不适用于作为成员的引用。

\mySubsubsection{}{There's more...}

实例的const限定符可以通过\verb|const_cast|转换移除,但这只应在了解该实例最初并不是声明为常量的情况下使用。


