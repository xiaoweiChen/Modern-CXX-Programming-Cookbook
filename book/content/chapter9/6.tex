
在前两个示例中,了解了constexpr函数,允许在所有输入都在编译时可用的情况下,在编译时函数,以及立即函数(C++20),这些函数保证总是编译时执行(否则,会产生错误)。constexpr函数的一个重要方面是常量表达式上下文;这些是在其中所有表达式和函数都在编译时执行的代码路径。常量计算上下文对于更有效地优化代码非常有用。另一方面,从constexpr函数中调用立即函数仅在C++23中成为可能。本示例中,将介绍如何利用常量表达式上下文。


\mySubsubsection{}{How to do it...}

为了判断函数上下文是否为常量表达式,以便提供编译时实现,可以使用以下方法:

\begin{itemize}
\item
C++20可以使用标准库中的\verb|std::is_constant_evaluated()|函数,该函数位于\verb|<type_traits>|头文件中,并与常规的if语句结合使用:

\begin{cpp}
constexpr double power(double base, int exponent)
{
    if(std::is_constant_evaluated())
    {
        double result = 1.0;
        if (exponent == 0)
        {
            return result;
        }
        else if (exponent > 0) {
            for (int i = 0; i < exponent; i++) {
                result *= base;
            }
        }
        else {
            exponent = -exponent;
            for (int i = 0; i < exponent; i++) {
                result *= base;
            }
            result = 1.0 / result;
        }
        return result;
    }
    else
    {
        return std::pow(base, exponent);
    }
}
int main()
{
    constexpr double a = power(10, 5); // compile-time eval
    std::cout << a << '\n';
    double b = power(10, 5);           // runtime eval
    std::cout << b << '\n';
}
\end{cpp}

\item
C++23可以使用if consteval语句,这是\verb|if(std::is_constant_evaluated()|语句的简化版:

\begin{cpp}
constexpr double power(double base, int exponent)
{
    if consteval
    {
        double result = 1.0;
        if (exponent == 0)
        {
            return result;
        }
        else if (exponent > 0) {
            for (int i = 0; i < exponent; i++) {
                result *= base;
            }
        }
        else {
            exponent = -exponent;
            for (int i = 0; i < exponent; i++) {
                result *= base;
            }
            result = 1.0 / result;
        }
        return result;
    }
    else
    {
        return std::pow(base, exponent);
    }
}
\end{cpp}
\end{itemize}

\mySubsubsection{}{How it works...}

C++20标准提供了一个名为\verb|std::is_constant_evaluated()|的标准库函数(位于\verb|<type_traits>|头文件中),可以检测调用是否发生在constexpr函数内的常量表达式上下文中。

此函数通常与常规if语句一起使用,如上一节提供的例子中计算数字的幂值。该实现的关键点包括:

\begin{itemize}
\item
在常量表达式上下文中,使用可以在编译时由编译器执行的算法来优化代码。

\item
非常量表达式上下文(即运行时)中,调用std::pow()函数来计算幂。
\end{itemize}

然而,这个函数和常量表达式上下文有一些“陷阱”:

\begin{itemize}
\item
因为函数的参数在编译时已知,并不意味着上下文为常量表达式。在下面的代码中,第一次调用constexpr函数power()是在常量表达式上下文中。但第二次不是,即使所有参数在编译时已知且函数声明为constexpr:

\begin{cpp}
constexpr double a = power(10, 5); // [1] compile-time eval
double b = power(10, 5);           // [2] runtime eval
\end{cpp}

\item
如果与constexpr if语句一起使用,\verb|std::is_constant_evaluated()|函数总是为true(像GCC和Clang这样的编译器会为此细微错误提供警告):

\begin{cpp}
constexpr double power(double base, int exponent)
{
    if constexpr (std::is_constant_evaluated())
    {
    }
}
\end{cpp}
\end{itemize}

这是一个报告的错误示例:

\begin{shell}
prog.cc: In function 'constexpr double power(double, int)':
prog.cc:10:45: warning: 'std::is_constant_evaluated' always evaluates to true in 'if constexpr' [-Wtautological-compare]
   10 |     if constexpr (std::is_constant_evaluated())
      |                   ~~~~~~~~~~~~~~~~~~~~~~~~~~^~
\end{shell}

C++23标准提供了比\verb|std::is_constant_evaluated()|函数更好的替代方案,即if consteval语句。这有几个优点:

\begin{itemize}
\item
不需要包含头文件

\item
避免了关于使用哪种形式的if语句的混淆

\item
允许在常量表达式上下文中调用立即函数
\end{itemize}

C++23的power函数实现改变为:

\begin{cpp}
constexpr double power(double base, int exponent)
{
    if consteval
    {
        /* ... */
    }
    else
    {
        return std::pow(base, exponent);
    }
}
\end{cpp}

if consteval语句始终需要花括号。否定形式也是可能的,可以使用!或者not关键字。下面的代码中,每对语句都是等价的:

\begin{cpp}
if !consteval {/*statement*/}          // [1] equivalent to [2]
if consteval {} else {/*statement*/}   // [2]
if not consteval {/*statement1*/}      // [3] equivalent to [4]
else {/*statement2*/}
if consteval {/*statement2*/}          // [4]
else {/*statement1*/}
\end{cpp}

if consteval语句还允许在constexpr函数的常量表达式上下文中调用立即函数。来看一个C++20的例子:

\begin{cpp}
consteval int plus_one(int const i)
{
    return i + 1;
}
consteval int plus_two(int i)
{
    return plus_one(i) + 1;
}
constexpr int plus_two_alt(int const i)
{
    if (std::is_constant_evaluated())
    {
        return plus_one(i) + 1;
    }
    else
    {
        return i + 2;
    }
}
\end{cpp}

\verb|plus_one()|函数是一个立即函数,可以从同样是立即函数的\verb|plus_two()|函数中调用。不可能从\verb|plus_two_alt()|函数中调用,constexpr函数并且调用\verb|plus_one()|函数是常量的,因为这不是一个常量表达式。

这个问题通过C++23中的if consteval语句得到了解决。这使得在常量表达式上下文中调用立即函数成为可能:

\begin{cpp}
constexpr int plus_two_alt(int const i)
{
    if consteval
    {
        return plus_one(i) + 1;
    }
    else
    {
        return i + 2;
    }
}
\end{cpp}

随着if consteval语句的可用性,\verb|std::is_constant_evaluated()|函数变得过时。实际上,可以使用if consteval语句来实现:

\begin{cpp}
constexpr bool is_constant_evaluated() noexcept
{
    if consteval {
        return true;
    } else {
        return false;
    }
}
\end{cpp}

\begin{myNotic}
当使用C++23编译器时,应该总是优先选择if consteval语句,而非过时的\verb|std::is_constant_evaluated()|。
\end{myNotic}


