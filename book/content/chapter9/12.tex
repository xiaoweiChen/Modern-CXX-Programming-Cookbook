
C++ 语言定义了六个执行比较的关系操作符:==, !=, <, <=, >, 和 >=。尽管 != 可以基于 == 实现,<=, >= 和 > 可以基于 < 实现,但如果想让用户定义类型支持等值比较,仍需实现 == 和 != 操作符;如果想支持排序,则还需实现 <, <=, > 和 >= 操作符。

如果希望类型——称之为 T——可比较,则需要实现 6 个函数;如果希望它们与另一种类型 U 可比较,则需要实现 12 个函数;如果还希望 U 类型的值与你的 T 类型可比较,则需要实现 18 个函数,依此类推。新的 C++20 标准通过引入一个新的比较操作符,即三向比较操作符,减少了这个数量到一两个,或者根据与其他类型的比较而定的多个。这个新操作符用符号 <=> 表示,因此通常称为宇宙飞船操作符。这个新操作符可编写更少的代码,更好地描述关系的强度,避免手动实现基于其他比较操作符所带来的性能问题。

\mySubsubsection{}{Getting ready}

定义或实现三向比较操作符时,有必要包含头文件 <compare>。这个新的 C++20 头文件是标准通用工具库的一部分,提供了实现比较所需的类、函数和概念。

\mySubsubsection{}{How to do it...}

为了在 C++20 中最优地实现比较,按以下步骤操作:

\begin{itemize}
\item
如果希望类型支持等值比较(即 == 和 !=),只需实现 == 操作符并返回布尔值。可以默认实现,以便编译器执行成员级比较:

\begin{cpp}
class foo
{
    int value;
public:
    foo(int const v):value(v){}
    bool operator==(foo const&) const = default;
};
\end{cpp}

\item
如果希望类型同时支持等值比较和排序,并且默认的成员级比较适用,只需定义 <=> 操作符,返回 auto,并默认其实现:

\begin{cpp}
class foo
{
    int value;
public:
    foo(int const v) :value(v) {}
    auto operator<=>(foo const&) const = default;
};
\end{cpp}

\item
如果希望类型同时支持等值比较和排序并且需要执行自定义比较,那么实现 == 操作符(用于等值比较)和 <=> 操作符(用于排序):

\begin{cpp}
class foo
{
    int value;
public:
    foo(int const v) :value(v) {}
    bool operator==(foo const& other) const
    { return value == other.value; }
    auto operator<=>(foo const& other) const
    { return value <=> other.value; }
};
\end{cpp}
\end{itemize}

实现三向比较操作符时,请遵循以下指导原则:

\begin{itemize}
\item
只实现三向比较操作符,但在比较值时总是使用 <, <=, > 和 >= 这些二向比较操作符。

\item
即使你想让比较的第一个操作数不是类型,也要将三向比较操作符实现为成员函数。

\item
只有希望在两个参数上进行隐式转换(比较两个都不是类型实例)时,才将三向比较操作符实现为非成员函数。
\end{itemize}

\mySubsubsection{}{How it works...}

新的三向比较操作符类似于 C 语言中的 memcmp() 和 strcmp() 函数,以及 std::string::compare() 方法。这些函数接受两个参数并返回一个整数值,如果第一个小于第二个,则该值小于零;如果相等,则为零;如果第一个大于第二个,则大于零。三向比较操作符不返回整数,而是返回比较类别类型的一个值。

这可能是以下之一:

\begin{itemize}
\item
\verb|std::strong_ordering| 表示三向比较的结果支持所有六个关系操作符,不允许不可比较的值(意味着 a < b, a == b 和 a > b 至少有一个为真),并且暗示可替代性。这是一种属性,即如果 a == b 且 f 是一个只读取比较相关状态(可通过参数的公共常量成员访问)的函数,则 f(a) == f(b)。

\item
\verb|std::weak_ordering| 支持所有六个关系操作符,不允许不可比较的值(意味着 a < b, a == b, 和 a > b 都不能为真),但也不暗示可替代性。弱排序的一个典型例子是不区分大小写的字符串类型。

\item
\verb|std::partial_ordering| 支持所有六个关系操作符,但不暗示可替代性,并且可能具有不可比较的值(例如,浮点 NaN 无法与其他值进行比较)。
\end{itemize}

\verb|std::strong_ordering| 类型是所有这些类别类型中最强大的。不能从任何其他类别隐式转换而来,但它可以隐式转换为 \verb|std::weak_ordering| 和 \verb|std::partial_ordering|。\verb|std::weak_ordering| 也可以隐式转换为 \verb|std::partial_ordering|。这些属性总结在下表中:

% Please add the following required packages to your document preamble:
% \usepackage{longtable}
% Note: It may be necessary to compile the document several times to get a multi-page table to line up properly
\begin{longtable}{|l|l|l|l|l|}
\hline
\textbf{类别} & \textbf{操作符} & \textbf{可替代性} & \textbf{可比较值} & \textbf{隐式转换} \\ \hline
\endfirsthead
%
\endhead
%
std::strong\_ordering  & ==, !=, \textless{}, \textless{}=, \textgreater{}, \textgreater{}= & 是 & 是 & $\downarrow$ \\ \hline
std::weak\_ordering    & ==, !=, \textless{}, \textless{}=, \textgreater{}, \textgreater{}= & 否  & 是 & $\downarrow$ \\ \hline
std::partial\_ordering & ==, !=, \textless{}, \textless{}=, \textgreater{}, \textgreater{}= & 否  & 否  &              \\ \hline
\end{longtable}

\begin{center}
表9.2:类型的属性
\end{center}

比较类别具有可以隐式与字面值零比较的值(但不能与零整数变量比较),其值列于下表:

% Please add the following required packages to your document preamble:
% \usepackage{multirow}
% \usepackage{longtable}
% Note: It may be necessary to compile the document several times to get a multi-page table to line up properly
\begin{longtable}{|l|lll|l|}
\hline
\multirow{2}{*}{\textbf{类别}} &
\multicolumn{3}{l|}{\textbf{数值}} &
\multirow{2}{*}{\textbf{非数值}} \\ \cline{2-4}
& \multicolumn{1}{l|}{-1}   & \multicolumn{1}{l|}{0}          & 1       &           \\ \hline
\endfirsthead
%
\endhead
%
strong\_ordering &
\multicolumn{1}{l|}{小于} &
\multicolumn{1}{l|}{\begin{tabular}[c]{@{}l@{}}相等\end{tabular}} &
大于 &
\\ \hline
weak\_ordering    & \multicolumn{1}{l|}{小于} & \multicolumn{1}{l|}{相等} & 大于 &           \\ \hline
partial\_ordering & \multicolumn{1}{l|}{小于} & \multicolumn{1}{l|}{相等} & 大于 & 无序 \\ \hline
\end{longtable}

\begin{center}
表9.3:与0隐式可比较的比较类别的值
\end{center}

为了更好地理解,来看一个例子:

\begin{cpp}
class cost_unit_t
{
    // data members
    public:
    std::strong_ordering operator<=>(cost_unit_t const & other) const noexcept = default;
};
class project_t : public cost_unit_t
{
    int         id;
    int         type;
    std::string name;
public:
    bool operator==(project_t const& other) const noexcept
    {
        return (cost_unit_t&)(*this) == (cost_unit_t&)other &&
        name == other.name &&
        type == other.type &&
        id == other.id;
    }
    std::strong_ordering operator<=>(project_t const & other) const noexcept
    {
        // compare the base class members
        if (auto cmp = (cost_unit_t&)(*this) <=> (cost_unit_t&)other;
        cmp != 0)
            return cmp;
        // compare this class members in custom order
        if (auto cmp =  name.compare(other.name); cmp != 0)
            return cmp < 0 ? std::strong_ordering::less :
                std::strong_ordering::greater;
        if (auto cmp = type <=> other.type; cmp != 0)
            return cmp;
        return id <=> other.id;
    }
};
\end{cpp}

\verb|cost_unit_t| 是一个基类,包含某些(未指定的)数据成员,并定义了 <=> 操作符,是编译器默认实现的。编译器不仅会提供 <, <=, > 和 >= 操作符,还会提供 == 和 != 操作符。\verb|project_t| 派生自 \verb|cost_unit_t|,包含几个数据字段:项目的标识符、类型和名称。但对于这种类型,不能默认实现操作符,因为不想按成员顺序比较,而是按照自定义顺序:首先是名称,然后是类型,最后是标识符。这种情况下,实现了返回布尔值的 == 操作符,用于测试成员字段的等值性,以及返回 \verb|std::strong_ordering| 的 <=> 操作符,该操作符使用自身来比较其两个参数的值。

接下来的代码展示了一个 \verb|employee_t| 的类型,该类型模拟公司中的员工。一个员工可以有经理,一个担任经理的员工可以管理其他人:

\begin{cpp}
struct employee_t
{
    bool is_managed_by(employee_t const&) const { /* ... */ }
    bool is_manager_of(employee_t const&) const { /* ... */ }
    bool is_same(employee_t const&) const { /* ... */ }
    bool operator==(employee_t const & other) const
    {
        return is_same(other);
    }
    std::partial_ordering operator<=>(employee_t const& other) const noexcept
    {
        if (is_same(other))
            return std::partial_ordering::equivalent;
        if (is_managed_by(other))
            return std::partial_ordering::less;
        if (is_manager_of(other))
            return std::partial_ordering::greater;
        return std::partial_ordering::unordered;
    }
};
\end{cpp}

\verb|is_same()|, \verb|is_manager_of()| 和 \verb|is_managed_by()| 返回两个员工之间的关系。然而,可能存在没有关系的员工;例如,不同团队的员工,或者没有上下级结构的团队。这里,可以实现等值性和排序。但是,不能相互比较所有员工,所以 <=> 操作符必须返回一个 \verb|std::partial_ordering| 值。如果值代表相同的员工,则返回值为 \verb|partial_ordering::equivalent|;如果当前员工受提供的员工管理,则返回 \verb|partial_ordering::less|;如果当前员工是提供的员工的管理者,则返回 \verb|partial_ordering::greater|;而其他情况,返回 \verb|partial_ordering::unordered|。

再看一个例子,以了解三向比较操作符的工作方式。下面的示例中,ipv4 类建模了一个 IPv4 地址。支持与其他 ipv4 类型的对象以及无符号长整数值(因为有一个 \verb|to_ulong()| 方法可以将 IP 地址转换成一个 32 位无符号整数值)进行比较:

\begin{cpp}
struct ipv4
{
    explicit ipv4(unsigned char const a=0, unsigned char const b=0,
    unsigned char const c=0, unsigned char const d=0) noexcept :
        data{ a,b,c,d }
    {}
    unsigned long to_ulong() const noexcept
    {
        return
        (static_cast<unsigned long>(data[0]) << 24) |
        (static_cast<unsigned long>(data[1]) << 16) |
        (static_cast<unsigned long>(data[2]) << 8) |
        static_cast<unsigned long>(data[3]);
    }
    auto operator<=>(ipv4 const&) const noexcept = default;
    bool operator==(unsigned long const other) const noexcept
    {
        return to_ulong() == other;
    }
    std::strong_ordering
    operator<=>(unsigned long const other) const noexcept
    {
        return to_ulong() <=> other;
    }
private:
    std::array<unsigned char, 4> data;
};
\end{cpp}

这里重载了 <=> 操作符并允许其默认实现,但还显式地实现了 == 操作符和 <=> 操作符的重载版本,这些版本比较 ipv4 对象与无符号长整数值。由于这些操作符的存在,可以创建以下的表达式:

\begin{cpp}
ipv4 ip(127, 0, 0, 1);
if(ip == 0x7F000001) {}
if(ip != 0x7F000001) {}
if(0x7F000001 == ip) {}
if(0x7F000001 != ip) {}
if(ip < 0x7F000001)  {}
if(0x7F000001 < ip)  {}
\end{cpp}

这里需要注意两点:一是只重载了 == 操作符,但也可以使用 != 操作符;二是重载了 == 操作符和 <=> 操作符来比较 ipv4 值与无符号长整数值,也可以比较无符号长整数值与 ipv4 值。因为编译器执行对称的重载解析,对于形式为 a@b 的表达式,其中 @ 是一个双向关系操作符,编译器会对 a@b, a<=>b, 和 b<=>a 进行名称查找。下表显示了关系操作符的所有可能变换:


% Please add the following required packages to your document preamble:
% \usepackage{longtable}
% Note: It may be necessary to compile the document several times to get a multi-page table to line up properly
\begin{longtable}{|l|l|l|}
\hline
a == b & b == a    &           \\ \hline
\endfirsthead
%
\endhead
%
a != b & !(a == b) & !(b == a) \\ \hline
a \textless{}=\textgreater b & 0 \textless{}=\textgreater (b \textless{}=\textgreater a) &                                                  \\ \hline
a \textless b                & (a \textless{}=\textgreater b) \textless 0                & 0 \textgreater (b \textless{}=\textgreater a)    \\ \hline
a \textless{}= b             & (a \textless{}=\textgreater b) \textless{}= 0             & 0 \textgreater{}= (b \textless{}=\textgreater a) \\ \hline
a \textgreater b             & (a \textless{}=\textgreater b) \textgreater 0             & 0 \textless (b \textless{}=\textgreater a)       \\ \hline
a \textgreater{}= b          & (a \textless{}=\textgreater b) \textgreater{}= 0          & 0 \textless{}= (b \textless{}=\textgreater a)    \\ \hline
\end{longtable}

\begin{center}
表9.4:关系操作符可能的转换
\end{center}

这减少了为了支持不同形式的比较,必须明确提供的重载数量。三向比较操作符可以作为成员函数或非成员函数来实现,通常应该优先选择成员实现。

只有在两个参数上都进行隐式转换时,才应该使用非成员形式。下面是一个例子:

\begin{cpp}
struct A { int i; };
struct B
{
    B(A a) : i(a.i) { }
    int i;
};
inline auto
operator<=>(B const& lhs, B const& rhs) noexcept
{
    return lhs.i <=> rhs.i;
}
assert(A{ 2 } > A{ 1 })
\end{cpp}

虽然 <=> 操作符是为类型 B 定义的,由于它是非成员函数,并且 A 可以隐式转换为 B,因此可以对 A 类型的对象进行比较。

