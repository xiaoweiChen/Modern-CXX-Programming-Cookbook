开发人员操作位标志(bit flag)是很常见的事情,因为他们与操作系统 API(通常用 C 语言)打交道,这些 API 接受各种类型的参数(如选项或样式)形式的位标志,或者使用的库也做类似的事情,又或者仅仅是因为某些类型的问题可以通过位标志来解决。

可以考虑位操作的替代方案,比如:定义一个为每个选项/标志准备一个元素的数组,或者定义一个带有成员和函数的结构体来模拟位标志,但这些方法往往更加复杂;而且当需要将表示位标志的数值传递给函数时,仍然需要将数组或结构体转换成位序列。因此,C++ 标准提供了名为 std::bitset 的容器,用于处理固定大小的位序列。

\mySubsubsection{}{Getting ready}

对于本示例,需要熟悉位运算(AND、OR、XOR、NOT 和移位——即将数字的二进制表示中的每一位向左或向右移动)。如果需要了解更多关于这些内容的信息,\url{https://en.wikipedia.org/wiki/Bitwise_operation} 是一个很好的起点。

bitset 类在 <bitset> 头文件中可用,位于 std 命名空间内。bitset 表示一个固定大小的位序列,其大小在编译时确定。本示例中,大多数例子将使用 8 位的 bitset。

\mySubsubsection{}{How to do it...}

要构造一个 std::bitset 类型实例,可以使用以下构造函数:

\begin{itemize}
\item
创建一个所有位都设置为 0 的空 bitset:

\begin{cpp}
std::bitset<8> b1;            // [0,0,0,0,0,0,0,0]
\end{cpp}

\item
从数值创建一个 bitset:

\begin{cpp}
std::bitset<8> b2{ 10 };      // [0,0,0,0,1,0,1,0]
\end{cpp}

\item
从包含 '0' 和 '1' 的字符串创建一个 bitset:

\begin{cpp}
std::bitset<8> b3{ "1010"s }; // [0,0,0,0,1,0,1,0]
\end{cpp}

\item
从包含任意两个字符表示 '0' 和 '1' 的字符串创建一个 bitset;必须指定哪个字符代表 0(第四个参数,'o')以及哪个字符代表 1(第五个参数,'x'):

\begin{cpp}
std::bitset<8> b4
    { "ooooxoxo"s, 0, std::string::npos, 'o', 'x' };
    // [0,0,0,0,1,0,1,0]
\end{cpp}
\end{itemize}

要测试集合中的单个位或整个集合的特定值,可以使用以下方法:

\begin{itemize}
\item
count() 获取设置为 1 的位数:

\begin{cpp}
std::bitset<8> bs{ 10 };
std::cout << "has " << bs.count() << " 1s" << '\n';
\end{cpp}

\item
any() 检查是否有至少一位设置为 1:

\begin{cpp}
if (bs.any()) std::cout << "has some 1s" << '\n';
\end{cpp}

\item
all() 检查所有位是否都设置为 1:

\begin{cpp}
if (bs.all()) std::cout << "has only 1s" << '\n';
\end{cpp}

\item
none() 检查所有位是否都设置为 0:

\begin{cpp}
if (bs.none()) std::cout << "has no 1s" << '\n';
\end{cpp}

\item
test() 检查单个位的值(该函数的唯一参数是位的位置):

\begin{cpp}
if (!bs.test(0)) std::cout << "even" << '\n';
\end{cpp}

\item
operator[] 访问并测试单个位:

\begin{cpp}
if(!bs[0]) std::cout << "even" << '\n';
\end{cpp}
\end{itemize}

要修改 bitset 的内容,可以使用以下任一方法:

\begin{itemize}
\item
成员操作符 |=, \verb|&=|, \verb|^=|, 和 \verb|~| 分别执行二进制操作 OR、AND、XOR 和 NOT。或者,也可以使用非成员操作符 |, \&, 和 \^{}:

\begin{cpp}
std::bitset<8> b1{ 42 }; // [0,0,1,0,1,0,1,0]
std::bitset<8> b2{ 11 }; // [0,0,0,0,1,0,1,1]
auto b3 = b1 | b2;       // [0,0,1,0,1,0,1,1]
auto b4 = b1 & b2;       // [0,0,0,0,1,0,1,0]
auto b5 = b1 ^ b2;       // [0,0,1,0,0,0,0,1]
auto b6 = ~b1;           // [1,1,0,1,0,1,0,1]
\end{cpp}

\item
成员操作符 <{}<=, <{}<, >{}>=, 和 >{}> 执行移位操作:

\begin{cpp}
auto b7 = b1 << 2;       // [1,0,1,0,1,0,0,0]
auto b8 = b1 >> 2;       // [0,0,0,0,1,0,1,0]
\end{cpp}

\item
flip() 切换整个集合或单个位从 0 到 1 或从 1 到 0:

\begin{cpp}
b1.flip();               // [1,1,0,1,0,1,0,1]
b1.flip(0);              // [1,1,0,1,0,1,0,0]
\end{cpp}

\item
set() 将整个集合或单个位更改为 true 或指定的值:

\begin{cpp}
b1.set(0, true);         // [1,1,0,1,0,1,0,1]
b1.set(0, false);        // [1,1,0,1,0,1,0,0]
\end{cpp}

\item
reset() 将整个集合或单个位更改为 false:

\begin{cpp}
b1.reset(2);             // [1,1,0,1,0,0,0,0]
\end{cpp}
\end{itemize}

要将 bitset 转换为数值或字符串,可以使用以下方法:

\begin{itemize}
\item
\verb|to_ulong()| 和 \verb|to_ullong()| 将 bitset 转换为无符号长整型或无符号长长整型。如果值不能在输出类型中表示,则这些操作会抛出 \verb|std::overflow_error| 异常。请参考以下示例:

\begin{cpp}
std::bitset<8> bs{ 42 };
auto n1 = bs.to_ulong();  // n1 = 42UL
auto n2 = bs.to_ullong(); // n2 = 42ULL
\end{cpp}

\item
\verb|to_string()| 将 bitset 转换为 \verb|std::basic_string|。默认情况下,结果是一个包含 '0' 和 '1' 的字符串,可以为这两个值指定不同的字符:

\begin{cpp}
auto s1 = bs.to_string();         // s1 = "00101010"
auto s2 = bs.to_string('o', 'x'); // s2 = "ooxoxoxo"
\end{cpp}
\end{itemize}

\mySubsubsection{}{How it works...}

如果曾经使用过 C 或者 C 类似的 API,那么很可能已经写过或者至少见过操纵位来定义样式、选项或其他类型值的代码:

\begin{itemize}
\item
定义位标志;这些可以是枚举、类中的静态常量,或者是使用 C 风格的 \verb|#define| 引入的宏。通常,有一个标志表示没有值(样式、选项等)。由于这些应该是位标志,值是 2 的幂。

\item
从集合(即数值)中添加和删除标志。添加位标志是通过按位或操作符完成的(value |= FLAG),而删除位标志是通过按位与操作符和取反后的标志完成的(\verb|value &= ~FLAG|)。

\item
测试标志是否已添加到集合中(\verb|value & FLAG == FLAG|)。

\item
调用带有标志作为参数的函数。
\end{itemize}

下面展示了一个简单的示例,其中的标志定义了控件边框样式,该控件可以在左侧、右侧、顶部或底部有边框,或者这些组合中的一个,包括没有边框:

\begin{cpp}
#define BORDER_NONE   0x00
#define BORDER_LEFT   0x01
#define BORDER_TOP    0x02
#define BORDER_RIGHT  0x04
#define BORDER_BOTTOM 0x08
void apply_style(unsigned int const style)
{
    if (style & BORDER_BOTTOM) { /* do something */ }
}
// 初始化时不带标志
unsigned int style = BORDER_NONE;
// 设置一个标志
style = BORDER_BOTTOM;
// 添加更多标志
style |= BORDER_LEFT | BORDER_RIGHT | BORDER_TOP;
// 删除一些标志
style &= ~BORDER_LEFT;
style &= ~BORDER_RIGHT;
// 测试是否设置了标志
if ((style & BORDER_BOTTOM) == BORDER_BOTTOM) {}
// 将标志作为参数传递给函数
apply_style(style);
\end{cpp}

标准的 std::bitset 类型作为 C++ 中对 C 风格位集操作的一种替代,能够编写更健壮和安全的代码,通过成员函数抽象了位操作(尽管仍然需要识别集合中每个位所代表的意义):

\begin{itemize}
\item
添加和删除标志是通过 set() 和 reset() 方法完成的,这些方法将由位置指示的位的值设置为 1 或 0(或 true 和 false);或者,可以使用索引操作符达到同样的目的。

\item
通过 test() 方法测试位是否已设置。

\item
从整数或字符串转换通过构造函数完成,而转换回整数或字符串则通过成员函数完成,这样就可以在期望整数的地方(如函数参数)使用 bitset 的值。
\end{itemize}

注意,从字符序列构建 bitset 的构造函数——无论是 \verb|std::basic_string|、const char*(或其他字符类型),还是 C++26 中的 \verb|std::basic_string_view|,都可能抛出异常:如果字符不是指定的零或一位,则抛出 \verb|std::invalid_argument| 异常;如果序列的起始偏移量超出序列的末端,则抛出 \verb|std::out_of_range| 异常。

除了这些操作之外,bitset 类还有其他方法用于对位执行位运算、移位、测试等,这些已在上一节中展示。

概念上,std::bitset 是一个数值的表示,允许访问和修改各个位。然而,bitset 内部拥有一个整数数组,其会在这个数组上执行位操作。bitset 的大小不受数值类型大小的限制;大小任意,但需要是编译时常量。

上一节中关于控件边框样式的示例可以使用 std::bitset :

\begin{cpp}
struct border_flags
{
    static const int left = 0;
    static const int top = 1;
    static const int right = 2;
    static const int bottom = 3;
};
// 初始化时不带标志
std::bitset<4> style;
// 设置一个标志
style.set(border_flags::bottom);
// 设置更多标志
style
    .set(border_flags::left)
    .set(border_flags::top)
    .set(border_flags::right);
// 删除一些标志
style[border_flags::left] = 0;
style.reset(border_flags::right);
// 测试是否设置了标志
if (style.test(border_flags::bottom)) {}
// 将标志作为参数传递给函数
apply_style(style.to_ulong());
\end{cpp}

这只是可能的一种实现方式,例如:\verb|border_flags| 类可以是一个枚举。但使用作用域枚举会需要显式的类型转换到 int,不同的解决方案可能各有优缺点。可以将尝试编写一个替代方案作为一个练习。

\mySubsubsection{}{There's more...}

bitset 可以从一个整数创建,并且可以使用 \verb|to_ulong()| 或 \verb|to_ullong()| 方法将其值转换回整数。但如果 bitset 的大小大于这些数值类型,并且超出请求的数值类型大小的任何位设置为 1,那么这些方法将抛出一个 \verb|std::overflow_error| 异常。因为该值无法在无符号长整型或无符号长长整型中表示。为了提取所有的位,需要执行以下操作:

\begin{itemize}
\item
清除超出无符号长整型或无符号长长整型大小的位。

\item
将值转换为无符号长整型或无符号长长整型。

\item
按照无符号长整型或无符号长长整型中的位数移位 bitset。

\item
重复上述步骤直到所有位都被提取。
\end{itemize}

这些操作如下所示:

\begin{cpp}
template <size_t N>
std::vector<unsigned long> bitset_to_vectorulong(std::bitset<N> bs)
{
    auto result = std::vector<unsigned long> {};
    auto const size = 8 * sizeof(unsigned long);
    auto const mask = std::bitset<N>{ static_cast<unsigned long>(-1)};
    auto totalbits = 0;
    while (totalbits < N)
    {
        auto value = (bs & mask).to_ulong();
        result.push_back(value);
        bs >>= size;
        totalbits += size;
    }
    return result;
}
\end{cpp}

为了举例说明,可创建如下 bitset:

\begin{cpp}
std::bitset<128> bs =
    (std::bitset<128>(0xFEDC) << 96) |
    (std::bitset<128>(0xBA98) << 64) |
    (std::bitset<128>(0x7654) << 32) |
    std::bitset<128>(0x3210);
std::cout << bs << '\n';
\end{cpp}

如果打印其内容,将会得到如下输出:

\begin{shell}
000000000000000011111110110111000000000000000000101110101001100000000000000000000111011001010100
00000000000000000011001000010000
\end{shell}

然而,当使用 \verb|bitset_to_vectorulong()| 函数将这个集合转换为一系列无符号长整型值,并打印其十六进制表示时,可得到如下输出:

\begin{cpp}
auto result = bitset_to_vectorulong(bs);
for (auto const v : result)
    std::cout << std::hex << v << '\n';
\end{cpp}

\begin{shell}
3210
7654
ba98
fedc
\end{shell}

对于那些在编译时无法确定 bitset 大小的情况,替代方案是使用 std::vector<bool>,将在下一个示例中讨论。





