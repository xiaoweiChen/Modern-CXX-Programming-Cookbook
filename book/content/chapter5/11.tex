标准容器提供了 begin() 和 end() 成员函数,来获取容器第一个和最后一个元素之后的迭代器。实际上,这些函数共有四组。除了 begin() 和 end() 之外,容器还提供了 cbegin() 和 cend() 返回常量迭代器,rbegin() 和 rend() 返回可变反向迭代器,以及 crbegin() 和 crend() 返回常量反向迭代器。C++11/C++14 中,所有这些都有非成员等价函数,可以与标准容器、数组以及对其进行特化的自定义类型一起使用。C++17 中,甚至增加了更多的非成员函数:std::data(),返回包含容器元素的内存块的指针;std::size(),返回容器或数组的大小;以及 std::empty(),返回给定容器是否为空。这些非成员函数旨在用于泛型代码,可以在代码中的任何地方使用。C++20 中,引入了 std::ssize() 非成员函数,用于返回容器或数组的大小作为带符号整数。

\mySubsubsection{}{Getting ready}

本示例中,将使用在上一个示例中实现的 dummy\_array 类及其迭代器作为例子。

非成员 begin() 和 end() 函数以及其他变体,以及非成员 data()、size() 和 empty() 函数都在 <iterator> 头文件的 std 命名空间中可用,该头文件会隐式地随着以下任何一个头文件的包含而包含:<array>、<deque>、<forward\_list>、<list>、<map>、<regex>、<set>、<string>、<unordered\_map>、<unordered\_set> 和 <vector>。

本示例中,将提到 std::begin() 和 std::end() 函数,但讨论的内容同样适用于其他函数:std::cbegin() 和 std::cend()、std::rbegin() 和 std::rend()、std::crbegin() 和 std::crend()。

\mySubsubsection{}{How to do it...}

使用非成员 std::begin() 和 std::end() 函数及其他变体版本,以及 std::data()、std::size() 和 std::empty() 函数:

\begin{itemize}
\item
标准容器:

\begin{cpp}
std::vector<int> v1{ 1, 2, 3, 4, 5 };
auto sv1 = std::size(v1);  // sv1 = 5
auto ev1 = std::empty(v1); // ev1 = false
auto dv1 = std::data(v1);  // dv1 = v1.data()
for (auto i = std::begin(v1); i != std::end(v1); ++i)
    std::cout << *i << '\n';
std::vector<int> v2;
std::copy(std::cbegin(v1), std::cend(v1),
          std::back_inserter(v2));
\end{cpp}

\item
数组:

\begin{cpp}
int a[5] = { 1, 2, 3, 4, 5 };
auto pos = std::find_if(std::crbegin(a), std::crend(a),
                        [](int const n) {return n % 2 == 0; });
auto sa = std::size(a);  // sa = 5
auto ea = std::empty(a); // ea = false
auto da = std::data(a);  // da = a
\end{cpp}

\item
提供相应成员函数的自定义类型,即 begin() 和 end()、data()、empty() 或 size():

\begin{cpp}
dummy_array<std::string, 5> sa;
dummy_array<int, 5> sb;
sa[0] = "1"s;
sa[1] = "2"s;
sa[2] = "3"s;
sa[3] = "4"s;
sa[4] = "5"s;
std::transform(
    std::begin(sa), std::end(sa),
    std::begin(sb),
    [](std::string const & s) {return std::stoi(s); });
// sb = [1, 2, 3, 4, 5]
auto sa_size = std::size(sa); // sa_size = 5
\end{cpp}

\item
泛型代码,其中容器的类型未知:

\begin{cpp}
template <typename F, typename C>
void process(F&& f, C const & c)
{
    std::for_each(std::begin(c), std::end(c),
    std::forward<F>(f));
}
auto l = [](auto const e) {std::cout << e << '\n'; };
process(l, v1); // std::vector<int>
process(l, a);  // int[5]
process(l, sa); // dummy_array<std::string, 5>
\end{cpp}

\end{itemize}

\mySubsubsection{}{How it works...}

这些非成员函数是在不同版本的标准中引入的,但在 C++17 中都修改为返回 \verb|constexpr auto|:

\begin{itemize}
\item
C++11 引入了 std::begin() 和 std::end()

\item
C++14 引入了 std::cbegin()、std::cend()、std::rbegin()、std::rend()、std::crbegin() 和 std::crend()

\item
C++17 引入了 std::data()、std::size() 和 std::empty()

\item
C++20 引入了 std::ssize()
\end{itemize}

begin() 和 end() 函数族有针对容器类和数组的重载:

\begin{itemize}
\item
对于容器,调用相应的成员函数并返回结果

\item
对于数组,返回指向数组第一个或最后一个元素之后位置的指针
\end{itemize}

std::begin() 和 std::end() 的实现如下:

\begin{cpp}
template<class C>
constexpr auto inline begin(C& c) -> decltype(c.begin())
{
    return c.begin();
}
template<class C>
constexpr auto inline end(C& c) -> decltype(c.end())
{
    return c.end();
}
template<class T, std::size_t N>
constexpr T* inline begin(T (&array)[N])
{
    return array;
}
template<class T, std::size_t N>
constexpr T* inline begin(T (&array)[N])
{
    return array+N;
}
\end{cpp}

对于那些没有对应的 begin() 和 end() 成员但仍然可以迭代的容器,可以提供自定义特化。实际上,标准库为 std::initializer\_list 和 std::valarray 提供了这样的特化。

\begin{myTip}
特化必须定义在原始类或函数模板定义所在的同一命名空间内,如果想特化任何一组 std::begin() 和 std::end(),必须在 std 命名空间内进行。
\end{myTip}

C++17 引入的其他用于容器访问的非成员函数也有多个重载:

\begin{itemize}
\item
std::data() 有几个重载;对于C类型,返回 c.data();对于数组,返回数组本身;对于 std::initializer\_list<T>,返回 il.begin():

\begin{cpp}
template <class C>
constexpr auto data(C& c) -> decltype(c.data())
{
    return c.data();
}
template <class C>
constexpr auto data(const C& c) -> decltype(c.data())
{
    return c.data();
}
template <class T, std::size_t N>
constexpr T* data(T (&array)[N]) noexcept
{
    return array;
}
template <class E>
constexpr const E* data(std::initializer_list<E> il) noexcept
{
    return il.begin();
}
\end{cpp}

\item
std::size() 有两个重载;对于C类型,返回 c.size();对于数组,返回大小 N:

\begin{cpp}
template <class C>
constexpr auto size(const C& c) -> decltype(c.size())
{
    return c.size();
}
template <class T, std::size_t N>
constexpr std::size_t size(const T (&array)[N]) noexcept
{
    return N;
}
\end{cpp}

\item
std::empty() 有几个重载;对于C类型,返回 c.empty();对于数组,返回 false;对于 std::initializer\_list<T>,返回 il.size() == 0:

\begin{cpp}
template <class C>
constexpr auto empty(const C& c) -> decltype(c.empty())
{
    return c.empty();
}
template <class T, std::size_t N>
constexpr bool empty(const T (&array)[N]) noexcept
{
    return false;
}
template <class E>
constexpr bool empty(std::initializer_list<E> il) noexcept
{
    return il.size() == 0;
}
\end{cpp}
\end{itemize}

C++20 中,作为对 std::size() 的补充,引入了 std::ssize() 非成员函数,用于返回给定容器或数组中的元素数量作为带符号整数。std::size() 返回无符号整数,但在某些情况下需要带符号值。例如,C++20 类 std::span 表示对连续对象序列的视图,其 size() 成员函数返回带符号整数,而标准库容器的 size() 成员函数返回无符号整数。

std::span 的 size() 成员函数返回带符号整数的原因是,值 -1 应该表示编译时未知大小的类型的哨兵。混合使用带符号和无符号算术可能导致难以发现的代码错误。std::ssize() 有两个重载:对于C类型,返回静态转换为带符号整数(通常是 std::ptrdiff\_t)的 c.size();对于数组,返回元素的数量 N:

\begin{cpp}
template <class C>
constexpr auto ssize(const C& c)
    -> std::common_type_t<std::ptrdiff_t,
                          std::make_signed_t<decltype(c.size())>>
{
    using R = std::common_type_t<std::ptrdiff_t,
                        std::make_signed_t<decltype(c.size())>>;
    return static_cast<R>(c.size());
}
template <class T, std::ptrdiff_t N>
constexpr std::ptrdiff_t ssize(const T (&array)[N]) noexcept
{
    return N;
}
\end{cpp}

上述片段展示了 std::ssize() 函数针对容器和数组的可能实现。

\mySubsubsection{}{There's more...}

这些非成员函数主要旨在用于模板代码中,当容器类型未知且可能是标准容器、数组或自定义类型时。使用这些函数的非成员版本能够编写更简单且更少的代码,这些代码可以适用于所有这些类型的容器。

然而,这些函数的使用不应该仅限于泛型代码。虽然这更多是一个偏好的问题,但养成一致性的好习惯,在代码的所有地方都使用也不错。所有这些方法都有轻量级的实现,编译器很可能会进行内联,所以使用相应的成员函数将不会产生任何开销。

