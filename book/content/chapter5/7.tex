前面的示例中,探讨了用于在一个范围内搜索和排序的标准通用算法。算法库提供了许多其他的通用算法,其中包括几个用于用值填充一个范围的算法。这个示例中,将了解这些算法是什么,以及应该如何使用。

\mySubsubsection{}{Getting ready}

本示例中的所有例子都使用了 \verb|std::vector|,像所有的通用算法一样,在本示例中看到的算法接受迭代器来定义一个范围的边界,因此也可以与任何标准容器、数组或定义了向前迭代器的自定义类型所代表的序列一起使用。

除了可以在 \verb|<numeric>| 头文件中找到的 \verb|std::iota()| 之外,其他所有算法都在 \verb|<algorithm>| 头文件中。

\mySubsubsection{}{How to do it...}

要向一个范围赋值,可以使用以下任何一个标准算法:

\begin{itemize}
\item
使用 \verb|std::fill()| 给一个范围内的所有元素赋同一个值;范围由第一个和最后一个向前迭代器定义:

\begin{cpp}
std::vector<int> v(5);
std::fill(v.begin(), v.end(), 42);
// v = {42, 42, 42, 42, 42}
\end{cpp}

\item
使用 \verb|std::fill_n()| 给一个范围内的一定数量的元素赋值;范围由第一个向前迭代器和一个计数器定义,该计数器指示应给多少个元素赋指定的值:

\begin{cpp}
std::vector<int> v(10);
std::fill_n(v.begin(), 5, 42);
// v = {42, 42, 42, 42, 42, 0, 0, 0, 0, 0}
\end{cpp}

\item
使用 \verb|std::generate()| 给一个范围内的每个元素赋函数返回的值;范围由第一个和最后一个向前迭代器定义,函数为范围中的每个元素调用一次:

\begin{cpp}
std::random_device rd{};
std::mt19937 mt{ rd() };
std::uniform_int_distribution<> ud{1, 10};
std::vector<int> v(5);
std::generate(v.begin(), v.end(),
              [&ud, &mt] {return ud(mt); });
\end{cpp}

\item
使用 \verb|std::generate_n()| 给一个范围内的一定数量的元素赋函数返回的值;范围由第一个向前迭代器和一个计数器定义,该计数器指示应给多少个元素赋值,函数为每个元素调用一次:

\begin{cpp}
std::vector<int> v(5);
auto i = 1;
std::generate_n(v.begin(), v.size(), [&i] { return i*i++; });
// v = {1, 4, 9, 16, 25}
\end{cpp}

\item
使用 \verb|std::iota()| 给一个范围内的元素赋按顺序递增的值;范围由第一个和最后一个向前迭代器定义,值从一个初始指定值开始递增:

\begin{cpp}
std::vector<int> v(5);
std::iota(v.begin(), v.end(), 1);
// v = {1, 2, 3, 4, 5}
\end{cpp}
\end{itemize}

\mySubsubsection{}{How it works...}

\verb|std::fill()| 和 \verb|std::fill_n()| 的工作方式类似,不过在范围的指定方式上有所不同:前者通过第一个和最后一个迭代器指定,后者通过第一个迭代器和一个计数器指定。第二个算法返回一个迭代器,如果计数器大于零,则该迭代器指向最后一个赋值元素之后的位置;否则,返回一个指向范围第一个元素的迭代器。

\verb|std::generate()| 和 \verb|std::generate_n()| 也相似,仅在范围的指定方式上不同。前者接受两个迭代器,定义范围的下限和上限,而后者接受一个指向第一个元素的迭代器和一个计数器。像 \verb|std::fill_n()| 一样,\verb|std::generate_n()| 也返回一个迭代器,如果计数器大于零,则该迭代器指向最后一个赋值元素之后的位置;否则,返回一个指向范围第一个元素的迭代器。这些算法为范围内的每个元素调用一个指定的函数,并将返回的值赋给元素。生成函数不接受任何参数,不能将参数值传递给函数,因为它们设计用于初始化一个范围的元素。如果需要使用元素的值来生成新值,应使用 \verb|std::transform()|。

\verb|std::iota()| 的名称来源于 APL 编程语言中的 ι(iota) 函数,最初是 STL 的一部分,但直到 C++11 才纳入标准库。此函数接受一个范围的第一个和最后一个迭代器,以及一个初始值,该值被赋给范围的第一个元素。然后,使用前缀递增操作符 \verb|++| 为范围中的其余元素生成按顺序递增的值。

\begin{myNotic}
STL 是标准模板库的缩写,是由 Alexander Stepanov 最初为 C++ 设计的一个软件库,在 C++ 语言标准化之前就已经存在。后来,用来建模 C++ 标准库,提供容器、迭代器、算法和函数。不应将其与 C++ 标准库混淆,因为这两者是不同的实体。
\end{myNotic}

\mySubsubsection{}{There's more...}

本示例中看到的例子使用了整数,以便更容易理解。也可以提供一个现实生活中的例子,帮助各位读者更好地理解这些算法如何用于更复杂的场景。

考虑一个给定两种颜色的函数,生成一系列中间点,代表一个渐变。颜色对象有三个值,分别对应红色、绿色和蓝色通道:

\begin{cpp}
struct color
{
    unsigned char red   = 0;
    unsigned char blue  = 0;
    unsigned char green = 0;
};
\end{cpp}

编写一个函数,该函数接受起始颜色和结束颜色,以及要生成的点数,并返回一个颜色对象的向量。内部使用 \verb|std::generate_n()| 生成值:

\begin{cpp}
std::vector<color> make_gradient(color const& c1, color const& c2, size_t points)
{
    std::vector<color> colors(points);
    auto rstep = static_cast<double>(c2.red - c1.red) / points;
    auto gstep = static_cast<double>(c2.green - c1.green) / points;
    auto bstep = static_cast<double>(c2.blue - c1.blue) / points;
    auto r = c1.red;
    auto g = c1.green;
    auto b = c1.blue;
    std::generate_n(colors.begin(),
                    points,
                    [&r, &g, &b, rstep, gstep, bstep] {
        color c {
            static_cast<unsigned char>(r),
            static_cast<unsigned char>(g),
            static_cast<unsigned char>(b)
        };
        r += rstep;
        g += gstep;
        b += bstep;
        return c;
    });
    return colors;
}
\end{cpp}

可以这样使用这个函数:

\begin{cpp}
color white { 255, 255, 255 };
color black { 0, 0, 0 };
std::vector<color> grayscale = make_gradient(white, black, 256);
std::for_each(
    grayscale.begin(), grayscale.end(),
    [](color const& c) {
        std::cout <<
        static_cast<int>(c.red) << ", "
        << static_cast<int>(c.green) << ", "
        << static_cast<int>(c.blue) << '\n';
    });
\end{cpp}

虽然运行这段代码的输出有 256 行(每行代表一个点),这里展示其中一部分:

\begin{shell}
255, 255, 255
254, 254, 254
253, 253, 253
…
1, 1, 1
0, 0, 0
\end{shell}


