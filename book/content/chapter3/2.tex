C++ 最重要的现代特性之一是 lambda 表达式,也称为 lambda 函数或简称 lambda。Lambda 表达式能够定义匿名函数对象,这些对象可以捕获作用域内的变量,并调用或作为参数传递给函数。其可避免了定义命名函数或函数对象的需要,Lambda 在许多方面都很有用。本示例中,将介绍如何将它们与标准算法一起使用。

\mySubsubsection{}{Getting ready}

本示例中,将介绍接受一个函数或谓词作为参数的标准算法,该函数或谓词会应用到迭代过程中遇到的元素上。需要知道什么是一元和二元函数,以及什么是谓词和比较函数。还需熟悉函数类型实例,因为 lambda 表达式实际上是函数类型实例的语法糖。

\mySubsubsection{}{How to do it...}

应该优先使用 lambda 表达式来传递回调给标准算法,而不是使用函数或函数对象:

\begin{itemize}
\item
如果需要在一个地方使用 lambda,可以在调用的地方定义匿名 lambda 表达式:

\begin{cpp}
auto numbers =
    std::vector<int>{ 0, 2, -3, 5, -1, 6, 8, -4, 9 };
auto positives = std::count_if(
    std::begin(numbers), std::end(numbers),
    [](int const n) {return n > 0; });
\end{cpp}

\item
如果需要在多个地方调用 lambda,可以定义一个命名 lambda,即分配给一个变量(通常使用 auto 指定类型):

\begin{cpp}
auto ispositive = [](int const n) {return n > 0; };
auto positives = std::count_if(
    std::begin(numbers), std::end(numbers), ispositive);
\end{cpp}

\item
如果需要 lambda 仅在参数类型上有所不同(C++14),则可以使用泛型 lambda 表达式:

\begin{cpp}
auto positives = std::count_if(
std::begin(numbers), std::end(numbers),
    [](auto const n) {return n > 0; });
\end{cpp}
\end{itemize}

\mySubsubsection{}{How it work...}

第二点中展示的非泛型lambda表达式接受一个常量整数,并在其大于0时返回true,否则返回false。编译器定义了一个具有调用操作符的无名函数对象,其签名与lambda表达式的签名相同:

\begin{cpp}
struct __lambda_name__
{
    bool operator()(int const n) const { return n > 0; }
};
\end{cpp}

编译器定义无名函数对象的方式取决于如何定义lambda表达式,包括是否捕获变量、使用mutable说明符或异常说明,或者具有尾随返回类型。前面显示的\verb|__lambda_name__|函数对象实际上是编译器生成的简化版本,还定义了默认的复制构造函数、移动构造函数、默认析构函数,以及删除的赋值操作符。

\begin{myNotic}
lambda表达式实际上是一个类(class),编译器需要实例化该类型实例。lambda表达式的实例称为lambda闭包。
\end{myNotic}

下面的示例中,计算一个范围内大于等于5且小于等于10的元素数量:

\begin{cpp}
auto numbers = std::vector<int>{ 0, 2, -3, 5, -1, 6, 8, -4, 9 };
auto minimum { 5 };
auto maximum { 10 };
auto inrange = std::count_if(
    std::begin(numbers), std::end(numbers),
    [minimum, maximum](int const n) {
        return minimum <= n && n <= maximum;});
\end{cpp}

lambda通过复制(即按值)捕获两个变量,minimum和maximum。编译器创建的无名函数实例,加上前面提到的默认和删除的特殊成员后,这个类(class)看起来会像这样:

\begin{cpp}
class __lambda_name_2__
{
    int minimum_;
    int maximum_;
    public:
    explicit __lambda_name_2__(int const minimum, int const maximum) :
    minimum_( minimum), maximum_( maximum)
    {}
    __lambda_name_2__(const __lambda_name_2__&) = default;
    __lambda_name_2__(__lambda_name_2__&&) = default;
    __lambda_name_2__& operator=(const __lambda_name_2__&)
    = delete;
    ~__lambda_name_2__() = default;
    bool operator() (int const n) const
    {
        return minimum_ <= n && n <= maximum_;
    }
};
\end{cpp}

lambda表达式可通过复制(或按值)或引用捕获变量,而且两种方式的不同组合是可能的。但不能多次捕获同一个变量,只能在捕获列表的开始处使用\&或=。

\begin{myNotic}
lambda表达式可以访问以下几种类型的变量:从封闭作用域捕获的变量、lambda参数、其内局部声明的变量、lambda在类内声明并且捕获了指针时的类数据成员,以及具有静态存储持续性的变量,例如:全局变量。

lambda只能从封闭函数作用域捕获变量,不能捕获具有静态存储持续性的变量(命名空间作用域中声明的变量或带有static或external说明符的变量)。
\end{myNotic}

下表展示了lambda捕获语义的组合:

% Please add the following required packages to your document preamble:
% \usepackage{longtable}
% Note: It may be necessary to compile the document several times to get a multi-page table to line up properly
\begin{longtable}{|l|l|}
\hline
\textbf{Lambda}    & \textbf{描述}                                                                        \\ \hline
\endfirsthead
%
\endhead
%
{[}{]}()\{\}       & 不捕获任何东西。                                                                  \\ \hline
{[}\&{]}()\{\}     & 按引用捕获所有。                                                           \\ \hline
{[}={]}()\{\}      & 按值捕获所有。自C++20起,隐式捕获指针this已弃用。 \\ \hline
{[}\&x{]}()\{\}    &  只按引用捕获x。                                                    \\ \hline
{[}x{]}()\{\}      & 只按值复制捕获x。                                                                    \\ \hline
{[}\&x...{]}()\{\} & 按引用捕获扩展包x。                                                     \\ \hline
{[}x...{]}()\{\}   & 按值复制捕获扩展包x。                                                          \\ \hline
{[}\&, x{]}()\{\}  & 按引用捕获所有,除了按值复制捕获的x。                     \\ \hline
{[}=, \&x{]}()\{\} & 按值捕获所有,除了按引用捕获的x。                     \\ \hline
{[}\&, this{]}()\{\} & 按引用捕获所有,除了按值捕获的指针this(this按值捕获)。 \\ \hline
{[}x, x{]}()\{\}   & 错误;x捕获两次。                                                                 \\ \hline
{[}\&, \&x{]}()\{\}  & 错误;所有都按引用捕获,不能再指定按引用捕获x。                         \\ \hline
{[}=, =x{]}()\{\}  & 错误;所有都按值复制捕获,不能再指定按值捕获x。    \\ \hline
{[}\&this{]}()\{\} & 错误;指针this总是按值捕获。                                         \\ \hline
{[}\&, ={]}()\{\}  & 错误;不能同时按值和按引用捕获所有。                             \\ \hline
{[}x=expr{]}()\{\} & x是lambda闭包的数据成员,由表达式expr初始化。            \\ \hline
{[}\&x=expr{]}()\{\} & x是lambda闭包的引用数据成员,由表达式expr初始化。                                 \\ \hline
\end{longtable}

\begin{center}
表3.1: lambda捕获示例及其解释
\end{center}

C++17中lambda表达式的一般形式如下所示:

\begin{cpp}
[capture-list](params) mutable constexpr exception attr -> ret
{ body }
\end{cpp}

 此语法中显示的所有部分实际上都是可选的,但捕获列表和函数主体除外,二者都可以为空。如果不需要参数,则可以省略参数列表。由于编译器可以从返回表达式的类型推断出返回类型,因此无需指定返回类型。mutable说明符(告诉编译器lambda可以实际修改按值捕获的变量,但这不同于按值捕获,因为变化仅在lambda内部可见)、constexpr说明符(指示编译器生成一个constexpr调用操作符)、异常说明符和属性都是可选的。

\begin{myNotic}
最简单的lambda表达式是\verb|[]{}|,通常写作\verb|[](){}|。
\end{myNotic}

上表中的最后两个例子是广义lambda捕获的形式。这些是在C++14中引入的,可以捕获具有移动语义的变量,也可以用于在lambda中定义的新类型实例。下面的例子展示了如何使用广义lambda捕获来按移动方式捕获变量:

\begin{cpp}
auto ptr = std::make_unique<int>(42);
auto l = [lptr = std::move(ptr)](){return ++*lptr;};
\end{cpp}

类型中的lambda需要捕获类数据成员时,可以采用几种方式:

\begin{itemize}
\item
以形式\verb|[x=expr]|单独捕获数据成员:

\begin{cpp}
struct foo
{
    int         id;
    std::string name;
    auto run()
    {
        return [i=id, n=name] { std::cout << i << ' ' << n << '\n'; };
    }
};
\end{cpp}

\item
以形式\verb|[=]|捕获整个对象(通过\verb|[=]|隐式捕获指针this在C++20中已弃用):

\begin{cpp}
struct foo
{
    int         id;
    std::string name;
    auto run()
    {
        return [=] { std::cout << id << ' ' << name << '\n'; };
    }
};
\end{cpp}

\item
通过捕获this指针捕获整个对象。如果需要调用类的其他方法,这是必需的。当实例本身按值捕获时,可用\verb|[*this]|捕获;当指针按值捕获时,可用\verb|[this]|捕获。如果实例可能在捕获后,但在调用lambda之前超出作用范围(原始实例析构),这会有很大的不同:

\begin{cpp}
struct foo
{
    int         id;
    std::string name;
    auto run()
    {
        return[this]{ std::cout << id << ' ' << name << '\n'; };
    }
};
auto l = foo{ 42, "john" }.run();
l(); // does not print 42 john
\end{cpp}
\end{itemize}

这里,正确的捕获应该是\verb|[*this]|,以便实例按值复。即使临时对象已经超出作用范围,lambda也会输出42 john。

C++20标准对捕获指针this进行了几项更改:

\begin{itemize}
\item
当使用\verb|[=]|时,隐式捕获this已弃用。这将导致编译器发出弃用警告。

\item
当捕获所有内容时,通过\verb|[=, this]|显式按值复制捕获this指针的方法,并只能通过\verb|[this]|捕获this指针。
\end{itemize}

lambda表达式仅在参数方面有所不同,可以像模板一样以泛型方式书写lambda,可使用auto说明符来指定类型参数(不涉及模板语法)。

C++23之前,可以在lambda表达式的可选异常说明符和可选尾随返回类型之间指定属性。这样的属性适用于类型,而不是函数调用操作符。然而,像\verb|[[nodiscard]]|或\verb|[[noreturn]]|这样的属性只在函数上有意义,而不在类型上有意义。

因此,从C++23开始,这一限制发生了变化,使得属性也可以进行指定:

\begin{itemize}
\item
lambda引入符及其可选捕获之后,或者

\item
模板参数列表及其可选requires子句之后。
\end{itemize}

在lambda声明的这些部分中声明的属性适用于函数调用操作符,而不是类型。

来看一个例子:

\begin{cpp}
auto linc = [](int a) [[deprecated]] { return a+1; };
linc(42);
\end{cpp}

\verb|[[deprecated]]|属性适用于lambda的类型,编译代码片段时不会产生警告。在C++23中,可以这样写:

\begin{cpp}
auto linc = [][[nodiscard,deprecated]](int a) { return a+1; };
linc(42);
\end{cpp}

有了这个改变,\verb|[[nodiscard]]|和\verb|[[deprecated]]|这两个属性都适用于lambda类型的函数调用操作符。这会导致发出两个警告:一个使用了已弃用的函数,另一个忽略了返回类型。






