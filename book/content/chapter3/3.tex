
前一节中,介绍了如何编写lambda表达式并将其与标准算法一起使用。C++中lambda是对未命名函数对象的语法糖,这些函数对象是实现了调用操作符的类。但就像其他函数一样,也可以用模板来实现。C++14引入了通用lambda,这种lambda不需要为其参数指定实际类型,而是使用auto说明符。虽然没有以这个名字来称呼,但通用lambda实际上是lambda模板。在希望使用相同的lambda,但带有不同类型的参数的情况下非常有用。此外,C++20标准更进一步,支持明确地定义模板lambda,有助于解决一些通用lambda难以处理的情况。

\mySubsubsection{}{Getting ready}

建议继续阅读本节之前先阅读前一节的内容,以熟悉C++中的lambda表达式。

\mySubsubsection{}{How to do it...}

C++14起,可以编写通用lambda:

\begin{itemize}
\item
使用auto说明符代替lambda表达式参数的实际类型

\item
需要使用多个仅因参数类型不同的lambda时
\end{itemize}

下面的例子展示了一个通用lambda与std::accumulate()算法一起使用,首先是整数vector,然后是字符串vector:

\begin{cpp}
auto numbers =
    std::vector<int>{0, 2, -3, 5, -1, 6, 8, -4, 9};
using namespace std::string_literals;
auto texts =
    std::vector<std::string>{"hello"s, " "s, "world"s, "!"s};
auto lsum = [](auto const s, auto const n) {return s + n;};
auto sum = std::accumulate(
    std::begin(numbers), std::end(numbers), 0, lsum);
    // sum = 22
auto text = std::accumulate(
    std::begin(texts), std::end(texts), ""s, lsum);
    // sum = "hello world!"s
\end{cpp}

C++20起可以编写模板lambda:

\begin{itemize}
\item
通过在捕获子句后面使用尖括号中的模板参数列表(例如<template T>)

\item
当希望:

\begin{itemize}
\item
仅限于某些类型使用通用lambda,比如容器,或是满足某个概念的类型。

\item
确保通用lambda的两个或更多参数确实具有相同的类型。

\item
获取通用参数的类型,例如:可以创建该类型的实例,调用静态方法,或使用其迭代器类型。

\item
通用lambda中执行完美转发。
\end{itemize}
\end{itemize}

下面的例子展示了只能使用std::vector调用的模板lambda:

\begin{cpp}
std::vector<int> vi { 1, 1, 2, 3, 5, 8 };
auto tl = []<typename T>(std::vector<T> const& vec)
{
    std::cout << std::size(vec) << '\n';
};
tl(vi); // OK, prints 6
tl(42); // error
\end{cpp}

\mySubsubsection{}{How it work...}

上一节的第一个例子中,定义了一个命名lambda表达式——赋值给变量的lambda表达式。然后,这个变量可作为参数传递给std::accumulate()函数。

这个通用算法接受定义了范围的开始和结束迭代器,累加的初始值,以及一个旨在将范围内的每个值累加到总数上的函数。这个函数接收一个表示当前累加值的第一个参数,和一个表示要累加到总数上的当前值的第二个参数,并返回新累加值。请注意,这里没有使用加法这个词,因为其不仅限于加法,还可以用于计算乘积、连接或其他聚合值的操作。

这个例子中的两次std::accumulate()调用几乎相同;只有参数类型不同:

\begin{itemize}
\item
第一次调用中,传递的是指向整数范围的迭代器(来自vector<int>),0作为初始和,以及一个将两个整数相加并返回加和的lambda。这会产生范围中所有整数的总和;这个例子中,结果是22。

\item
第二次调用中,传递的是指向字符串范围的迭代器(来自vector<string>),空字符串作为初始值,以及一个将两个字符串连接起来并返回结果的lambda。这会产生一个包含范围内所有字符串依次连接的结果;这个例子中,结果是hello world!。
\end{itemize}

尽管通用lambda可以在调用的地方匿名定义,但这样做并不合理,因为通用lambda的主要用途(正如前面提到的,是一个lambda表达式模板)是为了重用。

定义这个lambda表达式后,可多次调用std::accumulate()。并未指定lambda参数的具体类型(如int或std::string),而是使用了auto说明符,让编译器自行推断。

遇到具有auto说明符参数类型的lambda表达式时,编译器会生成一个具有调用操作符模板的匿名函数对象。对于本例中的通用lambda表达式:

\begin{cpp}
struct __lambda_name__
{
    template<typename T1, typename T2>
    auto operator()(T1 const s, T2 const n) const { return s + n; }
    __lambda_name__(const __lambda_name__&) = default;
    __lambda_name__(__lambda_name__&&) = default;
    __lambda_name__& operator=(const __lambda_name__&) = delete;
    ~__lambda_name__() = default;
};
\end{cpp}

调用操作符是一个模板,每个使用了auto说明符的lambda参数都有一个类型参数。操作符的返回类型也是auto,所以编译器将根据返回值的类型进行推断。这个操作符模板将根据编译器在使用通用lambda的上下文,识别出的实际类型进行实例化。

C++20的模板lambda是C++14通用lambda的改进,使某些场景更容易处理。上一节的第二个例子中,其中lambda的使用限制为std::vector类型的参数。另一个例子想确保lambda的两个参数有相同的类型。在C++20之前,这很难做到,但有了模板lambda,就变得非常容易:

\begin{cpp}
auto tl = []<typename T>(T x, T y)
{
    std::cout << x << ' ' << y << '\n';
};
tl(10, 20);   // OK
tl(10, "20"); // error
\end{cpp}

另一个需要模板lambda的情况是,需要知道参数的类型,以便创建该类型的实例或调用其静态成员。对于通用lambda,解决方案如下:

\begin{cpp}
struct foo
{
    static void f() { std::cout << "foo\n"; }
};
auto tl = [](auto x)
{
    using T = std::decay_t<decltype(x)>;
    T other;
    T::f();
};
tl(foo{});
\end{cpp}

要求使用\verb|std::decay_t|和decltype。decltype是一个类型说明符,产生指定表达式的类型,主要用于编写模板。另一方面,std::decay是从\verb|<type_traits>|中提供的工具,执行与按值传递函数参数时相同类型的转换。

C++20中,同样的lambda可以简化为如下形式:

\begin{cpp}
auto tl = []<typename T>(T x)
{
    T other;
    T::f();
};
\end{cpp}

类似的情况发生在通用lambda中执行完美转发时,这需要使用decltype来确定参数的类型:

\begin{cpp}
template <typename ...T>
void foo(T&& ... args)
{ /* ... */ }
auto tl = [](auto&& ...args)
{
    return foo(std::forward<decltype(args)>(args)...);
};
tl(1, 42.99, "lambda");
\end{cpp}

使用模板lambda,可以将其重写为更加简单的方式:

\begin{cpp}
auto tl = []<typename ...T>(T && ...args)
{
    return foo(std::forward<T>(args)...);
};
\end{cpp}

从这些例子中可以看出,模板lambda是对通用lambda的改进,使得处理本示例中提到的场景变得更加容易。







