本章中,已经多次讨论了折叠操作,是一种将二元函数应用于生成单一值的操作。介绍可变参数函数模板时见过这种操作,并将在高阶函数中再次见到。许多情况下,可变参数函数模板中参数包的展开是一种折叠操作。为了简化这类可变参数函数模板的编写,C++17引入了折叠表达式,可以在二元操作符上对参数包进行折叠。本示例中,将介绍如何使用折叠表达式来简化可变参数函数模板的编写。

\mySubsubsection{}{Getting ready}

本示例基于前一节的可变参数函数模板add(),实现是一个左折叠操作,再次展示该函数:

\begin{cpp}
template <typename T>
T add(T value)
{
    return value;
}
template <typename T, typename... Ts>
T add(T head, Ts... rest)
{
    return head + add(rest...);
}
\end{cpp}

\mySubsubsection{}{How to do it...}

要在二元操作符上对参数包进行折叠,请使用以下形式:

\begin{itemize}
\item
一元左折叠 (... op pack):

\begin{cpp}
template <typename... Ts>
auto add(Ts... args)
{
    return (... + args);
}
\end{cpp}

\item
二元左折叠 (init op ... op pack):

\begin{cpp}
template <typename... Ts>
auto add_to_one(Ts... args)
{
    return (1 + ... + args);
}
\end{cpp}

\item
一元右折叠 (pack op ...):

\begin{cpp}
template <typename... Ts>
auto add(Ts... args)
{
    return (args + ...);
}
\end{cpp}

\item
二元右折叠 (pack op ... op init):

\begin{cpp}
template <typename... Ts>
auto add_to_one(Ts... args)
{
    return (args + ... + 1);
}
\end{cpp}
\end{itemize}

\begin{myNotic}
括号是折叠表达式的一部分,不能省略。
\end{myNotic}

\mySubsubsection{}{How it work...}

编译器遇到折叠表达式时,会其展开成以下表达式:

% Please add the following required packages to your document preamble:
% \usepackage{longtable}
% Note: It may be necessary to compile the document several times to get a multi-page table to line up properly
\begin{longtable}{|l|l|}
\hline
\textbf{表达式}   & \textbf{展开}                                 \\ \hline
\endfirsthead
%
\endhead
%
(... op pack)         & ((pack\$1 op pack\$2) op ...) op pack\$n             \\ \hline
(init op ... op pack) & (((init op pack\$1) op pack\$2) op ...) op pack\$n   \\ \hline
(pack op ...)         & pack\$1 op (... op (pack\$n-1 op pack\$n))           \\ \hline
(pack op ... op init) & pack\$1 op (... op (pack\$n-1 op (pack\$n op init))) \\ \hline
\end{longtable}

\begin{center}
表3.2: 折叠表达式的形式
\end{center}

当使用二元形式时,省略号两边的操作符必须相同,初始化值中不得包含未展开的参数包。

以下二元操作符支持与折叠表达式一起使用:

% Please add the following required packages to your document preamble:
% \usepackage{longtable}
% Note: It may be necessary to compile the document several times to get a multi-page table to line up properly
\begin{longtable}{|l|l|l|l|l|l|l|l|l|l|l|l|}
\hline
+ &
- &
* &
/ &
\% &
\textasciicircum{} &
\& &
| &
= &
\textless{} &
\textgreater{} &
\textless{}\textless{} \\ \hline
\endfirsthead
%
\endhead
%
\textgreater{}\textgreater{} &
+= &
-= &
*= &
/= &
\%= &
\textasciicircum{}= &
\&= &
|= &
\textless{}\textless{}= &
\textgreater{}\textgreater{}= &
== \\ \hline
!= &
\textless{}= &
\textgreater{}= &
\&\& &
|| &
, &
.* &
-\textgreater{}*. &
&
&
&
\\ \hline
\end{longtable}

\begin{center}
表3.3: 支持与折叠表达式一起使用的二元操作符
\end{center}

使用一元形式时,只有 *、+、\verb|&|、|、\verb|&&|、|| 和 ,(逗号)这样的操作符允许与空参数包一起使用。这种情况下,空包的值如下:

% Please add the following required packages to your document preamble:
% \usepackage{longtable}
% Note: It may be necessary to compile the document several times to get a multi-page table to line up properly
\begin{longtable}{|l|l|}
\hline
\textbf{操作符} & \textbf{空包值} \\ \hline
\endfirsthead
%
\endhead
%
+                 & 0                         \\ \hline
*                 & 1                         \\ \hline
\&                & -1                        \\ \hline
|                 & 0                         \\ \hline
\&\&              & true                      \\ \hline
||                & false                     \\ \hline
,                 & void()                    \\ \hline
\end{longtable}

\begin{center}
表3.4: 可以与空参数包一起使用操作符
\end{center}

现在有了之前实现的函数模板(考虑左折叠版本),可以编写以下代码:

\begin{cpp}
auto sum = add(1, 2, 3, 4, 5);         // sum = 15
auto sum1 = add_to_one(1, 2, 3, 4, 5); // sum = 16
\end{cpp}

调用add(1, 2, 3, 4, 5),将产生以下函数:

\begin{cpp}
int add(int arg1, int arg2, int arg3, int arg4, int arg5)
{
    return ((((arg1 + arg2) + arg3) + arg4) + arg5);
}
\end{cpp}

由于现代编译器的优化非常激进,这个函数可以内联,最终会得到一个如auto sum = 1 + 2 + 3 + 4 + 5这样的表达式。

\mySubsubsection{}{There's more...}

折叠表达式适用于所有支持的二元操作符的重载,但不适用于任意二元函数。可以通过提供一个包装类型来实现一个解决方法,该类型将持有某个值,以及该包装类型的重载操作符:

\begin{cpp}
template <typename T>
struct wrapper
{
    T const & value;
};
template <typename T>
constexpr auto operator<(wrapper<T> const & lhs, wrapper<T> const & rhs)
{
    return wrapper<T> {lhs.value < rhs.value ? lhs.value : rhs.value};
}
\end{cpp}

这里,wrapper 是一个简单的类模板,保存类型为 T 的值的一个常量引用。为此类模板提供了重载的 operator<;此重载不返回布尔值,而是返回一个 wrapper 类型实例以保存两个参数中较小的值。这里展示的可变参数函数模板 min() ,使用重载的 operator< 折叠扩展为 wrapper 类型模板实例的参数包:

\begin{cpp}
template <typename... Ts>
constexpr auto min(Ts&&... args)
{
    return (wrapper<Ts>{args} < ...).value;
}
auto m = min(3, 1, 2); // m = 1
\end{cpp}

min() 函数可展开为类似于以下的内容:

\begin{cpp}
template<>
inline constexpr int min<int, int, int>(int && __args0,
                                        int && __args1,
                                        int && __args2)
{
    return
    operator<(wrapper_min<int>{__args0},
        operator<(wrapper_min<int>{__args1},
            wrapper_min<int>{__args2})).value;
}
\end{cpp}

可以看到对二元操作符 < 的级联调用,这些调用返回一个 Wrapper<int> 值。否则,无法使用折叠表达式实现 min() 函数。以下实现无效:

\begin{cpp}
template <typename... Ts>
constexpr auto minimum(Ts&&... args)
{
    return (args < ...);
}
\end{cpp}

调用 min(3, 1, 2),编译器会将其转换为:

\begin{cpp}
template<>
inline constexpr bool minimum<int, int, int>(int && __args0,
                                             int && __args1,
                                             int && __args2)
{
    return __args0 < (static_cast<int>(__args1 < __args2));
}
\end{cpp}

结果是一个返回布尔值的函数,而不是实际的整数值,即所提供的参数之间的最小值。
