
泛型代码是避免编写重复代码的关键,C++中可通过模板实现。类、函数和变量都可以模板化。尽管模板经常视为复杂且累赘的代名词,但其使得创建通用库成为可能,例如标准库,并可以编写更少且更好的代码。

模板是C++语言的一等公民,详细讲解可能需要一整本书。事实上,这本书中的多个示例涉及模板的各种方面。本示例中,将介绍编写函数模板的基础知识。

\mySubsubsection{}{How to do it...}

要创建函数模板,请按照以下步骤操作:

\begin{itemize}
\item
创建一个函数模板,在函数声明前加上template关键字,后面跟上尖括号中的模板参数列表:

\begin{cpp}
template <typename T>
T minimum(T a, T b)
{
    return a <= b ? a : b;
}
minimum(3, 4);
minimum(3.99, 4.01);
\end{cpp}

\item
为了特化一个函数模板,保留模板参数列表为空,并在函数签名中用实际类型或值替换模板参数:

\begin{cpp}
template <>
const char* minimum(const char* a, const char* b)
{
    return std::strcmp(a, b) <= 1 ? a : b;
}
minimum("abc", "acxyz");
\end{cpp}

\item
为了重载一个函数模板,提供另一个定义,可以是模板,也可以是非模板:

\begin{cpp}
template <typename T>
std::basic_string<T> minimum(std::basic_string<T> a,
std::basic_string<T> b) // [1]
{
    return a.length() <= b.length() ? a : b;
}
std::string minimum(std::string a, std::string b) // [2]
{
    return a.length() <= b.length() ? a : b;
}
minimum(std::string("def"), std::string("acxyz")); // calls [2]
minimum(std::wstring(L"def"), std::wstring(L"acxyz")); // calls [1]
\end{cpp}

\item
为了确保特定函数模板或函数模板的某个特化版本不可调用(从重载集中删除),可将其声明为已删除:

\begin{cpp}
template <typename T>
T* minimum(T* a, T* b) = delete;
int a = 3;
int b = 4;
minimum(&a, &b); // error
\end{cpp}
\end{itemize}

\mySubsubsection{}{How it work...}

至少从表面上看,函数模板与其他函数只有细微的不同。通过模板语法引入,并可以使用类型、值甚至其他模板进行参数化。由于模板只是创建实际代码的蓝图,函数模板基本上是一系列函数的蓝图。模板只存在于源代码中,直到使用它们。

编译器根据模板的使用情况实例化模板,通过编译器替换模板参数来完成,这个过程称为模板实例化。例如,对于前面展示的minimum<T>函数模板,当以minimum<int>(1, 2)调用时,编译器将int类型代入T参数。有两种形式的实例化:

\begin{itemize}
\item
隐式实例化发生在编译器,根据代码中使用的模板生成代码时,例如:minimum<T>函数在整个代码中可用int和double值调用,则会生成两个重载(一个带整数参数,一个带double参数)。这就是隐式实例化,如下面的代码片段所示:

\begin{cpp}
minimum<int>(1, 2);  // 显式int模板参数
minimum(3.99, 4.50); // 推导出double模板参数
\end{cpp}

\item
显式实例化发生在请求编译器即使该实例化,一个例子是在创建库(二进制)文件时,因为未实例化的模板(仅仅是蓝图)不会放入目标文件中。以下展示了char类型的minimum<T>函数的显式实例化示例。如果显式实例化不在与模板相同的命名空间中定义,名称必须在显式实例化定义时完全限定:

\begin{cpp}
template char minimum(char a, char b);
\end{cpp}

\end{itemize}

模板可以有不同的参数类型,参数在template关键字后的尖括号中提供:

\begin{itemize}
\item
类型模板参数,其中参数是类型的占位符。这是前面部分所有示例的情况。

\item
非类型模板参数,其中参数是结构类型的一个值。整数类型、浮点类型(C++20)、指针类型、枚举类型和左值引用类型都是结构类型。下面的示例中,T是类型模板参数,而S是非类型模板参数:

\begin{cpp}
template <typename T, std::size_t S>
std::array<T, S> make_array()
{
    return std::array<T, S>{};
}
\end{cpp}
\end{itemize}

C++17起,非类型模板参数可以使用auto关键字声明:

\begin{cpp}
template <typename T, auto S>
std::array<T, S> make_array()
{
    return std::array<T, S>{};
}
\end{cpp}

\begin{itemize}
\item
双重模板参数,其中参数的类型是另一种类型。下面的示例中,trimin函数模板有两个模板参数,一个是类型模板参数T,另一个是双重模板参数M:

\begin{cpp}
template <typename T>
struct Minimum
{
    T operator()(T a, T b)
    {
        return a <= b ? a : b;
    }
};
template <typename T, template <typename> class M>
T trimin(T a, T b, T c)
{
    return M<T>{}(a, M<T>{}(b, c));
}
trimin<int, Minimum>(5, 2, 7);
\end{cpp}
\end{itemize}

虽然模板允许为多种类型(或更一般地说,为多种模板参数)编写一个实现,但有时为不同类型提供修改后的实现也有意义,甚至是必要的。为某些模板参数提供替代实现的过程称为特化,特化的模板称为主模板。有两种可能的形式:

\begin{itemize}
\item
偏特化是指只为一些模板参数提供不同的实现。

\item
全特化是指为整个一组模板参数提供不同的实现。
\end{itemize}

函数模板仅支持全特化,偏特化仅适用于类模板。“How to do it”部分提供了一个全特化的例子,用const char*类型特化了minimum<T>函数模板。这里,不是按字典顺序比较函数的两个参数,而是根据其长度决定哪个更“小”。

函数模板可以像其他函数一样重载。当存在多个重载版本,既有模板也有非模板时,编译器会优先选择非模板重载。前面已经提供了示例,这里只展示函数声明:

\begin{cpp}
template <typename T>
std::basic_string<T> minimum(std::basic_string<T> a, std::basic_string<T> b);
std::string minimum(std::string a, std::string b);
minimum(std::string("def"), std::string("acxyz"));
minimum(std::wstring(L"def"), std::wstring(L"acxyz"));
\end{cpp}

第一次调用minimum函数时,参数是std::string类型,将调用非模板重载。第二次调用时,参数是std::wstring类型,由于函数模板是唯一的匹配重载,将调用其std::wstring实例化。

调用函数模板时指定模板参数并不总是必要的。以下两次调用等价:

\begin{cpp}
minimum(1, 2);
minimum<int>(1, 2);
\end{cpp}

许多情况下,编译器可以从函数的调用中推导出模板参数。这个例子中,由于两个函数参数都是整数,编译器可以确定模板参数应该是int类型,没必要显式指定。然而,也有一些情况下编译器无法推导出类型。在这种情况下,必须显式提供类型:

\begin{cpp}
minimum(1, 2u); // 错误,模棱两可的模板参数T
\end{cpp}

两个参数一个是int,另一个是unsigned int。编译器不知道T类型应该是int还是unsigned int。为了解决这种模棱两可的情况,必须显式提供模板参数:

\begin{cpp}
minimum<unsigned>(1, 2u); // OK
\end{cpp}

推导模板参数时,编译器会在模板参数和用于调用函数的参数之间进行比较。为了使比较成功并且编译器能够成功推导出所有参数,这些参数必须具有一定的结构。然而,深入探讨这个过程超出了本示例的范围。可以查阅其他资源,包括我的书《Template Metaprogramming with C++》的第4章,也对此进行了详细讨论。

正如引言中提到的,模板是一个很大的主题,不可能在一个示例中完全讲透。接下来的两个示例中,将介绍带有可变数量参数的函数模板。

