
上一个示例中,实现了两个高阶函数,map 和 fold,并看到了使用它们的各种示例。最后,了解了如何管道化,以在原始数据经过多次转换后产生最终值。管道化是一种组合形式,可从两个或更多给定的函数创建一个新函数。提到的例子中,并没有组合函数;只是用另一个函数产生的结果来调用一个函数,但在本示例中,将介绍如何真正地将函数组合成一个新的函数。为了简化,只考虑一元函数(只接受一个参数的函数)。

\mySubsubsection{}{Getting ready}

继续之前,建议阅读前一个示例。虽然与理解本示例不强相关,但本示例会引用在那里实现的 map 和 fold 函数。

\mySubsubsection{}{How to do it...}

要将一元函数组合成一个高阶函数:

\begin{itemize}
\item
为了组合两个函数,提供一个函数,该函数接受两个函数 f 和 g 作为参数,并返回一个新函数(一个 lambda),该函数返回 f(g(x)),其中 x 是组合函数的参数:

\begin{cpp}
template <typename F, typename G>
auto compose(F&& f, G&& g)
{
    return [=](auto x) { return f(g(x)); };
}
auto v = compose(
    [](int const n) {return std::to_string(n); },
    [](int const n) {return n * n; })(-3); // v = "9"
\end{cpp}

\item
为了组合可变数量的函数,提供上述函数的一个可变参数模板重载版本:

\begin{cpp}
template <typename F, typename... R>
auto compose(F&& f, R&&... r)
{
    return [=](auto x) { return f(compose(r...)(x)); };
}
auto n = compose(
    [](int const n) {return std::to_string(n); },
    [](int const n) {return n * n; },
    [](int const n) {return n + n; },
    [](int const n) {return std::abs(n); })(-3); // n = "36"
\end{cpp}
\end{itemize}

\mySubsubsection{}{How it work...}

将两个一元函数组合成一个新函数相对简单。创建一个模板函数,就像前面例子中的 compose(),有两个参数——f 和 g,代表函数,并返回一个接受一个参数 x 的函数,该函数返回 f(g(x))。重要的是,g 函数返回的值的类型应该与 f 函数的参数类型相同。compose 函数的返回值是一个闭包,也就是一个 lambda 表达式的实例。

实际上,能够组合多个函数非常有用,可以通过编写 compose() 函数的可变参数模板版本来实现。

可变参数模板通过展开参数包来隐含编译时递归。这个实现与 compose() 的第一个版本非常相似,除了以下几点:

\begin{itemize}
\item
接受可变数量的函数作为参数。

\item
返回的闭包通过展开参数包递归地调用 compose();只剩下两个函数时,递归结束,调用先前实现的重载函数。
\end{itemize}

即使代码看起来像是发生了递归,但这并不是真正的递归,这可以称为编译时递归。但每次展开都会调用另一个具有相同名称但参数数量不同的方法,这并不构成递归。

实现了这些可变参数模板重载,就可以重写前一示例中的最后一个例子:

\begin{cpp}
auto s = compose(
    [](std::vector<int> const & v) {
        return fold_left(std::plus<>(), v, 0); },
    [](std::vector<int> const & v) {
        return mapf([](int const i) {return i + i; }, v); },
    [](std::vector<int> const & v) {
        return mapf([](int const i) {return std::abs(i); }, v); })(vnums);
\end{cpp}

有了一个初始的整数vector,可通过将 std::abs() 应用于每个元素将其映射到一个只有正值的vector。然后,通过将每个元素的值加倍,将结果映射到新的vector。最后,将结果vector中的值加到初始值 0 上,将这些值折叠在一起。

\mySubsubsection{}{There's more...}

组合通常由点(.)或星号(*)表示,如 f . g 或 f * g。实际上,可以在 C++ 中做类似的事情,通过重载操作符 *(尝试重载点操作符是没有意义的)。类似于 compose() 函数,操作符 * 应该能够处理任何数量的参数;因此,将有两个重载,就像 compose() 情况下的一样:

\begin{itemize}
\item
第一个重载接受两个参数,并调用 compose() 返回一个新的函数。

\item
第二个重载是可变参数模板函数,再次通过展开参数包调用操作符 *。
\end{itemize}

根据这些考虑,可以如下实现操作符 *:

\begin{cpp}
template <typename F, typename G>
auto operator*(F&& f, G&& g)
{
    return compose(std::forward<F>(f), std::forward<G>(g));
}
template <typename F, typename... R>
auto operator*(F&& f, R&&... r)
{
    return operator*(std::forward<F>(f), r...);
}
\end{cpp}

现在可以通过应用操作符 * 来简化函数的实际组合:

\begin{cpp}
auto n =
    ([](int const n) {return std::to_string(n); } *
    [](int const n) {return n * n; } *
    [](int const n) {return n + n; } *
    [](int const n) {return std::abs(n); })(-3); // n = "36"
auto c =
    [](std::vector<int> const & v) {
        return fold_left(std::plus<>(), v, 0); } *
    [](std::vector<int> const & v) {
        return mapf([](int const i) {return i + i; }, v); } *
    [](std::vector<int> const & v) {
        return mapf([](int const i) {return std::abs(i); }, v); };
auto vnums = std::vector<int>{ 2, -3, 5 };
auto s = c(vnums); // s = 20
\end{cpp}

一开始可能不直观,但函数是以相反的顺序而不是文本中显示的顺序应用的。第一个例子中,保留了参数的绝对值。然后,结果翻倍,再将该操作的结果乘以自身。最后,结果转换为字符串。对于提供的参数 -3,最终结果是字符串 "36"。

