编译是指源代码被转换成机器码并组织成目标文件的过程,这些目标文件随后会链接起来生成可执行文件。实际上,编译器一次只能处理一个文件(称为翻译单元),该文件由预处理器(负责处理预处理指令的编译器部分)从单个源文件及其包含的所有头文件生成,但这只是编译代码时发生过程的简化描述。本章讨论与预处理和编译相关的话题,重点介绍条件编译的各种方法,同时也涉及其他现代话题,如使用属性提供实现定义的语言扩展。

本章包括的如下内容:

\begin{itemize}
\item
有条件地编译源代码

\item
使用间接模式进行预处理字符串化和连接

\item
使用 static\_assert 进行编译时断言检查

\item
使用 enable\_if 条件性地编译类和函数

\item
使用 constexpr if 在编译时选择分支

\item
使用属性向编译器提供元数据
\end{itemize}
