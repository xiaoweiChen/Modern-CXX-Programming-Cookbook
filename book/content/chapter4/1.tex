
条件编译是一种简单的机制,允许开发人员维护单一的代码库,但只将代码的某些部分考虑进编译过程,以生成不同的可执行文件,通常是为运行在不同平台或硬件上,或者依赖不同的库或库版本。常见的例子包括基于编译器、平台(x86、x64、ARM 等)、配置(调试或发布)、或用户定义的具体条件来使用或忽略代码。本示例中,将介绍条件编译如何工作。

\mySubsubsection{}{Getting ready}

条件编译技术广泛用于许多目的。本示例中,将了解几个例子,并解释其工作原理。这项技术并不局限于这些例子。对于本示例的范围,只会考虑三大编译器:GCC、Clang 和 VC++。

\mySubsubsection{}{How to do it...}

要条件性地编译代码段,可以使用 \verb|#if|、\verb|#ifdef| 和 \verb|#ifndef| 指令(与 \verb|#elif|、\verb|#else| 和 \verb|#endif| 指令一起使用)。条件编译的一般形式为:

\begin{cpp}
#if condition1
    text1
#elif condition2
    text2
#elif condition3
    text3
#else
    text4
#endif
\end{cpp}

这里的条件意味着使用 \verb|defined identifier| 或 \verb|defined (identifier)| 语法检查宏是否已定义,也可以使用以下形式:

\begin{cpp}
#ifdef identifier1
    text1
#elifdef identifier2
    text2
#endif
#ifndef identifier1
    text1
#elifndef identifier2
    text2
#endif
\end{cpp}

\verb|#elifdef| 和 \verb|#elifndef| 指令在 C++23 中引入。

为了定义条件编译的宏,可以使用以下任一方法:

\begin{itemize}
\item
在源代码中使用 \verb|#define| 指令:

\begin{cpp}
#define VERBOSE_PRINTS
#define VERBOSITY_LEVEL 5
\end{cpp}

\item
使用特定于每个编译器的编译器命令行选项。最常用的编译器的例子如下:

\begin{itemize}
\item
Visual C++,使用 /Dname 或 /Dname=value(其中 /Dname 等同于 /Dname=1),例如 \verb|cl /DVERBOSITY_LEVEL=5|。

\item
GCC 和 Clang,使用 -D name 或 -D name=value(其中 -D name 等同于 -D name=1),例如 \verb|gcc -D VERBOSITY_LEVEL=5|。
\end{itemize}
\end{itemize}

以下是条件编译的一些例子:

\begin{itemize}
\item
头文件保护以避免重复定义(由于同一个翻译单元中多次包含相同的头文件):

\begin{cpp}
#ifndef UNIQUE_NAME
#define UNIQUE_NAME
class widget { };
#endif
\end{cpp}

\item
跨平台应用程序的编译器特定代码。以下是一个示例,用于在控制台打印一条消息,显示编译器名称:

\begin{cpp}
void show_compiler()
{
#if defined _MSC_VER
    std::cout << "Visual C++\n";
#elif defined __clang__
    std::cout << "Clang\n";
#elif defined __GNUC__
    std::cout << "GCC\n";
#else
    std::cout << "Unknown compiler\n";
#endif
}
\end{cpp}

\item
针对多个架构的目标特定代码,例如,条件性地编译适用于多个编译器和架构的代码:

\begin{cpp}
void show_architecture()
{
#if defined _MSC_VER
#if defined _M_X64
    std::cout << "AMD64\n";
#elif defined _M_IX86
    std::cout << "INTEL x86\n";
#elif defined _M_ARM
    std::cout << "ARM\n";
#else
    std::cout << "unknown\n";
#endif
#elif defined __clang__ || __GNUC__
#if defined __amd64__
    std::cout << "AMD64\n";
#elif defined __i386__
    std::cout << "INTEL x86\n";
#elif defined __arm__
    std::cout << "ARM\n";
#else
    std::cout << "unknown\n";
#endif
#else
#error Unknown compiler
#endif
}
\end{cpp}

\item
配置特定代码,条件性地编译用于调试和发布的代码:

\begin{cpp}
void show_configuration()
{
#ifdef _DEBUG
    std::cout << "debug\n";
#else
    std::cout << "release\n";
#endif
}
\end{cpp}

\item
要检查语言或特性是否可用,可使用预定义宏 \verb|__cpp_xxx| 对于语言特性(如 \verb|__cpp_constexpr|、\verb|__cpp_constinit| 或 \verb|__cpp_modules|)和 \verb|__cpp_lib_xxx| 对于特性(如 \verb|__cpp_lib_concepts|、\verb|__cpp_lib_expected| 或 \verb|__cpp_lib_jthread|)。特性宏是在 C++20 中引入的,可以在 <version> 头文件中找到:

\begin{cpp}
#ifdef __cpp_consteval
#define CONSTEVAL consteval
#else
#define CONSTEVAL constexpr
#endif
CONSTEVAL int twice(int const n)
{
    return n + n;
}
int main()
{
    twice(42);
}
\end{cpp}

\item
要检查某个头文件或源文件是否可用以包含,可以使用 \verb|__has_include| 指令(C++17)。以下示例检查 <optional> 头文件是否存在:

\begin{cpp}
#if __has_include(<optional>)
#include <optional>
template<class T> using optional_t = std::optional<T>;
#elif
#include "myoptional.h"
template<class T> using optional_t = my::optional<T>;
#endif
\end{cpp}

\item
要检查某个属性是否受支持(以及从哪个版本开始支持),可以使用 \verb|__has_cpp_attribute| 指令(C++20):

\begin{cpp}
#if defined(__has_cpp_attribute)
#if __has_cpp_attribute(deprecated)
#define DEPRECATED(msg) [[deprecated(msg)]]
#endif
#endif
DEPRECATED("This function is deprecated.")
void func() {}
\end{cpp}
\end{itemize}

\mySubsubsection{}{How it works...}

讨论编译之前,首先澄清一个经常会遇到的术语:翻译单元,这是编译的基本单位。是将源文件(一个 .cpp 文件)的内容和整个直接或间接包含的头文件合并的结果,但不包括本示例中描述的条件预处理语句排除的文本。

使用预处理指令 \verb|#if|、\verb|#ifndef|、\verb|#ifdef|、\verb|#elif|、\verb|#else| 和 \verb|#endif| 时,编译器最多会选择一个分支,该分支的内容将被包含在翻译单元中进行编译。这些指令的主体可以是文本,包括其他预处理指令:

\begin{itemize}
\item
\verb|#if|、\verb|#ifdef| 和 \verb|#ifndef| 必须有一个匹配的 \verb|#endif|。

\item
\verb|#if| 指令可以有多个 \verb|#elif| 指令,但只能有一个 \verb|#else|,并且必须是 \verb|#endif| 前的最后一个。

\item
\verb|#if|、\verb|#ifdef|、\verb|#ifndef|、\verb|#elif|、\verb|#else| 和 \verb|#endif| 可以嵌套。

\item
\verb|#if| 指令要求一个常量表达式,而 \verb|#ifdef| 和 \verb|#ifndef| 要求一个标识符。

\item
定义操作符可以用于预处理常量表达式,但仅限于 \verb|#if| 和 \verb|#elif| 指令中。

\item
如果标识符已定义,则 \verb|defined(identifier)| 认为是真;否则,认为是假。

\item
定义为空文本的标识符,也是已定义。

\item
\verb|#ifdef identifier| 等价于 \verb|#if defined(identifier)|。

\item
\verb|#ifndef identifier| 等价于 \verb|#if !defined(identifier)|。

\item
\verb|defined(identifier)| 和 \verb|defined identifier| 等价。
\end{itemize}

头文件保护是条件编译最常见的形式,这种方法用于防止同一翻译单元中多次包含头文件的内容(尽管每次都会扫描头文件以检测应包含的内容)。头文件中的代码会以防止多次包含的方式受到保护,如前一节给出的例子所示。考虑到给定的例子,这种方式的工作原理是,如果 \verb|UNIQUE_NAME| 宏(前一节中的通用名称)未定义,则 \verb|#if| 指令后的代码直到 \verb|#endif| 都会包含在翻译单元中编译。这时,\verb|UNIQUE_NAME| 宏就会通过 \verb|#define| 指令定义。下次在同一翻译单元中包含头文件时,\verb|UNIQUE_NAME| 宏已经定义,因此 \verb|#if| 指令体内的代码不会包含在翻译单元中,从而避免了重复。

\begin{myNotic}
注意,宏名称在整个应用程序中必须唯一;否则,只有第一个使用该宏的头文件中的代码会编译。其他使用相同名称的头文件中的代码将忽略。通常情况下,宏的名称基于定义它的头文件的名称。
\end{myNotic}

另一个重要的条件编译的例子是跨平台代码,这需要考虑不同的编译器和架构,通常是 Intel x86、AMD64 或 ARM ,但编译器要也会为其支持的平台定义自己的宏。"How to do it..." 部分的示例展示了如何为多个编译器和架构条件性地编译代码。

\begin{myNotic}
注意,上述示例中只考虑了几种架构。实际上,有多种宏可以用来识别相同的架构。确保在代码中使用这些类型的宏之前阅读每个编译器的文档。
\end{myNotic}

配置特定代码也通过宏和条件编译来处理。像 GCC 和 Clang 这样的编译器在使用 -g 标志时不会为调试配置定义特殊的宏。Visual C++ 在调试配置中确实定义了 \verb|_DEBUG|,如“How to do it...”的最后一个示例所示。对于其他编译器,需要显式地定义一个宏来识别这样的调试配置。

特性测试是条件编译的一个重要应用场景,特别是在那些提供多平台(Windows、Linux 等)和编译器版本(C++11、C++14、C++17 等)支持的库中。库的实现者通常需要检查某个特定的语言特性和语言属性是否可用。这可以通过一组预定义的宏来实现:

\begin{itemize}
\item
\verb|__cplusplus|:指示正在使用的 C++ 标准版本。会展开为以下值之一:对于 C++11 之前的版本为 199711L,对于 C++11 为 201103L,对于 C++14 为 201402L,对于 C++17 为 201703L,对于 C++20 为 202002L。本书撰写时,C++23 的值尚未定义。

\item
\verb|__cpp_xxx| 宏,用于确定是否支持某个语言特性。例子包括 \verb|__cpp_concepts|、\verb|__cpp_consteval|、\verb|__cpp_modules| 等。

\item
\verb|__cpp_lib_xxx| 宏,用于确定是否支持某个特性。例子包括 \verb|__cpp_lib_any|、\verb|__cpp_lib_optional|、\verb|__cpp_lib_constexpr_string| 等。这些宏定义在 C++20 引入的 <version> 头文件中。
\end{itemize}

随着新功能的添加,\verb|__cpp_xxx| 语言特性宏和 \verb|__cpp_lib_xxx| 库特性宏的集合正在扩大新的宏。这里展示的宏列表太长无法全部列出,但可以在 \url{https://en.cppreference.com/w/cpp/feature_test}查阅完整的列表。

除了这些宏之外,还有两个指令 \verb|__has_include| 和 \verb|__has_cpp_attribute|,可以在 \verb|#if/#elif| 表达式中使用,以确定某个头文件或源文件是否存在,或者编译器是否支持某个属性。所有这些宏和指令都是有用的工具,可以帮助确定某个特定的功能是否存在,能够编写跨平台和编译器版本的可移植代码。

\mySubsubsection{}{There's more...}

进行条件编译时,可能希望显示警告或完全停止编译。可以通过两个诊断宏的帮助来实现:

\begin{itemize}
\item
\verb|#error| 显示一条消息到控制台并停止程序的编译。

\item
自 C++23 起可用 \verb|#warning| 显示一条消息到控制台,但不停止程序的编译。
\end{itemize}

下面的代码展示了使用这些指令的例子:

\begin{cpp}
#ifdef _WIN64
#error "64-bit not supported"
#endif
#if __cplusplus < 201703L
#warning "Consider upgrading to a C++17 compiler"
#endif
\end{cpp}

虽然 \verb|#warning| 自 C++23 起才可用,但许多编译器作为扩展提供了对这个指令的支持。


