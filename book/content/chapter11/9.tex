
框架提供了支持,可以将固件(fixture)用作测试套件中所有测试的可重用组件。还提供了支持来设置测试运行的全局环境。本示例中,将逐步指导,了解如何定义和使用测试固件,以及如何设置测试环境。

\mySubsubsection{}{How to do it...}

要创建和使用测试固件,请按照以下步骤操作:

\begin{enumerate}
创建一个派生自 \verb|testing::Test|的类型:

\begin{cpp}
class TestFixture : public testing::Test
{
};
\end{cpp}

\item
使用构造函数初始化装置,使用析构函数清理装置:

\begin{cpp}
protected:
    TestFixture()
    {
        std::cout << "constructing fixture\n";
        data.resize(10);
        std::iota(std::begin(data), std::end(data), 1);
    }
    ~TestFixture()
    {
        std::cout << "destroying fixture\n";
    }
\end{cpp}

\item
或者,重写虚方法 \verb|SetUp()| 和 \verb|TearDown()|。

\item
向类型中添加成员数据和函数,以便在测试中使用:

\begin{cpp}
protected:
    std::vector<int> data;
\end{cpp}

\item
使用 \verb|TEST_F| 宏定义使用固件的测试,并将固件类名指定为测试套件名:

\begin{cpp}
TEST_F(TestFixture, TestData)
{
    ASSERT_EQ(data.size(), 10);
    ASSERT_EQ(data[0], 1);
    ASSERT_EQ(data[data.size()-1], data.size());
}
\end{cpp}
\end{enumerate}

为了自定义测试运行环境的设置,请按照以下步骤操作:

\begin{enumerate}
\item
创建一个派生自 \verb|testing::Environment| 的类型:

\begin{cpp}
class TestEnvironment : public testing::Environment
{
};
\end{cpp}

\item
重写虚方法 \verb|SetUp()| 和 \verb|TearDown()| 来执行设置和清理操作:

\begin{cpp}
public:
    virtual void SetUp() override
    {
        std::cout << "environment setup\n";
    }
    virtual void TearDown() override
    {
        std::cout << "environment cleanup\n";
    }
    int n{ 42 };
\end{cpp}

\item
调用 \verb|RUN_ALL_TESTS()| 之前,通过调用 \verb|testing::AddGlobalTestEnvironment()| 将环境注册到框架中:

\begin{cpp}
int main(int argc, char **argv)
{
    testing::InitGoogleTest(&argc, argv);
    testing::AddGlobalTestEnvironment(new TestEnvironment{});
    return RUN_ALL_TESTS();
}
\end{cpp}
\end{enumerate}

\mySubsubsection{}{How it works...}

测试固件使得用户能够在多个测试之间共享数据配置,固件实例不是在测试之间共享。每个关联于文本函数的测试都会创建一个新的固件实例。框架为每个来自固件的测试执行以下操作:

\begin{enumerate}
\item
创建一个新的固件实例。

\item
调用其 \verb|SetUp()| 虚方法。

\item
运行测试。

\item
调用装置的 \verb|TearDown()| 虚方法。

\item
销毁固件实例。
\end{enumerate}

可以使用两种方式来设置和清理固件实例:使用构造函数和析构函数,或者使用 \verb|SetUp()| 和 \verb|TearDown()| 虚方法。大多数情况下推荐前者,但使用虚方法在某些情况下也合适:

\begin{itemize}
\item
当清理操作抛出异常时,不允许异常从析构函数中传出。

\item
如果需要在清理过程中使用断言宏并且使用了 \verb|--gtest_throw_on_failure| 标志,该标志决定在发生失败时抛出哪些宏。

\item
如果需要调用可能会在派生类中被覆写的虚方法,不应该从构造函数或析构函数中调用虚方法。
\end{itemize}

使用固件的测试必须使用 \verb|TEST_F| 宏定义(其中 \verb|_F| 表示固件)。尝试使用 \verb|TEST| 宏声明将导致编译器错误。

测试运行的环境也可以自定义。机制类似于测试固件:从基础类 \verb|testing::Environment| 派生,并重写 \verb|SetUp()| 和 \verb|TearDown()| 虚函数。这些派生环境类的实例必须通过 \verb|testing::AddGlobalTestEnvironment()| 注册到框架中,但这必须在运行测试之前完成。可以注册任意数量的实例,此时 \verb|SetUp()| 方法会按注册顺序为实例调用,而 \verb|TearDown()| 方法则按相反顺序调用。必须传递动态实例化的实例给这个函数。框架接管这些实例的所有权,并在程序终止前删除它们;因此,不要自己删除它们。

环境实例不可供测试访问,也不打算向测试提供数据。其目的是定制运行测试的全局环境。


