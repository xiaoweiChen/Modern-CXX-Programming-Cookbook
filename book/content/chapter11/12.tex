
Catch2 框架使用传统的测试用例和测试函数样式或带有场景和“给定-当-然后”(given-when-then)部分的行为驱动开发(BDD)样式编写测试。测试定义为测试用例的不同部分,并且可以根据需要嵌套得非常深。无论偏好哪种风格,测试都仅使用两个基本宏来定义。本示例将极少这两个宏是什么,以及它们是如何工作的。

\mySubsubsection{}{How to do it...}

要使用传统样式,包括测试用例和测试函数,编写测试,请按以下步骤操作:

\begin{itemize}
\item
使用 \verb|TEST_CASE| 宏定义一个带有名称(作为字符串)的测试用例,可选地,还可以提供关联标签列表:

\begin{cpp}
TEST_CASE("test construction", "[create]")
{
    // define sections here
}
\end{cpp}

\item
使用 SECTION 宏在测试用例内部定义一个测试函数,名称也是字符串:

\begin{cpp}
TEST_CASE("test construction", "[create]")
{
    SECTION("test constructor")
    {
        auto p = point3d{ 1,2,3 };
        REQUIRE(p.x() == 1);
        REQUIRE(p.y() == 2);
        REQUIRE(p.z() == 4);
    }
}
\end{cpp}

\item
如果希望重用设置和清理代码或以层次结构组织测试,则可以定义嵌套部分:

\begin{cpp}
TEST_CASE("test operations", "[modify]")
{
    SECTION("test methods")
    {
        SECTION("test offset")
        {
            auto p = point3d{ 1,2,3 };
            p.offset(1, 1, 1);
            REQUIRE(p.x() == 2);
            REQUIRE(p.y() == 3);
            REQUIRE(p.z() == 3);
        }
    }
}
\end{cpp}
\end{itemize}

要使用 BDD 样式编写测试,请按以下步骤操作:

\begin{itemize}
\item
使用 SCENARIO 宏定义场景,并为其指定名称:

\begin{cpp}
SCENARIO("modify existing object")
{
    // define sections here
}
\end{cpp}

\item
使用 GIVEN、WHEN 和 THEN 宏在场景中定义嵌套部分,并分别为每个部分指定名称:

\begin{cpp}
SCENARIO("modify existing object")
{
    GIVEN("a default constructed point")
    {
        auto p = point3d{};
        REQUIRE(p.x() == 0);
        REQUIRE(p.y() == 0);
        REQUIRE(p.z() == 0);
        WHEN("increased with 1 unit on all dimensions")
        {
            p.offset(1, 1, 1);
            THEN("all coordinates are equal to 1")
            {
                REQUIRE(p.x() == 1);
                REQUIRE(p.y() == 1);
                REQUIRE(p.z() == 1);
            }
        }
    }
}
\end{cpp}
\end{itemize}

要执行测试,请按以下步骤操作:

\begin{itemize}
\item
要执行程序中的所有测试(除隐藏的测试外),请在没有以下代码描述的命令行参数的情况下运行测试程序。

\item
要仅执行一组特定的测试用例,请提供一个过滤器作为命令行参数。这可以包含测试用例名称、通配符、标签名称和标签表达式:

\begin{shell}
chapter11ca_02.exe "test construction"
test construction
test constructor
-------------------------------------------------
f:\chapter11ca_02\main.cpp(7)
.................................................
f:\chapter11ca_02\main.cpp(12): FAILED:
REQUIRE( p.z() == 4 )
with expansion:
3 == 4
=================================================
test cases: 1 | 1 failed
assertions: 6 | 5 passed | 1 failed
\end{shell}

\item
要仅执行某个特定部分(或一组部分),请使用命令行参数 \verb|--section| 或 \verb|-c| 加上部分名称(可以多次使用以指定多个部分):

\begin{cpp}
chapter11ca_02.exe "test construction" --section "test origin"
Filters: test construction
==================================================
All tests passed (3 assertions in 1 test case)
\end{cpp}

\item
要指定测试用例的执行顺序,请使用命令行参数 \verb|--order| 加上以下值:decl(声明顺序)、lex(按名称字典序排列)或 rand(使用 \verb|std::random_shuffle()| 确定的随机顺序)。以下是对此的一个示例:

\begin{shell}
chapter11ca_02.exe --order lex
\end{shell}
\end{itemize}

\mySubsubsection{}{How it works...}

测试用例自我注册,除了定义测试用例和测试函数外,开发者无需设置测试程序。测试函数可定义为测试用例的部分(使用 SECTION 宏),并且可以嵌套。

SECTION的嵌套深度没有限制。测试用例和测试函数,从现在起将称为SECTION,形成了一棵树结构,测试用例位于根节点,最内层的部分作为叶子节点。当测试程序运行时,执行的是叶子部分。每个叶子部分都在与其他叶子部分隔离的状态下执行。然而,执行路径从根测试用例开始,向下延伸至最内层的部分。路径上的所有代码在每次运行时都会完全执行。当多个部分共享来自父部分或测试用例的公共代码时,相同的代码会为每个部分执行一次,而不会在执行之间共享数据。一方面,这消除了对特殊装置方法的需求;另一方面,允许多个装置为每个部分服务(路径上遇到的所有内容),这是许多测试框架所缺乏的功能。

编写测试用例的 BDD 样式由相同的两个宏提供动力,即 \verb|TEST_CASE| 和 SECTION,以及测试部分的能力。实际上,宏 SCENARIO 是 \verb|TEST_CASE| 的重新定义,而 GIVEN、WHEN、\verb|AND_WHEN|、THEN 和 \verb|AND_THEN| 是 SECTION 的重新定义:

\begin{cpp}
#define SCENARIO( ... ) TEST_CASE( "Scenario: " __VA_ARGS__ )
#define GIVEN(desc)     INTERNAL_CATCH_DYNAMIC_SECTION("    Given: " << desc)
#define AND_GIVEN(desc) INTERNAL_CATCH_DYNAMIC_SECTION("And given: " << desc)
#define WHEN(desc)      INTERNAL_CATCH_DYNAMIC_SECTION("     When: " << desc)
#define AND_WHEN(desc)  INTERNAL_CATCH_DYNAMIC_SECTION(" And when: " << desc)
#define THEN(desc)      INTERNAL_CATCH_DYNAMIC_SECTION("     Then: " << desc)
#define AND_THEN(desc)  INTERNAL_CATCH_DYNAMIC_SECTION("      And: " << desc)
\end{cpp}

当执行测试程序时,所有定义的测试都会运行。但这不包括使用以 . 开头的名字或以点开头的标签指定的隐藏测试。可以通过提供命令行参数 [.] 或 [hide] 强制运行隐藏测试。

可以使用名称或标签来过滤要执行的测试用例。下表显示了一些可能的选项:

% Please add the following required packages to your document preamble:
% \usepackage{longtable}
% Note: It may be necessary to compile the document several times to get a multi-page table to line up properly
\begin{longtable}{|l|l|}
\hline
\textbf{参数}   & \textbf{说明}                                     \\ \hline
\endfirsthead
%
\endhead
%
"test construction" & 名称为 test construction 的测试用例                   \\ \hline
test*               & 所有以 test 开头的测试用例                      \\ \hline
$\sim$"test construction"          & 除名为 test construction 的测试用例外的所有测试用例                                      \\ \hline
$\sim$*equal*       & 不包含单词 equal 的所有测试用例 \\ \hline
a* $\sim$ab* abc                   & 所有以 a 开头的测试用例,但排除以 ab 开头的,abc 除外 \\ \hline
{[}modify{]}        & 标记为 {[}modify{]} 的所有测试用例                   \\ \hline
{[}modify{]},{[}compare{]}{[}op{]} & \begin{tabular}[c]{@{}l@{}}标记为 {[}modify{]} 或同时标记为 {[}compare{]} 和 {[}op{]} 的\\所有测试用例\end{tabular}  \\ \hline
-\#sourcefile       & 来自 sourcefile.cpp 文件的所有测试                  \\ \hline
\end{longtable}

\begin{center}
表 11.5: 测试用例执行过滤器示例
\end{center}

通过使用命令行参数 \verb|--section| 或 \verb|-c| 指定一个或多个部分名称,也可以执行特定的测试函数,但此选项不支持通配符。如果指定了要运行的部分,从根测试用例到选定部分的整个测试路径都将执行。此外,如果没有首先指定测试用例或一组测试用例,那么所有测试用例都将被执行,尽管只有其中匹配的部分会执行。

