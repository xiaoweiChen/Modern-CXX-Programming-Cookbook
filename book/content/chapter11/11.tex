
Catch2 是一个多范式测试框架,适用于 C++ 和 Objective-C。Catch2 的命名延续了其第一版框架 Catch,后者代表 C++ Automated Test Cases in Headers(C++ 中的自动化测试用例头文件)。允许开发者使用传统的测试函数组合成测试用例的方式编写测试,或者采用行为驱动开发(Behavior-Driven Development, BDD)风格,使用“给定-当-然后”(given-when-then)部分。测试是自我注册的,框架提供了多种断言宏;其中最常用的有两个:一个是致命的(即,REQUIRE),另一个是非致命的(即,CHECK)。它们对左右两边的值进行表达式分解,如果测试失败,这些值会记录下来。与第一个版本不同,Catch2 不再支持 C++03。目前版本的 Catch2 是 v3,与 Catch2 v2 相比有显著的变化,例如库不再是一个单一头文件库,而是作为一个常规库工作(需要编译),并且需要 C++14 编译器。

\begin{myNotic}
将介绍如何使用 Catch2 版本 3 编写单元测试。
\end{myNotic}

\mySubsubsection{}{Getting ready}

Catch2 测试框架具有基于宏的 API。只需使用提供的宏编写测试,但想要很好地使用该框架,建议对宏有一个良好的理解。

\mySubsubsection{}{How to do it...}

为了使用 Catch2 测试框架,请按照以下步骤操作:

\begin{enumerate}
\item
从 \url{https://github.com/catchorg/Catch2} 克隆或下载 Git 存储库。

\item
下载存储库后,解压存档。
\end{enumerate}

使用 Catch2 v3 版本,有两种选择:

\begin{itemize}
\item
在测试项目中使用合并后的(amalgamated)头文件和源文件,这些文件称为 \verb|catch_amalgamated.hpp| 和 \verb|catch_amalgamated.cpp|。位于 Catch2 库的 extras 文件夹中,如果需要,可以将它们复制到测试项目中。这样做的一大优点是不必处理 CMake 脚本,但代价是增加构建时间。

\item
使用 CMake 将 Catch2 添加为项目的静态库。
\end{itemize}

要使用 Catch2 及其合并文件创建第一个测试程序,请按照以下步骤操作:

\begin{enumerate}
\item
创建一个新的空 C++ 项目。

\item
将 \verb|catch_amalgamated.hpp| 和 \verb|catch_amalgamated.cpp| 文件从 Catch2 库的 extras 文件夹复制到测试项目中。

\item
将 \verb|catch_amalgamated.cpp| 源文件添加到项目中,以便与其他源文件(包含测试)一起编译。

\item
向项目中添加一个新源文件:

\begin{cpp}
#include "catch_amalgamated.hpp"
TEST_CASE("first_test_case", "[learn][catch]")
{
    SECTION("first_test_function")
    {
        auto i{ 42 };
        REQUIRE(i == 42);
    }
}
\end{cpp}

\item
构建并运行项目。
\end{enumerate}

要使用 CMake 集成创建第一个 Catch2 测试程序,请按照以下步骤操作:

\begin{enumerate}
\item
打开控制台/命令提示符,并将目录更改为已克隆/解压的 Catch2 文件的位置。

\item
使用以下命令构建库。在 Unix 系统上,运行:

\begin{shell}
cmake -Bbuild -H. -DBUILD_TESTING=OFF
sudo cmake --build build/ --target install
\end{shell}

在 Windows 系统上,以管理员权限运行命令提示符,执行以下命令:

\begin{shell}
cmake -Bbuild -H. -DBUILD_TESTING=OFF
cmake --build build/ --target install
\end{shell}

\item
为 C++ 测试项目创建一个新文件夹(称为 Test)。

\item
向此文件夹中添加一个新源文件(名为 main.cpp):

\begin{cpp}
#include <catch2/catch_test_macros.hpp>
TEST_CASE("first_test_case", "[learn][catch]")
{
    SECTION("first_test_function")
    {
        auto i{ 42 };
        REQUIRE(i == 42);
    }
}
\end{cpp}

\item
向 Test 文件夹中添加一个 CMakeLists.txt文件:

\begin{cmake}
find_package(Catch2 3 REQUIRED)
add_executable(Test main.cpp)
target_link_libraries(Test PRIVATE Catch2::Catch2WithMain)
\end{cmake}

\item
运行 cmake.exe 生成/构建项目。
\end{enumerate}

\begin{myNotic}
使用 CMake 设置项目的方法有很多种。本示例中,提供了一个小例子,也可以在 GitHub 存储库的源文件中找到。熟悉 CMake 的读者可能会发现比这里提供的更好的方法。可以通过在线资源了解更多关于 CMake 的信息。
\end{myNotic}

\mySubsubsection{}{How it works...}

Catch2 能够编写自我注册的测试用例函数,甚至提供一个默认的 main() 函数实现,以便可以专注于测试代码并减少设置代码的编写量。测试用例划分为独立运行的部分。框架并不遵循“设置-测试-清理”架构的风格。相反,测试用例部分(或更确切地说是最内层的部分,部分可以嵌套)作为测试单元被执行,同时还有所包围的部分。这使得对固件的需求变得多余,可以在多个级别上重复使用数据和设置及清理代码。

测试用例和部分通过字符串而非标识符(如同大多数测试框架那样)来识别。测试用例还可以打上标签,以便根据标签执行或列出测试。测试结果以文本形式打印给人类阅读,也可以导出为 XML,使用特定于 Catch2 的模式或 JUNIT ANT 模式以方便与持续交付系统的集成。测试的执行可以参数化,以便在失败时中断(在 Windows 和 macOS 上),这样可以添加调试器并检查程序。

框架易于安装和使用。如“How to do it…”部分所述,有两种替代方案:

\begin{itemize}
\item
使用合并文件 \verb|catch_amalgamated.hpp| 和 \verb|catch_amalgamated.cpp|。这是所有头文件和源文件的合并。优点是不必担心构建 Catch2 库,只需要将这些文件复制到所需位置(通常是在项目内部),在包含测试的文件中包含 \verb|catch_amalgamated.hpp| 头文件,并将 \verb|catch_amalgamated.cpp| 与其他源文件一起构建。缺点是增加了构建时间。

\item
将 Catch2 用作静态库。这要求使用之前构建库。显式地将头文件和库文件添加到项目中,或者为此目的使用 CMake。这种方法的优势在于减少了构建时间。
\end{itemize}

上一节中展示的示例代码包含以下部分:

\begin{enumerate}
\item
\verb|#include "catch_amalgamated.hpp"| 包含了库的合并头文件,这是所有库头文件的合并。另一方面,如果您使用的是构建版本,只需包含所需的特定头文件,例如 \verb|<catch2/catch_test_macros.hpp>|。虽然可以包含 \verb|<catch2/catch_all.hpp>|,但这将包含所有库头文件,不建议这样用。一般来说,应该只包含所需的头文件。

\item
\verb|TEST_CASE("first_test_case", "[learn][catch]")| 定义了一个名为 \verb|first_test_case| 的测试用例,有两个相关的标签:learn 和 catch。标签用于选择运行或仅列出测试用例。多个测试用例可以带有相同的标签。

\item
\verb|SECTION("first_test_function")| 定义了一个部分,即一个名为 \verb|first_test_function| 的测试函数,作为外部测试用例的一部分。

\item
\verb|REQUIRE(i == 42);| 是一个断言,告诉测试如果条件不满足则失败。
\end{enumerate}

运行此程序的输出如下:

\begin{shell}
=========================================================
All tests passed (1 assertion in 1 test cases)
\end{shell}

\mySubsubsection{}{There's more...}

如前所述,框架能够使用带有“给定-当-然后”(given-when-then)部分的行为驱动开发(BDD)风格编写测试。这是通过几个别名实现的:\verb|SCENARIO| 对应 \verb|TEST_CASE|,\verb|GIVEN|、\verb|WHEN|、\verb|AND_WHEN|、\verb|THEN| 和 \verb|AND_THEN| 对应 \verb|SECTION|。使用这种风格,可以重写之前展示的测试:

\begin{cpp}
SCENARIO("first_scenario", "[learn][catch]")
{
    GIVEN("an integer")
    {
        auto i = 0;
        WHEN("assigned a value")
        {
            i = 42;
            THEN("the value can be read back")
            {
                REQUIRE(i == 42);
            }
        }
    }
}
\end{cpp}

成功执行时,程序输出以下输出:

\begin{shell}
=========================================================
All tests passed (1 assertion in 1 test cases)
\end{shell}

失败时(假设我们得到了错误的条件:i == 0),失败的表达式以及左右两边的值将在输出中输出出来:

\begin{shell}
---------------------------------------------------------------
f:\chapter11ca_01\main.cpp(11)
...............................................................
f:\chapter11ca_01\main.cpp(13): FAILED:
REQUIRE( i == 0 )
with expansion:
42 == 0
===============================================================
test cases: 1 | 1 failed
assertions: 1 | 1 failed
\end{shell}

这里展示的输出,以及后续示例中的其他片段,都是从实际控制台输出中稍微修剪或压缩过的,以便更容易地在页面中列出。


