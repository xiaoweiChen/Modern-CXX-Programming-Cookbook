
框架为提供了自定义测试日志,和测试报告中显示的内容以及格式化结果的能力。目前支持两种格式:人类可读格式(或HRF)和XML(还包括JUNIT格式用于测试日志),也可以创建并添加自己的格式。

\begin{myNotic}
人类可读格式是指可以阅读的数据编码形式。无论是以ASCII还是Unicode编码的文本,都用于此目的。
\end{myNotic}

输出内容的配置既可以在运行时通过命令行开关进行,也可以在编译时通过各种API完成。测试执行期间,框架会收集所有的事件到日志中。最后,会产生一个报告,该报告代表了执行情况的摘要,包含不同程度的细节。发生失败的情况下,报告包含了关于位置和原因的详细信息,包括实际值和预期值。这有助于开发人员快速识别错误。这个示例中,将看到如何控制写入日志和报告的内容及其格式,将在运行时使用命令行选项来实现。

\mySubsubsection{}{Getting ready}

对于本示例中的例子,将使用以下测试模块:

\begin{cpp}
#define BOOST_TEST_MODULE Controlling output
#include <boost/test/included/unit_test.hpp>
BOOST_AUTO_TEST_CASE(test_case)
{
    BOOST_TEST(true);
}
BOOST_AUTO_TEST_SUITE(test_suite)
BOOST_AUTO_TEST_CASE(test_case)
{
    int a = 42;
    BOOST_TEST(a == 0);
}
BOOST_AUTO_TEST_SUITE_END()
\end{cpp}

下一节将介绍如何通过命令行选项控制测试日志和测试报告的输出。

\mySubsubsection{}{How to do it...}

要控制测试日志的输出,请按照以下步骤操作:

\begin{itemize}
\item
使用 \verb|--log_format=<format>| 或 \verb|-f <format>| 命令行选项来指定日志格式。可能的格式有HRF(默认值)、XML和JUNIT。

\item
使用 \verb|--log_level=<level>| 或 \verb|-l <level>| 命令行选项来指定日志级别。日志级别包括错误(HRF和XML的默认值)、警告、全部和成功(JUNIT的默认值)。

\item
使用 \verb|--log_sink=<stream or file name>| 或 \verb|-k <stream or file name>| 命令行选项来指定框架应将测试日志写入的位置。可能的选项有stdout(HRF和XML的默认值)、stderr或任意文件名(JUNIT的默认值)。
\end{itemize}

要控制测试报告的输出,请按照以下步骤操作:

\begin{itemize}
\item
使用 \verb|--report_format=<format>| 或 \verb|-m <format>| 命令行选项来指定报告格式。格式有HRF(默认值)和XML。

\item
使用 \verb|--report_level=<level>| 或 \verb|-r <level>| 命令行选项来指定报告级别。格式有确认(默认值)、无(不产生报告)、简短和详细。

\item
使用 \verb|--report_sink=<stream or file name>| 或 \verb|-e <stream or file name>| 命令行选项来指定框架应将报告日志写入的位置。选项有stderr(默认值)、stdout或任意文件名。
\end{itemize}

\mySubsubsection{}{How it works...}

在控制台/终端运行测试模块时,会看到测试日志和测试报告,测试报告紧随测试日志之后。对于前面展示的测试模块,默认输出如下所示。前三行代表测试日志,而最后一行代表测试报告:

\begin{shell}
Running 2 test cases...
f:/chapter11bt_05/main.cpp(14): error: in "test_suite/test_case":
check a == 0 has failed [42 != 0]
*** 1 failure is detected in the test module "Controlling output"
\end{shell}

测试日志和测试报告的内容可以以多种格式提供,默认格式为HRF;框架还支持XML格式,对于测试日志,还支持JUNIT格式。这是一种旨在供自动化工具使用的格式,如持续构建或集成工具。除了这些选项外,还可以通过实现派生自 \verb|boost::unit_test::unit_test_log_formatter| 的类来自定义测试日志的格式。

下面的例子展示了如何使用XML格式化测试日志(第一个例子)和测试报告(第二个例子)(每个格式化部分用粗体标出):

\begin{shell}
chapter11bt_05.exe -f XML
<TestLog><Error file="f:/chapter11bt_05/main.cpp"

line="14"><![CDATA[check a == 0 has failed [42 != 0]]]>

</Error></TestLog>

*** 1 failure is detected in the test module "Controlling output"
chapter11bt_05.exe -m XML
Running 2 test cases...
f:/chapter11bt_05/main.cpp(14): error: in "test_suite/test_case":
check a == 0 has failed [42 != 0]
<TestResult><TestSuite name="Controlling output" result="failed"

assertions_passed="1" assertions_failed="1" warnings_failed="0"

expected_failures="0" test_cases_passed="1"

test_cases_passed_with_warnings="0" test_cases_failed="1"

test_cases_skipped="0" test_cases_aborted="0"></TestSuite>

</TestResult>
\end{shell}

日志或报告级别代表输出的详细程度。日志的详细程度级别的可能值如下表所示,按从最低到最高级别排序。表格中较高的级别包含其上所有级别的消息:

% Please add the following required packages to your document preamble:
% \usepackage{longtable}
% Note: It may be necessary to compile the document several times to get a multi-page table to line up properly
\begin{longtable}{|l|l|}
\hline
\textbf{级别} & \textbf{报告的消息}                                       \\ \hline
\endfirsthead
%
\endhead
%
nothing        & 没有任何日志记录。                                                        \\ \hline
fatal\_error & \begin{tabular}[c]{@{}l@{}}系统或用户致命错误及所有描述 REQUIRE 级别断言失败的消息\\(例如 BOOST\_TEST\_REQUIRE 和 BOOST\_REQUIRE)。\end{tabular} \\ \hline
system\_error  & 系统非致命错误。                                                  \\ \hline
cpp\_exception & 未捕获的 C++ 异常。                                                   \\ \hline
error          & CHECK 级别断言失败(BOOST\_TEST 和 BOOST\_CHECK\_)。     \\ \hline
warning        & WARN 级别断言失败 (BOOST\_TEST\_WARN 和 BOOST\_WARN\_). \\ \hline
message        & BOOST\_TEST\_MESSAGE 生成的消息。                               \\ \hline
test\_suite    & 每个测试单元开始和结束状态的通知。            \\ \hline
all/success    & 所有消息,包括通过的断言。                            \\ \hline
\end{longtable}

\begin{center}
表 11.2:日志详细程度级别的可能值
\end{center}

测试报告的可用格式如下表所述:

% Please add the following required packages to your document preamble:
% \usepackage{longtable}
% Note: It may be necessary to compile the document several times to get a multi-page table to line up properly
\begin{longtable}{|l|l|}
\hline
\textbf{级别} & \textbf{描述}                                                             \\ \hline
\endfirsthead
%
\endhead
%
no             & 不产生报告。                                                           \\ \hline
confirm &
\begin{tabular}[c]{@{}l@{}}通过的测试:\\ *** 未检测到错误。\\ 跳过的测试:\\ ***  测试套件<name>被跳过; 请参阅标准输出以获取详情。\\ 中断的测试:\\ ***  测试套件<name>被中断;请参阅标准输出以获取详情。\\ 断言失败但测试未失败:\\ *** 在测试套件<name>中检测到错误;请参阅标准输出以获取详情。\\ 失败的测试:\\ ***在测试套件<name>中检测到 N 个失败。\\ 部分预期失败的测试:\\ *** 在测试套件 <name>中检测到 N 个失败(预期 M 个失败)。\end{tabular} \\ \hline
detailed &
\begin{tabular}[c]{@{}l@{}}结果以层次结构方式报告(每个测试单元作为父测试单元的一部分报告),\\但仅显示相关信息。没有断言失败的测试用例不会在报告中生成条目。\\ 测试用例/套件 \textless{}name\textgreater 已通过/被跳过/被中断/失败:\\ N 个断言中有 M 个通过\\ N 个断言中有 M 个失败\\ N 个警告中有 M 个失败\\ 预期 X 个失败\end{tabular} \\ \hline
short          & 类似于详细,但只报告主测试套件的信息。 \\ \hline
\end{longtable}

\begin{center}
表 11.3:测试报告的可用格式
\end{center}

标准输出流(stdout)是测试日志的默认写入位置,标准错误流(stderr)是测试报告的默认位置。然而,测试日志和测试报告都可以重定向到另一个流或文件。

除了这些选项外,还可以使用 \verb|--report_memory_leaks_to=<file name>| 命令行选项指定用于报告内存泄漏的单独文件。如果没有此选项且检测到内存泄漏,则报告给标准错误流。

\mySubsubsection{}{There's more...}

除了本示例中讨论的选项外,框架还提供了编译时API来控制输出。有关这些API的全面描述以及本示例中描述的功能,请查阅框架文档 \url{https://www.boost.org/doc/libs/1_83_0/libs/test/doc/html/index.html}。


