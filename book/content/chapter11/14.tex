
像本书中讨论的其他测试框架一样,Catch2 会以可读的格式将测试程序的执行结果输出到标准输出流 stdout。此外,还支持使用 XML 格式报告或写入文件等选项。本示例中,将介绍使用 Catch2 时控制输出的主要选项。

\mySubsubsection{}{Getting ready}

为了举例说明如何修改测试程序的执行输出,可以使用以下测试用例:

\begin{cpp}
EST_CASE("case1")
{
    SECTION("function1")
    {
        REQUIRE(true);
    }
}
TEST_CASE("case2")
{
    SECTION("function2")
    {
        REQUIRE(false);
    }
}
\end{cpp}

运行这两个测试用例的输出如下:

\begin{shell}
----------------------------------------------------------
case2
    function2
----------------------------------------------------------
f:\chapter11ca_04\main.cpp(14)
..........................................................
f:\chapter11ca_04\main.cpp(16): FAILED:
REQUIRE( false )
==========================================================
test cases: 2 | 1 passed | 1 failed
assertions: 2 | 1 passed | 1 failed
\end{shell}

接下来的部分中,将了解一些控制 Catch2 测试程序输出的各种选项。

\mySubsubsection{}{How to do it...}

使用 Catch2 控制测试程序的输出时:

\begin{itemize}
\item
使用命令行参数 \verb|-r| 或 \verb|--reporter <reporter>| 来指定用于格式化和结构化结果的报告器。框架提供的默认选项有 console、compact、xml 和 junit:

\begin{shell}
chapter11ca_04.exe -r junit
<?xml version="1.0" encoding="UTF-8"?>
<testsuites>
    <testsuite name="chapter11ca_04.exe" errors="0"
               failures="1"
               tests="2" hostname="tbd"
               time="0.002039"
               timestamp="2020-05-02T21:17:04Z">
        <testcase classname="case1" name="function1"
                  time="0.00016"/>
        <testcase classname="case2"
                  name="function2" time="0.00024">
            <failure message="false" type="REQUIRE">
                at f:\chapter11ca_04\main.cpp(16)
            </failure>
        </testcase>
    <system-out/>
    <system-err/>
    </testsuite>
</testsuites>
\end{shell}

\item
使用命令行参数 \verb|-s| 或 \verb|--success| 显示成功的测试用例的结果:

\begin{shell}
chapter11ca_04.exe -s
--------------------------------------------------
case1
    function1
--------------------------------------------------
f:\chapter11ca_04\main.cpp(6)
..................................................
f:\chapter11ca_04\main.cpp(8):
PASSED:
    REQUIRE( true )
--------------------------------------------------
case2
    function2
--------------------------------------------------
f:\chapter11ca_04\main.cpp(14)
..................................................
f:\chapter11ca_04\main.cpp(16):
FAILED:
    REQUIRE( false )
==================================================
test cases: 2 | 1 passed | 1 failed
assertions: 2 | 1 passed | 1 failed
\end{shell}

\item
使用命令行参数 \verb|-o| 或 \verb|--out <filename>| 将所有输出发送到文件而不是标准流:

\begin{shell}
chapter11ca_04.exe -o test_report.log
\end{shell}

\item
使用命令行参数 \verb|-d| 或 \verb|--durations <yes/no>| 显示每个测试用例执行所需的时间:

\begin{shell}
chapter11ca_04.exe -d yes
0.000 s: scenario1
0.000 s: case1
--------------------------------------------------
case2
    scenario2
--------------------------------------------------
f:\chapter11ca_04\main.cpp(14)
..................................................
f:\chapter11ca_04\main.cpp(16):
FAILED:
REQUIRE( false )
0.003 s: scenario2
0.000 s: case2
0.000 s: case2
==================================================
test cases: 2 | 1 passed | 1 failed
assertions: 2 | 1 passed | 1 failed
\end{shell}
\end{itemize}

\mySubsubsection{}{How it works...}

除了默认用于报告测试程序执行结果的人类可读格式外,Catch2 框架还支持两种 XML 格式:

\begin{itemize}
\item
特定于 Catch2 的 XML 格式(使用 \verb|-r xml| 指定)

\item
遵循 JUNIT ANT 任务结构的 JUNIT 类似 XML 格式(使用 \verb|-r junit| 指定)
\end{itemize}

前者报告器在单元测试执行且结果可用时流式传输 XML 内容,可以用作 XSLT 转换的输入,以生成实例的 HTML 报告。后者报告器需要收集所有程序执行数据,以便在输出前构建报告。JUNIT XML 格式对于第三方工具(如持续集成服务器)很有用。

几个报告器以独立头文件的形式提供,需要包含在测试程序的源代码中(所有报告器的头文件命名格式为 \verb|catch_reporter_*.hpp|)。这些可用的报告器包括:

\begin{itemize}
\item
TeamCity 报告器(使用 \verb|-r teamcity| 指定),将 TeamCity 服务消息写入标准输出流。仅适用于与 TeamCity 的集成。这是一个流式报告器;数据在可用时即写入。

\item
Automake 报告器(使用 \verb|-r automake| 指定),写入 automake 通过 make check 所期望的元标签。

\item
测试任意协议(或简称 TAP)报告器(使用 \verb|-r tap| 指定)。

\item
SonarQube 报告器(使用 \verb|-r sonarqube| 指定),使用 SonarQube 泛用测试数据 XML 格式写入。
\end{itemize}

下面的例子展示了如何包含 TeamCity 头文件,以使用 TeamCity 报告器生成报告:

\begin{cpp}
#include <catch2/catch_test_macros.hpp>
#include <catch2/reporters/catch_reporter_teamcity.hpp>
\end{cpp}

测试报告的默认目标是标准输出流 stdout(即使是显式写入 stderr 的数据最终也会重定向到 stdout),也可以将输出写入文件。这些格式化选项可以组合使用:

\begin{shell}
hapter11ca_04.exe -r junit -o test_report.xml
\end{shell}

此命令指定了报告应使用 JUNIT XML 格式,并保存到名为 \verb|test_report.xml| 的文件中。

