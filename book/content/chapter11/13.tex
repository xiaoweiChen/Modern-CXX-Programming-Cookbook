
与其它测试框架不同,Catch2 并未提供大量断言宏。主要提供了两个宏:REQUIRE,在失败时产生致命错误并停止测试用例的执行;CHECK,在失败时产生非致命错误并继续执行测试用例。本示例中,还会介绍一些附加宏的使用。

\mySubsubsection{}{How to do it...}

下面列出了使用 Catch2 框架进行断言的可用选项:

\begin{itemize}
\item
使用 CHECK(expr) 检查 expr 是否为真,即使失败也会继续执行;使用 REQUIRE(expr) 确保 expr 为真,若失败则停止测试用例的执行:

\begin{cpp}
int a = 42;
CHECK(a == 42);
REQUIRE(a == 42);
\end{cpp}

\item
使用 \verb|CHECK_FALSE(expr)| 和 \verb|REQUIRE_FALSE(expr)| 确保 expr 为假,并在失败时产生非致命或致命错误:

\begin{cpp}
int a = 42;
CHECK_FALSE(a > 100);
REQUIRE_FALSE(a > 100);
\end{cpp}

\item
使用浮点数匹配器 WithinAbs、WithinRel 和 WithinUPL 比较浮点数(推荐替代已废弃的 Approx 类型):

\begin{cpp}
double a = 42.5;
CHECK_THAT(42.0, Catch::Matchers::WithinAbs(a, 0.5));
REQUIRE_THAT(42.0, Catch::Matchers::WithinAbs(a, 0.5));
CHECK_THAT(42.0, Catch::Matchers::WithinRel(a, 0.02));
REQUIRE_THAT(42.0, Catch::Matchers::WithinRel(a, 0.02));
\end{cpp}

\item
使用 \verb|CHECK_NOTHROW(expr)|/\verb|REQUIRE_NOTHROW(expr)| 验证 expr 不抛出异常,\verb|CHECK_THROWS(expr)|/\verb|REQUIRE_THROWS(expr)| 验证 expr 会抛出异常,\verb|CHECK_THROWS_AS(expr, exctype)|/\verb|REQUIRE_THROWS_AS(expr, exctype)| 验证 expr 抛出的 exctype 类型异常,或者 \verb|CHECK_THROWS_WITH(expression, string or string matcher)|/\\\verb|REQUIRE_THROWS_WITH(expression, string or string matcher)| 验证 expr 抛出一个其描述与指定字符串匹配的表达式:

\begin{cpp}
void function_that_throws()
{
    throw std::runtime_error("error");
}
void function_no_throw()
{
}
SECTION("expressions")
{
    CHECK_NOTHROW(function_no_throw());
    REQUIRE_NOTHROW(function_no_throw());

    CHECK_THROWS(function_that_throws());
    REQUIRE_THROWS(function_that_throws());

    CHECK_THROWS_AS(function_that_throws(), std::runtime_error);
    REQUIRE_THROWS_AS(function_that_throws(), std::runtime_error);

    CHECK_THROWS_WITH(function_that_throws(), "error");
    REQUIRE_THROWS_WITH(function_that_throws(),
    Catch::Matchers::ContainsSubstring("error"));
}
\end{cpp}

\item
使用 \verb|CHECK_THAT(value, matcher expression)|/\verb|REQUIRE_THAT(expr, matcher expression)| 检查指定值是否使给定的匹配器表达式为真:

\begin{cpp}
std::string text = "this is an example";
CHECK_THAT(
    text,
    Catch::Matchers::ContainsSubstring("EXAMPLE", Catch::CaseSensitive::No));
REQUIRE_THAT(
    text,
    Catch::Matchers::StartsWith("this") &&
    Catch::Matchers::ContainsSubstring("an"));
\end{cpp}

\item
使用 FAIL(message) 报告消息并使测试用例失败,WARN(message) 记录消息但不停止测试用例的执行,INFO(message) 将消息记录到缓冲区并在下一个断言失败时报告该消息。
\end{itemize}

\mySubsubsection{}{How it works...}

REQUIRE/CHECK 宏族将表达式分解为其左侧和右侧项,并在失败时报告失败的位置(源文件和行号)、表达式及左右两侧的值:

\begin{shell}
f:\chapter11ca_03\main.cpp(19): FAILED:
    REQUIRE( a == 1 )
with expansion:
    42 == 1
\end{shell}

然而,这些宏不支持使用逻辑操作符(如 \verb|&&| 和 ||)组成的复杂表达式。以下是一个错误的例子:

\begin{cpp}
REQUIRE(a < 10 || a %2 == 0);   // error
\end{cpp}

解决这个问题的方法是创建一个变量来保存表达式的计算结果,并在断言宏中使用。不过,这样就会失去输出表达式元素展开的能力:

\begin{cpp}
auto expr = a < 10 || a % 2 == 0;
REQUIRE(expr);
\end{cpp}

另一种方法是使用另一组括号,这样做也会阻止表达式的分解:

\begin{cpp}
REQUIRE((a < 10 || a %2 == 0)); // OK
\end{cpp}

两组断言,即 \verb|CHECK_THAT|/\verb|REQUIRE_THAT| 和 \verb|CHECK_THROWS_WITH|/\verb|REQUIRE_THROWS_WITH|,可以与匹配器一起工作。匹配器是可扩展和可组合的组件,用于执行值匹配。该框架提供了多种匹配器:

\begin{itemize}
\item
字符串: StartsWith, EndsWith, ContainsSubstring, Equals和 Matches

\item
std::vector: Contains, VectorContains, Equals, UnorderedEquals和 Approx

\item
浮点值: WithinAbs, WithinULP, WithinRel和 IsNaN

\item
范围类型(从版本 3.0.1 开始包含):IsEmpty、SizeIs、Contains、AllMatch、AnyMatch、NoneMatch、AllTrue、AnyTrue、NoneTrue、RangeEquals 和 UnorderedRangeEquals

\item
异常:Message 和 MessageMatches
\end{itemize}

\begin{myTip}
Contains() 和 VectorContains() 的区别在于 Contains() 在另一个vecotr中搜索vector,而 VectorContains() 是在vector中搜索单个元素。
\end{myTip}

正如前面列表中提到的,有几个匹配器针对浮点数。这些匹配器包括:

\begin{itemize}
\item
\verb|WithinAbs()|:接受小于或等于目标数值并具有指定误差范围(介于 0 和 1 之间的百分比)的浮点数的匹配器:

\begin{cpp}
REQUIRE_THAT(42.0, WithinAbs(42.5, 0.5));
\end{cpp}

\item
\verb|WithinRel()|:接受大约等于目标值并具有给定容差的浮点数的匹配器:

\begin{cpp}
REQUIRE_THAT(42.0, WithinRel(42.4, 0.01));
\end{cpp}

\item
\verb|WithinULP()|:接受与目标值相差不超过给定 ULP 的浮点数的匹配器:

\begin{cpp}
REQUIRE_THAT(42.0, WithinRel(target, 4));
\end{cpp}

\end{itemize}

这些匹配器也可以组合在一起使用:

\begin{cpp}
REQUIRE_THAT(a,
    Catch::Matchers::WithinRel(42.0, 0.001) ||
    Catch::Matchers::WithinAbs(42.0, 0.000001));
\end{cpp}

比较浮点数的一种过时的方式是使用 Catch 命名空间中的 Approx类型,通过构造 double 值的值重载等式/不等式和比较操作符。可以指定两个值之间的差异或被视为相等的最大误差,作为给定值的百分比。这通过成员函数 epsilon() 设置。值必须在 0 和 1 之间(例如,值 0.05 表示 5\%),epsilon 的默认值设置为 \verb|std::numeric_limits<float>::epsilon()*100|。

可以创建自己的匹配器,无论是为了扩展现有框架的功能还是为了与自定义类型一起工作。有两种方式可以创建自定义匹配器:旧版 v2 方式和新版 v3 方式。

以旧方式创建自定义匹配器需要两个步骤:

\begin{enumerate}
\item
从 Catch::MatcherBase<T> 派生的匹配器类,其中 T 是正在比较的类型。有两个虚函数必须重写:match(),接收一个值以匹配并返回一个布尔值指示匹配是否成功;describe(),不接收参数但返回一个描述匹配器的字符串。

\item
从测试代码中调用的构建函数。
\end{enumerate}

下面的例子定义了一个针对本章中多次出现的 point3d 类的匹配器,用于检查给定的三维点是否位于三维空间的一条直线上:

\begin{cpp}
class OnTheLine : public Catch::Matchers::MatcherBase<point3d>
{
    point3d const p1;
    point3d const p2;
public:
    OnTheLine(point3d const & p1, point3d const & p2):
        p1(p1), p2(p2)
    {}
    virtual bool match(point3d const & p) const override
    {
        auto rx = p2.x() - p1.x() != 0 ?
                  (p.x() - p1.x()) / (p2.x() - p1.x()) : 0;
        auto ry = p2.y() - p1.y() != 0 ?
                  (p.y() - p1.y()) / (p2.y() - p1.y()) : 0;
        auto rz = p2.z() - p1.z() != 0 ?
                  (p.z() - p1.z()) / (p2.z() - p1.z()) : 0;
        return
            Catch::Approx(rx).epsilon(0.01) == ry &&
            Catch::Approx(ry).epsilon(0.01) == rz;
    }
protected:
    virtual std::string describe() const
    {
        std::ostringstream ss;
        ss << "on the line between " << p1 << " and " << p2;
        return ss.str();
    }
};
inline OnTheLine IsOnTheLine(point3d const & p1, point3d const & p2)
{
    return OnTheLine {p1, p2};
}
\end{cpp}

以新方式创建自定义匹配器需要以下步骤:

\begin{enumerate}
\item
从 Catch::Matchers::MatcherGenericBase 派生的匹配器类。这个类必须实现两个方法:bool match(…) const,执行匹配;以及虚函数 string describe() const 的重写,不接收参数但返回一个描述匹配器的字符串。尽管这些函数与旧风格中使用的函数非常相似,但有一个关键区别:match() 函数对参数传递方式没有要求。所以它可以按值或可变引用接收参数,也可以是一个函数模板。这一优势使得编写更复杂的匹配器成为可能,比如能够比较类似范围的类型的匹配器。

\item
一个从测试代码中调用的构建函数。
\end{enumerate}

用新风格编写的同一个比较 point3d 值的匹配器:

\begin{cpp}
class OnTheLine : public Catch::Matchers::MatcherGenericBase
{
    point3d const p1;
    point3d const p2;
public:
    OnTheLine(point3d const& p1, point3d const& p2) :
        p1(p1), p2(p2)
    {
    }
    bool match(point3d const& p) const
    {
        auto rx = p2.x() - p1.x() != 0 ?
                  (p.x() - p1.x()) / (p2.x() - p1.x()) : 0;
        auto ry = p2.y() - p1.y() != 0 ?
                  (p.y() - p1.y()) / (p2.y() - p1.y()) : 0;
        auto rz = p2.z() - p1.z() != 0 ?
                  (p.z() - p1.z()) / (p2.z() - p1.z()) : 0;
        return
            Catch::Approx(rx).epsilon(0.01) == ry &&
            Catch::Approx(ry).epsilon(0.01) == rz;
    }
protected:
    std::string describe() const override
    {
#ifdef __cpp_lib_format
        return std::format("on the line between ({},{},{}) and ({},{},{})", p1.x(), p1.y(), p1.z(), p2.x(), p2.y(), p2.z());
#else
        std::ostringstream ss;
        ss << "on the line between " << p1 << " and " << p2;
        return ss.str();
#endif
    }
};
\end{cpp}

以下测试用例包含了如何使用此自定义匹配器:

\begin{cpp}
TEST_CASE("matchers")
{
    SECTION("point origin")
    {
        point3d p { 2,2,2 };
        REQUIRE_THAT(p, IsOnTheLine(point3d{ 0,0,0 }, point3d{ 3,3,3 }));
    }
}
\end{cpp}

这个测试确保点 \verb|{2,2,2}| 位于由点 \verb|{0,0,0}| 和 \verb|{3,3,3}| 定义的直线上,方法是使用之前实现的 IsOnTheLine() 自定义匹配器。
