
上一节介绍了存储字符和字符字符串的各种数据类型。之所以存在如此多的类型,是因为随着时间的发展出现了多种字符集。

最常用的字符集是 ASCII 和 Unicode。自语言创建以来,所有编译器和目标平台都支持ASCII,但Unicode的支持在 Windows 和 Unix/Linux 系统上的发展速度和形式却各不相同。本示例中,将介绍如何将不同编码的文本输出到标准输出控制台。

\mySubsubsection{}{How to do it...}

要将文本写入标准输出控制台,可以使用以下方法:

\begin{itemize}
\item
输出 ASCII 字符,可使用 std::cout:

\begin{cpp}
std::cout << "C++\n";
\end{cpp}

\item
Linux 上输出 UTF-8 编码的 Unicode 字符,同样可使用 std::cout:

\begin{tcolorbox}[ breakable,colback = blue!5!white, colframe=black!7!white]
\scriptsize{
std::cout <{}< "Erling Håland\verb|\n|"; \\
std::cout <{}< "Thomas Müller\verb|\n|"; \\
std::cout <{}< "Στέφανος Τσιτσιπάς\verb|\n|"; \\
std::string monkeys  = U"\includegraphics[width=0.07\textwidth]{content/chapter2/images/code-2.png}";\\
std::cout <{}< monkeys <{}< '\verb|\n|';
}
\end{tcolorbox}

对于使用 char8\_t 数据类型存储的 UTF-8 字符串,可以使用 std::cout,但必须将底层类型重新表示为 char 数组:

\begin{tcolorbox}[ breakable,colback = blue!5!white, colframe=black!7!white]
\scriptsize{
std::cout <{}< reinterpret\_cast<const char*>(u8"Στέφανος Τσιτσιπάς\verb|\n|");
}
\end{tcolorbox}

\item
Windows 系统上输出 UTF-8 编码的 Unicode 字符,可使用 C++20 中的 char8\_t 字符和 std::u8string 字符串。之前的版本中,可以使用 char 和 std::string。需要在标准输出前调用 Windows API SetConsoleOutputCP(CP\_UTF8):

\begin{tcolorbox}[ breakable,colback = blue!5!white, colframe=black!7!white]
\scriptsize{
SetConsoleOutputCP(CP\_UTF8);
std::cout <{}< reinterpret\_cast<const char*>(u8"Erling Håland\verb|\n|"); \\
std::cout <{}< reinterpret\_cast<const char*>(u8"Thomas Müller\verb|\n|"); \\
std::cout <{}< reinterpret\_cast<const char*>(u8"Στέφανος Τσιτσιπάς\verb|\n|"); \\
std::cout <{}< reinterpret\_cast<const char*>(u8"\includegraphics[width=0.07\textwidth]{content/chapter2/images/code-2.png}\verb|\n|");
}
\end{tcolorbox}

\myGraphic{0.5}{content/chapter2/images/3.png}{图2.3: 上述代码的输出}

\item
Windows 系统上输出 UTF-16 编码的 Unicode 字符,可使用 wchar\_t 字符和 std::wstring 字符串。输出前需要调用 \verb|_setmode(_fileno(stdout), _O_U16TEXT)|:

\begin{tcolorbox}[ breakable,colback = blue!5!white, colframe=black!7!white]
\scriptsize{
auto mode = \_setmode(\_fileno(stdout), \_O\_U16TEXT); \\
std::wcout <{}< L"Erling Håland\verb|\n|"; \\
std::wcout <{}< L"Thomas Müller\verb|\n|"; \\
std::wcout <{}< L"Στέφανος Τσιτσιπάς\verb|\n|"; \\
\_setmode(\_fileno(stdout), mode);
}
\end{tcolorbox}

\myGraphic{0.5}{content/chapter2/images/4.png}{图2.4: 上述代码片段的输出}

\end{itemize}

\mySubsubsection{}{How it work...}

ASCII 编码半个世纪以来一直是最常见的字符编码格式,包括 128 个字符,包括英语的小写和大写字母、10 个十进制数字以及符号。该集合中的前 32 个字符是非打印字符,称为控制字符。C++ 语言完全支持 ASCII 字符集。可以使用 std::cout 将 ASCII 字符打印到标准输出。

 ASCII 编码仅包括英文字母,为了支持其他语言和字母表进行了各种尝试。其中一个方法是代码页的概念。ASCII 编码只需要 7 位来编码 128 个字符,可以使用 8 位数据类型额外编码 128 个字符,所以索引 128–255 的字符可以映射到其他语言或字母表,这样的映射称为代码页,如 IBM 代码页、DOS 代码页、Windows 代码页等。 可以在 \url{https://en.wikipedia.org/wiki/Code_page}了解更多信息。下表列出了几个 Windows 代码页:

% Please add the following required packages to your document preamble:
% \usepackage{longtable}
% Note: It may be necessary to compile the document several times to get a multi-page table to line up properly
\begin{longtable}{|l|l|l|}
\hline
\textbf{代码页} & \textbf{名称}      & \textbf{支持的语言}           \\ \hline
\endfirsthead
%
\endhead
%
1250 & Windows 中欧  & \begin{tabular}[c]{@{}l@{}}捷克语、波兰语、斯洛伐克语、匈牙利语、斯洛文尼亚语、\\塞尔维亚-克罗地亚语、罗马尼亚语、阿尔巴尼亚语 \end{tabular}                     \\ \hline
1251 & Windows 西里尔       & \begin{tabular}[c]{@{}l@{}}俄语、白俄罗斯语、乌克兰语、保加利亚语、马其顿语、\\塞尔维亚语  \end{tabular}                                   \\ \hline
1252 & Windows 西欧        & \begin{tabular}[c]{@{}l@{}}西班牙语、葡萄牙语、法语、德语、丹麦语、挪威语、瑞典语、\\芬兰语、冰岛语、法罗语等。\end{tabular} \\ \hline
1253               & Windows 希腊      & 希腊语                                  \\ \hline
1254               & Windows 土耳其    & 土耳其语                                 \\ \hline
1255               & Windows 希伯来     & 希伯来语                                 \\ \hline
1256               & Windows 阿拉伯     & 阿拉伯语、波斯语、乌尔都语、英语、法语 \\ \hline
1257               & Windows 波罗的海     & 爱沙尼亚语、拉脱维亚语、立陶宛语           \\ \hline
1258               & Windows 越南  & 越南语                             \\ \hline
\end{longtable}

\begin{center}
表2.10: Windows 代码页的一部分
\end{center}

为了介绍其工作原理,看用一个代码片段举例说明。索引 224 或 0xE0(以十六进制表示)在不同的代码页中映射到不同的字符:

% Please add the following required packages to your document preamble:
% \usepackage{longtable}
% Note: It may be necessary to compile the document several times to get a multi-page table to line up properly
\begin{longtable}{|l|l|l|l|l|l|l|l|l|}
\hline
\textbf{250} & \textbf{1251} & \textbf{1252} & \textbf{1253} & \textbf{1254} & \textbf{1255} & \textbf{1256} & \textbf{1257} & \textbf{1258} \\ \hline
\endfirsthead
%
\endhead
%
ŕ            & а             & à             & ΰ             & à             & \includegraphics[width=0.018\textwidth]{content/chapter2/images/table-1.png}            & à             & ą             & à             \\ \hline
\end{longtable}

\begin{center}
表2.11: 不同 Windows 代码页中索引 224 对应的字符
\end{center}

\begin{myNotic}
编码术语中,字符映射的数值称为代码点(codepoint)。例子中,224 是一个代码点,而 a、à 或 ą 是在不同代码页中映射到此代码点的具体字符。
\end{myNotic}

 Windows可以通过调用 SetConsoleOutputCP() 激活与运行进程关联的控制台的一个代码页。下面的代码片段显示了一个示例,可打印出从 1250 到 1258(前面列出的那些)所有代码页中映射到 224 代码点的字符:

\begin{cpp}
char c = 224;
for (int codepage = 1250; codepage <= 1258; codepage++)
{
    SetConsoleOutputCP(codepage);
    std::cout << c << ' ';
}
\end{cpp}

运行此程序的输出结果如下一图像所示,可以看到这里打印的字符正是根据表2.11预期的结果。

\myGraphic{0.5}{content/chapter2/images/5.png}{图2.5: 使用不同代码页打印代码点 224}

虽然代码页提供了简单的切换方法,但并不能支持包含数百或数千个字符或象形文字的语言或书写系统,比如中文或埃及象形文字。为此,需要另一种标准,即 Unicode。这种编码标准旨在代表世界上大多数现行和过去的书写系统,以及其他符号,如近年来因短信而变得极其流行的表情符号。目前,Unicode 标准定义了近 150,000 个字符。

Unicode 字符可以存储在几种编码中,最受欢迎的是 UTF-8 和 UTF-16。还有 UTF-32 和 GB18030;后者不是 Unicode 规范的一部分,但仅在中国使用,并完全实现了 Unicode。

UTF-8 是一种与 ASCII 兼容的可变长度字符编码标准,使用 1、2、3 或 4 个字节来编码所有可表示的代码点。代码点使用得越多,其编码使用的字节就越少。ASCII 编码的 128 个代码点由单个字节表示,UTF-8 完全“向后”兼容 ASCII。所有其他 Unicode 代码点都使用多个字节编码:范围 128–2047 的代码点使用 2 个字节,范围 2048–65535 的代码点使用 3 个字节,范围 65536–1114111 的代码点使用 4 个字节。编码中的第一个字节称为引导字节,提供了编码代码点所使用的字节数的信息。所以,UTF-8 是一个非常高效的编码系统,也是万维网的首选,几乎所有网页都使用这种编码。

UTF-16 也是一种可以编码所有 Unicode 代码点的可变长度字符编码,使用一个或两个 16 位代码单元,与 ASCII 不兼容。UTF-16 是 Windows 操作系统以及 Java 和 JavaScript 编程语言使用的编码。

UTF-32 是一种较少使用的编码系统,是一种固定长度的编码,每个代码点使用 32 位。由于所有的 Unicode 代码点最多只需要 21 位,前导的 11 位总是 0。这使得其空间利用率很低,是其主要缺点。其主要优点是可以在一个序列中找到第 N 个代码点的时间相同,而像 UTF-8 和 UTF-16 可变长度编码查找的时间是线性的。

编译器通常假定源文件使用 UTF-8 编码,GCC、Clang 和 MSVC 都是这种情况。

Linux 发行版原生支持 UTF-8。所以可将字符串字面量写入输出控制台,如 "Στέφανος Τσιτσιπάς"。因为终端支持 UTF-8,会得到预期的结果:

\begin{tcolorbox}[ breakable,colback = blue!5!white, colframe=black!7!white]
\scriptsize{
std::cout <{}< "Στέφανος Τσιτσιπάς";
}
\end{tcolorbox}

另外,直接的宽字符串如 L"Στέφανος Τσιτσιπάς" 并不能立即生效。为了得到预期的结果,需要设置一个区域对象。默认的 C 区域不知道如何将宽字符转换为 UTF-8。为了实现这一转换,需要另一个区域。这里有两个选项:

\begin{itemize}
\item
初始化一个与环境配置相匹配的区域对象,通常支持 UTF-8:

\begin{tcolorbox}[ breakable,colback = blue!5!white, colframe=black!7!white]
\scriptsize{
std::locale utf8(""); \\
std::wcout.imbue(utf8); \\
std::wcout <{}< L"Στέφανος Τσιτσιπάς\verb|\n|";
}
\end{tcolorbox}

\item
使用特定的区域初始化区域实例,比如美国英语:

\begin{tcolorbox}[ breakable,colback = blue!5!white, colframe=black!7!white]
\scriptsize{
std::locale utf8("en\_US.UTF-8"); \\
std::wcout.imbue(utf8); \\
std::wcout <{}< L"Στέφανος Τσιτσιπάς\verb|\n|";
}
\end{tcolorbox}

\end{itemize}

\begin{myNotic}
区域在第7章中有详细讨论。
\end{myNotic}

Windows 命令提示符 (cmd.exe) 不支持 UTF-8,Windows 10 添加了一个名为“使用 Unicode UTF-8 提供全球语言支持”的测试功能,深藏于区域设置中。而且目前报告表示,这可能会阻止某些应用程序正常工作。要向命令提示符写入 UTF-8 内容,首先调用 SetConsoleOutputCP() 设置正确的代码页,并传递 CP\_UTF8 作为参数(或者其数值 65001):

\begin{tcolorbox}[ breakable,colback = blue!5!white, colframe=black!7!white]
\scriptsize{
SetConsoleOutputCP(CP\_UTF8); \\
std::cout <{}< reinterpret\_cast<const char*>(u8"Erling Håland\verb|\n|"); \\
std::cout <{}< reinterpret\_cast<const char*>(u8"Thomas Müller\verb|\n|"); \\
std::cout <{}< reinterpret\_cast<const char*>(u8"Στέφανος Τσιτσιπάς\verb|\n|"); \\
td::u8string monkeys = u8"\includegraphics[width=0.07\textwidth]{content/chapter2/images/code-2.png}";\\
std::cout <{}< reinterpret\_cast<const char*>(monkeys.c\_str());
}
\end{tcolorbox}

输出 UTF-16需要调用 \verb|_setmode()|(<io.h>)设置(标准输出控制台)转换模式为 UTF-16,需要设置 \verb|_O_U16TEXT| 参数。函数返回先前的转换模式,以便在写入所需内容后恢复转换模式。

使用\verb|_O_TEXT| 设置文本模式(输入时 CR-LF 组合会转换成单一的 LF,输出时 LF 字符会转换成 CR-LF):

\begin{tcolorbox}[ breakable,colback = blue!5!white, colframe=black!7!white]
\scriptsize{
auto mode = \_setmode(\_fileno(stdout), \_O\_U16TEXT); \\
std::wcout <{}< L"Erling Håland\verb|\n|"; \\
std::wcout <{}< L"Thomas Müller\verb|\n|"; \\
std::wcout <{}< L"Στέφανος Τσιτσιπάς\verb|\n|"; \\
\_setmode(\_fileno(stdout), mode);
}
\end{tcolorbox}

\begin{myNotic}
为了能正确输出,命令提示符应用程序需要使用一种真型字体,如 Lucida Console 或 Consolas,而不是只支持 ASCII 的光栅字体。
\end{myNotic}

从 Windows 10 开始,Windows 提供了一个新的终端应用程序——Windows Terminal,内置支持 UTF-8。以下代码无需先调用 SetConsoleOutputCP() 就能输出预期结果:

\begin{cpp}
std::cout << reinterpret_cast<const char*>(u8"Erling Håland\n");
\end{cpp}

与其他编程语言不同,对 Unicode 的支持并不是 C++ 的强项。本节提供了在控制台应用程序中使用 Unicode 的基础知识。实践中,事情可能会变得更加复杂,需要额外的支持。为了更深入地了解这一主题,建议查阅更多的(在线)资料。