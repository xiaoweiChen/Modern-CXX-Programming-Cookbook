

上一节介绍了伪随机数库及其组件,以及如何生成不同统计分布的数字。其中,忽略了一个重要因素,就是伪随机数生成器的正确初始化。

通过仔细分析(超出了本书的范畴),可以证明梅森旋转引擎倾向于重复生成某些值而遗漏其他值,因此生成的数字不是均匀分布,而是二项式或泊松分布。本示例中,将介绍如何初始化生成器,以便生成真正均匀分布的伪随机数。

\mySubsubsection{}{Getting ready}

阅读上一节,了解伪随机数库。

\mySubsubsection{}{How to do it...}

为了正确初始化伪随机数生成器以生成均匀分布的伪随机数序列,请执行以下步骤:

\begin{enumerate}
\item
使用 \verb|std::random_device| 生成随机数作为随机种子:

\begin{cpp}
std::random_device rd;
\end{cpp}

\item
为引擎的所有内部位生成随机数据:

\begin{cpp}
std::array<int, std::mt19937::state_size> seed_data {};
std::generate(std::begin(seed_data), std::end(seed_data),
              std::ref(rd));
\end{cpp}

\item
从之前生成的伪随机数据创建 \verb|std::seed_seq| 实例:

\begin{cpp}
std::seed_seq seq(std::begin(seed_data), std::end(seed_data));
\end{cpp}

\item
创建引擎实例,并初始化代表引擎内部状态的所有位;例如,\verb|mt19937| 具有 19,937 位的内部状态:

\begin{cpp}
auto eng = std::mt19937{ seq };
\end{cpp}

\item
根据应用程序的需求使用适当的分布:

\begin{cpp}
auto dist = std::uniform_real_distribution<>{ 0, 1 };
\end{cpp}
\end{enumerate}

\mySubsubsection{}{How it work...}

上一节中展示的代码中,使用了 \verb|std::mt19937| 引擎来生成伪随机数。虽然梅森旋转引擎的速度比其他引擎慢,但可以生成最长的非重复数字序列,并具有最佳的频谱特性。问题在于 \verb|mt19937| 引擎的内部状态有 624 个 32 位整数,而在上一节的示例中,只初始化了其中的一个。

使用伪随机数库时,请记住以下经验法则。

\begin{myNotic}
为了获得最佳结果,生成数字之前,引擎的整个内部状态必须正确初始化。
\end{myNotic}

伪随机数库为此特定目的提供了名为 \verb|std::seed_seq| 的类型,可以使用 32 位整数种子生成器,并在 32 位空间内均匀地生成请求的数量的整数。

“How to do it...”中定义了一个名为 \verb|seed_data| 的数组,其中包含等于 \verb|mt19937| 发生器内部状态的 32 位整数的数量——即 624 个整数,使用 \verb|std::random_device| 生成的随机数初始化了数组。后来,该数组用于设置 \verb|std::seed_seq|的种子,后者可用于设置 \verb|mt19937| 发生器的种子。