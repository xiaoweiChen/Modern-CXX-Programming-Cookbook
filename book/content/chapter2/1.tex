
C++ 编程语言定义了多种算术类型,可以执行算术操作(加法、减法、乘法、除法、取模),包括字符、整数和浮点数类型。有些类型从 C 编程语言继承而来,有些是 C++ 标准添加的。算术类型的常见问题是,大多数大小随目标平台的不同而变化,标准仅保证一个最小值。本章中,将介绍各种整数和浮点数类型。

\mySubsubsection{}{How to do it...}

根据需要使用数字类型:

\begin{itemize}
\item
表示整数值(当范围不是很重要时),可以使用 int 类型。这是基本(有符号)整数类型,通常大小为 32 位,但并不保证。可用于表示人的年龄、日期(日、月、年)、电影或书籍的评分、集合中的项目数量等:

\begin{cpp}
int age = 42;
int attendance = 96321;
\end{cpp}

\item
需要对可能值的范围或内存表示形式施加限制时,使用符号性(有符号/无符号)和大小(short/long/long long)修饰符,例如:不能使用无符号整数表示为负值,避免混合使用有符号和无符号整数。另外,如果优化存储某些值所使用的内存(比如:表示日期),可以使用 short int(至少是 16 位)。如果需要表示大数值(比如:文件大小),可以使用 unsigned long long int(至少是 64 位):

\begin{cpp}
unsigned int length = 32;
short year = 2023; // same as short int
unsigned long long filesize = 3'758'096'384;
\end{cpp}

\item
为了表示不能为负值的标准数组索引、集合中的元素数量(如标准容器)、或 sizeof 操作符的结果,可以使用 std::size\_t(至少 16 位的无符号整数类型)。标准容器定义了一个名为 size\_type 的成员类型别名用于容器的大小和索引,这个类型通常是 std::size\_t 的同义词:

\begin{cpp}
std::size_t items = arr.size();
\end{cpp}

\item
为了存储指针算术运算的结果,或者呈现可以为负值的 C 风格数组索引,可用 std::ptrdiff\_t。C++ 标准容器定义了一个 difference\_type 的成员类型别名,来存储迭代器之间的差异,该类型通常为 std::ptrdiff\_t 的同义词。

\item
需要存储需要保证范围的值时,可用 std::int8\_t、std::int16\_t、std::int32\_t 或 std::int64\_t 类型。虽然些可选,但在主流架构上都有定义。

\item
需要存储非负值或对具有保证范围的值进行位操作时,可使用 std::uint8\_t、std::uint16\_t、std::uint32\_t 或 std::uint64\_t 类型。

\item
需要存储需要保证范围的值,同时希望优化最快的访问速度时,使用 std::int\_fast8\_t、std::int\_fast16\_t、std::int\_fast32\_t 或 std::int\_fast64\_t 类型(或其无符号对应类型)。需要确定这些在目标架构上的可用性。

\item
需要存储需要保证范围的值,同时希望优化内存使用时,使用 std::int\_least8\_t、std::int\_least16\_t、std::int\_least32\_t 或 std::int\_least64\_t(或其无符号对应类型)。这些也需要确定在目标架构上的可用性。

\item
为了表示实数,可以使用 double 类型;默认的浮点类型,大小为 64 位。名称表示双精度类型,相对于单精度类型(使用 32 位),后者由 float 类型实现。还有一种扩展精度类型,称为 long double。标准未指定其实际精度,至少与 double 类型相同。某些编译器上,可以是四倍精度(使用 128 位),不过 VC++ 将其视为 double:

\begin{cpp}
double price = 4.99;
float temperature = 36.5;
long double pi = 3.14159265358979323846264338327950288419716939937510L;
\end{cpp}
\end{itemize}

\mySubsubsection{}{How it work...}

C++ 语言整数类型 int 有几个修饰符,用于符号性和大小。int 类型是有符号的,因此 int 和 signed int 是相同的类型。使用修饰符时,实际上可以省略 int 的使用,以下类型等价:

% Please add the following required packages to your document preamble:
% \usepackage{longtable}
% Note: It may be necessary to compile the document several times to get a multi-page table to line up properly
\begin{longtable}{|l|l|}
\hline
\textbf{类型}      & \textbf{等价于} \\ \hline
\endfirsthead
%
\endhead
%
signed             & int                    \\ \hline
unsigned           & unsigned int           \\ \hline
short              & short int              \\ \hline
signed short       & short int              \\ \hline
signed short int   & short int              \\ \hline
long               & long int               \\ \hline
long long          & long long int          \\ \hline
unsigned short     & unsigned short int     \\ \hline
unsigned long      & unsigned long int      \\ \hline
unsigned long long & unsigned long long int \\ \hline
\end{longtable}

\begin{center}
表 2.1: 整数类型等价表
\end{center}

此表并未列出所有可能的组合,类型的修饰符顺序未指定,任何顺序都可以。下表列出了几种相同的类型:

% Please add the following required packages to your document preamble:
% \usepackage{multirow}
% \usepackage{longtable}
% Note: It may be necessary to compile the document several times to get a multi-page table to line up properly
\begin{longtable}{|l|l|}
\hline
\textbf{类型}          & \textbf{等效类型}                \\ \hline
\endfirsthead
%
\endhead
%
long long unsigned int & \multirow{5}{*}{unsigned long long int} \\ \cline{1-1}
long unsigned long int &                                         \\ \cline{1-1}
int long long unsigned &                                         \\ \cline{1-1}
unsigned long long int &                                         \\ \cline{1-1}
int long unsigned long &                                         \\ \hline
\end{longtable}

\begin{center}
表 2.2: 带修饰符的整数类型等价表
\end{center}

虽然顺序未定义,但常见的是以符号性修饰符开始,然后是大小修饰符,最后是 int 类型。因此,前一表格左侧类型的规范形式是 unsigned long long int。

无论整数类型的符号性或大小如何,都可能发生溢出或下溢的过程。当尝试存储大于数据类型最大值时会发生溢出。反之,即当尝试存储小于数据类型最小值时会发生下溢。

以 short 类型为例。这是一个有符号整数类型,可以存储 -32,768 到 32,767 范围内的值。如果想存储 32,768 会发生什么?由于这个值大于最大值,所以会发生溢出。十进制的 32,767 在二进制中是 01111111 11111111,下一个值是 10000000 00000000,在 16 位表示中是 -32,768 十进制。下表展示了溢出和下溢的情况:

% Please add the following required packages to your document preamble:
% \usepackage{longtable}
% Note: It may be necessary to compile the document several times to get a multi-page table to line up properly
\begin{longtable}{|l|l|}
\hline
\textbf{要存储的值} & \textbf{实际存储的值} \\ \hline
\endfirsthead
%
\endhead
%
-32771                  & 32765                 \\ \hline
-32770                  & 32766                 \\ \hline
-32769                  & 32767                 \\ \hline
-32768                  & -32768                \\ \hline
…                       & …                     \\ \hline
32767                   & 32767                 \\ \hline
32768                   & -32768                \\ \hline
32769                   & -32767                \\ \hline
32770                   & -32765                \\ \hline
\end{longtable}

\begin{center}
表 2.3: short int 类型值的溢出和下溢示例
\end{center}

同样的例子,以不同的形式展示在图表中:

\myGraphic{1.0}{content/chapter2/images/1.png}{图 2.1: short int 类型值的溢出和下溢示例}

如果使用 unsigned short 类型,而不是 short 类型,同样的问题也会出现。unsigned short 的范围是 0 到 65,535。尝试存储 65,536 将导致存储值为 0。同样,尝试存储 65,537 将导致存储值为 1。存储的值与数据类型可以存储的值的数量之间,进行了模运算。对于 unsigned short 类型来说,这个值是 $2^{16}$ 或 65,536。对于下溢,结果的发生方式相似。值 -1 变为 65,535,-2 变为 65,534,以此类推。这相当于将负值加到 65,536 上,然后执行模 65,536 的运算。溢出和下溢的情况如下表所示:

% Please add the following required packages to your document preamble:
% \usepackage{longtable}
% Note: It may be necessary to compile the document several times to get a multi-page table to line up properly
\begin{longtable}{|l|l|}
\hline
\textbf{要存储的值} & \textbf{实际存储的值} \\ \hline
\endfirsthead
%
\endhead
%
-2                      & 65534                 \\ \hline
-1                      & 65535                 \\ \hline
0                       & 0                     \\ \hline
…                       & …                     \\ \hline
65535                   & 65535                 \\ \hline
65536                   & 0                     \\ \hline
65537                   & 1                     \\ \hline
\end{longtable}

\begin{center}
表 2.4: unsigned short int 类型值的溢出和下溢示例
\end{center}

同样地,下图中也举例演示了这些值的情况:

\myGraphic{1.0}{content/chapter2/images/2.png}{图 2.2: unsigned short int 类型值的溢出和下溢示例}

C++ 中整数类型的一个问题是大小没有明确定义。唯一明确规定的大小是 char 类型(及其有符号和无符号修饰符),必须是 1 字节。对于其余类型,适用以下关系:

\begin{cpp}
1 == sizeof(char) <= sizeof(short) <= sizeof(int) <= sizeof(long) <= sizeof(long long)
\end{cpp}

主流平台上,short 为 16 位,int 和 long 都为 32 位,而 long long 为 64 位。然而,有些平台上的 long 和 long long 都为 64 位,或者 int 为 16 位。为了弥合这种异构性,C++11 标准引入了一系列固定宽度的整数类型。这些类型定义在 <cstdint> 头文件中,并分为两类:

\begin{itemize}
\item
一类是可选的,某些平台上不可用。这些类型的名字指定了确切的位数:

\begin{itemize}
\item
int8\_t 和 uint8\_t 是 8 位

\item
int16\_t 和 uint16\_t 是 16 位

\item
int32\_t 和 uint32\_t 是 32 位

\item
int64\_t 和 uint64\_t 是 64 位

\item
intptr\_t 和 uintptr\_t,大小足以存储指向 void 的指针
\end{itemize}

\item
另一类是强制性的,所有平台上都可用。这些类型又分为两组:

\begin{itemize}
\item
一组为了快速访问而优化;称为 int\_fastX\_t 和 uint\_fastX\_t,其中 X 是 8、16、32 或 64,表示位数。这些类型提供了访问速度最快的整数类型,至少有 X 位宽。

\item
一组为了节省内存而优化;称为 int\_leastX\_t 和 uint\_leastX\_t,其中 X 是 8、16、32 或 64,表示位数。这些类型提供了表示最小的整数类型,至少有 X 位宽。
\end{itemize}
\end{itemize}

实际上,大多数编译器将 8 位类型(int8\_t、uint8\_t、int\_least8\_t、uint\_least8\_t、int\_fast8\_t 和 uint\_fast8\_t)视为与有符号 char 和无符号 char 。所以在不同的系统上,使用这些类型的程序的行为,与使用其他固定宽度整数类型的程序不同:

\begin{cpp}
std::int8_t x = 42;
std::cout << x << '\n'; // [1] prints *
std::int16_t y = 42;
std::cout << y << '\n'; // [2] prints 42
\end{cpp}

x 和 y 都是固定宽度的整数类型,都赋值为 42。当打印到控制台时,x 将输出为 * 而不是 42,这种行为取决于系统。

开发者可能希望避免使用 8 位的固定宽度整数类型,而倾向于使用 int16\_t/uint16\_t 或者其中一个快速/内存最小版本。

\begin{myNotic}
编写数字字面量时,可以使用单引号(')作为数字分隔符。这使得读取大数字更加容易,还可以视觉上进行比较。可以用于十进制、十六进制、八进制和二进制数字:

\begin{cpp}
auto a = 4'234'871'523ll;        // 4234871523
auto b = 0xBAAD'F00D;            // 3131961357
auto c = 00'12'34;               // 668
auto d = 0b1011'01011'0001'1001; // 46361
\end{cpp}

数字分隔符会在确定数字值时忽略,其位置无关紧要:

\begin{cpp}
auto e = 1'2'3'4'5;
\end{cpp}
\end{myNotic}








