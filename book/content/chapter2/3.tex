
数字类型和字符串类型互相转换是一种常见的操作。C++11 前,将数字转换为字符串及反向转换的支持较少,开发者大多依赖于类型不安全的函数,并且通常会编写自己的实用函数以避免重复编写相同的代码。随着 C++11 的推出,标准库提供了用于在数字和字符串之间进行转换的实用函数。这个示例中,将介绍如何使用现代 C++ 标准函数在数字和字符串相互转换。

\mySubsubsection{}{Getting ready}

本示例中提到的所有实用函数都可以在 <string> 头文件中找到。

\mySubsubsection{}{How to do it...}

需要在数字和字符串之间进行转换时,可以使用以下标准转换函数:

\begin{itemize}
\item
将整数或浮点类型转换为字符串类型:std::to\_string() 或 std::to\_wstring():

\begin{cpp}
auto si = std::to_string(42);      // si="42"
auto sl = std::to_string(42L);     // sl="42"
auto su = std::to_string(42u);     // su="42"
auto sd = std::to_wstring(42.0);   // sd=L"42.000000"
auto sld = std::to_wstring(42.0L); // sld=L"42.000000"
\end{cpp}

\item
将字符串类型转换为整数类型:std::stoi()、std::stol()、std::stoll()、std::stoul() 或 std::stoull():

\begin{cpp}
auto i1 = std::stoi("42");                  // i1 = 42
auto i2 = std::stoi("101010"L, nullptr, 2); // i2 = 42
auto i3 = std::stoi("052", nullptr, 8);     // i3 = 42
auto i4 = std::stoi("0x2A"L, nullptr, 16);  // i4 = 42
\end{cpp}

\item
将字符串类型转换为浮点类型:std::stof()、std::stod() 或 std::stold():

\begin{cpp}
// d1 = 123.45000000000000
auto d1 = std::stod("123.45");
// d2 = 123.45000000000000
auto d2 = std::stod("1.2345e+2");
// d3 = 123.44999980926514
auto d3 = std::stod("0xF.6E6666p3");
\end{cpp}
\end{itemize}

\mySubsubsection{}{How it work...}

要将整数或浮点类型转换为字符串类型,可用 std::to\_string() 函数(转换为 std::string)或 std::to\_wstring() 函数(转换为 std::wstring)。这些函数在 <string> 头文件中可用,并且对有符号和无符号整数以及实数类型都有重载。产生的结果与调用带有相应格式说明符的 std::sprintf() 和 std::swprintf() 时相同,以下是这两个函数对 int 类型的重载示例:

\begin{cpp}
std::string to_string(int value);
std::wstring to_wstring(int value);
\end{cpp}

除了 int 类型之外,这两个函数还对 long、long long、unsigned int、unsigned long、unsigned long long、float、double 和 long double 类型有重载。

当涉及到相反方向的转换时,有一整套函数名格式为 ston(字符串转数字)的函数,其中 n 可以是 i(整数)、l(长整数)、ll(长长整数)、ul(无符号长整数)或 ull(无符号长长整数)。下面列表展示了 stoi 函数及其两个重载——一个接受 std::string 参数,另一个接受 std::wstring 参数。此外,还有类似的函数 stol、stoll、stoul、stoull、stof、stod 和 stold:

\begin{cpp}
int stoi(const std::string& str, std::size_t* pos = 0, int base = 10);
int stoi(const std::wstring& str, std::size_t* pos = 0, int base = 10);
\end{cpp}

字符串到整数类型的转换函数的工作方式是,丢弃非空白字符前的所有空白字符,然后尽可能多地取字符来形成一个有符号或无符号数(视情况而定),最后将该数转换为相应的整数类型(stoi() 返回整数,stoul() 返回无符号长整数等)。以下示例中,结果都是整数 42(最后一个示例除外,其结果为 -42):

\begin{cpp}
auto i1 = std::stoi("42");             // i1 = 42
auto i2 = std::stoi("   42");          // i2 = 42
auto i3 = std::stoi("   42fortytwo");  // i3 = 42
auto i4 = std::stoi("+42");            // i4 = 42
auto i5 = std::stoi("-42");            // i5 = -42
\end{cpp}

有效的整数可以由以下部分组成:

\begin{itemize}
\item
符号,加号 (+) 或减号 (-)(可选)

\item
前缀 0 表示八进制基数(可选)

\item
前缀 0x 或 0X 表示十六进制基数(可选)

\item
数字序列
\end{itemize}

可选的八进制前缀 0 只在指定的基数为 8 或 0 时应用。同样地,可选的十六进制前缀 0x 或 0X 只在指定的基数为 16 或 0 时可用。

将字符串转换为整数的函数有三个参数:

\begin{itemize}
\item
输入字符串。

\item
当不是空指针时,将接收处理过的字符数的指针。这包括已丢弃的前导空白字符、符号和基数前缀,不应将其与整数值的位数混淆。

\item
表示基数的数字,默认为 10。
\end{itemize}

输入字符串中的有效数字取决于基数。对于二进制,唯一有效的数字是 0 和 1;对于五进制,有效数字是 01234。对于十一进制,有效数字是 0-9 和字符 A 和 a。这一直持续到三十六进制,其有效字符为 0-9、A-Z 和 a-z。

以下是各种基数的数字字符串转换为十进制整数的其他示例,结果要么是 42,要么是 -42:

\begin{cpp}
auto i6 = std::stoi("052", nullptr, 8);      // i6 = 42
auto i7 = std::stoi("052", nullptr, 0);      // i7 = 42
auto i8 = std::stoi("0x2A", nullptr, 16);    // i8 = 42
auto i9 = std::stoi("0x2A", nullptr, 0);     // i9 = 42
auto i10 = std::stoi("101010", nullptr, 2); // i10 = 42
auto i11 = std::stoi("22", nullptr, 20);    // i11 = 42
auto i12 = std::stoi("-22", nullptr, 20);  // i12 = -42
auto pos = size_t{ 0 };
auto i13 = std::stoi("42", &pos);      // i13 = 42,  pos = 2
auto i14 = std::stoi("-42", &pos);     // i14 = -42, pos = 3
auto i15 = std::stoi("  +42dec", &pos);// i15 = 42,  pos = 5
\end{cpp}

需要注意的是,如果转换失败,转换函数会抛出异常。可能抛出两种异常:

\begin{itemize}
\item
如果无法执行转换,则抛出 std::invalid\_argument:

\begin{cpp}
try
{
    auto i16 = std::stoi("");
}
catch (std::exception const & e)
{
    // prints "invalid stoi argument"
    std::cout << e.what() << '\n';
}
\end{cpp}

\item
如果转换后的值超出结果类型的范围(或底层函数将 errno 设置为 ERANGE),则抛出 std::out\_of\_range:

\begin{cpp}
try
{
    // OK
    auto i17 = std::stoll("12345678901234");
    // throws std::out_of_range
    auto i18 = std::stoi("12345678901234");
}
catch (std::exception const & e)
{
    // prints "stoi argument out of range"
    std::cout << e.what() << '\n';
}
\end{cpp}
\end{itemize}

将字符串转换为浮点类型的另一组函数非常相似,没有用于数字基数的参数。有效的浮点值在输入字符串中,可以以不同的形式表示:

\begin{itemize}
\item
十进制浮点表达式(可选符号、带有可选小数点的十进制数字序列、可选的 e 或 E 后跟带有可选符号的指数)

\item
二进制浮点表达式(可选符号、0x 或 0X 前缀、带有可选小数点的十六进制数字序列、可选的 p 或 P 后跟带有可选符号的指数)

\item
无穷大表达式(可选符号后跟不区分大小写的 INF 或 INFINITY)

\item
非数字表达式(可选符号后跟不区分大小写的 NAN,以及可能的其他字母数字字符)
\end{itemize}

以下是将字符串转换为双精度浮点数的各种示例:

\begin{cpp}
auto d1 = std::stod("123.45");         // d1 =  123.45000000000000
auto d2 = std::stod("+123.45");        // d2 =  123.45000000000000
auto d3 = std::stod("-123.45");        // d3 = -123.45000000000000
auto d4 = std::stod("  123.45");       // d4 =  123.45000000000000
auto d5 = std::stod("  -123.45abc");   // d5 = -123.45000000000000
auto d6 = std::stod("1.2345e+2");      // d6 =  123.45000000000000
auto d7 = std::stod("0xF.6E6666p3");   // d7 =  123.44999980926514
auto d8 = std::stod("INF");            // d8 = inf
auto d9 = std::stod("-infinity");      // d9 = -inf
auto d10 = std::stod("NAN");           // d10 = nan
auto d11 = std::stod("-nanabc");       // d11 = -nan
\end{cpp}

以二进制科学记数法表示的浮点数,如 0xF.6E6666p3。为了清晰理解,这里提供了一个简短的描述,但建议查阅其他参考资料以获取详细信息(例如 \url{https://en.cppreference.com/w/cpp/language/floating_literal})。以二进制科学记数法表示的浮点常量由几个部分组成:

\begin{itemize}
\item
十六进制前缀 0x。

\item
整数部分;此示例中为 F,即十进制中的 15。

\item
小数部分;此示例中为 6E6666,即二进制中的 011011100110011001100110。将其转换为十进制,需要加上 2 的负幂:1/4 + 1/8 + 1/32 + 1/64 + 1/128 + ....

\item
后缀,表示 2 的幂;在此示例中,p3 表示 2 的三次方。
\end{itemize}

十进制等效值的确定方法是将有效数字(由整数部分和小数部分组成)乘以基数的指数次幂。

对于给定的以二进制表示的十六进制浮点字面量,有效数字为 15.4312499...(注意,第七位之后的数字未显示),基数为 2,指数为 3。因此,结果为 15.4312499... * 8,即 123.44999980926514。


















