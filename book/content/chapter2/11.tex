
标准库中的字符串类型是一种通用实现,缺乏很多实用方法,如转换大小写、去除空白、分割等,这些方法可以满足开发者的不同需求,有些第三方库提供了丰富的字符串功能集合。本示例中,将实现几个简单但有用的函数,目的是展示如何使用字符串方法和标准泛型算法来操纵字符串,同时也提供一些可在应用程序中重用的代码。

本示例中,将实现一个小的字符串工具库:

\begin{itemize}
\item
将字符串转换为小写或大写

\item
反转字符串

\item
从字符串的开头和/或结尾修剪空白字符

\item
从字符串的开头和/或结尾修剪特定字符集中的字符

\item
删除字符串中任意位置的字符

\item
使用特定的分隔符对字符串进行分词
\end{itemize}

开始实现之前,先来看一些前提条件。

\mySubsubsection{}{Getting ready}

支持所有的标准字符串类型——即 \verb|std::string|、\verb|std::wstring|、\verb|std::u16string| 和 \verb|std::u32string|。

为了避免使用

\verb|std::basic_string<CharT, std::char_traits<CharT>, std::allocator<CharT>>|

这样的长名称,将使用以下字符串和字符串流的别名模板:

\begin{cpp}
template <typename CharT>
using tstring =
    std::basic_string<CharT, std::char_traits<CharT>,
                      std::allocator<CharT>>;
template <typename CharT>
using tstringstream =
    std::basic_stringstream<CharT, std::char_traits<CharT>,
                            std::allocator<CharT>>;
\end{cpp}

为了实现这些字符串辅助函数,需要包含用于字符串的头文件 <string> 和标准通用算法的头文件 <algorithm>。

本示例中的所有例子中,将使用 C++14 的标准自定义字面量操作符,为此需要显式使用 \verb|std::string_literals| 命名空间。

\mySubsubsection{}{How to do it...}

\begin{enumerate}
\item
要将字符串转换为大写或小写,可以使用通用算法 \verb|std::transform()| 对字符串中的字符使用 \verb|toupper()| 或 \verb|tolower()|:

\begin{cpp}
template<typename CharT>
inline tstring<CharT> to_upper(tstring<CharT> text)
{
    std::transform(std::begin(text), std::end(text),
    std::begin(text), toupper);
    return text;
}
template<typename CharT>
inline tstring<CharT> to_lower(tstring<CharT> text)
{
    std::transform(std::begin(text), std::end(text),
    std::begin(text), tolower);
    return text;
}
\end{cpp}

\item
要反转字符串,可以使用通用算法 \verb|std::reverse()|:

\begin{cpp}
template<typename CharT>
inline tstring<CharT> reverse(tstring<CharT> text)
{
    std::reverse(std::begin(text), std::end(text));
    return text;
}
\end{cpp}

\item
修剪字符串的开头、结尾或两者,可用 \verb|std::basic_string|的 \verb|find_first_not_of()| 和 \verb|find_last_not_of()|成员函数:

\begin{cpp}
template<typename CharT>
inline tstring<CharT> trim(tstring<CharT> const & text)
{
    auto first{ text.find_first_not_of(' ') };
    auto last{ text.find_last_not_of(' ') };
    return text.substr(first, (last - first + 1));
}
template<typename CharT>
inline tstring<CharT> trimleft(tstring<CharT> const & text)
{
    auto first{ text.find_first_not_of(' ') };
    return text.substr(first, text.size() - first);
}
template<typename CharT>
inline tstring<CharT> trimright(tstring<CharT> const & text)
{
    auto last{ text.find_last_not_of(' ') };
    return text.substr(0, last + 1);
}
\end{cpp}

\item
要修剪字符串中给定字符集中的字符,可以使用接受定义要查找的字符集字符串参数的 \verb|find_first_not_of()| 和 \verb|find_last_not_of()| 函数(重载版本):

\begin{cpp}
template<typename CharT>
inline tstring<CharT> trim(tstring<CharT> const & text,
tstring<CharT> const & chars)
{
    auto first{ text.find_first_not_of(chars) };
    auto last{ text.find_last_not_of(chars) };
    return text.substr(first, (last - first + 1));
}
template<typename CharT>
inline tstring<CharT> trimleft(tstring<CharT> const & text,
tstring<CharT> const & chars)
{
    auto first{ text.find_first_not_of(chars) };
    return text.substr(first, text.size() - first);
}
template<typename CharT>
inline tstring<CharT> trimright(tstring<CharT> const &text,
tstring<CharT> const &chars)
{
    auto last{ text.find_last_not_of(chars) };
    return text.substr(0, last + 1);
}
\end{cpp}

\item
要从字符串中删除字符,可以使用 \verb|std::remove_if()| 和 \verb|std::basic_string::erase()|:

\begin{cpp}
template<typename CharT>
inline tstring<CharT> remove(tstring<CharT> text,
CharT const ch)
{
    auto start = std::remove_if(
                    std::begin(text), std::end(text),
                    [=](CharT const c) {return c == ch; });
    text.erase(start, std::end(text));
    return text;
}
\end{cpp}

\item
要基于指定的分隔符拆分字符串,可以使用 \verb|std::getline()| 从 \verb|std::basic_stringstream| 中读取内容。从转义流中提取token压入vector中:

\begin{cpp}
template<typename CharT>
inline std::vector<tstring<CharT>> split
(tstring<CharT> text, CharT const delimiter)
{
    auto sstr = tstringstream<CharT>{ text };
    auto tokens = std::vector<tstring<CharT>>{};
    auto token = tstring<CharT>{};
    while (std::getline(sstr, token, delimiter))
    {
        if (!token.empty()) tokens.push_back(token);
    }
    return tokens;
}
\end{cpp}
\end{enumerate}

\mySubsubsection{}{How it work...}

实现库中的工具函数的两种选择:

\begin{itemize}
\item
函数将修改使用引用传递的字符串

\item
函数不更改原始字符串,会返回一个新的字符串
\end{itemize}

第二种选择的优势在于保留了原始字符串,这在很多情况下有用。否则,首先需要复制字符串并更改副本。本示例提供的实现采取了第二种方法。

“How to do it...”中首先实现的功能是 \verb|to_upper()| 和 \verb|to_lower()|,可将字符串的内容转换为大写或小写。最简单的方式是使用标准算法 \verb|std::transform()|。这是一个通用算法,将一个函数应用于由开始和结束迭代器定义的范围内的每个元素,并将结果存储在另一个只需指定开始迭代器的范围内。输出范围可以与输入范围相同,应用的函数是 \verb|toupper()| 或 \verb|tolower()|:

\begin{cpp}
auto ut{ string_library::to_upper("this is not UPPERCASE"s) };
// ut = "THIS IS NOT UPPERCASE"
auto lt{ string_library::to_lower("THIS IS NOT lowercase"s) };
// lt = "this is not lowercase"
\end{cpp}

接下来是 \verb|reverse()|,该函数反转字符串的内容,使用标准算法 \verb|std::reverse()|,反转由开始和结束迭代器定义的范围内的元素:

\begin{cpp}
auto rt{string_library::reverse("cookbook"s)}; // rt = "koobkooc"
\end{cpp}

涉及到修剪时,字符串可以在开头、结尾或两端修剪。这里,实现了三种不同的函数:\verb|trim()| 用于修剪两端,\verb|trimleft()| 用于修剪字符串开头,\verb|trimright()| 用于修剪字符串结尾。第一个版本的函数只修剪空格。为了找到要修剪的部分,使用了 \verb|std::basic_string| 的 \verb|find_first_not_of()| 和 \verb|find_last_not_of()|。这两个函数会返回字符串中不是指定字符的第一个和最后一个字符。随后,对 \verb|std::basic_string| 的 \verb|substr()| 方法的调用返回了一个新的字符串。该方法接受字符串中的一个索引和要复制到新字符串中的元素数量:

\begin{cpp}
auto text1{"   this is an example   "s};
auto t1{ string_library::trim(text1) };
// t1 = "this is an example"
auto t2{ string_library::trimleft(text1) };
// t2 = "this is an example   "
auto t3{ string_library::trimright(text1) };
// t3 = "   this is an example"
\end{cpp}

有时,从字符串中修剪其他字符然后是空格可能很有用。需要为修剪函数提供了重载版本,指定了要移除的字符集(该集合作为字符串指定)。实现与先前版本非常相似,\verb|find_first_not_of()| 和 \verb|find_last_not_of()| 都有接受包含要排除搜索字符串的重载:

\begin{cpp}
auto chars1{" !%\n\r"s};
auto text3{"!!  this % needs a lot\rof trimming  !\n"s};
auto t7{ string_library::trim(text3, chars1) };
// t7 = "this % needs a lot\rof trimming"
auto t8{ string_library::trimleft(text3, chars1) };
// t8 = "this % needs a lot\rof trimming  !\n"
auto t9{ string_library::trimright(text3, chars1) };
// t9 = "!!  this % needs a lot\rof trimming"
\end{cpp}

如果需要从字符串中移除字符,修剪方法就不够用了,因其仅处理字符串开头和结尾处的一段连续字符。为此,实现了一个简单的 \verb|remove()| 方法,使用了标准算法 \verb|std::remove_if()|。

\verb|std::remove()| 和 \verb|std::remove_if()| 可能一开始看起来不够直观。通过重新排列由起始和结束迭代器定义范围内的内容(使用移动赋值),来移除满足条件的元素。需要移除的元素会放置在范围的末尾,函数会返回一个指向代表已移除元素的范围中首个元素的迭代器。这个迭代器实际上定义了修改后的范围的新终点。如果没有元素移除,返回的迭代器就是原始范围的终点迭代器。此返回迭代器的值随后用来调用 \verb|std::basic_string::erase()| ,该函数真正地擦除由两个迭代器定义的字符串内容。例子中,这两个迭代器分别是 \verb|std::remove_if()| 返回的是迭代器和字符串的终点:

\begin{cpp}
auto text4{"must remove all * from text**"s};
auto t10{ string_library::remove(text4, '*') };
// t10 = "must remove all  from text"
auto t11{ string_library::remove(text4, '!') };
// t11 = "must remove all * from text**"
\end{cpp}

最后实现的是 \verb|split()|,根据指定的分隔符拆分字符串的内容。实现这个功能有多种方法,可使用 \verb|std::getline()| 函数。这个函数从输入流中读取字符直到遇到指定的分隔符,并将字符放入一个字符串中。

开始从输入缓冲区读取之前,会在输出字符串上调用 \verb|erase()| 以清空其内容。循环中调用此方法会产生放入vector中的token,结果会跳过空格:

\begin{cpp}
auto text5{"this text will be split   "s};
auto tokens1{ string_library::split(text5, ' ') };
// tokens1 = {"this", "text", "will", "be", "split"}
auto tokens2{ string_library::split(""s, ' ') };
// tokens2 = {}
\end{cpp}

这里展示了两个文本拆分的例子。第一个例子中,\verb|text5| 变量中的文本拆分为单词,正如前面提到的,忽略空标记。第二个例子中,拆分空字符串会产生一个空的 \verb|token| vector。

\mySubsubsection{}{There's more...}

较新的标准中,为了帮助用户避免定义一些广泛使用的函数,已经在 \verb|std::basic_string| 类模板中添加了几个辅助方法。方法列在下表中:

% Please add the following required packages to your document preamble:
% \usepackage{longtable}
% Note: It may be necessary to compile the document several times to get a multi-page table to line up properly
\begin{longtable}{|l|l|l|}
\hline
\textbf{函数} & \textbf{C++ 版本} & \textbf{描述}                                       \\ \hline
\endfirsthead
%
\endhead
%
starts\_with      & C++20                & 检查字符串是否以指定的前缀开始 \\ \hline
ends\_with        & C++20                & 检查字符串是否以指定的后缀结束   \\ \hline
contains          & C++23                & 检查字符串是否包含指定的子字符串 \\ \hline
\end{longtable}

\begin{center}
表 2.13: 新增的字符串函数(basic\_string的成员函数)
\end{center}

这些成员函数的使用在下面的代码中有所举例:

\begin{cpp}
std::string text = "The Lord of the Rings";
if(text.starts_with("The")) {}
if(text.ends_with("Rings")) {}
if(text.contains("Lord")) {}
\end{cpp}

