

字面量是不能更改的内置常量类型(数值型、布尔型、字符型、字符串型和指针型),C++定义了一系列前缀和后缀来指定字面量(前缀/后缀是字面量的一部分)。C++11 允许通过定义称为字面量操作符的函数来创建自定义字面量,这些操作符引入了用于指定字面量的后缀,功能仅适用于数值、字符和字符串类型。

C++允许开发者创建自定义字面量。本示例中,将介绍如何创建自定义“熟”字面量。

\mySubsubsection{}{Getting ready}

用户自定义字面量可以有两种形式:生(raw)和熟(cooked)。生字面量不会编译器处理,而熟字面量则是由编译器处理过的值(例如,处理字符串中的转义序列或识别出整数 2898 来自字面量 0xBAD)。生字面量仅对整数和浮点数类型可用,而熟字面量也适用于字符和字符串字面量。

\mySubsubsection{}{How to do it...}

要创建熟自定义字面量,应遵循以下步骤:

\begin{enumerate}
\item
在单独的命名空间中定义字面量以避免名称冲突。

\item
始终以前缀下划线(\verb|_|)开始用户自定义后缀。

\item
为熟字面量定义一种以下形式的字面量操作符( \verb|char8_t| 自 C++20 起可用)。注意,在下面的列表中,T 不是指代类型的模板参数,而只是操作符返回类型的占位符:

\begin{cpp}
T operator "" _suffix(unsigned long long int);
T operator "" _suffix(long double);
T operator "" _suffix(char);
T operator "" _suffix(wchar_t);
T operator "" _suffix(char8_t); // since C++20
T operator "" _suffix(char16_t);
T operator "" _suffix(char32_t);
T operator "" _suffix(char const *, std::size_t);
T operator "" _suffix(wchar_t const *, std::size_t);
T operator "" _suffix(char8_t const *, std::size_t); // C++20
T operator "" _suffix(char16_t const *, std::size_t);
T operator "" _suffix(char32_t const *, std::size_t);
\end{cpp}

以下示例创建了一个用于指定千字节的用户自定义字面量:

\begin{cpp}
namespace compunits
{
    constexpr size_t operator "" _KB(unsigned long long const size)
    {
        return static_cast<size_t>(size * 1024);
    }
}
auto size{ 4_KB };         // size_t size = 4096;
using byte = unsigned char;
auto buffer = std::array<byte, 1_KB>{};
\end{cpp}
\end{enumerate}

\mySubsubsection{}{How it work...}

编译器遇到带有用户自定义后缀 \verb|_X| 的自定义字面量时(对于第三方后缀,总是以前导下划线开头,没有前导下划线的后缀保留给标准库),会进行未限定名称查找以识别名为 \verb|operator "" _X| 的函数。如果找到,则根据字面量的类型和字面量操作符的类型调用它;否则,编译器将报错。

“How to do it...”的示例中,字面量操作符称为 \verb|operator "" _KB|,并且有一个 \verb|unsigned long long int| 类型的参数。这是字面量操作符处理整数类型时唯一可能的整数类型,对于浮点数自定义字面量,参数类型必须是 \verb|long double|,因为对于数值类型,字面量操作符必须能够处理尽可能大的值。此字面量操作符返回一个 \verb|constexpr| 值,可以在需要编译时常量的地方使用,比如:指定数组的大小。

当编译器识别到自定义字面量,并需要调用相应的用户自定义字面量操作符时,将根据以下规则选择重载集中的重载:

\begin{itemize}
\item
对于整数字面量:按以下顺序调用操作符:\verb|unsigned long long| 的操作符、\verb|const char*| 的原始字面量操作符或字面量操作符模板。

\item
对于浮点数字面量:按以下顺序调用操作符:\verb|long double| 的操作符、\verb|const char*| 的原始字面量操作符或字面量操作符模板。

\item
对于字符字面量:调用适当的操作符,具体取决于字符类型(\verb|char|, \verb|wchar_t|, \verb|char16_t| 和 \verb|char32_t|)。

\item
对于字符串字面量:调用适当的操作符,具体取决于字符串类型,该操作符接受指向字符串的指针及字符串的大小。
\end{itemize}

以下示例中,定义了一套单位和数量系统。希望操作千克、件、升等单位。这在订单处理系统非常有用,可为每一件商品指定数量和单位。

以下是在 \verb|units| 命名空间中定义的类型:

\begin{itemize}
\item
枚举类,用于可能的单位类型(千克、米、升和件):

\begin{cpp}
enum class unit { kilogram, liter, meter, piece, };
\end{cpp}

\item
类型模板,用于指定特定单位的数量(如 3.5 千克或 42 件):

\begin{cpp}
template <unit U>
class quantity
{
    const double amount;
    public:
    constexpr explicit quantity(double const a) : amount(a)
    {}
    explicit operator double() const { return amount; }
};
\end{cpp}

\item
\verb|quantity| 类型模板定义了 \verb|operator+| 和 \verb|operator-|,以对数量进行增减:

\begin{cpp}
template <unit U>
constexpr quantity<U> operator+(quantity<U> const &q1,
                                quantity<U> const &q2)
{
    return quantity<U>(static_cast<double>(q1) +
    static_cast<double>(q2));
}
template <unit U>
constexpr quantity<U> operator-(quantity<U> const &q1,
                                quantity<U> const &q2)
{
    return quantity<U>(static_cast<double>(q1) –
    static_cast<double>(q2));
}
\end{cpp}

\item

\end{itemize}
定义了用于创建 \verb|quantity| 字面量的字面量操作符,这些操作符定义在一个名为 \verb|unit_literals| 的内部命名空间中。这是为了避免与其他命名空间中的字面量发生符号冲突。

如果确实发生了冲突,开发者可以使用适当的命名空间在需要定义字面量的作用域中,选择他们应该使用的字面量:

\begin{cpp}
namespace unit_literals
{
    constexpr quantity<unit::kilogram> operator "" _kg(
    long double const amount)
    {
        return quantity<unit::kilogram> { static_cast<double>(amount) };
    }
    constexpr quantity<unit::kilogram> operator "" _kg(
    unsigned long long const amount)
    {
        return quantity<unit::kilogram> { static_cast<double>(amount) };
    }
    constexpr quantity<unit::liter> operator "" _l(
    long double const amount)
    {
        return quantity<unit::liter> { static_cast<double>(amount) };
    }
    constexpr quantity<unit::meter> operator "" _m(
    long double const amount)
    {
        return quantity<unit::meter> { static_cast<double>(amount) };
    }
    constexpr quantity<unit::piece> operator "" _pcs(
    unsigned long long const amount)
    {
        return quantity<unit::piece> { static_cast<double>(amount) };
    }
}
\end{cpp}

仔细观察可以发现,上面定义的字面量操作符并不相同:

\begin{itemize}
\item
\verb|_kg| 同时为整数和浮点数字面量定义;可以创建像 \verb|1_kg| 和 \verb|1.0_kg| 这样的整数和浮点数值。

\item
\verb|_l| 和 \verb|_m| 仅定义为浮点数字面量;只能为这些单位定义带有浮点数的数量字面量,如 \verb|4.5_l| 和 \verb|10.0_m|。

\item
\verb|_pcs| 仅定义为整数字面量;只能定义整数数量的件数,如 \verb|42_pcs|。
\end{itemize}

有了这些字面量操作符,就可以操作各种数量。以下示例展示了有效和无效的操作:

\begin{cpp}
using namespace units;
using namespace unit_literals;
auto q1{ 1_kg };    // OK
auto q2{ 4.5_kg };  // OK
auto q3{ q1 + q2 }; // OK
auto q4{ q2 - q1 }; // OK
// error, cannot add meters and pieces
auto q5{ 1.0_m + 1_pcs };
// error, cannot have an integer number of liters
auto q6{ 1_l };
// error, can only have an integer number of pieces
auto q7{ 2.0_pcs}
\end{cpp}

q1 是 1 千克的数量,是一个整数值。由于\verb|operator "" _kg(unsigned long long const)|,所以可以从整数 1 正确创建字面量。同样,q2 是 4.5 千克的数量;这是一个实数值。由于 \verb|operator "" _kg(long double)|,所以可以从双精度浮点值 4.5 创建字面量。

另一方面,\verb|q6| 是 1 升的数量。由于没有\verb|operator "" _l(unsigned long long)|,所以无法创建字面量。需要一个接受 \verb|unsigned long long| 的重载,但不存在这样的重载。同样,q7 是 2.0 件的数量,但件数字面量只能从整数值创建,这会产生另一个编译错误。

\mySubsubsection{}{There's more...}

虽然自定义字面量自 C++11 起就可用,但标准字面量操作符仅从 C++14 起可用。后续标准版本中还添加了更多的标准用户自定义字面量。以下是这些标准字面量操作符的列表:

\begin{itemize}
\item
\verb|operator""s| 用于定义 \verb|std::basic_string| 字面量和 \verb|operator""sv|(自 C++17 起)用于定义 \verb|std::basic_string_view| 字面量:

\begin{cpp}
using namespace std::string_literals;
auto s1{  "text"s }; // std::string
auto s2{ L"text"s }; // std::wstring
auto s3{ u"text"s }; // std::u16string
auto s4{ U"text"s }; // std::u32string
using namespace std::string_view_literals;
auto s5{ "text"sv }; // std::string_view
\end{cpp}

\item
\verb|operator""h|, \verb|operator""min|, \verb|operator""s|, \verb|operator""ms|, \verb|operator""us| 和 \verb|operator""ns| 用于创建 \verb|std::chrono::duration| 值:

\begin{cpp}
using namespace std::chrono_literals;
// std::chrono::duration<long long>
auto timer {2h + 42min + 15s};
\end{cpp}

\item
\verb|operator""y| 用于创建 \verb|std::chrono::year| 字面量,\verb|operator""d| 用于创建表示月份中某一天的 \verb|std::chrono::day| 字面量,这两个都是在 C++20 中新增的:

\begin{cpp}
using namespace std::chrono_literals;
auto year { 2020y }; // std::chrono::year
auto day { 15d };    // std::chrono::day
\end{cpp}

\item
\verb|operator""if|, \verb|operator""i| 和 \verb|operator""il| 用于创建 \verb|std::complex<float>|, \verb|std::complex<double>| 和 \verb|std::complex<long double>| 值:

\begin{cpp}
using namespace std::complex_literals;
auto c{ 12.0 + 4.5i }; // std::complex<double>
\end{cpp}
\end{itemize}

标准字面量可在多个命名空间中使用。例如,用于字符串的 ""s 和 ""sv 字面量定义在命名空间 \verb|std::literals::string_literals| 中。

literals 和 string\_literals 都是内联命名空间,可以使用 \verb|using namespace| \verb|std::literals|、\verb|using namespace| \verb|std::string_literals| 或 \verb|using namespace| \verb|std::literals::string_literals| 访问字面量。前面的例子中,选择了第二种形式。















