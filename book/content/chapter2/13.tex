
标准库提供了名为 \verb|std::regex_search()| 的算法,可以将正则表达式与字符串的任何部分匹配,就和 \verb|regex_match()| 一样。然而,这个函数不允许在输入字符串中搜索正则表达式出现的所有情况,所以需要使用标准库中的迭代器。

本示例中,将介绍如何使用正则表达式解析字符串的内容,需要考虑解析包含名称-值对的文本文件的问题,格式为 \verb|name = value|,但以 \# 开头的行表示注释,必须忽略。以下是一个例子:

\begin{shell}
#remove # to uncomment a line
timeout=120
server = 127.0.0.1
#retrycount=3
\end{shell}

\mySubsubsection{}{Getting ready}

接下来的例子中,text 是一个变量:

\begin{cpp}
auto text {
    R"(
    #remove # to uncomment a line
    timeout=120
    server = 127.0.0.1
    #retrycount=3
    )"s};
\end{cpp}

目的是简化代码,实际应用中可能会从文件或其他来源读取文本。

\mySubsubsection{}{How to do it...}

在字符串中搜索正则表达式的出现情况,应该按照以下步骤操作:

\begin{enumerate}
\item
包含头文件 <regex> 和 <string> 以及用于字符串的标准用户定义字面量的命名空间 \verb|std::string_literals|(C++14):

\begin{cpp}
#include <regex>
#include <string>
using namespace std::string_literals;
\end{cpp}

\item
使用原字符串字面量(对于 \verb|std::wregex| 使用原宽字符串字面量)来指定正则表达式,以避免转义反斜杠。以下正则表达式验证了前面提出的数据格式:

\begin{cpp}
auto pattern {R"(^(?!#)(\w+)\s*=\s*([\w\d]+[\w\d._,\-:]*)$)"s};
\end{cpp}

\item
创建 \verb|std::regex/std::wregex| 类型实例(取决于使用的字符集)来封装正则表达式:

\begin{cpp}
auto rx = std::regex{pattern};
\end{cpp}

\item
要搜索给定文本中正则表达式的首次出现,可用通用算法 \verb|std::regex_search()|(示例 1):

\begin{cpp}
auto match = std::smatch{};
if (std::regex_search(text, match, rx))
{
    std::cout << match[1] << '=' << match[2] << '\n';
}
\end{cpp}

\item
要查找给定文本中正则表达式的所有出现,可用迭代器 \verb|std::regex_iterator|(示例 2):

\begin{cpp}
auto end = std::sregex_iterator{};
for (auto it=std::sregex_iterator{ std::begin(text),
                                   std::end(text), rx };
    it != end; ++it)
{
    std::cout << '\'' << (*it)[1] << "'='"
    << (*it)[2] << '\'' << '\n';
}
\end{cpp}

\item
要遍历匹配的所有子表达式,可用迭代器 \verb|std::regex_token_iterator|(示例 3):

\begin{cpp}
auto end = std::sregex_token_iterator{};
for (auto it = std::sregex_token_iterator{
                 std::begin(text), std::end(text), rx };
    it != end; ++it)
{
    std::cout << *it << '\n';
}
\end{cpp}
\end{enumerate}

\mySubsubsection{}{How it work...}

可以解析前面显示输入文件的简单正则表达式:

\begin{shell}
^(?!#)(\w+)\s*=\s*([\w\d]+[\w\d._,\-:]*)$
\end{shell}

此正则表达式会忽略所有以 \# 开头的行;对于那些不以 \# 开头的行,匹配一个名字后面跟着等号,然后是一个值,该值可以由字母数字字符和其他几个字符(下划线、点、逗号等)组成。该正则表达式的具体含义解释如下

% Please add the following required packages to your document preamble:
% \usepackage{longtable}
% Note: It may be necessary to compile the document several times to get a multi-page table to line up properly
\begin{longtable}{|l|l|}
\hline
\textbf{部分}        & \textbf{描述}                                                                   \\ \hline
\endfirsthead
%
\endhead
%
\textasciicircum{}   & 行首                                                                          \\ \hline
(?!\#)               & 负向前瞻,确保不能匹配 \# 字符 \\ \hline
(\textbackslash{}w)+ & 捕获组,至少包含一个单词字符的标识符          \\ \hline
\textbackslash{}s*   & 任意空白字符                                                                       \\ \hline
=                    & 等号                                                                            \\ \hline
\textbackslash{}s*   & 任意空白字符                                                                       \\ \hline
({[}\textbackslash{}w\textbackslash{}d{]}+{[}\textbackslash{}w\textbackslash{}d.\_,\textbackslash{}-:{]}*) &
\begin{tabular}[c]{@{}l@{}}捕获组,以字母数字字符开头,也可以包含点、\\逗号、反斜杠、破折号、冒号或下划线的值 \end{tabular} \\ \hline
\$                   & 行尾                                                                             \\ \hline
\end{longtable}

\begin{center}
表 2.15: 解析正则表达式
\end{center}

可以使用 \verb|std::regex_search()| 在输入文本中搜索匹配项。这个算法有多个重载版本,工作方式相同。必须指定要处理的字符范围、一个输出 \verb|std::match_results| 类型实例(将包含匹配结果)、一个表示正则表达式及其匹配标志的 \verb|std::basic_regex| 类型实例(标志定义了搜索方式)。如果找到匹配,则函数返回 true;否则返回 false。

前面部分的第一个示例(见第四条列表项)中,match 是 \verb|std::smatch| 的一个类型实例,\verb|std::smatch| 是 \verb|std::match_results| 的类型别名,模板类型为 \verb|string::const_iterator|。如果找到了匹配项,此实例将在一系列值中包含所有匹配子表达式的匹配信息;索引 0 处的子匹配始终是整个匹配。

索引 1 处的子匹配是第一个匹配的子表达式,索引 2 处的子匹配是第二个匹配的子表达式,依此类推。正则表达式中有两个捕获组(即子表达式),在成功的情况下,\verb|std::match_results| 将会有三个子匹配。表示名称的标识符位于索引 1,等号后的值位于索引 2,所以这段代码会输出以下内容:

\myGraphic{0.3}{content/chapter2/images/10.png}{图 2.9: 第一个示例的输出}

\verb|std::regex_search()| 算法无法遍历文本中所有可能的匹配项,需要使用迭代器,就是 \verb|std::regex_iterator| 的作用所在。不仅可以遍历所有的匹配项,还能访问匹配项的所有子匹配项。

迭代器是在构造时和每次递增时调用 \verb|std::regex_search()|,调用结果为 \verb|std::match_results|。默认构造函数创建一个表示序列末尾的迭代器,可以用来测试何时应停止遍历匹配项。

前面部分的第二个示例(见第五条列表项)中,首先创建一个序列末尾迭代器,然后开始遍历所有可能的匹配项。构造时会调用 \verb|std::regex_match()|,如果找到匹配项,可以通过当前迭代器访问其结果。这将持续进行直到找不到匹配项(即到达序列末尾):

\myGraphic{0.3}{content/chapter2/images/11.png}{图 2.10: 第二个示例的输出}

\verb|std::regex_iterator| 的替代方案是 \verb|std::regex_token_iterator|。工作原理类似于 \verb|std::regex_iterator|,实际上内部包含了一个这样的迭代器,能够访问匹配项中的特定子表达式。 “How to do it...”的第三个示例(见第六条列表项)中已经有所展示。首先创建一个序列末尾迭代器,然后循环遍历匹配项,直到达到序列末尾。构造函数中,没有指定通过迭代器访问的子表达式的索引,所以默认值为 0。此程序将输出所有的匹配项:

\myGraphic{0.3}{content/chapter2/images/12.png}{图 2.11: 第三个示例的输出}

如果只想访问第一个子表达式(这种情况下就是名称),只需要在token迭代器的构造函数中指定子表达式的索引即可:

\begin{cpp}
auto end = std::sregex_token_iterator{};
for (auto it = std::sregex_token_iterator{ std::begin(text),
               std::end(text), rx, 1 };
    it != end; ++it)
{
    std::cout << *it << '\n';
}
\end{cpp}

这次,得到的输出只包含名称,如下图所示:

\myGraphic{0.3}{content/chapter2/images/13.png}{图 2.12: 输出仅包含名称}

关于token迭代器的一个有趣之处在于,如果子表达式的索引为 -1,会返回字符串中未匹配的部分,返回一个对应的 \verb|std::match_results| 类型实例,代表上次匹配与序列结束间的字符序列:

\begin{cpp}
auto end = std::sregex_token_iterator{};
for (auto it = std::sregex_token_iterator{ std::begin(text),
               std::end(text), rx, -1 };
    it != end; ++it)
{
    std::cout << *it << '\n';
}
\end{cpp}

这个程序将输出以下内容:

\myGraphic{0.4}{content/chapter2/images/14.png}{图 2.13: 包括空行的输出}

注意,输出中的空行对应于空token。









